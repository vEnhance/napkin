\chapter{Multivariable Calculus Done Correctly}
As I have ranted about before, linear algebra is done wrong
by the extensive use of matrices to obscure the structure of a linear map.
Similar problems occur with multivariable calculus, so here I would like to set 
the record straight.

Once that's done, I will tell you what a differential form is,
and you'll finally know all those stupid $dx$'s and $dy$'s really mean.
(They weren't just there for decoration!)

Since we are doing this chapter using morally correct linear algebra,
it's imperative you're comfortable with linear maps,
and in particular the dual space $V^\vee$ which we will repeatedly use.

\section{Preliminaries}
\prototype{$V = \RR^n$, and $\norm{(x_1, \dots, x_n)} = \sqrt{x_1^2 + \dots + x_n^2}$.}
In this chapter, all vector spaces are \vocab{normed} and finite-dimensional over $\RR$.
By ``normed'' I mean there is an ``absolute value'' function $\norm{-}_V : V \to \RR_{\ge 0}$
satisfying the triangle inequality and scaling by constants: 
\[
	\norm{v_1+v_2}_V \le \norm{v_1}_V+\norm{v_2}_V
	\qquad\text{and}\qquad
	\norm{cv}_V = c\norm{v}_V \quad\text{(for $c \ge 0$)}.
\]
In particular, $\norm{0_V}_V = 0$.

The norm can be used as a metric on $V$ (by taking $d(x,y) = \norm{x-y}$),
thus we can talk about continuous maps,  open sets, etc.
The typical example, of course, is $V = \RR^n$ as above.


\section{The Total Derivative}
\prototype{If $f(x,y) = x^2+y^2$, then $(Df)_{(x,y)} = 2x\ee_1^\vee + 2y\ee_2^\vee$.}
First, let $f : (a,b) \to \RR$.
You might recall from high school calculus that for every point $p \in \RR$,
we defined $f'(p)$ as the derivative at the point $p$ (if it existed), which we interpreted as the \emph{slope} of
the ``tangent line''.

\begin{center}
	\begin{asy}
		import graph;
		size(150,0);

		real f(real x) {return 3-2/(x+2.5);}
		graph.xaxis("$x$");
		graph.yaxis();
		draw(graph(f,-2,2,operator ..), heavygray, Arrows);

		real p = -1;
		real h = 1000 * (f(p+0.001)-f(p));
		real r = 0.9;
		draw( (p+r,f(p)+r*h)--(p-r,f(p)-r*h), red);
		dot( (p, f(p)) );
		draw( (p, f(p))--(p,0), dashed);
		dot("$p$", (p, 0), dir(-90));
		label("$f'(p)$", (p+r/2, f(p) + h*r/2), dir(115));
	\end{asy}
\end{center}

That's fine, but I claim that the ``better'' way to interpret
the derivative at that point is as a \emph{linear map},
that is, as a \emph{function}.
If $f'(p) = 1.5$,
then the derivative tells me that if I move $\eps$ away from $p$
then I should expect $f$ to change by about $1.5\eps$.
In other words,
\begin{moral}
The derivative of $f$ at $p$ approximates $f$ near $p$ by a \emph{linear function}.
\end{moral}

What about more generally?
Suppose I have a function like $f : \RR^2 \to \RR$, say 
\[ f(x,y) = x^2+y^2 \]
for concreteness or something.
For a point $p \in \RR^2$, the ``derivative'' of $f$ at $p$ ought to represent a linear map
that approximates $f$ at that point $p$.
That means I want a linear map $T : \RR^2 \to \RR$ such that
\[ f(p + v) \approx f(p) + T(v) \]
for small displacements $v \in \RR^2$.

Even more generally, if $f : U \to W$ with $U \subseteq V$ open,
then the derivative at $p \in U$ ought to be so that
\[ f(p + v) \approx f(p) + T(v) \in W. \]
(We need $U$ open so that for small enough $v$, $p+v \in U$ as well.)
In fact this is exactly what we're doing earlier with $f'(p)$ in high school.

\missingfigure{2D image}

The only difference is that, by an unfortunate coincidence,
a linear map $\RR \to \RR$ can be represented by just its slope.
And in the unending quest to make everything a number so that it can be AP tested,
we immediately forgot all about what we were trying to do in the first place
and just defined the derivative of $f$ to be a \emph{number} instead of a \emph{function}.

\begin{moral}
	The fundamental idea of Calculus is the local approximation of functions by linear functions.
	The derivative does exactly this.
\end{moral}
Jean Dieudonn\'e as quoted in \cite{ref:pugh} continues:
\begin{quote}
	In the classical teaching of Calculus, this idea is immediately obscured
	by the accidental fact that, on a one-dimensional vector space,
	there is a one-to-one correspondence between linear forms and numbers,
	and therefore the derivative at a point is defined as a number instead of a linear form.
	This \textbf{slavish subservience to the shibboleth of numerical interpretation at any cost}
	becomes much worse . . .
\end{quote}

So let's do this right.
The only thing that we have to do is say what ``$\approx$'' means, and for
this we use the norm of the vector space.
\begin{definition}
	Let $U \subseteq V$ be open.
	Let $f : U \to W$ be a continuous function, and $p \in U$.
	Suppose there exists a linear map $T : V \to W$ such that
	\[
		\lim_{\norm{v} \to 0}
		\frac{\norm{f(p + v) - f(p) - T(v)}_W}{\norm{v}_V} = 0.
	\]
	Then $T$ is the \vocab{total derivative} of $f$ at $p$.
	We denote this by $(Df)_p$, and say $f$ is \vocab{differentiable at $p$}.

	If $(Df)_p$ exists at every point, we say $f$ is \vocab{differentiable}.
\end{definition}

\begin{ques}
	Check that $V = W = \RR$, this is equivalent to the single-variable definition.
	(What are the linear maps from $V$ to $W$?)
\end{ques}
\begin{example}[Total Derivative of $f(x,y) = x^2+y^2$]
	Let $V = \RR^2$ with standard basis $\ee_1$, $\ee_2$ and let $W = \RR$,
	and let $f\left( x \ee_1 + y \ee_2 \right) = x^2+y^2$.  Let $p = a\ee_1 + b\ee_2$.
	Then, we claim that \[ (Df)_p : \RR^2 \to \RR \quad\text{by}\quad
	v \mapsto 2a \cdot \ee_1^\vee(v) + 2b \cdot \ee_2^\vee(v). \]
\end{example}
Here, the notation $\ee_1^\vee$ and $\ee_2^\vee$ makes sense,
because by definition $(Df)_p \in V^\vee$: these are functions from $V$ to $\RR$!

Let's check this manually with the limit definition.
Set $v = xe_1 + ye_2$, and note that the norm on $V$ is $\norm{(x,y)}_V = \sqrt{x^2+y^2}$
while the norm on $W$ is just the absolute value $\norm{c}_W = \left\lvert c \right\rvert$.
Then we compute
\begin{align*}
	\frac{\norm{f(p + v) - f(p) - T(v)}_W}{\norm{v}_V} 
	&= \frac{\left\lvert (a+x)^2 + (b+y)^2 - (a^2+b^2) - (2ax+2by) \right\rvert}{\sqrt{x^2+y^2}} \\
	&= \frac{x^2+y^2}{\sqrt{x^2+y^2}} = \sqrt{x^2+y^2} \\
	&\to 0
\end{align*}
as $\norm{v} \to 0$.
Thus, for $p = ae_1 + be_2$ we indeed have $(Df)_p = 2a \cdot e_1^\vee + 2b \cdot e_2^\vee$.

\begin{remark}
	As usual, differentiability implies continuity.
\end{remark}

\section{The Projection Principle}
Before proceeding I need to say something really important.
\begin{theorem}[Projection Principle]
	\label{thm:project_principle}
	Let $W$ be an $n$-dimensional real vector space with basis $w_1, \dots, w_n$.
	Then there is a bijection between continuous functions $f : V \to W$ and
	$n$-tuples of continuous $f_1, f_2, \dots, f_n : V \to \RR$
	by projection onto the $i$th basis element, i.e.\ 
	\[ f(v) = f_1(v)w_1 + \dots + f_n(v)w_n. \]
\end{theorem}
\begin{proof}
	Obvious.
\end{proof}
The theorem remains true if one replaces ``continuous'' by ``differentiable'', ``smooth'', ``arbitrary'',
or most other reasonable words. Translation:
\begin{moral}
To think about a function $f : V \to \RR^{n}$,
it suffices to think about each coordinate separately.
\end{moral}
For this reason, we'll most often be interested in functions $f : V \to \RR$.
That's why the dual space $V^\vee$ is so important.

\section{Total and Partial Derivatives}
\prototype{If $f(x,y) = x^2+y^2$, then $(Df) : (x,y) \mapsto 2x\ee_1^\vee + 2y\ee_2^\vee$, and
$\fpartial fx = 2x$, $\fpartial fy = 2y$.}
Let $U \subseteq V$ be open and let $V$ have a basis $e_1$, \dots, $e_n$.
Suppose $f : U \to \RR$ is a function which is differentiable everywhere,
meaning $(Df)_p \in V^\vee$ exists for every $p$.
In that case, one can consider $Df$ as \emph{itself} a function:
\begin{align*}
	Df : U &\to V^\vee \\
	p &\mapsto (Df)_p.
\end{align*}
This is a little crazy: to every \emph{point} in $U$ we associate a \emph{function} in $V^\vee$.
We say $Df$ is the \vocab{total derivative} of $f$ at $p$,
to reflect how much information we're dealing with.

Let's apply the projection principle now to $Df$.
Since we picked a basis $e_1$, \dots, $e_n$ of $V$,
there is a corresponding dual basis
$e_1^\vee$, $e_2^\vee$, \dots, $e_n^\vee$.
The Projection Principle tells us that $Df$ can thus be thought of as just $n$ functions, so we can write
\[ Df = \psi_1 e_1^\vee + \dots + \psi_n e_n^\vee.  \]
In fact, we can even describe what the $\psi_i$ are.
\begin{definition}
	The \vocab{$i^{\text{th}}$ partial derivative} of $f : U \to \RR$, denoted 
	\[ \fpartial{f}{e_i}: U \to \RR \]
	is defined by
	\[
		\fpartial{f}{e_i} (p)
		\defeq \lim_{t \to 0} \frac{f(p + te_i) - f(v)}{t}.
	\]
\end{definition}
You can think of it as ``$f'$ along $e_i$''.
\begin{ques}
	Check that if $Df$ exists, then \[ (Df)_p(e_i) = \fpartial{f}{e_i}(p). \]
\end{ques}
\begin{remark}
	Of course you can write down a definition of $\fpartial{f}{v}$
	for any $v$ (rather than just the $e_i$).
\end{remark}

From the above remarks, we can derive that
\[
	\boxed{
	Df =
	\frac{\partial f}{\partial e_1} \cdot e_1^\vee
	+ \dots + 
	\frac{\partial f}{\partial e_n} \cdot e_n^\vee .
	}
\]
and so given a basis of $V$, we can think of $Df$ as just
the $n$ partials.
\begin{remark}
Keep in mind that each $\frac{\partial f}{\partial e_i}$ is a function from $U$ to the \emph{reals}.
That is to say,
\[
	(Df)_p =
	\underbrace{\frac{\partial f}{\partial e_1}(p)}_{\in \RR} \cdot e_1^\vee
	+ \dots + 
	\underbrace{\frac{\partial f}{\partial e_n}(p)}_{\in \RR} \cdot e_n^\vee
	\in V^\vee.
\]
\end{remark}


\begin{example}[Partial Derivatives of $f(x,y) = x^2+y^2$]
	Let $f : \RR^2 \to \RR$ by $(x,y) \mapsto x^2+y^2$.
	Then in our new language, 
	\[ Df : (x,y) \mapsto 2x \cdot \ee_1^\vee + 2y \cdot \ee_2^\vee. \]
	Thus the partials are
	\[
		\frac{\partial f}{\partial x} : (x,y) \mapsto 2x \in \RR
		\quad\text{and}\quad
		\frac{\partial f}{\partial y} : (x,y) \mapsto 2y \in \RR
	\]
\end{example}

With all that said, I haven't really said much about how to
find the total derivative itself.
For example, if I told you
\[ f(x,y) = x \sin y + x^2y^4 \]
you might want to be able to compute $Df$ without going through
that horrible limit definition I told you about earlier.

Fortunately, it turns out you already know how to compute partial derivatives,
because you had to take AP Calculus at some point in your life.
It turns out for most reasonable functions, this is all you'll ever need.
\begin{theorem}[Continuous Partials Implies Differentiable]
	Let $U \subset V$ be open and pick any basis $e_1, \dots, e_n$.
	Let $f : U \to W$ and suppose that $\fpartial{f}{e_i}$ is defined
	for each $i$ and moreover is \emph{continuous}.
	Then $f$ is differentiable and given by
	\[ Df = \sum \fpartial{f}{e_i} \cdot e_i^\vee. \]
\end{theorem}
\begin{proof}
	Not going to write out the details, but\dots
	given $v = t_1e_1 + \dots + t_ne_n$,
	the idea is to just walk from $p$ to $p+t_1e_1$, $p+t_1e_1+t_2e_2$, \dots,
	up to $p+t_1e_1+t_2e_2+\dots+t_ne_n = p+v$,
	picking up the partial derivatives on the way.
	Do some calculation.
\end{proof}

\begin{remark}
	The continuous condition cannot be dropped. The function
	\[
			f(x,y)
		=
		\begin{cases}
			\frac{xy}{x^2+y^2} & (x,y) \neq (0,0) \\
			0 & (x,y) = (0,0).
		\end{cases}
	\]
	is the classical counterexample -- the total derivative $Df$ does not exist at zero,
	even though both partials do.
\end{remark}

\begin{example}
	[Actually Computing a Total Derivative]
	Let $f(x,y) = x \sin y + x^2y^4$. Then
	\begin{align*}
		\fpartial fx (x,y) &= \sin y + y^4 \cdot 2x \\
		\fpartial fy (x,y) &= x \cos y + x^2 \cdot 4y^3.
	\end{align*}
	Then $Df = \fpartial fx \ee_1^\vee + \fpartial fy \ee_2^\vee$,
	which I won't bother to write out.
\end{example}

The example $f(x,y) = x^2+y^2$ is the same thing.
That being said, who cares about $x \sin y + x^2y^4$ anyways?

\section{A Word on Higher Derivatives}
Let $U \subseteq V$ be open, and take $f : U \to W$, so that $Df : U \to \Hom(V,W)$.

Well, $\Hom(V,W)$ can also be thought of as a normed vector space in its own right:
it turns out that one can define an operator norm on it by setting
\[ \norm{T} \defeq \sup \left\{ \frac{\norm{T(v)f}_W}{\Norm{v}_W} \mid v \neq 0_V \right\}. \] 
So $\Hom(V,W)$ can be thought of as a normed vector space as well.
Thus it makes sense to write
\[ D(Df) : U \to \Hom(V,\Hom(V,W)) \]
which we abbreviate as $D^2 f$. Dropping all doubt and plunging on,
\[ D^3f : U \to \Hom(V, \Hom(V,\Hom(V,W))). \]
I'm sorry.
As consolation, we at least know that $\Hom(V,W) \simeq V^\vee \otimes W$ in a natural way,
so we can at least condense this as
\[ D^kf : V \to (V^\vee)^{\otimes k} \otimes W \]
rather than writing a bunch of $\Hom$'s.
\begin{remark}
	If $k=2$, $W = \RR$, then $D^2f(v) \in (V^\vee)^{\otimes 2}$,
	so it can be represented as an $n \times n$ matrix, which for some reason is called Hessian.
\end{remark}
The most important property of the second derivative is that
\begin{theorem}
	[Symmetry of $Df^2$]
	Let $f : U \to W$ with $U \subseteq V$ open.
	If $(D^2f)_p$ exists at some $p \in U$, then it is symmetric, meaning
	\[ (D^2f)_p(v_1, v_2) = (D^2f)_p(v_2, v_1). \]
\end{theorem}
I'll just quote this without proof, because double derivatives make my head spin.
An important corollary of this theorem:
\begin{corollary}
	[Clairaut's Theorem: Mixed Partials are Symmetric]
	Let $f : U \to \RR$ with $U \subseteq V$ open.
	For any point $p$ such that the quantities are defined,
	\[
		\frac{\partial}{\partial e_i}
		\frac{\partial}{\partial e_j}
		f(p)
		=
		\frac{\partial}{\partial e_j}
		\frac{\partial}{\partial e_i}
		f(p).
	\]
\end{corollary}

\section{Exterior Derivatives}
Let's now get a handle on what $dx$ means. This is something called a \emph{differential form}.
Fix a real vector space $V$ of dimension $n$, and let $\ee_1$, \dots, $\ee_n$ be a standard basis.

First, given a function $f : V \to \RR$,
we define 
\[ df \defeq Df = \sum_i f_i e_i^\vee \]
In particular, suppose $V = \RR^n$ and $f(x_1, \dots, x_n) = x_1$ (i.e.\ $f = \ee_1^\vee$). Then:
\begin{ques}
	Show that \[ d(\ee_1^\vee) = \ee_1^\vee. \]
\end{ques}
\begin{abuse}
	Unfortunately, someone somewhere decided it would be a good idea to use ``$x_1$'' to denote $\ee_1^\vee$
	(because \emph{obviously}\footnote{Sarcasm.} $x_1$ means
	``the function that takes $(x_1, \dots, x_n) \in \RR^n$ to $x_1$'')
	and thus decided that \[ dx_1 \defeq \ee_1^\vee. \]
	This notation is so entrenched that I have no choice but to grudgingly accept it.
	\label{abuse:dx}
\end{abuse}
\begin{remark}
	This is the reason why we use the notation $\frac{df}{dx}$ in calculus now:
	given, say $f : \RR \to \RR$ by $f(x) = x^2$, it is indeed true that
	\[ df = 2x \cdot \ee_1^\vee = 2x \cdot dx. \]
\end{remark}

More generally, $df$ is in a class of objects called a $k$-form.
\begin{definition}
	We define a \vocab{differential $k$-form} $\alpha$ to be a smooth (infinitely differentiable)
	map $\alpha : V \to \Lambda^k(V^\vee)$.
	(Here $\Lambda^k(V^\vee)$ is the wedge product.)
\end{definition}
Like with $Df$, we'll use $\alpha_p$ instead of $\alpha(p)$.

\begin{example}
	[$k$-forms for $k=0,1$]
	\listhack
	\begin{enumerate}[(a)]
		\item A $0$-form is just a function $V \to \RR$.
		\item A $1$-form is a function $V \to V^\vee$.
		For example, the total derivative $df$ of a function $V \to \RR$ is a $1$-form.
		\item Let $V = \RR^3$ with standard basis $\ee_1$, $\ee_2$, $\ee_3$.
		Then a typical $2$-form is given by
		\[
			\alpha_p
			=
			f(p) \ee_1^\vee \wedge \ee_2^\vee
			+ g(p) \ee_1^\vee \wedge \ee_3^\vee
			+ h(p) \ee_2^\vee \wedge \ee_3^\vee
			\in \Lambda^2(V).
		\]
	\end{enumerate}
\end{example}

Now, by the projection principle (\Cref{thm:project_principle}) we only have to specify
a function on each of $\binom nk$ basis elements of $\Lambda^k(V^\vee)$.
So, take any basis $\{e_i\}$ of $V$, and 
take the usual basis for $\Lambda^k(V^\vee)$ of elements
\[ e_{i_1}^\vee \wedge e_{i_2}^\vee \wedge \dots \wedge e_{i_k}^\vee. \]
Thus, a general $k$-form takes the shape
\[ \alpha_p = \sum_{1 \le i_1 < \dots < i_k \le n} 
	f_{i_1, \dots, i_k}(p)
	e_{i_1}^\vee \wedge e_{i_2}^\vee \wedge \dots \wedge e_{i_k}^\vee. \]
Since this is a huge nuisance to write, we will abbreviate this to just
\[ \alpha = \sum_I f_I de_I \]
where we understand the sum runs over $I = (i_1, \dots, i_k)$,
and $de_I$ represents $e_{i_1}^\vee \wedge \dots \wedge e_{i_k}^\vee$.

Now, we saw that we can take a $0$-form $f : V \to \RR$ to a $1$-form $df : V \to V^\vee$.
More generally, we can define the \vocab{exterior derivative} in terms of our basis $e_1$, \dots, $e_n$ as follows:
if $\alpha = \sum_I f_I de_I$ then we set
\[ d\alpha \defeq \sum_I df_I \wedge de_I = \sum_I \sum_j \fpartial{f_I}{e_j} de_j \wedge de_I. \]
(We'll show this doesn't depend on the choice of basis in a moment.)
\begin{example}[Computing some exterior derivatives]
	Let $V = \RR^3$ with standard basis $\ee_1$, $\ee_2$, $\ee_3$.
	Let $f(x,y,z) = x^4 + y^3 + 2xz$.
	Then we compute
	\[ df = Df = (4x^3+2z) \; dx + 3y^2 \; dy + 2x \; dz. \]
	Next, we can evaluate $d(df)$ as prescribed: it is
	\begin{align*}
		d^2f &= (12x^2 \; dx + dz) \wedge dx + (6y \; dy) \wedge dy + (dx \wedge dz) \\
		&= 12x^2 (dx \wedge dx) + 2(dz \wedge dx) + 6y (dy \wedge dy) + 2(dx \wedge dz) \\
		&= 2(dz \wedge dx) + 2(dx \wedge dz) \\
		&= 0.
	\end{align*}
	So surprisingly, $d^2f$ is the zero map.
	Here, we have exploited \Cref{abuse:dx} for the first time,
	in writing $dx$, $dy$, $dz$.
\end{example}
And in fact, this is always true in general:
\begin{theorem}[Exterior Derivative Vanishes]
	Let $\alpha$ be any $k$-form.
	Then $d^2(\alpha) = 0$.
	Even more succinctly, $d^2 = 0$.
	\label{thm:dd_zero}
\end{theorem}
The proof is left as \Cref{prob:dd_zero}.

Here are some other properties of $d$:
\begin{itemize}
	\ii As we just saw, $d^2 = 0$.
	\ii It is bilinear; $d(c\alpha+\beta) = c \cdot d\alpha + d\beta$.
	\ii For a $k$-form $\alpha$ and $\ell$-form $\beta$, one can show that
	\[ d(\alpha \wedge \beta) = d\alpha \wedge \beta + (-1)^k (\alpha \wedge d\beta). \]
\end{itemize}
In fact, one can show that the $df$ as defined in terms of bases above is
the \emph{unique} map sending $k$-forms to $(k+1)$-forms.
Actually, one way to define $df$ is to take as axioms the bulleted properties above
and then declare $d$ to be the unique solution to this functional equation.
In any case, this tells us that our definition of $d$ does not depend on the basis chosen.

\section{Problems}
\begin{problem}[Chain Rule]
	Let $V \taking f W \taking g X$ be differentiable maps between normed vector spaces,
	and let $h = g \circ f$.
	Prove the Chain Rule: for any point $p \in V$, we have
	\[ (Dh)_p = (Dg)_{f(p)} \circ (Df)_p. \]
\end{problem}
\begin{problem}
	\label{prob:dd_zero}
	Establish \Cref{thm:dd_zero}, which states that $d^2 = 0$.
\end{problem}
\begin{problem}
	Let $U \subseteq V$ be open, and $f : U \to \RR$ be differentiable $k$ times.
	Show that $(D^kf)_p$ is symmetric in its $k$ arguments, meaning for any $v_1, \dots, v_k \in V$
	and any permutation $\sigma$ on $\left\{ 1, \dots, k \right\}$ we have
	\[ (D^kf)_p(v_1, \dots, v_k) = (D^kf)_p(v_{\sigma(1)}, \dots, v_{\sigma(k)}). \]
\end{problem}


