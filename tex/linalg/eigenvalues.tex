\chapter{Eigen-things}
This chapter will develop the theory of eigenvalues and eigenvectors,
the so-called ``Jordan canonical form''.
(Later on we will use it to
define the characteristic polynomial.)

\section{Why you should care}
We know that a square matrix $T$ is really just
a linear map from $V$ to $V$.
What's the simplest type of linear map?
It would just be multiplication by some scalar $\lambda$,
which would have associated matrix (in any basis!)
\[
	T =
	\begin{bmatrix}
		\lambda & 0 & \dots & 0 \\
		0 & \lambda & \dots & 0 \\
		\vdots & \vdots & \ddots & \vdots \\
		0 & 0 & \dots & \lambda
	\end{bmatrix}.
\]
That's perhaps \emph{too} simple, though.
If we had a fixed basis $e_1, \dots, e_n$
then another very ``simple'' operation
would just be scaling each basis element $e_i$ by $\lambda_i$,
i.e.\ a \vocab{diagonal matrix} of the form
\[
	T = \begin{bmatrix}
		\lambda_1 & 0 & \dots & 0 \\
		0 & \lambda_2 & \dots & 0 \\
		\vdots & \vdots & \ddots & \vdots \\
		0 & 0 & \dots & \lambda_n
	\end{bmatrix}.
\]
These maps are more general.
Indeed, you can, for example, compute $T^{100}$ in a heartbeat:
the map sends $e_1 \to \lambda_1^{100} e_1$.
(Try doing that with an arbitrary $n \times n$ matrix.)

Of course, most linear maps are probably not that nice.
Or are they?
\begin{example}
	[Getting lucky]
	Let $V$ be some two-dimensional vector space
	with $e_1$ and $e_2$ as basis elements.
	Let's consider a map $T \colon V \to V$
	by $e_1 \mapsto 2e_1$ and $e_2 \mapsto e_1+3e_2$,
	which you can even write concretely as
	\[ T = \begin{bmatrix}
		2 & 1 \\
		0 & 3
	\end{bmatrix} \quad\text{in basis $e_1$, $e_2$}. \]
	This doesn't look anywhere as nice until we realize we can rewrite it as
	\begin{align*}
		e_1 &\mapsto 2e_1 \\
		e_1+e_2 &\mapsto 3(e_1+e_2).
	\end{align*}
	So suppose we change to the basis $e_1$ and $e_1 + e_2$.
	Thus in the new basis,
	\[ T = \begin{bmatrix}
		2 & 0 \\
		0 & 3
	\end{bmatrix} \quad\text{in basis $e_1$, $e_1+e_2$}. \]
	So our completely random-looking map,
	under a suitable change of basis,
	looks like the very nice maps we described before!
\end{example}
In this chapter, we will be \emph{making} our luck,
and we will see that our better understanding of matrices
gives us the right way to think about this.

\section{Warning on assumptions}
Most theorems in this chapter only work for
\begin{itemize}
	\ii finite-dimensional vector spaces $V$,
	\ii over a field $k$ which is \emph{algebraically closed}.
\end{itemize}
On the other hand, the definitions work fine without
these assumptions.

\section{Eigenvectors and eigenvalues}
Let $k$ be a field and $V$ a vector space over it.
In the above example, we saw that there were two very nice
vectors, $e_1$ and $e_1+e_2$, for which $V$ did something very simple.
Naturally, these vectors have a name.
\begin{definition}
	Let $T \colon V \to V$ and $v \in V$ a \emph{nonzero} vector.
	We say that $v$ is an \vocab{eigenvector} if $T(v) = \lambda v$
	for some $\lambda \in k$ (possibly zero, but remember $v \neq 0$).
	The value $\lambda$ is called an \vocab{eigenvalue} of $T$.

	We will sometimes abbreviate
	``$v$ is an eigenvector with eigenvalue $\lambda$''
	to just ``$v$ is a $\lambda$-eigenvector''.
\end{definition}
Of course, no mention to a basis anywhere.

\begin{example}
	[An example of an eigenvector and eigenvalue]
	Consider the example earlier with
	$T = \begin{bmatrix} 2 & 1 \\ 0 & 3 \end{bmatrix}$.
	\begin{enumerate}[(a)]
		\ii Note that $e_1$ and $e_1 + e_2$ are
		$2$-eigenvectors and $3$-eigenvectors.
		\ii Of course, $5e_1$ is also an $2$-eigenvector.
		\ii And, $7e_1 + 7e_2$ is also a $3$-eigenvector.
	\end{enumerate}
\end{example}
So you can quickly see the following observation.
\begin{ques}
	Show that the $\lambda$-eigenvectors, together with $\{0\}$
	form a subspace.
\end{ques}
\begin{definition}
	For any $\lambda$, we define the $\lambda$-\vocab{eigenspace}
	as the set of $\lambda$-eigenvectors together with $0$.
\end{definition}
This lets us state succinctly that
``$2$ is an eigenvalue of $T$
with one-dimensional eigenspace spanned by $e_1$''.

Unfortunately, it's not exactly true that eigenvalues always exist.
\begin{example}[Eigenvalues need not exist]
	Let $V = \RR^2$ and let $T$ be the map
	which rotates a vector by $90\dg$
	around the origin.
	Then $T(v)$ is not a multiple of $v$ for any $v \in V$,
	other than the trivial $v=0$.
\end{example}

However, it is true if we replace $k$ with an
algebraically closed field\footnote{A field is \vocab{algebraically closed}
	if all its polynomials have roots,
	the archetypal example being $\CC$.}.
\begin{theorem}[Eigenvalues always exist over algebraically closed fields]
	Suppose $k$ is an \emph{algebraically closed} field.
	Let $V$ be a finite dimensional $k$-vector space.
	Then if $T \colon V \to V$ is a linear map,
	there exists an eigenvalue $\lambda \in k$.
\end{theorem}
\begin{proof}
	(From \cite{ref:axler})
	The idea behind this proof is to consider ``polynomials'' in $T$.
	For example, $2T^2-4T+5$ would be shorthand for $2T(T(v)) - 4T(v) + 5v$.
	In this way we can consider ``polynomials'' $P(T)$;
	this lets us tie in the ``algebraically closed'' condition.
	These polynomials behave nicely:
	\begin{ques}
		Show that $P(T)+Q(T) = (P+Q)(T)$ and $P(T) \circ Q(T) = (P \cdot Q)(T)$.
	\end{ques}

	Let $n = \dim V < \infty$ and fix any nonzero vector $v \in V$,
	and consider vectors $v$, $T(v)$, \dots, $T^n (v)$.
	There are $n+1$ of them,
	so they must not be linearly dependent for dimension reasons;
	thus there is a nonzero polynomial $P$ such that $P(T)$
	is zero when applied to $v$.
	WLOG suppose $P$ is a monic polynomial,
	and thus $P(z) = (z-r_1)\dots(z-r_m)$ say.
	Then we get
	\[ 0 = (T - r_1 \id) \circ (T - r_2 \id) \circ \dots
		\circ (T - r_m \id)(v) \]
	which means at least one of $T - r_i \id$ is not injective,
	i.e.\ has a nontrivial kernel,
	which is the same as an eigenvector.
\end{proof}
So in general we like to consider algebraically closed fields.
This is not a big loss:
any real matrix can be interpreted as a complex matrix
whose entries just happen to be real, for example.

\section{The Jordan form}
So that you know exactly where I'm going,
here's the main theorem.
\begin{definition}
	A \vocab{Jordan block} is an $n \times n$ matrix of the following shape:
	\[
		\begin{bmatrix}
			\lambda & 1 & 0 & 0 & \dots & 0 & 0 \\
			0 & \lambda & 1 & 0 & \dots & 0 & 0 \\
			0 & 0 & \lambda & 1 & \dots & 0 & 0 \\
			0 & 0 & 0 & \lambda & \dots & 0 & 0 \\
			\vdots & \vdots & \vdots & \vdots & \ddots & \vdots & \vdots \\
			0 & 0 & 0 & 0 & \dots & \lambda & 1 \\
			0 & 0 & 0 & 0 & \dots & 0 & \lambda
		\end{bmatrix}.
	\]
	In other words, it has $\lambda$ on the diagonal,
	and $1$ above it.
	We allow $n = 1$,
	so $\begin{bmatrix} \lambda \end{bmatrix}$ is a Jordan block.
\end{definition}

\begin{theorem}
	[Jordan canonical form]
	Let $T \colon V \to V$ be a linear map
	of finite-dimensional vector spaces
	over an algebraically closed field $k$.
	Then we can choose a basis of $V$
	such that the matrix $T$ is ``block-diagonal''
	with each block being a Jordan block.

	Such a matrix is said to be in \vocab{Jordan form}.
	This form is unique up to rearranging the order of the blocks.
\end{theorem}
As an example, this means the matrix should look something like:
\[
	\begin{bmatrix}
		\lambda_1 & 1 \\
		0 & \lambda_1 \\
		&& \lambda_2 \\
		&&& \lambda_3 & 1 & 0 \\
		&&& 0 & \lambda_3 & 1 \\
		&&& 0 & 0 & \lambda_3 \\
		&&&&&& \ddots \\
		&&&&&&& \lambda_m & 1 \\
		&&&&&&& 0 & \lambda_m
	\end{bmatrix}
\]
\begin{ques}
	Check that diagonal matrices are the special case
	when each block is $1 \times 1$.
\end{ques}

What does this mean?
Basically, it means \emph{our dream is almost true}.
What happens is that $V$ can get broken down as a direct sum
\[ V = J_1 \oplus J_2 \oplus \dots \oplus J_m \]
and $T$ acts on each of these subspaces independently.
These subspaces correspond to the blocks in the matrix above.
In the simplest case, $\dim J_i = 1$,
so $J_i$ has a basis element $e$
for which $T(e) = \lambda_i e$;
in other words, we just have a simple eigenvalue.
But on occasion, the situation is not quite so simple,
and we have a block of size greater than $1$;
this leads to $1$'s just above the diagonals.

I'll explain later how to interpret the $1$'s,
when I make up the word \emph{descending staircase}.
For now, you should note that even if $\dim J_i \ge 2$,
we still have a basis element
which is an eigenvector with eigenvalue $\lambda_i$.

\begin{example}
	[A concrete example of Jordan form]
	Let $T : k^6 \to k^6$ and suppose $T$ is given by the matrix
	\[
		T = \begin{bmatrix}
			5 & 0 & 0 & 0 & 0 & 0 \\
			0 & 2 & 1 & 0 & 0 & 0 \\
			0 & 0 & 2 & 0 & 0 & 0 \\
			0 & 0 & 0 & 7 & 0 & 0 \\
			0 & 0 & 0 & 0 & 3 & 0 \\
			0 & 0 & 0 & 0 & 0 & 3 \\
		\end{bmatrix}.
	\]
	Reading the matrix, we can compute all the eigenvectors and eigenvalues:
	for any constants $a, b \in k$ we have
	\begin{align*}
		T(a \cdot e_1) &= 5a \cdot e_1 \\
		T(a \cdot e_2) &= 2a \cdot e_2 \\
		T(a \cdot e_4) &= 7a \cdot e_4 \\
		T(a \cdot e_5 + b \cdot e_6) &= 3\left[ a \cdot e_5 + b \cdot e_6 \right].
	\end{align*}
	The element $e_3$ on the other hand,
	is not an eigenvalue since $T(e_3) = e_2 + 2e_3$.
\end{example}

%Here's the idea behind the proof.
%We would like to be able to break down $V$ directly into components as above,
%but some of the $\lambda_i$'s from the different Jordan blocks could coincide,
%and this turns out to give a technical difficulty for detecting them.
%So instead, we're going to break down $V$ first into subspaces
%based on the values of $\lambda$'s;
%these are called the \emph{generalized eigenspaces}.
%In other words, the generalized eigenspace of a given $\lambda$
%is just all the Jordan blocks which have eigenvalue $\lambda$.
%Only after that will we break the generalized eigenspace
%into the individual Jordan blocks.
%
%The sections below record this proof in the other order:
%the next part deals with breaking generalized
%eigenspaces into Jordan blocks,
%and the part after that is the one where we break down $V$ into
%generalized eigenspaces.

\section{Nilpotent maps}
Bear with me for a moment.  First, define:
\begin{definition}
	A map $T: V \to V$ is \vocab{nilpotent} if $T^m$ is the zero map for some integer $m$.
	(Here $T^m$ means ``$T$ applied $m$ times''.)
\end{definition}
What's an example of a nilpotent map?
\begin{example}
	[The ``descending staircase'']
	Let $V = k^3$ have basis $e_1$, $e_2$, $e_3$.
	Then the map $T$ which sends
	\[ e_3 \mapsto e_2 \mapsto e_1 \mapsto 0 \]
	is nilpotent, since $T(e_1) = T^2(e_2) = T^3(e_3) = 0$,
	and hence $T^3(v) = 0$ for all $v \in V$.
\end{example}
The $3 \times 3$ descending staircase has matrix representation
\[ T = \begin{bmatrix}
		0 & 1 & 0 \\
		0 & 0 & 1 \\
		0 & 0 & 0
	\end{bmatrix}. \]
You'll notice this is a Jordan block.
\begin{exercise}
	Show that the descending staircase above
	has $0$ as its only eigenvalue.
\end{exercise}

That's a pretty nice example.
As another example, we can have multiple such staircases.
\begin{example}
	[Double staircase]
	Let $V = k^{\oplus 5}$ have basis $e_1$, $e_2$, $e_3$, $e_4$, $e_5$.
	Then the map
	\[ e_3 \mapsto e_2 \mapsto e_1 \mapsto 0 \text{ and }
		e_5 \mapsto e_4 \mapsto 0 \]
	is nilpotent.
\end{example}
Picture, with some zeros omitted for emphasis:
\[ T = \begin{bmatrix}
		0 & 1 & 0 &   &   \\
		0 & 0 & 1 &   &   \\
		0 & 0 & 0 &   &   \\
		  &   &   & 0 & 1 \\
		  &   &   & 0 & 0 \\
	\end{bmatrix}
\]
You can see this isn't really that different
from the previous example;
it's just the same idea repeated multiple times.
And in fact we now claim that \emph{all}
nilpotent maps have essentially that form.
\begin{theorem}
	[Nilpotent Jordan]
	Let $V$ be a finite-dimensional vector space
	over an algebraically closed field $k$.
	Let $T \colon V \to V$ be a nilpotent map.
	Then we can write $V = \bigoplus_{i=1}^m V_i$
	where each $V_i$ has a basis of the form
	$v_i$, $T(v_i)$, \dots, $T^{\dim V_i - 1}(v_i)$
	for some $v_i \in V_i$.
\end{theorem}
Hence:
\begin{moral}
	Every nilpotent map can be viewed as independent staircases.
\end{moral}
Each chain $v_i$, $T(v_i)$, $T(T(v_i))$, \dots is just one staircase.
The proof is given later, but first let me point out where this is going.

Here's the punch line.
Let's take the double staircase again.
Expressing it as a matrix gives, say
\[
	S = \begin{bmatrix}
		0 & 1 & 0 &   &   \\
		0 & 0 & 1 &   &   \\
		0 & 0 & 0 &   &   \\
		  &   &   & 0 & 1 \\
		  &   &   & 0 & 0
	\end{bmatrix}.
\]
Then we can compute
\[
	S + \lambda \id = \begin{bmatrix}
		\lambda & 1 & 0 &   &   \\
		0 & \lambda & 1 &   &   \\
		0 & 0 & \lambda &   &   \\
		  &   &   & \lambda & 1 \\
		  &   &   & 0 & \lambda
	\end{bmatrix}.
\]
It's a bunch of $\lambda$ Jordan blocks!
This gives us a plan to proceed: we need to break $V$ into
a bunch of subspaces such that $T - \lambda \id$ is nilpotent over each subspace.
Then Nilpotent Jordan will finish the job.

\section{Reducing to the nilpotent case}
\begin{definition}
	Let $T \colon V \to V$. A subspace $W \subseteq V$
	is called $T$-\vocab{invariant}
	if $T(w) \in W$ for any $w \in W$.
	In this way, $T$ can be thought of as a map $W \to W$.
\end{definition}
In this way, the Jordan form is a decomposition of $V$ into invariant subspaces.

Now I'm going to be cheap, and define:
\begin{definition}
	A map $T \colon V \to V$ is called \vocab{indecomposable}
	if it's impossible to write $V = W_1 \oplus W_2$
	where both $W_1$ and $W_2$ are nontrivial $T$-invariant spaces.
\end{definition}
Picture of a \emph{decomposable} map:
\[
	\begin{bmatrix}
		\multicolumn{2}{c|}{\multirow{2}{*}{$W_1$}} & 0 & 0 & 0  \\
		\multicolumn{2}{c|}{} & 0 & 0 & 0 \\ \hline
		0 & 0 & \multicolumn{3}{|c}{\multirow{3}{*}{$W_2$}} \\
		0 & 0 & \multicolumn{3}{|c}{} \\
		0 & 0 & \multicolumn{3}{|c}{}
	\end{bmatrix}
\]
As you might expect, we can break a space apart into ``indecomposable'' parts.
\begin{proposition}
	[Invariant subspace decomposition]
	Let $V$ be a finite-dimensional vector space.
	Given any map $T \colon V \to V$, we can write
	\[ V = V_1 \oplus V_2 \oplus \dots \oplus V_m \]
	where each $V_i$ is $T$-invariant,
	and for any $i$ the map $T \colon V_i \to V_i$ is indecomposable.
\end{proposition}
\begin{proof}
	Same as the proof that every integer is the product of primes.
	If $V$ is not decomposable, we are done.
	Otherwise, by definition write $V = W_1 \oplus W_2$
	and then repeat on each of $W_1$ and $W_2$.
\end{proof}

Incredibly, with just that we're almost done!
Consider a decomposition as above,
so that $T \colon V_1 \to V_1$ is an indecomposable map.
Then $T$ has an eigenvalue $\lambda_1$, so let $S = T - \lambda_1 \id$; hence $\ker S \neq \{0\}$.
\begin{ques}
	Show that $V_1$ is also $S$-invariant, so we can consider $S : V_1 \to V_1$.
\end{ques}
By \Cref{prob:endomorphism_eventual_lemma}, we have
\[ V_1 = \ker S^N \oplus \img S^N \]
for some $N$.
But we assumed $T$ was indecomposable,
so this can only happen if $\img S^N = \{0\}$ and $\ker S^N = V_1$
(since $\ker S^N$ contains our eigenvector).
Hence $S$ is nilpotent, so it's a collection of staircases.
In fact, since $T$ is indecomposable, there is only one staircase.
Hence $V_1$ is a Jordan block, as desired.

\section{(Optional) Proof of nilpotent Jordan}
The proof is just induction on $\dim V$.
Assume $\dim V \ge 1$, and let $W = T\im(V)$ be the image of $V$.
Since $T$ is nilpotent, we must have $W \subsetneq V$.
Moreover, if $W = \{0\}$ (i.e.\ $T$ is the zero map) then we're already done.
So assume $\{0\} \subsetneq W \subsetneq V$.

By the inductive hypothesis, we can select a good basis of $W$:
\begin{align*}
	\mathcal B' =
	\Big\{ & T(v_1), T(T(v_1)), T(T(T(v_1))), \dots \\
	& T(v_2), T(T(v_2)), T(T(T(v_2))), \dots \\
	& \dots, \\
	& T(v_\ell), T(T(v_\ell)), T(T(T(v_\ell))), \dots \Big\}
\end{align*}
for some $T(v_i) \in W$ (here we have taken advantage of the fact that each element of $W$ is itself of the form $T(v)$ for some $v$).

Also, note that there are exactly $\ell$ elements of $\mathcal B'$ which are in $\ker T$
(namely the last element of each of the $\ell$ staircases).
We can thus complete it to a basis $v_{\ell+1}, \dots, v_m$ (where $m = \dim \ker T$).
(In other words, the last element of each staircase plus the $m-\ell$ new ones are a basis for $\ker T$.)

Now consider
\begin{align*}
	\mathcal B =
	\Big\{ & v_1, T(v_1), T(T(v_1)), T(T(T(v_1))), \dots \\
	& v_2, T(v_2), T(T(v_2)), T(T(T(v_2))), \dots \\
	& \dots, \\
	& v_\ell, T(v_\ell), T(T(v_\ell)), T(T(T(v_\ell))), \dots \\
	& v_{\ell+1}, v_{\ell+2}, \dots, v_m \Big\}.
\end{align*}
\begin{ques}
Check that there are exactly $\ell + \dim W + (\dim \ker T - \ell) = \dim V$ elements.
\end{ques}
\begin{exercise}
	Show that all the elements are linearly independent.
	(Assume for contradiction there is some linear dependence,
	then take $T$ of both sides.)
\end{exercise}
Hence $\mathcal B$ is a basis of the desired form.

\section{Algebraic and geometric multiplicity}
\prototype{The matrix $T$ below.}
This is some convenient notation:
let's consider the matrix in Jordan form
\[
	T =
	\begin{bmatrix}
		7 & 1 \\
		0 & 7 \\
		& & 9 \\
		& & & 7 & 1 & 0 \\
		& & & 0 & 7 & 1 \\
		& & & 0 & 0 & 7
	\end{bmatrix}.
\]
We focus on the eigenvalue $7$,
which appears multiple times, so it is certainly ``repeated''.
However, there a two different senses in which you could say it is repeated.
\begin{itemize}
	\ii \emph{Algebraic}: You could say it is repeated five times,
	because it appears five times on the diagonal.
	\ii \emph{Geometric}: You could say it really only appears two times:
	because there are only two eigen\emph{vectors}
	with eigenvalue $7$, namely $e_1$ and $e_4$.

	Indeed, the vector $e_2$ for example has $T(e_2) = 7e_2 + e_1$,
	so it's not really an eigenvector!
	If you apply $T - 7\id$ to $e_2$ twice though,
	you do get zero.
\end{itemize}
\begin{ques}
	In this example,
	how many times do you need to apply $T - 7\id$ to $e_6$ to get zero?
\end{ques}
Both these notions are valid,
so we will name both.
To preserve generality,
we first state the ``intrinsic'' definition.
\begin{definition}
	Let $T \colon V \to V$ be a linear map and $\lambda$ a scalar.
	\begin{itemize}
		\ii The \vocab{geometric multiplicity}
		of $\lambda$ is the dimension $\dim V_\lambda$
		of the $\lambda$-eigenspace.

		\ii Define the \vocab{generalized eigenspace}
		$V^\lambda$ to be the subspace of $v$
		for which $(T-\lambda \id)^n(v) = 0$ for some $n \ge 1$.
		The \vocab{algebraic multiplicity} of $\lambda$ is the
		dimension $\dim V^\lambda$.
	\end{itemize}
	(Silly edge case: we allow ``multiplicity zero''
	if $\lambda$ is not an eigenvalue at all.)
\end{definition}
However in practice you should just count the Jordan blocks.
\begin{example}
	[An example of eigenspaces via Jordan form]
	Retain the matrix $T$ mentioned earlier and let $\lambda = 7$.
	\begin{itemize}
		\ii The eigenspace $V_\lambda$ has basis $e_1$ and $e_4$,
		so the geometric multiplicity is $2$.
		\ii The generalized eigenspace $V^\lambda$ has basis $e_1$, $e_2$,
		$e_4$, $e_5$, $e_6$ so the algebraic multiplicity is $5$.
	\end{itemize}
\end{example}

To be completely explicit, here is how you think of these in practice:
\begin{proposition}
	[Geometric and algebraic multiplicity vs Jordan blocks]
	Assume $T \colon V \to V$ is a linear map
	of finite-dimensional vector spaces,
	written in Jordan form.
	Let $\lambda$ be a scalar.
	Then
	\begin{itemize}
		\ii The geometric multiplicity of $\lambda$ is the number
		of Jordan blocks with eigenvalue $\lambda$;
		the eigenspace has one basis element per Jordan block.

		\ii The algebraic multiplicity of $\lambda$ is the
		sum of the dimensions of the Jordan blocks
		with eigenvalue $\lambda$;
		the eigenspace is the direct sum of the subspaces
		corresponding to those blocks.
	\end{itemize}
\end{proposition}

\begin{ques}
	Show that the geometric multiplicity
	is always less than or equal to the algebraic multiplicity.
\end{ques}

This actually gives us a tentative definition:
\begin{itemize}
	\ii The trace is the sum of the eigenvalues,
	counted with algebraic multiplicity.
	\ii The determinant is the product of the eigenvalues,
	counted with algebraic multiplicity.
\end{itemize}
This definition is okay,
but it has the disadvantage of requiring the ground
field to be algebraically closed.
It is also not the definition that is easiest
to work with computationally.
The next two chapters will give us a better definition.

\section\problemhead

\begin{problem}
	[Sum of algebraic multiplicities]
	Given a $2018$-dimensional complex vector space $V$
	and a map $T \colon V \to V$,
	what is the sum of the algebraic multiplicities
	of all eigenvalues of $T$?
	\begin{sol}
		It's just $\dim V = 2018$.
		After all, you are adding the dimensions of the Jordan blocks\dots
	\end{sol}
\end{problem}

\begin{problem}
	[The word ``diagonalizable'']
	A linear map $T \colon V \to V$ (where $\dim V$ is finite)
	is said to be \vocab{diagonalizable}
	if it has a basis $e_1$, \dots, $e_n$
	such that each $e_i$ is an eigenvector.
	\begin{enumerate}[(a)]
		\ii Explain the name ``diagonalizable''.
		\ii Suppose we are working over an algebraically closed field.
		Then show that that $T$ is diagonalizable if and only if
		for any $\lambda$, the geometric multiplicity of $\lambda$
		equals the algebraic multiplicity of $\lambda$.
	\end{enumerate}
	\begin{sol}
		(a): if you express $T$ as a matrix in such a basis,
		one gets a diagonal matrix.
		(b): this is just saying each Jordan block has dimension $1$,
		which is what we wanted.
		(We are implicitly using uniqueness of Jordan form here.)
	\end{sol}
\end{problem}

\begin{problem}
	[Switcharoo]
	Let $V$ be the $\CC$-vector space
	with basis $e_1$ and $e_2$.
	The map $T \colon V \to V$ sends $T(e_1) = e_2$ and $T(e_2) = e_1$.
	Determine the eigenspaces of $T$.
	\begin{sol}
		The $+1$ eigenspace is spanned by $e_1+e_2$.
		The $-1$ eigenspace is spanned by $e_1-e_2$.
	\end{sol}
\end{problem}

\begin{problem}
	[Writing a polynomial backwards]
	Define the complex vector space $V$
	of polynomials with degree at most $2$,
	say $V = \left\{ ax^2 + bx + c \mid a,b,c \in \CC \right\}$.
	Define $T \colon V \to V$ by
	\[ T(ax^2+bx+c) = cx^2+bx+a. \]
	Determine the eigenspaces of $T$.
	\begin{sol}
		The $+1$ eigenspace is spanned by $1+x^2$ and $x$.
		The $-1$ eigenspace is spanned by $1-x^2$.
	\end{sol}
\end{problem}

\begin{problem}
	[Differentiation of polynomials]
	Let $V = \RR[x]$ be the real vector space of all real polynomials.
	Note that $\frac{d}{dx} \colon V \to V$ is a linear map
	(for example it sends $x^3$ to $3x^2$).
	Which real numbers are eigenvalues of this map?
	\begin{hint}
		Only $0$ is.
		Look at degree.
	\end{hint}
	\begin{sol}
		Constant functions differentiate to zero,
		and these are the only $0$-eigenvectors.
		There can be no other eigenvectors,
		since if $\deg p > 0$ then $\deg p' = \deg p - 1$,
		so if $p'$ is a constant real multiple of $p$
		we must have $p' = 0$, ergo $p$ is constant.
	\end{sol}
\end{problem}

\begin{problem}
	[Differentiation of functions]
	Let $V$ be the real vector space of all
	infinitely differentiable functions $\RR \to \RR$.
	Note that $\frac{d}{dx} \colon V \to V$ is a linear map
	(for example it sends $\cos x$ to $-\sin x$).
	Which real numbers are eigenvalues of this map?
	\begin{hint}
		All of them are!
	\end{hint}
	\begin{sol}
		$e^{cx}$ is an example of a $c$-eigenvector for every $c$.
		If you know differential equations,
		these generate all examples!
	\end{sol}
\end{problem}
