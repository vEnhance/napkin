\chapter{Bonus: Cellular Homology (In Progress)}
We now introduce cellular homology, which essentially lets us compute
the homology groups of any CW complex we like.

\section{Degrees}
\prototype{$z \mapsto z^d$ has degree $d$.}
For any $n > 0$ and map $f : S^n \to S^n$, consider
\[ f_\ast : \underbrace{H_n(S^n)}_{\cong \ZZ} \to \underbrace{H_n(S^n)}_{\cong \ZZ} \]
which must be multiplication by some constant $d$.
This $d$ is called the \vocab{degree} of $f$, denoted $\deg f$.
\begin{ques}
	Show that $\deg(f \circ g) = \deg(f) \deg(g)$.
\end{ques}

\begin{example}
	[Degree]
	\listhack
	\begin{enumerate}[(a)]
		\ii For $n=1$, the map $z \mapsto z^k$ (viewing $S^1 \subseteq \CC$)
		has degree $k$.
		\ii A reflection map $(x_0, x_1, \dots, x_n) \mapsto (-x_0, x_1, \dots, x_n)$
		has degree $-1$; we won't prove this, but geometrically this should be clear.
		\ii The antipodal map $x \mapsto -x$ has degree $(-1)^{n+1}$
		since it's the composition of $n+1$ reflections as above.
		We denote this map by $-\id$.
	\end{enumerate}
\end{example}

Obviously, if $f$ and $g$ are homotopic, then $\deg f = \deg g$.
In fact, a theorem of Hopf says that this is a classifying invariant:
anytime $\deg f = \deg g$, we have that $f$ and $g$ are homotopic.

One nice application of this:
\begin{theorem}
	[Hairy Ball Theorem]
	If $n > 0$ is even, then $S^n$ doesn't have a continuous field
	of nonzero tangent vectors.
\end{theorem}
\begin{proof}
	If the vectors are nonzero then WLOG they have norm $1$;
	that is for every $x$ we have an orthogonal unit vector $v(x)$.
	Then we can construct a homotopy map $F : S^n \times [0,1] \to S^n$ by
	\[ (x,t) \mapsto (\cos \pi t)x + (\sin \pi t) v(x). \]
	which gives a homotopy from $\id$ to $-\id$.
	So $\deg(\id) = \deg(-\id)$, which means $1 = (-1)^{n+1}$
	so $n$ must be odd.
\end{proof}
Of course, the one can construct such a vector field whenever $n$ is odd.
For example, when $n=1$ such a vector field is drawn below.
\begin{center}
	\begin{asy}
		size(5cm);
		draw(unitcircle, blue+1);
		label("$S^1$", dir(100), dir(100), blue);
		void arrow(real theta) {
			pair P = dir(theta);
			dot(P);
			pair delta = 0.8*P*dir(90);
			draw( P--(P+delta), EndArrow );
		}
		arrow(0);
		arrow(50);
		arrow(140);
		arrow(210);
		arrow(300);
	\end{asy}
\end{center}


\section{Cellular Homology}
Before starting, we state the following lemma.
\begin{lemma}
	[CW Homology Groups]
	Let $X$ be a CW complex. Then
	\begin{align*}
		H_k(X^n, X^{n-1}) &\cong
		\begin{cases}
			\ZZ^{\oplus\text{\#$n$-cells of $X$}} & k = n \\
			0 & \text{otherwise}.
		\end{cases} \\
		\intertext{and}
		H_k(X^n) &\cong
		\begin{cases}
			H_k(X) & k \le n-1 \\
			0 & k \ge n+1.
		\end{cases}
	\end{align*}
\end{lemma}
\begin{proof}
	I'll prove just the case where $X$ is finite-dimensional for simplicity.
	The first part is immediate by noting that $(X^n, X^{n-1})$ is a good pair
	and $X^n/X^{n-1}$ is a wedge sum of two spheres.
	For the second part, fix $k$ and note that, as long as $n \le k-1$ or $n \ge k+2$,
	\[
		\underbrace{H_{k+1}(X^n, X^{n-1})}_{=0}
		\to H_k(X^{n-1})
		\to H_k(X^n)
		\to \underbrace{H_{k}(X^n, X^{n-1})}_{=0}.
	\]
	So we have isomorphisms
	\[ H_k(X^{k-1}) \cong H_k(X^{k-2}) \cong \dots \cong H_k(X^0) = 0 \]
	and
	\[ H_k(X^{k+1}) \cong H_k(X^{k+2}) \cong \dots \cong H_k(X). \qedhere \]
\end{proof}

So, we know that the groups $H_k(X^n, X^{n-1})$ are super nice:
they are free abelian with basis given by the cells of $X$.

In light of this, let's look at our long exact sequence and try to
string together maps between these.
Consider the following diagram.

\begin{diagram}
	\small
	&& \underbrace{H_3(X^2)}_{=0} &&&&&&&& \\
	&& \dTo~0 &&&&&&&& \\
	\boxed{H_4(X^4, X^3)} & \rTo^{\partial_4} & H_3(X^3) & \rSurj
		& \underbrace{H_3(X^4)}_{\cong H_3(X)} &
		\rTo~0 & \underbrace{H_3(X^4, X^3)}_{= 0} &&&& \\
	& \rdDotted~{d_4} & \dTo~0 &&&&&&&& \\
	&& \boxed{H_3(X^3, X^2)} &&&& \underbrace{H_1(X^0)}_{=0} &&&& \\
	&& \dTo^{\partial_3} & \rdDotted~{d_3} &&& \dTo~0 &&&& \\
	\underbrace{H_2(X^1)}_{=0} & \rTo~0 & H_2(X^2) & \rInj &
		\boxed{H_2(X^2, X^1)} & \rTo^{\partial_2} & H_1(X^1) & \rSurj &
		\underbrace{H_1(X^2)}_{\cong H_1(X)} & \rTo~0 &
		\underbrace{H_1(X^2, X^1)}_{=0} \\
	&& \dSurj &&& \rdDotted~{d_2} & \dInj &&&& \\
	&& \underbrace{H_2(X^3)}_{\cong H_2(X)} &&&& \boxed{H_1(X^1, X^0)} &&&& \\
	&& \dTo~0 &&&& \dTo^{\partial_1} & \rdDotted~{d_1} &&& \\
	&& \underbrace{H_2(X^3, X^2)}_{=0} && \underbrace{H_0(\varnothing)}_{=0}
		& \rTo~0 & H_0(X^0) & \rInj & \boxed{H_0(X^0, \varnothing)} & \rTo^{\partial_0} & \dots \\
	&&&&&& \dSurj &&&& \\
	&&&&&& \underbrace{H_0(X^1)}_{\cong H_0(X)} &&&& \\
	&&&&&& \dTo~0 &&&& \\
	&&&&&& \underbrace{H_0(X^1, X^0)}_{=0} &&&& \\
\end{diagram}
Here we have $X^{-1} = \varnothing$.
The idea is that we have taken all the exact sequences generated by adjacent
skeletons, and strung them together at the groups $H_k(X^k)$,
with half the exact sequences being laid out vertically
and the other half horizontally.

In that case, composition generates a sequence of dotted maps
between the $H_k(X^k, X^{k-1})$ as shown.
\begin{ques}
	Show that the composition of two adjacent dotted arrows is zero.
\end{ques}

So the sequence of arrows $\dots \to H_4(X^4, X^3) \to H_3(X^3, X^2) \to \dots$
that we obtain is a chain complex, called the \vocab{cellular chain complex};
as mentioned before before all the homology groups are free,
but these ones are especially nice because for most reasonable CW complexes,
they are also finitely generated
(unlike the massive $C_\bullet(X)$ that we had earlier).
In other words, the $H_k(X^k, X^{k-1})$ are especially nice ``concrete'' free groups
that one can actually work with.

The other reason we care is that in fact:
\begin{theorem}[Cellular Chain Complex Gives $H_n(X)$]
	\label{thm:cellular_chase}
	The $k$th homology group of the cellular chain complex
	is isomorphic to $H_k(X)$.
\end{theorem}
\begin{proof}
	Follows from the diagram; \Cref{prob:diagram_chase}.
\end{proof}

In fact, one can describe explicitly what the maps $d_n$ are.
Recalling that $H_k(X^k, X^{k-1})$ has a basis the $k$-cells of $X$, we
can write the following formula.
\begin{theorem}
	[Cellular Boundary Formula]
	Let $e^k$ be a $k$-cell, and let $e_\beta^{k-1}$
	denote all $(k-1)$-cells of $X$.
	Define $d_\beta$ to be the degree of the composed map
	\[ S^{n-1} = \partial D_\beta^n \xrightarrow{\text{attach}}
		X^{n-1} \surjto S_\beta^{n-1} \]
	where the first arrow is the attaching map for $e^k$
	and the second arrow is the quotient of collapsing
	$X^{n-1} - e^{k-1}_\beta$ to a point.
	Then \[ d_n : e^k \mapsto \sum_\beta d_\beta e_\beta^{k-1} \]
\end{theorem}


\todo{CP, Torus, Klein, RP2}

\section{Euler Characteristic}
Let $X$ be a finite CW complex.
We define
\[ \chi(X) = \sum_n (-1)^n \cdot \text{\# $n$-cells of $X$}. \]
This generalizes the familiar $V-E+F$ formula.

A priori, we might expect that this depends on the choice
of CW complex, but it turns that we can in fact write:
\begin{theorem}
	[Euler Characteristic via Betti Numbers]
	For any finite CW complex $X$ we have
	\[ \chi(X) = \sum_n (-1)^n \rank H_n(X). \]
\end{theorem}
Thus $\chi(X)$ does not depend on the choice of CW decomposition.
The numbers $\rank H_n(X)$ are called the \vocab{Betti numbers} of $X$.

\begin{proof}
	Harmless exercise.
	\todo{maybe actually write this one out}
\end{proof}

\section\problemhead
\begin{problem}
	Show that a non-surjective map $f : S^n \to S^n$ has degree zero.
	\begin{hint}
		The space $S^n - \{x_0\}$ is contractible.
	\end{hint}
\end{problem}

\begin{problem}[Moore Spaces]
	\gim
	Let $G_1$, $G_2$, \dots, $G_N$ be a sequence of
	finitely generated abelian groups.
	Construct a space $X$ such that
	\[
		\wt H_n(X)
		\cong
		\begin{cases}
			G_n & 1 \le n \le N \\
			0 & \text{otherwise}.
		\end{cases}
	\]
\end{problem}

\begin{problem}
	\label{prob:diagram_chase}
	Prove \Cref{thm:cellular_chase},
	showing that the homology groups of $X$
	coincide with the homology groups of the cellular chain complex.
	\begin{hint}
		You won't need to refer to any elements.
		Start with $H_2(X) \cong H_2(X^3) \cong
			H_2(X^2) / \ker \left[ H_2(X^2) \surjto H_2(X^3) \right]$, say.
		Take note of the marked injective and surjective arrows.
	\end{hint}
	\begin{sol}
		For concreteness, let's just look at the homology at $H_2(X^2, X^1)$
		and show it's isomorphic to $H_2(X)$.
		According to the diagram
		\begin{align*}
			H_2(X) &\cong H_2(X^3) \\
			&\cong H_2(X^2) / \ker \left[ H_2(X^2) \surjto H_2(X^3) \right] \\
			&\cong H_2(X^2) / \img \partial_3 \\
			&\cong \img\left[ H_2(X^2) \injto H_2(X^2, X^1) \right] / \img \partial_3 \\
			&\cong \ker(\partial_2) / \img\partial_3 \\
			&\cong \ker d_2 / \img d_3. \qedhere
		\end{align*}
	\end{sol}
\end{problem}


