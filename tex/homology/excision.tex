\chapter{Excision and relative homology}
\label{ch:relative_hom}
We have already seen how to use the Mayer-Vietoris sequence:
we started with a sequence
\[ \dots \to H_n(U \cap V) \to H_n(U) \oplus H_n(V) \to H_n(U+V) \to H_{n-1}(U \cap V) \to \dots \]
and its reduced version,
then appealed to the geometric fact that $H_n(U+V) \cong H_n(X)$.
This allowed us to algebraically make computations on $H_n(X)$.

In this chapter, we turn our attention to the long exact
sequence associated to the chain complex
\[ 0 \to C_n(A) \injto C_n(X) \surjto C_n(X,A) \to 0. \]
The setup will look a lot like the previous two chapters,
except in addition to $H_n \colon \catname{hTop} \to \catname{Grp}$
we will have a functor $H_n \colon \catname{hPairTop} \to \catname{Grp}$
which takes a pair $(X,A)$ to $H_n(X,A)$.
Then, we state (again without proof) the key geometric result,
and use this to make deductions.

\section{Motivation}

The main motivation is that:
\begin{moral}
	Relative homology is the algebraic analog of quotient space.
\end{moral}
So, for instance, when you see a map of pairs
$f \colon (X, A) \to (Y, B)$, you should think of $X/A \to Y/B$.

Which explains the ``reasonable guess''
that for spaces $A \subseteq X$, we have $H_n(X,A) \cong \wt H_n(X/A)$.

By \Cref{thm:good_pair}, the guess above is indeed true for most spaces. For example:
\begin{ques}
	Let $X = [0, 1]$ and $A = \{ 0, 1 \}$. Show that $H_1(X/A)$ and $H_1(X, A)$
	are isomorphic to $\ZZ$. (In this example, so is $\pi_1(X/A)$.)
\end{ques}

But not all. Similar to \Cref{ex:quotient_by_open_set}, if $A$ is not closed, weird
things can happen:
\begin{example}[$H_n(X, A)$ where $A$ is open in $X$]
	% https://math.stackexchange.com/a/2578207
	Let $X = D^2$ be the closed disk.

	If $A$ is reasonably nice, for instance $A = S^1$ the boundary of $X$, we have $H_2(X, A) \cong
	H_2(X/A) \cong \ZZ$.

	However, if $A = X \setminus \{ 0 \}$ where $0$ is the center of $X$, then
	$H_2(X, A)$ is still isomorphic to $\ZZ$; however $H_2(X/A) \cong 0$. (The latter
	isomorphism is harder to see, mainly because $X/A$ is a weird space --- it's not Hausdorff.)
\end{example}

Even when $A$ is closed in $X$, problems can still happen.
\begin{example}[The shrinking wedge of circles]
	% exercise 26 section 2.1, Hatcher
	Let $X$ be the interval $[0, 1]$, and $A \subseteq X$ be $A = \{ \frac{1}{n} \mid n \in \ZZ^+
	\} \cup \{ 0 \}$. In this case, the quotient $X/A$ would be isomorphic to the shrinking wedge of
	circles, as depicted below.
	\begin{center}
	\begin{asy}
		draw((0, 0)--(1, 0), L=Label("$X$", EndPoint, align=NE));
		dot((0, 0), red);
		for(int n=1; n<30; ++n){
			dot((1/n, 0), red);
		}

		draw((1.5, 0)--(2.5, 0), Arrow, L="$q$");
		for(int n=1; n<30; ++n){
			draw(shift(3, 0)*scale(1/(2*n))*shift(1, 0)*unitcircle);
		}
		dot((3, 0), red);
		label("$X/A$", (4.2, 0.5));
	\end{asy}
	\end{center}
	Note that in $X/A$, any open neighborhood of the red dot $A/A$ must contains all
	but finitely many circles.

	We claim that:
	\[ H_1(X, A) \not\cong \wt H_1(X/A). \]

	What could go wrong? Generally speaking, when you work algebraically then everything are finite,
	while in topology you have to consider things related to infinity.

	Consider the following $1$-simplex in $C(X/A)$, depicted in cyan.
	\begin{center}
	\begin{asy}
		dot((0, 0), red);
		for(int n=1; n<30; ++n){
			draw((1/n, 0)--(1/(n+1), 0), (n%2==1 ? cyan: black)+(n==1 ? roundcap: squarecap));
		}

		draw((1.5, 0)--(2.5, 0), Arrow, L="$q$");
		for(int n=1; n<30; ++n){
			draw(shift(3, 0)*scale(1/(2*n))*shift(1, 0)*unitcircle, n%2==1 ? cyan: black);
		}
		dot((3, 0), red);
		label("$X/A$", (4.2, 0.5));
	\end{asy}
	\end{center}

	Every element of $H(X, A)$ has a representative in $C(X)$ as a $1$-cycles, which comprises of
	finitely many $1$-simplices, each $1$-simplex is equivalent to a segment $[a, b]$ --- modulo a
	difference of a $1$-boundary. Thus, intuitively, every element of $H(X, A)$ can only cover
	``finitely many circles'' (or all but finitely many).

	We haven't had enough tools to formalize all these yet.
	Formally speaking, the quotient maps $q \colon X \to X/A$ and $q \colon A \to A/A$ induces $q_*
	\colon H_1(X, A) \to H_1(X/A, A/A)$, and $q_*$ is not injective.
\end{example}

Regardless, for nice spaces $A \subseteq X$ such that $H_n(X, A) \cong \wt H_n(X/A)$, we would be
able to compute $H_n(X)$ based on $H_n(A)$ and $\wt H_n(X/A)$ --- note that $A$ and $X/A$ is, in
some sense, smaller and simpler than $X$.
% ``The Equivalence of Simplicial and Singular Homology'' section in \cite{ref:hatcher} has another
% application.

\section{The long exact sequences}
Recall \Cref{thm:long_exact_rel}, which says that the sequences
\[ \dots \to H_n(A) \to H_n(X) \to H_n(X,A) \to H_{n-1}(A) \to \dots. \]
and
\[ \dots \to \wt H_n(A) \to \wt H_n(X) \to H_n(X,A) \to \wt H_{n-1}(A) \to \dots \]
are long exact.
By \Cref{prob:triple_long_exact} we even have a long exact sequence
\[
	\dots
	\to H_n(B,A)
	\to H_n(X,A)
	\to H_n(X,B)
	\to H_{n-1}(B,A)
	\to \dots.
\]
for $A \subseteq B \subseteq X$.

\begin{moral}
	This is the analog of the fact that $X/B$ is homeomorphic to $\frac{X/A}{B/A}$ --- we ``cancel
	the common factor in the fraction''.
\end{moral}

An application of the first long exact sequence above gives:
\begin{lemma}
	[Homology relative to contractible spaces]
	\label{lem:rel_contractible}
	Let $X$ be a topological space,
	and let $A \subseteq X$ be contractible.
	For all $n$, \[ H_n(X, A) \cong \wt H_n(X). \]
\end{lemma}
\begin{proof}
	Since $A$ is contractible, we have $\wt H_n(A) = 0$ for every $n$.
	For each $n$ there's a segment of the long exact sequence given by
	\[ \dots \to \underbrace{\wt H_n(A)}_{=0} \to \wt H_n(X) \to H_n(X,A)
	\to \underbrace{\wt H_{n-1}(A)}_{=0} \to \dots. \]
	So since $0 \to \wt H_n(X) \to H_n(X,A) \to 0$ is exact,
	this means $H_n(X,A) \cong \wt H_n(X)$.
\end{proof}

In particular, the theorem applies if $A$ is a single point.
The case $A = \varnothing$ is also worth noting.
We compile these results into a lemma:
\begin{lemma}
	[Relative homology generalizes absolute homology]
	Let $X$ be any space, and $\ast \in X$ a point. Then for all $n$,
	\[
		H_n(X, \{\ast\}) \cong \wt H_n(X)
		\qquad\text{and}\qquad
		H_n(X, \varnothing) = H_n(X).
	\]
\end{lemma}

\section{The category of pairs}
Since we now have an $H_n(X,A)$ instead of just $H_n(X)$,
a natural next step is to create a suitable category of \emph{pairs}
and give ourselves the same functorial setup as before.

\begin{definition}
	Let $\varnothing \neq A \subseteq X$ and $\varnothing \neq B \subseteq Y$
	be subspaces, and consider a map $f \colon X \to Y$.
	If $f\im(A) \subseteq B$ we write
	\[ f \colon (X,A) \to (Y,B). \]
	We say $f$ is a \vocab{map of pairs},
	between the pairs $(X,A)$ and $(Y,B)$.
\end{definition}
\begin{definition}
	We say that $f,g \colon (X,A) \to (Y,B)$ are \vocab{pair-homotopic} if they
	are ``homotopic through maps of pairs''.

	More formally, a \vocab{pair-homotopy}
	$f, g \colon (X,A) \to (Y,B)$ is a map $F \colon [0,1] \times X \to Y$,
	which we'll write as $F_t(X)$, such that
	$F$ is a homotopy of the maps $f,g \colon X \to Y$
	and each $F_t$ is itself a map of pairs.
\end{definition}

A typical $f, g \colon (X, A) \to (Y, B)$ that are pair-homotopic might look like this.
Note that for all $t \in [0, 1]$, we must have $F_t\im(A) \subseteq B$.
\begin{center}
\begin{asy}
	usepackage("xcolor");
	draw((0, 0)--(0, 1), L=Label("$(X, {\color{red} A})$", EndPoint));
	dot((0, 0), red);
	dot((0, 1), red);

	draw((0.2, 0.5)--(0.8, 0.5), Arrow, L="$q$");
	fill(shift(1, 0)*unitsquare, yellow+opacity(0.1));
	draw((1, 0)--(2, 0), red);
	draw((1, 1)--(2, 1), red);
	label("$(Y, {\color{red} B})$", (2, 1), align=N);

	draw((1.1, 0)--(1.3, 1), heavygreen, L=Label("$f(X)$"));
	dot((1.1, 0), red);
	dot((1.3, 1), red);

	draw((1.75, 0)--(1.5, 1), blue, L=Label("$g(X)$", Relative(0.8)));
	dot((1.75, 0), red);
	dot((1.5, 1), red);
\end{asy}
\end{center}

Thus, we naturally arrive at two categories:
\begin{itemize}
	\ii $\catname{PairTop}$, the category of \emph{pairs} of
	topological spaces, and
	\ii $\catname{hPairTop}$, the same category except
	with maps only equivalent up to homotopy.
\end{itemize}
\begin{definition}
	As before, we say pairs $(X,A)$ and $(Y,B)$ are
	\vocab{pair-homotopy equivalent}
	if they are isomorphic in $\catname{hPairTop}$.
	An isomorphism of $\catname{hPairTop}$ is a
	\vocab{pair-homotopy equivalence}.
\end{definition}

\begin{remark}
	Pair-homotopy equivalence of pairs is the natural generalization of homotopy equivalence of
	spaces, as defined in \Cref{def:homotopy_equiv}. In fact, if $A = B = \varnothing$ then we
	have $X$ is homotopy equivalent to $Y$ if and only if $(X, \varnothing)$ is pair-homotopy
	equivalent to $(Y, \varnothing)$.
\end{remark}

We can do the same song and dance as before with the prism operator to obtain:
\begin{lemma}[Induced maps of relative homology]
	We have a functor
	\[ H_n \colon \catname{hPairTop} \to \catname{Grp}. \]
\end{lemma}
That is, if $f \colon (X,A) \to (Y,B)$ then we obtain an induced map
\[ f_\ast \colon H_n(X,A) \to H_n(Y,B). \]
and if two such $f$ and $g$ are pair-homotopic
then $f_\ast = g_\ast$.

Now, we want an analog of contractible spaces for our pairs:
i.e.\ pairs of spaces $(X,A)$ such that $H_n(X,A) = 0$.
The correct definition is:
\begin{definition}
	Let $A \subseteq X$.
	We say that $A$ is a \vocab{deformation retract}\footnote{This might be called a
	\emph{deformation retraction in the weak sense} in other resources, such as \cite{ref:hatcher}}
	of $X$ if there is a map of pairs $r \colon (X, A) \to (A, A)$
	which is a pair-homotopy equivalence.
\end{definition}
\begin{example}
	[Examples of deformation retracts]
	\listhack
	\begin{enumerate}[(a)]
		\ii If a single point $p$ is a deformation retract of a space $X$,
		then $X$ is contractible, since the retraction $r \colon X \to \{\ast\}$
		(when viewed as a map $X \to X$)
		is homotopic to the identity map $\id_X \colon X \to X$.
		\ii The punctured disk $D^2 \setminus \{0\}$
		deformation retracts onto its boundary $S^1$.
		\ii More generally, $D^{n} \setminus \{0\}$
		deformation retracts onto its boundary $S^{n-1}$.
		\ii Similarly, $\RR^n \setminus \{0\}$
		deformation retracts onto a sphere $S^{n-1}$.
	\end{enumerate}
\end{example}
Of course in this situation we have that
\[ H_n(X,A) \cong H_n(A,A) = 0. \]

\begin{exercise}
	Show that if $A \subseteq V \subseteq X$,
	and $A$ is a deformation retract of $V$,
	then $H_n(X,A) \cong H_n(X,V)$ for all $n$.
	(Use \Cref{prob:triple_long_exact}. Solution in next section.)
\end{exercise}

\section{Excision}
Now for the key geometric result, which is the analog of
\Cref{thm:open_cover_homology} for our relative homology groups.
\begin{theorem}
	[Excision]
	Let $Z \subseteq A \subseteq X$ be subspaces such that
	the closure of $Z$ is contained in the interior of $A$.
	Then the inclusion $\iota (X \setminus Z, A \setminus Z) \injto (X,A)$
	(viewed as a map of pairs) induces an isomorphism of
	relative homology groups
	\[ H_n(X \setminus Z, A \setminus Z) \cong H_n(X,A). \]
\end{theorem}
This means we can \emph{excise} (delete) a subset $Z$ of $A$ in computing
the relative homology groups $H_n(X,A)$.
This should intuitively make sense:
since we are ``modding out by points in $A$'',
the internals of the set $A$ should not matter so much.

\begin{example}
	Excision may seem trivial (for a ``relative cycle modulo relative boundary'' in $H_n(X, A)$,
	just tweak the part that
	lies inside $A$ until it doesn't touch $Z$), until you realize that it isn't always possible ---
	you may accidentally cut a cycle apart! For example:
	\begin{center}
	\begin{asy}
		fill(unitcircle, lightcyan);
		fill(shift(2, 0)*unitcircle, lightcyan);
		picture p;
		fill(p, shift(1, 0)*scale(0.4)*unitcircle, orange);
		clip(p, unitcircle);
		add(p);
		picture p;
		fill(p, shift(1, 0)*scale(0.4)*unitcircle, orange);
		label("$X$", (2.9, 0.9));
		label("$A$", (1.3, 0.5), orange);
		clip(p, shift(2, 0)*unitcircle);
		add(p);
		dot("$Z$", (1, 0), red, align=NE);
		draw((0.6, 0.5)..(1, 0)..(1.5, 0.2));
	\end{asy}
	\end{center}
\end{example}

The main application of excision is to decide
when $H_n(X,A) \cong \wt H_n(X/A)$.
Answer:

\begin{theorem}
	[Relative homology $\implies$ quotient space]
	\label{thm:good_pair}
	Let $X$ be a space and $A$ be a closed subspace such that
	$A$ is a deformation retract of some open set $V \subseteq X$.
	Then the quotient map $q \colon X \to X/A$ induces an isomorphism
	\[ H_n(X,A) \cong H_n(X/A, A/A) \cong \wt H_n(X/A). \]
\end{theorem}

The key idea of the proof is: While it is not necessarily true that $H(X, A) \cong H(X/A, A/A)$
(indeed, we have seen two counterexamples earlier), if we cut out $A$, then we trivially have
$H(X-A, A-A) \cong H(X/A-A/A, A/A-A/A)$. Unfortunately, this group is not isomorphic to $H(X, A)$,
so we fix that using the set $V$ --- that is, $H(X-A, V-A) \cong H(X/A-A/A, V/A-A/A)$. The rest of
the work is to use excision theorem and deformation retract to show the left hand side is
isomorphic to $H(X, A)$, and the right hand side is isomorphic to $H(X/A)$.

\begin{proof}
	By hypothesis, we can consider the following maps of pairs:
	\begin{align*}
		r & \colon (V,A) \to (A,A)  \\
		q & \colon (X,A) \to (X/A, A/A) \\
		\widehat q &: (X-A, V-A) \to (X/A-A/A, V/A-A/A).
	\end{align*}
	Moreover, $r$ is a pair-homotopy equivalence.
	Considering the long exact sequence of a triple
	(which was \Cref{prob:triple_long_exact})
	we have a diagram
	\begin{center}
		\begin{tikzcd}[row sep=huge]
		H_n(V,A) \ar[r] \ar[d, "\cong"', "r"]
			& H_n(X,A) \ar["f", r]
			& H_n(X, V) \ar[r]
			& H_{n-1}(V,A) \ar[d, "\cong"', "r"] \\
		\underbrace{H_n(A,A)}_{=0} & & & \underbrace{H_{n-1}(A,A)}_{=0}
	\end{tikzcd}
	\end{center}
	where the isomorphisms arise since $r$ is a pair-homotopy equivalence.
	So $f$ is an isomorphism.
	Similarly the map
	\[ g \colon H_n(X/A, A/A) \to H_n(X/A, V/A) \]
	is an isomorphism.

	Now, consider the commutative diagram
	\begin{center}
		\begin{tikzcd}[sep=huge]
		H_n(X,A) \ar[r, "f"] \ar[d, "q_\ast"']
			& H_n(X,V)
			& H_n(X-A, V-A) \ar[l, "\text{Excise}"'] \ar[d, "\widehat{q}_\ast", "\cong"']
			\\
		H_n(X/A,A/A) \ar[r, "g"']
			& H_n(X/A,V/A)
			& \ar["\text{Excise}"', l] H_n(X/A-A/A, V/A-A/A)
	\end{tikzcd}
	\end{center}
	and observe that the rightmost arrow $\widehat{q}_\ast$ is an isomorphism,
	because outside of $A$ the map $\widehat q$ is the identity.
	We know $f$ and $g$ are isomorphisms,
	as are the two arrows marked with ``Excise'' (by excision).
	From this we conclude that $q_\ast$ is an isomorphism.
	Of course we already know that homology relative to a point
	is just the relative homology groups
	(this is the important case of \Cref{lem:rel_contractible}).
\end{proof}

\section{Some applications}
One nice application of excision is to compute $\wt H_n(X \vee Y)$.
\begin{theorem}[Homology of wedge sums]
	Let $X$ and $Y$ be spaces with basepoints $x_0 \in X$ and $y_0 \in Y$,
	and assuming each point is a deformation retract of some open neighborhood.
	Then for every $n$ we have
	\[
		\wt H_n(X \vee Y)
		= \wt H_n(X) \oplus \wt H_n(Y).
	\]
\end{theorem}
\begin{proof}
	Apply \Cref{thm:good_pair} with the subset $\{x_0, y_0\}$ of $X \amalg Y$,
	\begin{align*}
		\wt H_n (X \vee Y)
		\cong \wt H_n( (X \amalg Y) / \{x_0, y_0\} )
		&\cong H_n(X \amalg Y, \{x_0,y_0\}) \\
		&\cong H_n(X, \{x_0\}) \oplus H_n(Y, \{y_0\}) \\
		&\cong\wt H_n(X) \oplus \wt H_n(Y). \qedhere
	\end{align*}
\end{proof}

Another application is to give a second method
of computing $H_n(S^m)$.
To do this, we will prove that
\[ \wt H_n(S^m) \cong \wt H_{n-1}(S^{m-1}) \]
for any $n,m > 1$.
However,
\begin{itemize}
	\ii $\wt H_0(S^n)$ is $\ZZ$ for $n=0$ and $0$ otherwise.
	\ii $\wt H_n(S^0)$ is $\ZZ$ for $m=0$ and $0$ otherwise.
\end{itemize}
So by induction on $\min \{m,n\}$ we directly obtain that
\[
	\wt H_n(S^m) \cong
	\begin{cases}
		\ZZ & m=n \\
		0 & \text{otherwise}
	\end{cases}
\]
which is what we wanted.

To prove the claim, let's consider the exact sequence
formed by the pair $X = D^2$ and $A = S^1$.
\begin{example}[The long exact sequence for $(X,A) = (D^2, S^1)$]
	Consider $D^2$ (which is contractible) with boundary $S^1$.
	Clearly $S^1$ is a deformation retraction of $D^2 \setminus \{0\}$,
	and if we fuse all points on the boundary together we get $D^2 / S^1 \cong S^2$.
	So we have a long exact sequence
	\begin{center}
	\begin{tikzcd}
		\wt H_2(S^1) \ar[r] & \underbrace{\wt H_2(D^2)}_{=0} \ar[r] & \wt H_2(S^2) \ar[lld] \\
		\wt H_1(S^1) \ar[r] & \underbrace{\wt H_1(D^2)}_{=0} \ar[r] & \wt H_1(S^2) \ar[lld] \\
		\wt H_0(S^1) \ar[r] & \underbrace{\wt H_0(D^2)}_{=0} \ar[r] & \underbrace{\wt H_0(S^2)}_{=0}
	\end{tikzcd}
	\end{center}
	From this diagram we read that
	\[
		\dots, \quad
		\wt H_3(S^2) = \wt H_2(S^1), \quad
		\wt H_2(S^2) = \wt H_1(S^1), \quad
		\wt H_1(S^2) = \wt H_0(S^1).
	\]
\end{example}
More generally, the exact sequence for the pair $(X,A) = (D^m, S^{m-1})$
shows that $\wt H_n(S^m) \cong \wt H_{n-1}(S^{m-1})$,
which is the desired conclusion.

\section{Invariance of dimension}
Here is one last example of an application of excision.
\begin{definition}
	Let $X$ be a space and $p \in X$ a point.
	The $k$th \vocab{local homology group} of $p$ at $X$ is defined as
	\[ H_k(X, X \setminus \{p\}). \]
\end{definition}
Note that for any open neighborhood $U$ of $p$, we have by excision that
\[ H_k(X, X \setminus \{p\}) \cong H_k(U, U \setminus \{p\}). \]
Thus this local homology group only depends on the space near $p$.

\begin{theorem}
	[Invariance of dimension, Brouwer 1910]
	Let $U \subseteq \RR^n$ and $V \subseteq \RR^m$ be nonempty open sets.
	If $U$ and $V$ are homeomorphic, then $m = n$.
\end{theorem}
\begin{proof}
	Consider a point $x \in U$ and its local homology groups. By excision,
	\[ H_k(\RR^n, \RR^n \setminus \{x\}) \cong
		H_k(U, U \setminus \{x\}). \]
	But since $\RR^n \setminus \{x\}$ is homotopic to $S^{n-1}$,
	the long exact sequence of \Cref{thm:long_exact_rel} tells us
	that
	\[
		H_k(\RR^n, \RR^n \setminus \{x\})
		\cong
		\begin{cases}
			\ZZ & k = n \\
			0 & \text{otherwise}.
		\end{cases}
	\]
	Analogously, given $y \in V$ we have
	\[ H_k(\RR^m, \RR^m \setminus\{y\}) \cong H_k(V, V\setminus\{y\}). \]
	If $U \cong V$, we thus
	deduce that
	\[ H_k(\RR^n, \RR^n\setminus\{x\}) \cong H_k(\RR^m, \RR^m\setminus\{y\}) \]
	for all $k$.  This of course can only happen if $m=n$.
\end{proof}

\section\problemhead
\begin{problem}
	Let $X = S^1 \times S^1$ and $Y = S^1 \vee S^1 \vee S^2$.
	Show that \[ H_n(X) \cong H_n(Y) \] for every integer $n$.
\end{problem}

\begin{problem}[Hatcher \S2.1 exercise 18]
	Consider $\QQ \subset \RR$.
	Compute $\wt H_1(\RR, \QQ)$.
	\begin{hint}
		Use \Cref{thm:long_exact_rel}.
	\end{hint}
	\begin{sol}
		We have an exact sequence
		\[
			\underbrace{\wt H_1(\RR)}_{=0}
			\to \wt H_1(\RR, \QQ) \to \wt H_0(\QQ) \to
			\underbrace{\wt H_0(\RR)}_{=0}.
		\]
		Now, since $\QQ$ is path-disconnected
		(i.e.\ no two of its points are path-connected)
		it follows that $\wt H_0(\QQ)$ consists of
		countably infinitely many copies of $\ZZ$.
	\end{sol}
\end{problem}

\begin{sproblem}
	What are the local homology groups of a topological $n$-manifold?
\end{sproblem}

\begin{problem}
	Let \[ X = \{(x,y) \mid x \ge 0\} \subseteq \RR^2 \]
	denote the half-plane.
	What are the local homology groups of points in $X$?
	% https://math.stackexchange.com/questions/350667/local-homology-group-a-homeomorphism-takes-the-boundary-to-the-boundary
\end{problem}

\begin{problem}
	[Brouwer-Jordan separation theorem,
	generalizing Jordan curve theorem]
	\yod
	Let $X \subseteq \RR^n$ be a subset
	which is homeomorphic to $S^{n-1}$.
	Prove that $\RR^n \setminus X$
	has exactly two path-connected components.
	\begin{hint}
		For any $n$, prove by induction for $k=1,\dots,n-1$ that
		(a) if $X$ is a subset of $S^n$ homeomorphic to $D^k$
		then $\wt H_i(S^n \setminus X) = 0$;
		(b) if $X$ is a subset of $S^n$ homeomorphic to $S^k$
		then $\wt H_i(S^n \setminus X) = \ZZ$ for $i=n-k-1$
		and $0$ otherwise.
	\end{hint}
	\begin{sol}
		This is shown in detail in Section 2.B of Hatcher.
	\end{sol}
\end{problem}
