\chapter{Abelian categories}
\label{ch:abelian_categories}
In this chapter I'll translate some more familiar concepts into categorical language;
this will require some additional assumptions about our category,
culminating in the definition of a so-called ``abelian category''.
Once that's done, I'll be able to tell you what this ``diagram chasing'' thing is all about.

Throughout this chapter, ``$\injto$'' will be used for monic maps and ``$\surjto$'' for epic maps.

\section{Zero objects, kernels, cokernels, and images}
\prototype{In $\catname{Grp}$, the trivial group and homomorphism
are the zero objects and morphisms.
If $G$, $H$ are abelian then the cokernel
of $\phi \colon G \to H$ is $H/\img \phi$.}

A \vocab{zero object} of a category is an object $0$ which is both initial and terminal;
of course, it's unique up to unique isomorphism.
For example, in $\catname{Grp}$ the zero object is the trivial group,
in $\catname{Vect}_k$ it's the zero-dimensional vector space consisting of one point, and so on.
\begin{ques}
	Show that $\catname{Set}$ and $\catname{Top}$ don't have zero objects.
\end{ques}
For the rest of this chapter, all categories will have zero objects.

In a category $\AA$ with zero objects, any two objects $A$ and $B$ thus have a distinguished morphism
\[ A \to 0 \to B \]
which is called the \vocab{zero morphism} and also denoted $0$.
For example, in $\catname{Grp}$ this is the trivial homomorphism.

We can now define:
\begin{definition}
	Consider a map $A \taking f B$.
	The \vocab{kernel} is defined as the equalizer of this map and the map $A \taking 0 B$.
	Thus, it's a map $\ker f \colon \Ker f \injto A$ such that
	\begin{center}
	\begin{tikzcd}
		\Ker f \ar[rd, "0", dashed] \ar[d, "\ker f"', hook] \\
		A \ar[r, "f"'] & B
	\end{tikzcd}
	\end{center}
	commutes, and moreover any other map with the same property factors uniquely through $\Ker f$
	(so it is universal with this property).
	By \Cref{prob:equalizer_monic}, $\ker f$ is a monic morphism,
	which justifies the use of ``$\injto$''.
\end{definition}
Notice that we're using $\ker f$ to represent the map and $\Ker f$ to represent the object
Similarly, we define the cokernel, the dual notion:
\begin{definition}
	Consider a map $A \taking f B$.
	The \vocab{cokernel} of $f$ is a map $\coker f \colon B \surjto \Coker f$ such that
	\begin{center}
	\begin{tikzcd}
		A \ar[r, "f"] \ar[rd, "0"', dashed] & B \ar[d, "\coker f", surjective head] \\
		& \Coker f
	\end{tikzcd}
	\end{center}
	commutes, and moreover any other map with the same property factors
	uniquely through $\Coker f$ (so it is universal with this property).
	Thus it is the ``coequalizer'' of this map and the map $A \taking 0 B$.
	By the dual of \Cref{prob:equalizer_monic}, $\coker f$ is an epic morphism,
	which justifies the use of ``$\surjto$''.
\end{definition}
Think of the cokernel of a map $A \taking f B$ as ``$B$ modulo the image of $f$''.
\begin{example}
	[Cokernels]
	Consider the map $\Zc6 \to D_{12} = \left\langle r,s \mid r^6=s^2=1, rs=sr\inv\right\rangle$.
	Then the cokernel of this map in $\catname{Grp}$ is $D_{12} / \left\langle r \right\rangle \cong \Zc 2$.
\end{example}
This doesn't always work out quite the way we want since in general the image of
a homomorphism need not be normal in the codomain.
Nonetheless, we can use this to define:
\begin{definition}
	The \vocab{image} of $A \taking f B$ is the kernel of $\coker f$.
	We denote $\Img f = \Ker(\coker f)$.
	This gives a unique map $\img f \colon A \to \Img f$.
\end{definition}
When it exists, this coincides with our concrete notion of ``image''.
Picture:
\begin{center}
\begin{tikzcd}
	A \ar[rd, "\exists!"'] \ar[rr, "f"] \ar[rrrd, "0"', near start, dashed]
		&& B \ar[rd, surjective head, "\coker f"] \\
	& \Img f \ar[ur, hook] \ar[rr, "0", dashed] && \Coker f
\end{tikzcd}
\end{center}
Note that by universality of $\Img f$,
we find that there is a unique map
$\img f \colon A \to \Img f$ that makes the entire diagram commute.

\section{Additive and abelian categories}
\prototype{$\catname{Ab}$, $\catname{Vect}_k$, or more generally $\catname{Mod}_R$.}
We can now define the notion of an additive and abelian category,
which are the types of categories where this notion is most useful.

\begin{definition}
	An \vocab{additive category} $\AA$ is one such that:
	\begin{itemize}
		\ii $\AA$ has a zero object, and any two objects have a product.
		\ii More importantly: every $\Hom_\AA(A, B)$ forms an \emph{abelian group} (written additively)
		such that composition distributes over addition:
		\[ (g+h)\circ f = g\circ f + h\circ f
			\quad\text{and}\quad
			f\circ(g+h) = f\circ g + f \circ h. \]
		The zero map serves as the identity element for each group.
\end{itemize}
\end{definition}
In short:
\begin{moral}
	In an additive category, you can add two morphisms.
\end{moral}
Which is the only definition that makes sense anyway, we cannot talk about elements.

\begin{definition}
	An \vocab{abelian category} $\AA$ is one with the additional properties that
	for any morphism $A \taking f B$,
	\begin{itemize}
		\ii The kernel and cokernel exist, and
		\ii The morphism factors through the image so that $\img(f)$ is epic.
	\end{itemize}
	So, this yields a diagram
	\begin{center}
	\begin{tikzcd}
		\Ker(f) \ar[rd, hook, "\ker(f)"']
		&&
		\Img(f) \ar[rd, hook]
		&&
		\Coker(f) \\
		& A \ar[ru, "\img(f)", surjective head] \ar[rr, "f"', dashed]
		&& B \ar[ru, "\coker(f)"', surjective head]
	\end{tikzcd}
	\end{center}
\end{definition}

\begin{example}[Examples of abelian categories]
	\listhack
	\begin{enumerate}[(a)]
		\ii $\catname{Vect}_k$, $\catname{Ab}$ are abelian categories,
		where $f+g$ takes its usual meaning.
		\ii Generalizing this, the category $\catname{Mod}_R$ of $R$-modules is abelian.
		\ii $\catname{Grp}$ is not even additive, because there is no way to assign
		a commutative addition to pairs of morphisms.
	\end{enumerate}
\end{example}

From now on, you can basically forget about additive category, we will be working in abelian
category.

In general, once you assume a category is abelian, all the properties you would want
of these kernels, cokernels, \dots\ that you would guess hold true.
For example,
\begin{proposition}[Monic $\iff$ trivial kernel]
	A map $A \taking f B$ is monic if and only if its kernel is $0 \to A$.
	Dually, $A \taking f B$ is epic if and only if its cokernel is $B \to 0$.
\end{proposition}
\begin{proof}
	The easy direction is:
	\begin{exercise}
		Show that if $A \taking f B$ is monic, then $0 \to A$ is a kernel.
		(This holds even in non-abelian categories.)
	\end{exercise}
	Of course, since kernels are unique up to isomorphism, monic $\implies$ $0$ kernel.
	On the other hand, assume that $0 \to A$ is a kernel of $A \taking f B$.
	For this we can exploit the group structure of the underlying homomorphisms now.
	Assume the diagram
	\begin{center}
	\begin{tikzcd}
		Z \ar[r, "g", shift left] \ar[r, "h"', shift right] & A \ar[r, "f"] & B
	\end{tikzcd}
	\end{center}
	commutes.
	Then $(g - h) \circ f = g \circ f - h \circ f = 0$, and we've arrived at a commutative diagram.
	\begin{center}
	\begin{tikzcd}
		Z \ar[d, "g-h"'] \ar[rd, dashed, "0"] & \\
		A \ar[r, "f"'] & B
	\end{tikzcd}
	\end{center}
	But since $0 \to A$ is a kernel it follows that $g-h$ factors through $0$,
	so $g-h = 0 \implies g = h$, which is to say that $f$ is monic.
\end{proof}
\begin{proposition}[Isomorphism $\iff$ monic and epic]
	In an abelian category,
	a map is an isomorphism if and only if it is monic and epic.
\end{proposition}
\begin{proof}
	Omitted. (The Mitchell embedding theorem
	presented later implies this anyways for
	most situations we care about,
	by looking at a small sub-category.)
\end{proof}

\section{Exact sequences}
\label{sec:exact_sequences}
\prototype{$0 \to G \to G \times H \to H \to 0$ is exact.}
Exact sequences will seem exceedingly unmotivated until you learn about homology groups,
which is one of the most natural places that exact sequences appear.
In light of this, it might be worth trying to read the chapter on homology groups
simultaneously with this one.

First, let me state the definition for groups, to motivate the general categorical definition.
A sequence of groups
\[ G_0 \taking{f_1} G_1 \taking{f_2} G_2 \taking{f_3} \dots \taking{f_n} G_n \]
is \emph{exact} at $G_k$ if the image of $f_k$ is the kernel of $f_{k+1}$.
We say the entire sequence is exact if it's exact at $k=1,\dots,n-1$.
\begin{example}
	[Exact sequences]
	\listhack
	\begin{enumerate}[(a)]
		\ii The sequence
		\[ 0 \to \Zc 3
			\overset{\times 5}{\injto} \Zc{15}
			\surjto \Zc{5}
			\to 0 \]
		is exact.
		Actually, $0 \to G \injto G \times H \surjto H \to 0$ is exact in general.
		(Here $0$ denotes the trivial group.)
		\ii For groups, the map $0 \to A \to B$ is exact if and only if $A \to B$ is injective.
		\ii For groups, the map $A \to B \to 0$ is exact if and only if $A \to B$ is surjective.
	\end{enumerate}
\end{example}

If you look at the prototypical example, actually, a \vocab{short exact sequence} (an exact sequence
of the form $0 \to A \to B \to C \to 0$) is the most natural things ever:
\begin{moral}
	It's basically just an equation $C = B/A$.
\end{moral}
Whenever you see ``there is a short exact sequence $0 \to A \to B \to C \to 0$'', you
can mentally translate it to ``$C \cong B/A$''; but there's a slight difference:
A group has more structures than a number, so the sequence also contains the information of the
maps --- the map that identifies $A$ with a subgroup of $B$, and the map that identifies $C$
with the quotient group $B/A$.
\begin{example}[More exact sequences]
	\listhack
	\begin{enumerate}[(a)]
		\ii The sequence
			\[ 0 \to \ZZ \taking{\times 3} \ZZ \to \Zc{3} \to 0 \]
		is short exact.
		\ii So is
			\[ 0 \to \ZZ \taking{\times 5} \ZZ \to \Zc{5} \to 0. \]
	\end{enumerate}
	As you can see, the written equation ``$C \cong B/A$'' is not completely accurate, the map $A
	\to B$ also matters in determining what $C$ is. This also explains the common notation: the
	image of the map $\ZZ \taking{\times 3} \ZZ$ is usually written $3 \ZZ$, thus $\Zc{3} =
	\frac{\ZZ}{3 \ZZ}$.
\end{example}

Now, we want to mimic this definition in a general \emph{abelian} category $\AA$.
So, let's write down a criterion for when $A \taking f B \taking g C$ is exact.
First, we had better have that $g \circ f = 0$,
which encodes the fact that $\img(f) \subseteq \ker(g)$.
Adding in all the relevant objects, we get the commutative diagram below.
\begin{center}
\begin{tikzcd}
	A \ar[rd, "f"] \ar[rr, dashed, "0"] \ar[dd, "\img f"', surjective head] && C  \\
	& B \ar[ru, "g"] \\
	\Img f \ar[ru, hook, "\iota"] \ar[rr, dashed, "\exists!"] &&
	\Ker g \ar["0"', dashed, uu] \ar[lu, hook']
\end{tikzcd}
\end{center}
Here the map $A \surjto \Img f$ is epic since we are assuming $\AA$ is an abelian category.
So, we have that
\[ 0 = (g \circ \iota) \circ \img f = g \circ (\iota \circ \img f) = g \circ f = 0 \]
but since $\img f$ is epic, this means that $g \circ \iota = 0$.
So there is a \emph{unique} map $\Img f \to \Ker g$, and we require that this diagram commutes.
In short,
\begin{definition}
	Let $\AA$ be an abelian category. The sequence
	\[ \dots \to A_{n-1} \taking{f_n} A_n \taking{f_{n+1}} A_{n+1} \to \dots \]
	is \vocab{exact} at $A_n$ if $f_n \circ f_{n+1} = 0$ and
	the canonical map $\Img f_n \to \Ker f_{n+1}$ is an isomorphism.
	The entire sequence is exact if it is exact at each $A_i$.
	(For finite sequences we don't impose condition on the very first and very last object.)
\end{definition}

\begin{exercise}
	Show that, as before, $0 \to A \to B$ is exact $\iff$ $A \to B$ is monic.
\end{exercise}

\section{The Freyd-Mitchell embedding theorem}
We now introduce the Freyd-Mitchell embedding theorem,
which essentially says that any abelian category can be realized as a concrete one.

\begin{definition}
	A category is \vocab{small} if $\obj(\AA)$ is a set (as opposed to a class),
	i.e.\ there is a ``set of all objects in $\AA$''.
	For example, $\catname{Set}$ is not small because there is no set of all sets.
\end{definition}

\begin{theorem}
	[Freyd-Mitchell embedding theorem]
	Let $\AA$ be a small abelian category.
	Then there exists a ring $R$ (with $1$ but possibly non-commutative)
	and a full, faithful, exact functor onto the category of left $R$-modules.
\end{theorem}
Here a functor is \vocab{exact} if it preserves exact sequences.
This theorem is good because it means
\begin{moral}
	You can basically forget about all the weird definitions
	that work in any abelian category.
\end{moral}
Any time you're faced with a statement about an abelian category,
it suffices to just prove it for a ``concrete'' category
where injective/surjective/kernel/image/exact/etc.\
agree with your previous notions.

\begin{remark}
	The ``small'' condition is a technical obstruction
	that requires the objects $\AA$ to actually form a set.
	I'll ignore this distinction,
	because one can almost always work around it
	by doing enough set-theoretic technicalities.
\end{remark}

For example, let's prove:
\begin{lemma}
	[Short five lemma]
	In an abelian category, consider the commutative diagram
	\begin{center}
	\begin{tikzcd}
		0 \ar[r]
		& A \ar[r, hook, "p"] \ar[d, "\alpha", "\cong"']
			& B \ar[r, surjective head, "q"] \ar[d, "\beta"]
			& C \ar[r] \ar[d, "\gamma", "\cong"']
			& 0 \\
		0 \ar[r] & A' \ar[r, hook, "p'"'] & B' \ar[r, surjective head, "q'"'] & C' \ar[r] & 0
	\end{tikzcd}
	\end{center}
	and assume the top and bottom rows are exact.
	If $\alpha$ and $\gamma$ are isomorphisms, then so is $\beta$.
\end{lemma}

\begin{proof}
	We prove that $\beta$ is epic (with a similar proof to get monic).
%	One can show that it's possible to take a small subcategory of $\AA$
%	containing the $10$ elements and $13$ arrows above, as well as all necessary
%	kernels, cokernels, et cetera.
%	(Essentially, let $\BB_0$ be the diagram above and let $\BB_{i+1}$ add in any needed objects;
%	then $\bigcup \BB_i$ is a set-sized category).
	By the embedding theorem we can treat the category as $R$-modules over some $R$.
	This lets us do a so-called ``diagram chase'' where we move elements around the picture,
	using the concrete interpretation of our category as $R$-modules.

	Let $b'$ be an element of $B'$.
	Then $q'(b') \in C'$, and since $\gamma$ is surjective, we have a $c$ such that $\gamma(c) = b'$,
	and finally a $b \in B$ such that $q(b) = c$.
	Picture:
	\begin{center}
	\begin{tikzcd}
		b \in B \ar[r, "q", mapsto] \ar[d, "\beta"', dashed] & c \in C \ar[d, "\gamma", "\cong"', mapsto] \\
		b' \in B' \ar[r, mapsto, "q'"] & c' \in C'
	\end{tikzcd}
	\end{center}
	Now, it is not necessarily the case that $\beta(b) = b'$.
	However, since the diagram commutes we at least have that
	\[ q'(b') = q'(\beta(b)) \]
	so $b' - \beta(b) \in \Ker q' = \Img p'$, and there is an $a' \in A'$ such that
	$p'(a') = b' - \beta(b)$;
	use $\alpha$ now to lift it to $a \in A$.
	Picture:
	\begin{center}
	\begin{tikzcd}
		a \in A \ar[d, mapsto] & b \in B \\
		a' \in A' \ar[r, mapsto] & b' - \beta(b) \in B' \ar[r, mapsto] & 0 \in C
	\end{tikzcd}
	\end{center}
	Then, we have
	\[
		\beta(b + q(a)) = \beta b + \beta p a
		= \beta b + p' \alpha a
		= \beta b + (b' - \beta b)
		= b'
	\]
	so $b' \in \Img \beta$ which completes the proof that $\beta'$ is surjective.
\end{proof}

In general, proofs in the style above (whether or not they use the embedding theorem)
are sometimes referred to by the name \emph{diagram chasing}.
I'm not sure there's an exact definition for this term,
but the following quote is due to Aluffi \cite{ref:aluffi}:
\begin{quote}
	Proving the snake lemma [\Cref{prob:snake}]
	is something that should not be done in public,
	and it is notoriously useless to write down the
	details of the verification for others to read:
	the details are all essentially obvious,
	but they lead quickly to a notational quagmire.
	Such proofs are collectively known as the sport of diagram chase,
	best executed by pointing several fingers at different parts of a diagram on a blackboard,
	while enunciating the elements one is manipulating and stating their fate.
\end{quote}

\section{Breaking long exact sequences}
\prototype{First isomorphism theorem.}

In fact, it turns out that any exact sequence breaks into short exact sequences.
This relies on:
\begin{proposition}[``First isomorphism theorem'' in abelian categories]
	\label{prop:break_exact}
	Let $A \taking f B$ be an arrow of an abelian category.
	Then there is an exact sequence
	\[ 0 \to \Ker f \taking{\ker f} A \taking{\img f} \Img f \to 0. \]
\end{proposition}

\begin{example}
Let's analyze this theorem in our two examples of abelian categories:
\begin{enumerate}[(a)]
	\ii In the category of abelian groups,
	this is basically the first isomorphism theorem.
	\ii In the category $\catname{Vect}_k$,
	this amounts to the rank-nullity theorem, \Cref{thm:rank_nullity}.
\end{enumerate}
\end{example}
Thus, any exact sequence can be broken into short exact sequences, as
\begin{center}
	\begin{tikzcd}[sep=0.8em]
	&& 0 \ar[rd] && 0 && 0 \ar[rd] && 0 \\
	&&& C_{n} \ar[rd] \ar[ru] &&&& C_{n+2} \ar[ru] \ar[rd] \\
	{\color{red}\dots} \ar[rr, red] \ar[rd]
		&& {\color{red}A_{n-1}} \ar[rr, "f_{n-1}"', red] \ar[ru]
		&& {\color{red}A_n} \ar[rr, "f_n", red] \ar[rd]
		&& {\color{red}A_{n+1}} \ar[rr, "f_{n+1}"', red] \ar[ru]
		&& \dots
	\\
	& C_{n-1} \ar[ru] \ar[rd] &&&& C_{n+1} \ar[ru] \ar[rd] \\
	0 \ar[ru] && 0 && 0 \ar[ru] && 0
\end{tikzcd}
\end{center}
where $C_k = \img f_{k-1} = \ker f_k$ for every $k$.

\section\problemhead
\begin{problem}
	[Four lemma]
	In an abelian category, consider the commutative diagram
	\begin{center}
	\begin{tikzcd}
		A \ar[r, "p"] \ar[d, "\alpha"', surjective head]
			& B \ar[r, "q"] \ar[d, "\beta"', hook]
			& C \ar[r, "r"] \ar[d, "\gamma"']
			& D \ar[d, "\delta"', hook] \\
		A' \ar[r, "p'"'] & B' \ar[r, "q'"'] & C' \ar[r, "r'"'] & D'
	\end{tikzcd}
	\end{center}
	where the first and second rows are exact.
	Prove that if $\alpha$ is epic, and $\beta$ and $\delta$ are monic,
	then $\gamma$ is monic.
	\begin{sol}
		Let $c \in C$ with $\gamma(c) = 0$.
		We show $c = 0$.
		This proceeds in a diagram chase:
		\begin{itemize}
			\ii Note that $0 = r'(\gamma(c)) = \delta(r(c))$, and since $\delta$
			is injective, it follows that $r(c) = 0$.
			\ii Since the top row is exact,
			it follows $c = q(b)$ for some $b \in B$.
			\ii Then $q'(\beta(b)) = 0$,
			so if we let $b' = \beta(b)$,
			then $b' \in \ker(q')$.
			As the bottom row is exact, there exists $a'$ with $p'(a') = b'$.
			\ii Since $\alpha$ is injective,
			there is $a \in A$ with $\alpha(a) = a'$.
			\ii Since $\beta$ is injective,
			it follows that $p(a) = b$.
			\ii Since the top row is exact, and $b$ is in the image of $p$,
			it follows that $0 = q(b) = c$ as needed.
		\end{itemize}
	\end{sol}
\end{problem}

\begin{problem}
	[Five lemma]
	\onechili
	In an abelian category, consider the commutative diagram
	\begin{center}
	\begin{tikzcd}
		A \ar[r, "p"] \ar[d, "\alpha"', surjective head]
			& B \ar[r, "q"] \ar[d, "\beta"', "\cong"]
			& C \ar[r, "r"] \ar[d, "\gamma"']
			& D \ar[r, "s"] \ar[d, "\delta"', "\cong"]
			& E \ar[d, "\eps"', hook] \\
		A' \ar[r, "p'"'] & B' \ar[r, "q'"'] & C' \ar[r, "r'"'] & D' \ar[r, "s'"'] & E'
	\end{tikzcd}
	\end{center}
	where the two rows are exact,
	$\beta$ and $\delta$ are isomorphisms,
	$\alpha$ is epic, and $\eps$ is monic.
	Prove that $\gamma$ is an isomorphism.
\end{problem}

\begin{sproblem}
	[Snake lemma]
	\onechili
	In an abelian category, consider the diagram
	\begin{center}
	\begin{tikzcd}
		& A \ar[r, "f"] \ar[d, "a"] & B \ar[r, "g", surjective head] \ar[d, "b"] & C \ar[r] \ar[d, "c"] & 0 \\
		0 \ar[r] & A' \ar[r, hook, "f'"'] & B' \ar[r, "g'"'] & C'
	\end{tikzcd}
	\end{center}
	where the first and second rows are exact sequences.
	Prove that there is an exact sequence
	\[ \Ker a \to \Ker b \to \Ker c \to \Coker a \to \Coker b \to \Coker c. \]
	\label{prob:snake}
\end{sproblem}

\begin{problem}
	[An additive category that is not abelian]
	Consider a category, where:
	\begin{itemize}
		\ii the objects are pairs of abelian groups $(B, A)$ where $A$ is a subgroup of $B$.
		\ii the morphisms $(B, A) \to (B', A')$ are maps $f \colon B \to B'$ where
		$f\im(A) \subseteq A'$.
	\end{itemize}
	(You can think of this similar to the $\catname{PairTop}$ category, seen in
	\Cref{ch:relative_hom}. We use abelian groups here to make the category additive.)

	This category can be equivalently viewed as the category of short exact sequences
	$0 \to A \to B \to B/A \to 0$ of abelian groups.

	Show that the arrow $(X, 0) \to (X, X)$ is monic and epic, but not an
	isomorphism. Conclude that the category is not abelian.
\end{problem}
