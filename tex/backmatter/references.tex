\chapter{Reference Requests and Thoughts on Material Presentation}
I don't like resources too much: otherwise I wouldn't have written this.
But anyways\dots

\section{Enthusiastically Recommended Further Reading}
\subsection{Category Theory}
I enthusiastically recommend \cite{ref:msci}.
Beautifully written.
You might try reading chapters {2-4} in reverse order though:
I found that limits were much more intuitive than adjoints.
But your mileage may vary.

The category theory will make more sense as you learn
more examples of structures: it will help to have read,
say, the chapters on groups, rings, and modules.

\subsection{Analytic Number Theory}
Not covered at all in Napkin, because it would have taken too much
space, and the superb \cite{ref:analytic_NT} exists.
I strongly recommend these lecture notes.
They are highly accessible and delightful to read.

\section{Subjects I have (Strong) Opinions On}
\subsection{Linear Algebra}
See the rant in \Cref{sec:basis_evil}.
I think it's immoral to not think of a matrix as a linear map.
Hence I do not recommend most linear algebra books.

I've heard that \cite{ref:axler} follows this approach,
hence the appropriate name ``Linear Algebra Done Right''.
I followed with heavy modifications the proceedings of Math 55a,
see \cite{ref:55a}

\subsection{General Topology}
My personal view is that spaces should either be metric or the Zariski topology.

I essentially follow the approach of \cite{ref:pugh}, using metric topology first.
I find that metric spaces are far more intuitive, and are a much better
way to get a picture of what open / closed / compact etc.\ sets look like.
This is the approach history took;
general topology grew out of metric topology.

In my opinion, it's a very bad idea to start any general topology class
by defining what a general topological space is.
No one will have any idea what an open set should look like.

\todo{abstract alg, \dots}

\subsection{Abstract Algebra}
I teach groups before commutative rings but might convert.
Rings have better examples, don't have the confusion of multiplicative
notation for additive groups, and modding out by ideals makes more sense.
Oh well.

\subsection{Differential Geometry}
As linear algebra is butchered, so is multivariable calculus,
with partial derivatives and matrices superseding linear maps.

