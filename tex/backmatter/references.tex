\chapter{Pedagogical comments and references}
\label{ch:refs}
Here are some higher-level comments on the way specific topics were presented,
as well as pointers to further reading.

\section{Basic algebra and topology}
\subsection{Linear algebra and multivariable calculus}
Following the comments in \Cref{sec:basis_evil},
I think that most presentations of linear algebra and multivariable calculus
are \emph{immoral}, because they miss the two key ideas, namely:
\begin{itemize}
	\ii In linear algebra, we study \emph{linear maps} between spaces.
	\ii In calculus, we \emph{approximate functions at points by linear functions}.
\end{itemize}
In particular, I believe linear algebra should \emph{always} be taught
before multivariable calculus.
The fact that this is the opposite of what happens is a testament to how
atrociously these subjects are usually taught.

In particular, I do not recommend most linear algebra or
multivariable calculus books.

For linear algebra, I've heard that \cite{ref:axler} follows this approach,
hence the appropriate name ``Linear Algebra Done Right''.
I followed with heavy modifications the proceedings of Math 55a,
see \cite{ref:55a}.

For multivariable calculus and differential geometry,
I found the notes \cite{ref:manifolds} to be unusually well written.
I referred to it frequently while I was enrolled in Math 55b \cite{ref:55b}.

\subsection{General topology}
My personal view on spaces is that every space I ever work with
is either metrizable or is the Zariski topology.

I adopted the approach of \cite{ref:pugh}, using metric topology first.
I find that metric spaces are far more intuitive, and are a much better
way to get a picture of what open / closed / compact etc.\ sets look like.
This is the approach history took; general topology grew out of metric topology.

In my opinion, it's a very bad idea to start any general topology class
by defining what a general topological space is,
because it doesn't communicate a good picture of open and closed sets
to draw pictures of.

\subsection{Groups and commutative algebra}
I teach groups before commutative rings but might convert later.
Rings have better examples, don't have the confusion of multiplicative
notation for additive groups, and modding out by ideals is more intuitive.

There's a specific thing I have a qualm with in group theory:
the way that the concept of a normal subgroup is introduced.
Only \cite{ref:gowers} does something similar to what I do.
Most other people simply \emph{define} a normal subgroup $N$
as one with $gNg\inv$ and then proceed to define modding out,
without taking the time to explain where this definition comes from.
I remember distinctly this concept as the first time in learning math
where I didn't understand what was going on.
Only in hindsight do I see where this definition came from;
I tried hard to make sure my own presentation didn't have this issue.

I deliberately don't include a chapter on just commutative algebra;
other than the chapter on rings and ideals.
The reason is that I always found it easier to learn
commutative algebra theorems on the fly,
in the context of something like algebraic number theory or algebraic geometry.
For example, I finally understand why radicals and the Nullstellensatz were important
when I saw how they were used in algebraic geometry.
Before then, I never understood why I cared about them.


\section{Second-year topics}
\subsection{Complex analysis}
I picked the approach of presenting the Cauchy-Gorsat theorem is given
(rather than proving a weaker version by Stoke's theorem, or whatever),
and then deriving the key result that holomorphic functions are analytic
from it. I think this most closely mirrors the ``real-life'' use of complex
analysis, i.e.\ the computation of contour integrals.

The main reference for this chapter was \cite{ref:dartmouth}, which I recommend.

\subsection{Category theory}
I enthusiastically recommend \cite{ref:msci},
from which my chapters are based,
and which contains much more than I had time to cover.
You might try reading chapters {2-4} in reverse order though:
I found that limits were much more intuitive than adjoints.
But your mileage may vary.

The category theory will make more sense as you learn
more examples of structures: it will help to have read,
say, the chapters on groups, rings, and modules.

\subsection{Quantum algorithms}
The exposition given here is based off a full semester
at MIT taught by Seth Lloyd, in 18.435J \cite{ref:18-435}.
It is written in a far more mathematical perspective.

I only deal with finite-dimensional Hilbert spaces,
because that is all that is needed for Shor's algorithm,
which is the point of this chapter.
This is not an exposition intended for someone who wishes to seriously
study quantum mechanics (though it might be a reasonable first read):
the main purpose is to give students a little appreciation for
what this ``Shor's algorithm'' that everyone keeps talking about is.

\subsection{Representation theory}
I staunchly support teaching the representation of algebras first,
and then specializing to the case of groups by looking at $k[G]$.
The primary influence for the chapters here is \cite{ref:etingof},
and you might think of what I have here as just some selections
from the first four chapters of this source.

\subsection{Set theory}
Set theory is far off the beaten path.
The notes I have written are based off the class I took at Harvard College,
Math 145a \cite{ref:145a}.

My general impression is that the way I present set theory
(trying to remain intuitive and informal in what a topic full of minefields)
is not standard. Possible other reference: \cite{ref:miquel}.

\section{Advanced topics}
\subsection{Algebraic topology}
I cover the fundamental group $\pi_1$ first, because I think the subject is more
intuitive this way. A possible reference in this topic is \cite{ref:munkres}.
Only later do I do the much more involved homology groups.
The famous standard reference for algebraic topology is \cite{ref:hatcher},
which is what almost everyone uses these days.
But I also found \cite{ref:maxim752} to be very helpful,
particularly in the part about cohomology rings.

I don't actually do very much algebraic topology.
In particular, I think the main reason to learn algebraic topology
is to see the construction of the homology and cohomology groups
from the chain complex, and watch the long exact sequence in action.
The concept of a (co)chain complex comes up often in other contexts as well,
like the cohomology of sheaves or Galois cohomology.
Algebraic topology is by far the most natural one.

I use category theory extensively, because it makes life easier.

\subsection{Algebraic number theory}
I learned from \cite{ref:oggier_NT}, using \cite{ref:lenstra_chebotarev}
for the part about the Chebotarev density theorem.

When possible I try to keep the algebraic number theory chapter close at
heart to an ``olympiad spirit''.
Factoring in rings like $\ZZ[i]$ and $\ZZ[\sqrt{-5}]$
is very much an olympiad-flavored topic at heart:
one is led naturally to the idea of factoring in general rings of integers,
around which the presentation is built.
As a reward for the entire buildup, the exposition finishes
with the application of the Chebotarev density theorem to IMO 2003, Problem 6.

\subsection{Algebraic geometry}
My preferred introduction to algebraic geometry is \cite{ref:gathmann}.

At the outset, I think there are three main approaches:
\begin{itemize}
	\ii Only looking at affine and projective varieties,
	as part of an ``introductory'' class.
	\ii Studying affine and projective varieties closely
	and using them as the motivating example of a \emph{scheme},
	and then developing algebraic geometry from there.
	\ii Jumping straight into the definition of a scheme,
	as in the well-respected \cite{ref:vakil}.
\end{itemize}
I have gone with the second approach,
I think that if you don't know what a scheme is,
then you haven't learned algebraic geometry.
But on the other hand I think the definition of a scheme is woefully
difficult to digest without having a good handle first on varieties.
But these opinions are based on my personal experience of having
tried to learn the subject through all three approaches over a period of a year.
Your mileage may vary.

\section{Further topics}
\subsection{Analytic number theory}
I never had time to write up notes in Napkin for these.
If you're interested though, I recommend \cite{ref:analytic_NT} exists.
They are highly accessible and delightful to read.
The only real prerequisites are a good handle on Cauchy's residue formula.
