\chapter{Bonus: A hint of Pontryagin duality}
\label{ch:pontryagin}

In this short chapter we will give statements
about how to generalize our Fourier analysis
(a bonus chapter \Cref{ch:fourier})
to a much wider class of groups $G$.

\section{LCA groups}
\prototype{$\TT$, $\RR$.}
Earlier we played with $\RR$,
which is nice because in addition to being a topological space,
it is also an abelian group under addition.
These sorts of objects which are both groups and spaces have a name.
\begin{definition}
	A group $G$ is a \vocab{topological group}
	is a Hausdorff\footnote{Some authors omit the Hausdorff condition.}
	topological space equipped also with a group operation $(G, \cdot)$,
	such that both maps
	\begin{align*}
		G \times G &\to G \quad\text{ by }\quad (x,y) \mapsto xy \\
		G &\to G \quad\text{ by }\quad x \mapsto x\inv
	\end{align*}
	are continuous.
\end{definition}

For our Fourier analysis, we need some additional conditions.
\begin{definition}
	A \vocab{locally compact abelian (LCA) group} $G$
	is one for which the group operation is abelian,
	and moreover the topology is \emph{locally compact}:
	for every point $p$ of $G$,
	there exists a compact subset $K$ of $G$
	such that $K \ni p$, and $K$ contains some open neighborhood of $p$.
\end{definition}

Our previous examples all fall into this category:
\begin{example}
	[Examples of locally compact abelian groups]
	\listhack
	\begin{itemize}
		\ii Any finite group $Z$ with the discrete topology is LCA.
		\ii The circle group $\TT$ is LCA and also in fact compact.
		\ii The real numbers $\RR$ are an example of an LCA group
		which is \emph{not} compact.
	\end{itemize}
\end{example}

These conditions turn out to be enough
for us to define a measure on the space $G$.
The relevant theorem, which we will just quote:
\begin{theorem}
	[Haar measure]
	Let $G$ be a locally compact abelian group.
	We regard it as a measurable space using its Borel $\sigma$-algebra $\SB(G)$.
	There exists a measure $\mu \colon \SB(G) \to [0,\infty]$, called the \vocab{Haar measure},
	satisfying the following properties:
	\begin{itemize}
		\ii $\mu(gS) = \mu(S)$ for every $g \in G$ and measurable $S$.
		That means that $\mu$ is ``translation-invariant'' under translation by $G$.
		\ii $\mu(K)$ is finite for any compact set $K$.
		\ii if $S$ is measurable, then $\mu(S) = \inf\left\{ \mu(U) \mid U \supseteq S \text{ open} \right\}$.
		\ii if $U$ is open, then $\mu(U) = \sup\left\{ \mu(K) \mid K \subseteq U \text{ compact} \right\}$.
	\end{itemize}
	Moreover, it is unique up to scaling by a positive constant.
\end{theorem}

\begin{remark}
	Note that if $G$ is compact, then $\mu(G)$ is finite (and positive).
	For this reason the Haar measure on a LCA group $G$
	is usually normalized so $\mu(G) = 1$.
\end{remark}

For this chapter, we will only use the first two properties at all,
and the other two are just mentioned for completeness.
Note that this actually generalizes the chapter where
we constructed a measure on $\SB(\RR^n)$, since $\RR^n$ is an LCA group!

So, in short: if we have an LCA group, we have a measure $\mu$ on it.

\section{The Pontryagin dual}
Now the key definition is:
\begin{definition}
	Let $G$ be an LCA group.
	Then its \vocab{Pontryagin dual} is the abelian group
	\[ \wh G \defeq \left\{ \text{continuous group homomorphisms }
			\xi \colon G \to \TT \right\}. \]
	The maps $\xi$ are called \vocab{characters}.
	It can be itself made into an LCA group.\footnote{If you must
		know the topology, it is the \vocab{compact-open topology}:
		for any compact set $K \subseteq G$ and open set $U \subseteq \TT$,
		we declare the set of all $\xi$ with $\xi\im(K) \subseteq U$ to be open,
		and then take the smallest topology containing all such sets.
		We won't use this at all.}
\end{definition}
\begin{example}
	[Examples of Pontryagin duals]
	\listhack
	\begin{itemize}
		\ii $\wh{\ZZ} \cong \TT$,
		since group homomorphisms $\ZZ \to \TT$ are determined by the image of $1$.
		\ii $\wh{\TT} \cong \ZZ$.
		The characters are given by $\theta \mapsto n\theta$ for $n \in \ZZ$.
		\ii $\wh{\RR} \cong \RR$.
		This is because a nonzero continuous homomorphism
		$\RR \to S^1$ is determined by the fiber above $1 \in S^1$.
		(Algebraic topologists might see covering projections here.)
		\ii $\wh{\ZZ/n\ZZ} \cong \ZZ/n\ZZ$,
		characters $\xi$ being determined by the image $\xi(1) \in \TT$.
		\ii $\wh{G \times H} \cong \wh G \times \wh H$.
	\end{itemize}
\end{example}
\begin{exercise}
	[$\wh Z \cong Z$, for those who read \Cref{sec:FTFGAG}]
	If $Z$ is a finite abelian group, show that $\wh Z \cong Z$,
	using the results of the previous example.
	You may now recognize that the bilinear form
	$\cdot \colon Z \times Z \to \TT$
	is exactly a choice of isomorphism $Z \to \wh Z$.
	It is not ``canonical''.
\end{exercise}

True to its name as the dual,
and in analogy with $(V^\vee)^\vee \cong V$ for vector spaces $V$, we have:
\begin{theorem}
	[Pontryagin duality theorem]
	For any LCA group $G$, there is an isomorphism
	\[ G \cong \wh{\wh G} \qquad \text{by} \qquad
		x \mapsto \left( \xi \mapsto \xi(x) \right). \]
\end{theorem}

The compact case is especially nice.
\begin{proposition}
	[$G$ compact $\iff$ $\wh G$ discrete]
	Let $G$ be an LCA group.
	Then $G$ is compact if and only if $\wh G$ is discrete.
\end{proposition}
\begin{proof}
	\Cref{prob:LCA_compact}.
\end{proof}

\section{The orthonormal basis in the compact case}
Let $G$ be a compact LCA group,
and work with its Haar measure.
We may now let $L^2(G)$ be the space of
square-integrable functions to $\CC$, i.e.
\[ L^2(G) = \left\{ f \colon G \to \CC
	\quad\text{such that}\quad \int_G |f|^2 < \infty \right\}. \]
Thus we can equip it with the inner form
\[ \left< f,g \right> = \int_G f \cdot \ol{g}. \]
In that case, we get all the results we wanted before:
\begin{theorem}
	[Characters of $\wh G$ form an orthonormal basis]
	\label{thm:god}
	Assume $G$ is LCA and compact (so $\wh G$ is discrete).
	Then the characters
	\[ (e_\xi)_{\xi \in \wh G}
		\qquad\text{by}\qquad e_\xi(x) = e(\xi(x)) = \exp(2\pi i \xi(x)) \]
	form an orthonormal basis of $L^2(G)$.
	Thus for each $f \in L^2(G)$ we have
	\[ f = \sum_{\xi \in \wh G} \wh f(\xi) e_\xi \]
	where
	\[ \wh f(\xi) = \left< f, e_\xi \right>
		= \int_G f(x) \exp(-2\pi i \xi(x)) \; d\mu. \]
\end{theorem}
The sum $\sum_{\xi \in \wh G}$ makes sense since $\wh G$ is discrete.
In particular,
\begin{itemize}
	\ii Letting $G = Z$ for a finite group $G$
	gives ``Fourier transform on finite groups''.
	\ii The special case $G = \ZZ/n\ZZ$ has its
	\href{https://en.wikipedia.org/wiki/Discrete_Fourier_transform#Definition}%
	{own Wikipedia page}: the ``discrete-time Fourier transform''.
	\ii Letting $G = \TT$ gives the ``Fourier series'' earlier.
\end{itemize}

\section{The Fourier transform of the non-compact case}
If $G$ is LCA but not compact, then \Cref{thm:god} becomes false.
On the other hand, it's still possible to define $\wh G$.
We can then try to write the Fourier coefficients anyways:
let \[ \wh f(\xi) = \int_G f \cdot \ol{e_\xi} \; d\mu \]
for $\xi \in \wh G$ and $f \colon G \to \CC$.
The results are less fun in this case, but we still have, for example:
\begin{theorem}
	[Fourier inversion formula in the non-compact case]
	Let $\mu$ be a Haar measure on $G$.
	Then there exists a unique Haar measure $\nu$ on $\wh G$
	(called the \vocab{dual measure}) such that:
	whenever $f \in L^1(G)$ and $\wh f \in L^1(\wh G)$, we have
	\[ f(x) = \int_{\wh G} \wh f(\xi) \xi(x) \; d\nu \]
	for almost all $x \in G$ (with respect to $\mu$).
	If $f$ is continuous, this holds for all $x$.
\end{theorem}
So while we don't have the niceness of a full inner product from before,
we can still in some situations at least write $f$ as integral
in sort of the same way as before.

In particular, they have special names for a few special $G$:
\begin{itemize}
	\ii If $G = \RR$, then $\wh G = \RR$,
	yielding the
	``\href{https://en.wikipedia.org/wiki/Fourier_transform}{(continuous) Fourier transform}''.
	\ii If $G = \ZZ$, then $\wh G = \TT$,
	yielding the
	``\href{https://en.wikipedia.org/wiki/Discrete-time_Fourier_transform}{discrete time Fourier transform}''.
\end{itemize}

\section{Summary}
We summarize our various flavors of Fourier analysis
from the previous sections in the following table.
In the first part $G$ is compact,
in the second half $G$ is not.
\[
	\begin{array}{llll}
		\hline
		\text{Name} & \text{Domain }G & \text{Dual }\wh G
			& \text{Characters} \\ \hline
		\text{Binary Fourier analysis} & \{\pm1\}^n
			& S \subseteq \left\{ 1, \dots, n \right\}
			& \prod_{s \in S} x_s \\
		\text{Fourier transform on finite groups} & Z
			& \xi \in \wh Z \cong Z & e( i \xi \cdot x) \\
		\text{Discrete Fourier transform} & \ZZ/n\ZZ & \xi \in \ZZ/n\ZZ
			& e(\xi x / n) \\
		\text{Fourier series} & \TT \cong [-\pi, \pi]  & n \in \ZZ
			& \exp(inx) \\ \hline
		\text{Continuous Fourier transform} & \RR & \xi \in \RR
			& e(\xi x) \\
		\text{Discrete time Fourier transform} & \ZZ & \xi \in \TT \cong [-\pi, \pi]
			& \exp(i \xi n) \\
	\end{array}
\]
You might notice that the \textbf{various names are awful}.
This is part of the reason I got confused as a high school student:
every type of Fourier series above has its own Wikipedia article.
If it were up to me, we would just use the term ``$G$-Fourier transform'',
and that would make everyone's lives a lot easier.



\section{\problemhead}
\begin{problem}
	If $G$ is compact, so $\wh G$ is discrete,
	describe the dual measure $\nu$.
	\begin{hint}
		You can read it off \Cref{thm:god}.
	\end{hint}
	\begin{sol}
		It is the counting measure.
	\end{sol}
\end{problem}

\begin{problem}
	\label{prob:LCA_compact}
	Show that an LCA group $G$ is compact
	if and only if $\wh G$ is discrete.
	(You will need the compact-open topology for this.)
	\begin{hint}
		After Pontryagin duality,
		we need to show $G$ compact implies $\wh G$ discrete
		and $G$ discrete implies $\wh G$ compact.
		Both do not need anything fancy:
		they are topological facts.
	\end{hint}
\end{problem}

\endinput
\section{Peter-Weyl}
In fact, if $G$ is a Lie group, even if $G$ is not abelian
we can still give an orthonormal basis of $L^2(G)$
(the square-integrable functions on $G$).
It turns out in this case the characters are attached to complex
irreducible representations of $G$
(and in what follows all representations are complex).

The result is given by the Peter-Weyl theorem.
First, we need the following result:
\begin{lemma}
	[Compact Lie groups have unitary reps]
	Any finite-dimensional (complex) representation $V$ of a compact Lie group $G$
	is unitary, meaning it can be equipped with a $G$-invariant inner form.
	Consequently, $V$ is completely reducible:
	it splits into the direct sum of irreducible representations of $G$.
\end{lemma}
\begin{proof}
	Suppose $B \colon V \times V \to \CC$ is any inner product.
	Equip $G$ with a right-invariant Haar measure $dg$.
	Then we can equip it with an ``averaged'' inner form
	\[ \wt B(v,w) = \int_G B(gv, gw) \; dg. \]
	Then $\wt B$ is the desired $G$-invariant inner form.
	Now, the fact that $V$ is completely reducible follows from the fact
	that given a subrepresentation of $V$, its orthogonal complement
	is also a subrepresentation.
\end{proof}

The Peter-Weyl theorem then asserts that the finite-dimensional irreducible
unitary representations essentially give an orthonormal basis for $L^2(G)$,
in the following sense.  Let $V = (V, \rho)$ be such a representation of $G$,
and fix an orthonormal basis of $e_1$, \dots, $e_d$ for $V$ (where $d = \dim V$).
The $(i,j)$th \vocab{matrix coefficient} for $V$ is then given by
\[ G \taking{\rho} \GL(V) \taking{\pi_{ij}} \CC \]
where $\pi_{ij}$ is the projection onto the $(i,j)$th entry of the matrix.
We abbreviate $\pi_{ij} \circ \rho$ to $\rho_{ij}$.
Then the theorem is:
\begin{theorem}
	[Peter-Weyl]
	Let $G$ be a compact Lie group.
	Let $\Sigma$ denote the (pairwise non-isomorphic) irreducible finite-dimensional
	unitary representations of $G$.
	Then
	\[ \left\{ \sqrt{\dim V} \rho_{ij}
			\; \Big\vert \; (V, \rho) \in \Sigma,
			\text{ and } 1 \le i,j \le \dim V \right\}  \]
	is an orthonormal basis of $L^2(G)$.
\end{theorem}
Strictly, I should say $\Sigma$ is a set of representatives of
the isomorphism classes of irreducible unitary representations,
one for each isomorphism class.

In the special case $G$ is abelian,
all irreducible representations are one-dimensional.
A one-dimensional representation of $G$ is a map
$G \injto \GL(\CC) \cong \CC^\times$,
but the unitary condition implies it is actually a map $G \injto S^1 \cong \TT$,
i.e.\ it is an element of $\wh G$.
