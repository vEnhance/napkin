\chapter{Swapping order with Lebesgue integrals}
\section{Motivating limit interchange}
\prototype{$\mathbf{1}_\QQ$ is good!}

One of the issues with the Riemann integral is
that it behaves badly with respect to convergence of functions,
and the Lebesgue integral deals with this.
This is therefore often given as a poster child
on why the Lebesgue integral has better behaviors than the Riemann one.

We technically have already seen this:
consider the indicator function $\mathbf{1}_\QQ$,
which is not Riemann integrable by \Cref{prob:1QQ}.
But we can readily compute its Lebesgue integral over $[0,1]$, as
\[ \int_{[0,1]} \mathbf{1}_\QQ \; d\mu
	= \mu\left( [0,1] \cap \QQ \right) = 0 \]
since it is countable.

This \emph{could} be thought of as a failure of convergence
for the Riemann integral.
\begin{example}
	[$\mathbf{1}_\QQ$ is a limit of finitely supported functions]
	\label{ex:1QQindicator}
	We can define the sequence of functions $g_1$, $g_2$, \dots\ by
	\[ g_n(x) = \begin{cases}
			1 & (n!)x \text{ is an integer} \\
			0 & \text{else}.
		\end{cases} \]
	Then each $g_n$ is piecewise continuous
	and hence Riemann integrable on $[0,1]$ (with integral zero),
	but $\lim_{n \to \infty} g_n = \mathbf{1}_\QQ$ is not.
\end{example}

The limit here is defined in the following sense:
\begin{definition}
	Let $f$ and $f_1, f_2, \dots \colon \Omega \to \RR$ be a sequence of functions.
	Suppose that for each $\omega \in \Omega$, the sequence
	\[ f_1(\omega), \; f_2(\omega), \; f_3(\omega), \;, \dots \]
	converges to $f(\omega)$.
	Then we say $(f_n)_n$ \vocab{converges pointwise}
	to the limit $f$, written $\lim_{n \to \infty} f_n = f$.

	We can define $\liminf_{n \to \infty} f_n$
	and $\limsup_{n \to \infty} f_n$ similarly.
\end{definition}
This is actually a fairly weak notion of convergence, for example:
\begin{exercise}
	[Witch's hat]
	Find a sequence of continuous function on $[-1,1] \to \RR$
	which converges pointwise to the function $f$ given by
	\[ f(x) = \begin{cases}
			1 & x = 0 \\
			0 & \text{otherwise}.
		\end{cases} \]
\end{exercise}
This is why when thinking about the Riemann integral
it is commonplace to work with stronger conditions like
``uniformly convergent'' and the like.
However, with the Lebesgue integral, we can mostly not think about these!

\section{Overview}
The three big-name results for exchanging
pointwise limits with Lebesgue integrals is:
\begin{itemize}
	\ii Fatou's lemma: the most general statement possible,
	for any nonnegative measurable functions.
	\ii Monotone convergence: ``increasing limits'' just work.
	\ii Dominated convergence (actually Fatou-Lebesgue):
	limits that are not too big
	(bounded by some absolutely integrable function) just work.
\end{itemize}

\section{Fatou's lemma}
Without further ado:
\begin{lemma}
	[Fatou's lemma]
	Let $f_1, f_2, \dots \colon \Omega \to [0,+\infty]$
	be a sequence of \emph{nonnegative} measurable functions.
	Then $\liminf_n \colon \Omega \to [0,+\infty]$ is measurable and
	\[ \int_\Omega \left( \liminf_{n \to \infty} f_n \right) \; d\mu
		\le \liminf_{n \to \infty} \left( \int_\Omega f_n \; d\mu \right).  \]
	Here we allow either side to be $+\infty$.
\end{lemma}
Notice that there are \emph{no extra hypothesis}
on $f_n$ other than nonnegative: which makes this quite surprisingly versatile
if you ever are trying to prove some general result.

\section{Everything else}
The big surprise is how quickly all the ``big-name''
theorem follows from Fatou's lemma.
Here is the so-called ``monotone convergence theorem''.
\begin{corollary}
	[Monotone convergence theorem]
	Let $f$ and $f_1, f_2, \dots \colon \Omega \to [0,+\infty]$
	be a sequence of \emph{nonnegative}
	measurable functions such that $\lim_n f_n = f$
	and $f_n(\omega) \le f(\omega)$ for each $n$.
	Then $f$ is measurable and
	\[ \lim_{n \to \infty} \left( \int_\Omega f_n \; d\mu \right)
		= \int_\Omega f \; d\mu. \]
	Here we allow either side to be $+\infty$.
\end{corollary}
\begin{proof}
	We have
	\begin{align*}
		\int_\Omega f \; d\mu
		&= \int_\Omega \left( \liminf_{n \to \infty} f_n \right) \; d\mu \\
		&\le \liminf_{n \to \infty} \int_\Omega f_n \; d\mu \\
		&\le \limsup_{n \to \infty} \int_\Omega f_n \; d\mu \\
		&\le \int_\Omega f \; d\mu
	\end{align*}
	where the first $\le$ is by Fatou lemma,
	and the second by the fact that
	$\int_\Omega f_n \le \int_\Omega f$ for every $n$.
	This implies all the inequalities are equalities and we are done.
\end{proof}
\begin{remark}
	[The monotone convergence theorem does not require monotonicity!]
	In the literature it is much more common
	to see the hypothesis $f_1(\omega) \le f_2(\omega) \le \dots \le f(\omega)$
	rather than just $f_n(\omega) \le f(\omega)$ for all $n$,
	which is where the theorem gets its name.
	However as we have shown this hypothesis is superfluous!
	This is pointed out in \url{https://mathoverflow.net/a/296540/70654},
	as a response to a question entitled
	``Do you know of any very important theorems that remain unknown?''.
\end{remark}

\begin{example}
	[Monotone convergence gives $\mathbf{1}_\QQ$]
	This already implies \Cref{ex:1QQindicator}.
	Letting $g_n$ be the indicator function for $\frac1{n!}\ZZ$
	as described in that example, we have $g_n \le \mathbf{1}_\QQ$
	and $\lim_{n \to \infty} g_n(x) = \mathbf{1}_\QQ(x)$,
	for each individual $x$.
	So since $\int_{[0,1]} g_n \; d\mu = 0$ for each $n$,
	this gives $\int_{[0,1]} \mathbf{1}_\QQ = 0$ as we already knew.
\end{example}

The most famous result, though is the following.
\begin{corollary}
	[Fatou–Lebesgue theorem]
	Let $f$ and $f_1, f_2, \dots \colon \Omega \to \RR$
	be a sequence of measurable functions.
	Assume that $g \colon \Omega \to \RR$ is an
	\emph{absolutely integrable} function for which
	$|f_n(\omega)| \le |g(\omega)|$ for all $\omega \in \Omega$.
	Then the inequality
	\begin{align*}
		\int_\Omega \left( \liminf_{n \to \infty} f_n \right) \; d\mu
		&\le \liminf_{n \to \infty} \left( \int_\Omega f_n \; d\mu \right) \\
		&\le \limsup_{n \to \infty} \left( \int_\Omega f_n \; d\mu \right)
		\le \int_\Omega \left( \limsup_{n \to \infty} f_n \right) \; d\mu.
	\end{align*}
\end{corollary}
\begin{proof}
	There are three inequalities:
	\begin{itemize}
		\ii The first inequality follows by Fatou on $g + f_n$ which is nonnegative.
		\ii The second inequality is just $\liminf \le \limsup$.
		(This makes the theorem statement easy to remember!)
		\ii The third inequality follows by Fatou on $g - f_n$ which is nonnegative.
		\qedhere
	\end{itemize}
\end{proof}

\begin{exercise}
	Where is the fact that $g$ is absolutely integrable used in this proof?
\end{exercise}

\begin{corollary}
	[Dominated convergence theorem]
	Let $f_1, f_2, \dots \colon \Omega \to \RR$
	be a sequence of measurable functions
	such that $f = \lim_{n \to \infty} f_n$ exists.
	Assume that $g \colon \Omega \to \RR$ is an
	\emph{absolutely integrable} function for which
	$|f_n(\omega)| \le |g(\omega)|$ for all $\omega \in \Omega$.
	Then
	\[ \int_\Omega f \; d \mu
		= \lim_{n \to \infty} \left( \int_\Omega f_n \; d\mu \right). \]
\end{corollary}
\begin{proof}
	If $f(\omega) = \lim_{n \to \infty} f_n(\omega)$,
	then $f(\omega) = \liminf_{n \to \infty} f_n(\omega)
	= \limsup_{n \to \infty} f_n(\omega)$.
	So all the inequalities in the Fatou-Lebesgue theorem
	become equalities, since the leftmost and rightmost sides are equal.
\end{proof}
Note this gives yet another way to verify \Cref{ex:1QQindicator}.
In general, the dominated convergence theorem
is a favorite clich\'{e} for undergraduate exams,
because it is easy to create questions for it.
Here is one example showing how they all look.
\begin{example}
	[The usual Lebesgue dominated convergence examples]
	Suppose one wishes to compute
	\[ \lim_{n \to \infty}
		\left( \int_{(0,1)} \frac{n\sin(n\inv x)}{\sqrt x} \right) \; dx \]
	then one starts by observing that
	the inner term is bounded by the absolutely integrable function $x^{-1/2}$.
	Therefore it equals
	\begin{align*}
		\int_{(0,1)} \lim_{n \to \infty}
			\left( \frac{n\sin(n\inv x)}{\sqrt x} \right) \; dx
		&= \int_{(0,1)} \frac{x}{\sqrt x} \; dx \\
		&= \int_{(0,1)} \sqrt{x} \; dx = \frac23.
	\end{align*}
\end{example}

\section{Fubini and Tonelli}
\todo{TO BE WRITTEN}

\section{\problemhead}
\todo{problems}
