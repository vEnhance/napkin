\chapter{Characters}
Characters are basically the best thing ever.
To every representation $V$ of $A$ we will attach a
so-called character $\chi_V : A \to k$.
It will turn out that the characters of irreps of $V$
will determine the representation $V$ completely.
Thus an irrep is just specified by a set of $\dim A$ numbers.

\section{Definitions}
\begin{definition}
	Let $V = (V, \rho)$ be a finite-dimensional representation of $A$.
	The \vocab{character} $\chi_V : A \to k$ attached to
	$A$ is defined $\chi_V = \Tr \circ \rho$, i.e.\
	\[ \chi_V(a) \defeq \Tr\left( \rho(a) : V \to V \right). \]
\end{definition}
Since $\Tr$ and $\rho$ are additive, this is a $k$-linear map
(but it is not multiplicative).
Note also that $\chi_{V \oplus W} = \chi_V + \chi_W$
for any representations $V$ and $W$.

We are especially interested in the case $A = k[G]$, of course.
As usual, we just have to specify $\chi_V(g)$ for each
$g \in S_3$ to get the whole map $k[G] \to k$.
Thus we often think of $\chi_V$ as a function $G \to k$,
called a character of the group $G$.
Here is the case $G = S_3$:
\begin{example}
	[Character table of $S_3$]
	Let's consider the three irreps of $G = S_3$ from before.
	For $\CC_{\text{triv}}$ all traces are $1$;
	for $\CC_{\text{sign}}$ the traces are $\pm 1$ depending on sign
	(obviously, for one-dimensional maps $k \to k$ the trace ``is''
	just the map itself).
	For $\refl_0$ we take a basis $(1,0,-1)$ and $(0,1,-1)$, say,
	and compute the traces directly in this basis.
	\[
		\begin{array}{|r|rrrrrr|}
			\hline
			\chi_V(g) & \id & (1\;2) & (2\;3) & (3\;1) 
				& (1\;2\;3) & (3\;2\;1)  \\ \hline
			\Ctriv & 1 & 1 & 1 & 1 & 1 & 1 \\
			\CC_{\mathrm{sign}} & 1 & -1 & -1 & -1 & 1 & 1 \\
			\refl_0 & 2 & 0 & 0 & 0 & -1 & -1 \\ \hline
		\end{array}
	\]
\end{example}
The above table is called the \vocab{character table} of the group $G$.
The table above has certain mysterious properties,
which we will prove as the chapter progresses.
\begin{enumerate}[(I)]
	\ii The value of $\chi_V(g)$ only depends on the conjugacy class of $g$.
	\ii The number of rows equals the number of conjugacy classes.
	\ii The sum of the squares of any row is $6$ again!
	\ii The ``dot product'' of any two rows is zero.
\end{enumerate}

\begin{abuse}
	The name ``character'' for $\chi_V : G \to k$ is a bit of a misnomer.
	This $\chi_V$ is not multiplicative in any way,
	as the above example shows: one can almost think of it as
	an element of $k^{\oplus |G|}$.
\end{abuse}

\begin{ques}
	Show that $\chi_V(1_A) = \dim V$,
	so one can read the dimensions of the representations
	from the leftmost column of a character table.
\end{ques}

\section{The dual space modulo the commutator}
For any algebra, we first observe that since $\Tr(TS) = \Tr(ST)$,
we have for any $V$ that
\[ \chi_V(ab) = \chi_V(ba). \]
This explains observation (I) from earlier:
\begin{ques}
	Deduce that if $g$ and $h$ are in the same conjugacy class of a 
	group $G$, and $V$ is a representation of $\CC[G]$,
	then $\chi(g) = \chi(h)$.
\end{ques}
Now, given our algebra $A$ we define the \vocab{commutator} $[A,A]$
to be the (two-sided) ideal\footnote{%
	This means the ideal consists of sums elements of the form
	$a(xy-yx)b$ for $a,b \in A$.
}
generated by elements of the form $xy-yx$.
Thus $[A,A]$ is contained in the kernel of each $\chi_V$.
\begin{definition}
	The space $A / [A,A]$ is called the \vocab{abelianization} of $A$;
	for brevity we denote it as $A\ab$.
	We think of this as ``$A$ modulo the relation $ab=ba$ for each $a,b \in A$.''
\end{definition}
So we can think of each character $\chi_V$ as an element of $(A\ab)^\vee$.

\begin{example}
	[Examples of abelianizations]
	\listhack
	\begin{enumerate}[(a)]
		\ii If $A$ is commutative, then $[A,A] = \{0\}$
		and $A\ab = A$.
		\ii If $A = \Mat_k(d)$, then $[A,A]$ consists exactly
		of the $d \times d$ matrices of trace zero.
		(Proof: harmless exercise.)
		Consequently, $A\ab$ is one-dimensional.
		\ii Suppose $A = k[G]$.  We claim that $\dim A\ab$ is equal to the
		number of conjugacy classes of $A$.
		Indeed, an element of $A$ can be thought of as just 
		an arbitrary function $\xi : G \to k$.
		So an element of $A\ab$ is a function $\xi: G \to k$ such that
		$\xi(gh) = \xi(hg)$ for every $g,h \in G$.
		This is equivalent to functions from conjugacy classes of $G$ to $k$.
	\end{enumerate}
\end{example}

\begin{theorem}
	[Character of representations of algebras]
	Let $A$ be an algebra over an algebraically closed field. Then
	\begin{enumerate}[(a)]
		\ii Characters of pairwise non-isomorphic irreps are
		linearly independent as elements of $A\ab$.
		\ii If $A$ is finite-dimensional and semisimple,
		then the characters attached to irreps
		form a basis of $A\ab$.
	\end{enumerate}
	In particular, in (b) the number of irreps of $A$ equals $\dim A\ab$.
\end{theorem}
\begin{proof}
	Part (a) is more or less obvious by the density theorem.
	Suppose there is a linear dependence, so that for every $a$ we have
	\[ c_1 \chi_{V_1}(a) + c_2 \chi_{V_2}(a) + \dots + c_r \chi_{V_r} (a) = 0\]
	for some integer $r$.
	\begin{ques}
		Deduce that $c_1 = \dots = c_r = 0$ from the density theorem.
	\end{ques}
	For part (b), assume there are $r$ irreps
	we may assume that \[ A = \bigoplus_{i=1}^r \Mat(V_i) \]
	where $V_1$, \dots, $V_r$ are the irreps of $A$.
	Since we have already showed the characters are linearly independent
	we need only show that $\dim ( A / [A,A] ) = r$,
	which follows from the observation earlier that each $\Mat(V_i)$
	has a one-dimensional abelianization.
\end{proof}
Since $G$ has $\dim \CC[G]\ab$ conjugacy classes,
this completes the proof of (II).

\section{Orthogonality of characters}
Now we specialize to the case of finite groups $G$, represented over $\CC$.
\begin{definition}
	Let $\Classes(G)$ denote the set conjugacy classes of $G$.
\end{definition}
If $G$ has $r$ conjugacy classes, then it has $r$ irreps.
Each (finite-dimensional) representation $V$, irreducible or not, gives a
character $\chi_V$.
\begin{abuse}
	From now on, we will often regard $\chi_V$ as a function $G \to \CC$
	or as a function $\Classes(G) \to \CC$.
	So for example, we will write both $\chi_V(g)$ (for $g \in G$)
	and $\chi_V(C)$ (for a conjugacy class $C$);
	the latter just means $\chi_V(g_C)$ for any representative $g_C \in C$.
\end{abuse}
\begin{definition}
	Let $\FunCl(G)$ denote the set of functions $\Classes(G) \to \CC$
	viewed as a vector space over $\CC$.
	We endow it with the inner form
	\[
		\left< f_1, f_2 \right> = 
		\frac{1}{|G|}
		\sum_{g \in G} f_1(g) \ol{f_2(g)}.
	\]
\end{definition}
This is the same ``dot product'' that we mentioned at the beginning,
when we looked at the character table of $S_3$.
We now aim to prove the following orthogonality theorem,
which will imply (III) and (IV) from earlier.
\begin{theorem}[Orthogonality]
	For any finite-dimensional complex representations $V$ and $W$
	of $G$ we have
	\[ \left< \chi_V, \chi_W \right> = \dim \Homrep(W, V). \]
	In particular, if $V$ and $W$ are irreps then
	\[ \left< \chi_V, \chi_W \right> 
		=
		\begin{cases}
			1 & V  \cong W \\
			0 & \text{otherwise}.
		\end{cases}
	\]
\end{theorem}
\begin{corollary}[Irreps give an orthonormal basis]
	The characters associated to irreps
	form an \emph{orthonormal} basis of $\FunCl(G)$.
\end{corollary}

In order to prove this theorem, we have to define
the dual representation and the tensor representation,
which give a natural way to deal with the quantity $\chi_V(g)\ol{\chi_W(g)}$.
\begin{definition}
	Let $V = (V, \rho)$ be a representation of $G$.
	The \vocab{dual representation} $V^\vee$ is the representation on $V^\vee$
	with the action of $G$ given as follows: for each $\xi \in V^\vee$,
	the action of $g$ gives a $g \cdot \xi \in V^\vee$ specified by
	\[ v \xmapsto{g \cdot \xi} \xi\left( \rho(g\inv)(v) \right). \]
\end{definition}
\begin{definition}
	Let $V = (V, \rho_V)$ and $W = (W, \rho_W)$
	be \emph{group} representations of $G$.
	The \vocab{tensor product} of $V$ and $W$ is the group representation
	on $V \otimes W$ with the action of $G$ given on pure tensors by
	\[
		g \cdot (v \otimes w)
		= 
		(\rho_V(g)(v)) \otimes (\rho_W(g)(w)) \]
	which extends linearly to define the action of $G$ on all of $V \otimes W$.
\end{definition}
\begin{remark}
	Warning: the definition for tensors does \emph{not} extend to algebras.
	We might hope that $a \cdot (v \otimes w) = (a \cdot v) \otimes (a \cdot w)$
	would work, but this is not even linear in $a \in A$
	(what happens if we take $a=2$, for example?).
\end{remark}

\begin{theorem}
	[Character traces]
	If $V$ and $W$ are finite-dimensional representations of $G$,
	then for any $g \in G$.
	\begin{enumerate}[(a)]
		\ii $\chi_{V \oplus W}(g) = \chi_V(g) + \chi_W(g)$.
		\ii $\chi_{V \otimes W}(g) = \chi_V(g) \cdot \chi_W(g)$.
		\ii $\chi_{V^\vee}(g) = \ol{\chi_V(g)}$.
	\end{enumerate}
\end{theorem}
\begin{proof}
	Parts (a) and (b) follow from the identities
	$\Tr(S \oplus T) = \Tr(S) + \Tr(T)$
	and $\Tr(S \otimes T) = \Tr(S) \Tr(T)$.
	However, part (c) is trickier.
	As $(\rho(g))^{|G|} = \rho(g^{|G|}) = \rho(1_G) = \id_V$
	by Lagrange's theorem, we can diagonalize $\rho(g)$,
	say with eigenvalues $\lambda_1$, \dots, $\lambda_n$
	which are $|G|$th roots of unity,
	corresponding to eigenvectors $e_1$, \dots, $e_n$.
	Then we see that in the basis $e_1^\vee$, \dots, $e_n^\vee$,
	the action of $g$ on $V^\vee$ has eigenvalues
	$\lambda_1\inv$, $\lambda_2\inv$, \dots, $\lambda_n\inv$.
	So
	\[
		\chi_V(g) = \sum_{i=1}^n \lambda_i \quad\text{and}\quad
		\chi_{V^\vee}(g) = \sum_{i=1}^n \lambda_i\inv = \sum_{i=1}^n \ol{\lambda_i}
	\]
	where the last step follows from the identity $|z|=1 \iff z\inv = \ol z$.
\end{proof}
\begin{remark}
	[Warning]
	The identities (b) and (c) do not extend linearly to $\CC[G]$,
	i.e.\ it is not true for example that $\chi_V(a) = \ol{\chi_V(a)}$
	if we think of $\chi_V$ as a map $\CC[G] \to \CC$.
\end{remark}
\begin{proof}
	[Proof of orthogonality relation]
	The key point is that we can now reduce
	the sums of products to just a single character by
	\[ \chi_V(g) \ol{\chi_W(g)} = \chi_{V \otimes W^\vee} (g). \]
	So we can rewrite the sum in question as just
	\[
		\left< \chi_V, \chi_W \right>
		= \frac{1}{|G|} \sum_{g \in G} \chi_{V \otimes W^\vee} (g)
		= \chi_{V \otimes W^\vee}
		\left( \frac{1}{|G|} \sum_{g \in G} g \right).
	\]
	Let $P : V \otimes W^\vee \to V \otimes W^\vee$ be the
	action of $\frac{1}{|G|} \sum_{g \in G}$,
	so we wish to find $\Tr P$.
	\begin{exercise}
		Show that $P$ is idempotent.
		(Compute $P \circ P$ directly.)
	\end{exercise}
	Hence $V \otimes W^\vee = \ker P \oplus \img P$ (by \Cref{prob:idempotent})
	and $\img P$ is the subspace of elements which are fixed under $G$.
	From this we deduce that
	\[ \Tr P = \dim \img P =
		\dim \left\{ x \in V \otimes W^\vee
		\mid g \cdot x = x \; \forall g \in G  \right\}.
		\]
	Now, consider the usual isomorphism $V \otimes W^\vee \to \Hom(W, V)$.
	\begin{exercise}
		Let $g \in G$.
		Show that under this isomorphism, $T \in \Hom(W, V)$
		satisfies $g \cdot T = T$ if and only if
		$T(g \cdot w) = g \cdot T(w)$ for each $w \in W$.
		(This is just unwinding three or four definitions.)
	\end{exercise}
	Consequently, $\chi_{V \otimes W^\vee}(P) = \Tr P = \dim \Homrep(W,V)$
	as desired.
\end{proof}

The orthogonality relation gives us a fast and mechanical way to check
whether a finite-dimensional representation $V$ is irreducible.
Namely, compute the traces $\chi_V(g)$ for each $g \in G$,
and then check whether $\left< \chi_V, \chi_V \right> = 1$.
So, for example, we could have seen the three representations of
$S_3$ that we found were irreps directly from the character table.
Thus, we can now efficiently verify any time we have
a complete set of irreps.

\section{Examples of character tables}
\begin{example}
	[Dihedral group on $10$ elements]
	Let $D_{10} = \left< r,s \mid r^5 = s^2 = 1, rs = sr\inv \right>$.
	Let $\omega = \exp(\frac{2\pi i}{5})$.
	We write four representations of $D_{10}$:
	\begin{itemize}
		\ii $\Ctriv$, all elements of $D_{10}$ act as the identity.
		\ii $\Csign$, $r$ acts as the identity while $s$ acts by negation.
		\ii $V_1$, which is two-dimensional and given by
		$r \mapsto \begin{bmatrix} \omega & 0 \\ 0 & \omega^4 \end{bmatrix}$
		and $s \mapsto \begin{bmatrix} 0 & 1 \\ 1 & 0 \end{bmatrix}$.
		\ii $V_2$, which is two-dimensional and given by
		$r \mapsto \begin{bmatrix} \omega^2 & 0 \\ 0 & \omega^3 \end{bmatrix}$
		and $s \mapsto \begin{bmatrix} 0 & 1 \\ 1 & 0 \end{bmatrix}$.
	\end{itemize}
	We claim that these four representations are irreducible
	and pairwise non-isomorphic.
	We do so by writing the character table:
	\[
		\begin{array}{|c|rccr|}
			\hline
			D_{10} & 1 & r, r^4 & r^2, r^3 & sr^k \\ \hline
			\Ctriv & 1 & 1 & 1 & 1 \\
			\Csign & 1 & 1 & 1 & -1 \\
			V_1 & 2 & \omega+\omega^4 & \omega^2+\omega^3 & 0 \\
			V_2 & 2 & \omega^2+\omega^3 & \omega+\omega^4 & 0 \\ \hline
		\end{array}
	\]
	Then a direct computation shows the orthogonality relations,
	hence we indeed have an orthonormal basis.
	For example, $\left< \Ctriv, \Csign \right> = 1 + 2 \cdot 1 + 2 \cdot 1 + 5 \cdot (-1) = 0$.
\end{example}

\begin{example}
	[Character table of $S_4$]
	We now have enough machinery to to compute the character
	table of $S_4$, which has five conjugacy classes
	(corresponding to cycle types $\id$, $2$, $3$, $4$ and $2+2$).
	First of all, we note that it has two one-dimensional representations,
	$\Ctriv$ and $\Csign$, and these are the only ones
	(because there are only two homomorphisms $S_4 \to \CC^\times$).
	So thus far we have the table
	\[
		\begin{array}{|c|rrrrr|}
			\hline
			S_4 & 1 & (\bullet\;\bullet) & (\bullet\;\bullet\;\bullet)
				& (\bullet\;\bullet\;\bullet\;\bullet)
				& (\bullet\;\bullet)(\bullet\;\bullet)
				\\ \hline
			\Ctriv & 1 & 1 & 1 & 1 & 1 \\
			\Csign & 1 & -1 & 1 & -1 & 1 \\ 
			\vdots & \multicolumn{5}{|c|}{\vdots}
		\end{array}
	\]
	Note the columns represent $1+6+8+6+3=24$ elements.

	Now, the remaining three representations have dimensions
	$d_1$, $d_2$, $d_3$ with
	\[ d_1^2 + d_2^2 + d_3^2 = 4! - 2 = 22 \]
	which has only $(d_1, d_2, d_3) = (2,3,3)$ and permutations.
	Now, we can take the $\refl_0$ representation
	\[ \left\{ (w,x,y,z) \mid w+x+y+z=0 \right\} \]
	with basis $(1,0,0,-1)$, $(0,1,0,-1)$ and $(0,0,1,-1)$.
	This can be geometrically checked to be irreducible,
	but we can also do this numerically by computing the
	character directly (this is tedious):
	it comes out to have $3$, $-1$, $0$, $1$, $-1$
	which indeed gives norm
	\[
		\left< \chi_{\refl_0}, \chi_{\refl_0} \right>
		=
		\frac{1}{4!}
		\left( 
			\underbrace{3^2}_{\id}
			+ \underbrace{6\cdot(-1)^2}_{(\bullet\;\bullet)}
			+ \underbrace{8\cdot(0)^2}_{(\bullet\;\bullet\;\bullet)}
			+ \underbrace{6\cdot(1)^2}_{(\bullet\;\bullet\;\bullet\;\bullet)}
			+ \underbrace{3\cdot(-1)^2}_{(\bullet\;\bullet)(\bullet\;\bullet)}
		\right)
		= 1.
	\]
	Note that we can also tensor this with the sign representation,
	to get another irreducible representation
	(since $\Csign$ has all traces $\pm 1$, the norm doesn't change).
	Finally, we recover the final row using orthogonality
	(which we name $\CC^2$, for lack of a better name);
	hence the completed table is as follows.
	\[
		\begin{array}{|c|rrrrr|}
			\hline
			S_4 & 1 & (\bullet\;\bullet) & (\bullet\;\bullet\;\bullet)
				& (\bullet\;\bullet\;\bullet\;\bullet)
				& (\bullet\;\bullet)(\bullet\;\bullet)
				\\ \hline
			\Ctriv & 1 & 1 & 1 & 1 & 1 \\
			\Csign & 1 & -1 & 1 & -1 & 1 \\ 
			\CC^2 & 2 & 0 & -1 & 0 & 2 \\ 
			\refl_0 & 3 & -1 & 0 & 1 & -1 \\
			\refl_0 \otimes \Csign & 3 & 1 & 0 & -1 & -1 \\\hline
		\end{array}
	\]
\end{example}

\section\problemhead

\begin{dproblem}
	[Reading decompositions from characters]
	Let $W$ be a complex representation of a finite group $G$.
	Let $V_1$, \dots, $V_r$ be the complex irreps of $G$
	and set $n_i = \left< \chi_W, \chi_{V_i} \right>$.
	Prove that each $n_i$ is a positive integer and
	\[ W = \bigoplus_{i=1}^r V_i^{\oplus n_i}. \]
	\begin{hint}
		Obvious.
		Let $W = \bigoplus V_i^{m_i}$ (possible since $\CC[G]$ semisimple)
		thus $\chi_W = \sum_i m_i \chi_{V_i}$.
	\end{hint}
\end{dproblem}

\begin{problem}
	Consider complex representations of $G = S_4$.
	The representation $\refl_0 \otimes \refl_0$
	is $9$-dimensional, so it is clearly reducible.
	Compute its decomposition in terms of the five
	irreducible representations.
	\begin{hint}
		Use the previous problem, with $\chi_W = \chi_{\refl_0}^2$.
	\end{hint}
	\begin{sol}
		$\Csign \oplus \CC^2 \oplus \refl_0 \oplus (\refl_0\otimes\Csign)$.
	\end{sol}
\end{problem}

\begin{problem}
	[Tensoring by one-dimensional irreps]
	Let $V$ and $W$ be irreps of $G$, with $\dim W = 1$.
	Show that $V \otimes W$ is irreducible.
	\begin{hint}
		Characters. Note that $|\chi_W| = 1$ everywhere.
	\end{hint}
	\begin{sol}
		First, observe that $|\chi_W(g)|=1$ for all $g \in G$.
		\begin{align*}
			\left< \chi_{V \otimes W}, \chi_{V \otimes W} \right>
			&= \left< \chi_V \chi_W, \chi_V \chi_W \right> \\
			&= \frac{1}{|G|} \sum_{g \in G} 
			\left\lvert \chi_V(g) \right\rvert^2
			\left\lvert \chi_W(g) \right\rvert^2 \\
			&= \frac{1}{|G|} \sum_{g \in G} 
			\left\lvert \chi_V(g) \right\rvert^2 \\
			&= \left< \chi_V, \chi_V \right> = 1.
		\end{align*}
	\end{sol}
\end{problem}

\begin{problem}
	[Quaternions]
	Compute the character table of the quaternion group $Q_8$.
	\begin{hint}
		There are five conjugacy classes, $1$, $-1$
		and $\pm i$, $\pm j$, $\pm k$.
		Given four of the representations, orthogonality
		can give you the fifth one.
	\end{hint}
	\begin{sol}
		The table is given by
		\[
			\begin{array}{|c|rrrrr|}
				\hline
				Q_8 & 1 & -1 & \pm i & \pm j & \pm k \\ \hline
				\Ctriv & 1 & 1 & 1 & 1 & 1 \\
				\CC_i & 1 & 1 & 1 & -1 & -1 \\
				\CC_j & 1 & 1 & -1 & 1 & -1 \\
				\CC_k & 1 & 1 & -1 & -1 & 1 \\
				\CC^2 & 2 & -2 & 0 & 0 & 0 \\\hline
			\end{array}
		\]
		The one-dimensional representations (first four rows)
		follows by considering the homomorphism $Q_8 \to \CC^\times$.
		The last row is two-dimensional and can be recovered
		by using the orthogonality formula.
	\end{sol}
\end{problem}

\begin{sproblem}
	[Second orthogonality formula]
	\label{prob:second_orthog}
	\gim
	Let $g$ and $h$ be elements of a finite group $G$,
	and let $V_1$, \dots, $V_r$ be the irreps of $G$.
	Prove that
	\[
		\sum_{i = 1}^r \chi_{V_i}(g) \ol{\chi_{V_i}(h)}
		=
		\begin{cases}
			|Z(g)| & \text{if $g$ and $h$ are conjugates} \\
			0 & \text{otherwise}.
		\end{cases}
	\]
	Here, $Z(g) = \left\{ x \in G : xg = gx \right\}$
	is the center of $G$.
	\begin{hint}
		Write as 
		\[ \sum_{i=1}^r \chi_{V_i \otimes V_i^\vee} (gh\inv)
			= \chi_{\bigoplus_i V_i \otimes V_i^\vee}(gh\inv)
			= \chi_{\CC[G]}(gh\inv).
		\]
		Now look at the usual basis for $\CC[G]$.
	\end{hint}
\end{sproblem}
