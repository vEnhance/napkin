%%fakesection Load packages
% These are evan.sty
\usepackage{amsmath,amssymb,amsthm}
\usepackage{mathrsfs}
\usepackage[usenames,svgnames,dvipsnames]{xcolor}
\usepackage{hyperref}
\usepackage[nameinlink]{cleveref}
\usepackage{textcomp}
\usepackage{enumerate}
\usepackage[textsize=scriptsize,shadow]{todonotes}
\usepackage{mathtools}
\usepackage{microtype}

\usepackage[normalem]{ulem}
\usepackage{stmaryrd}
\usepackage{wasysym}
\usepackage{multirow}
\usepackage{prerex}

%%fakesection evan.sty macros
%Small commands
\newcommand{\cbrt}[1]{\sqrt[3]{#1}}
\newcommand{\floor}[1]{\left\lfloor #1 \right\rfloor}
\newcommand{\ceiling}[1]{\left\lceil #1 \right\rceil}
\newcommand{\mailto}[1]{\href{mailto:#1}{\texttt{#1}}}
\newcommand{\ol}{\overline}
\newcommand{\ul}{\underline}
\newcommand{\wt}{\widetilde}
\newcommand{\wh}{\widehat}
\newcommand{\eps}{\varepsilon}
%\renewcommand{\iff}{\Leftrightarrow}
%\renewcommand{\implies}{\Rightarrow}
\newcommand{\vocab}[1]{\textbf{\color{blue} #1}}
\newcommand{\catname}{\mathsf}
\newcommand{\hrulebar}{
  \par\hspace{\fill}\rule{0.95\linewidth}{.7pt}\hspace{\fill}
  \par\nointerlineskip \vspace{\baselineskip}
}
\newcommand{\half}{\frac{1}{2}}

%For use in author command
\newcommand{\plusemail}[1]{\\ \normalfont \texttt{\mailto{#1}}}

%More commands and math operators
\DeclareMathOperator{\cis}{cis}
\DeclareMathOperator*{\lcm}{lcm}
\DeclareMathOperator*{\argmin}{arg min}
\DeclareMathOperator*{\argmax}{arg max}

%Convenient Environments
\newenvironment{soln}{\begin{proof}[Solution]}{\end{proof}}
\newenvironment{parlist}{\begin{inparaenum}[(i)]}{\end{inparaenum}}
\newenvironment{gobble}{\setbox\z@\vbox\bgroup}{\egroup}

%Inequalities
\newcommand{\cycsum}{\sum_{\mathrm{cyc}}}
\newcommand{\symsum}{\sum_{\mathrm{sym}}}
\newcommand{\cycprod}{\prod_{\mathrm{cyc}}}
\newcommand{\symprod}{\prod_{\mathrm{sym}}}

%From H113 "Introduction to Abstract Algebra" at UC Berkeley
\newcommand{\CC}{\mathbb C}
\newcommand{\FF}{\mathbb F}
\newcommand{\NN}{\mathbb N}
\newcommand{\QQ}{\mathbb Q}
\newcommand{\RR}{\mathbb R}
\newcommand{\ZZ}{\mathbb Z}
\newcommand{\charin}{\text{ char }}
\DeclareMathOperator{\sign}{sign}
\DeclareMathOperator{\Aut}{Aut}
\DeclareMathOperator{\Inn}{Inn}
\DeclareMathOperator{\Syl}{Syl}
\DeclareMathOperator{\Gal}{Gal}
\DeclareMathOperator{\GL}{GL} % General linear group
\DeclareMathOperator{\SL}{SL} % Special linear group

%From Kiran Kedlaya's "Geometry Unbound"
\newcommand{\abs}[1]{\left\lvert #1 \right\rvert}
\newcommand{\norm}[1]{\left\lVert #1 \right\rVert}
\newcommand{\dang}{\measuredangle} %% Directed angle
\newcommand{\ray}[1]{\overrightarrow{#1}} 
\newcommand{\seg}[1]{\overline{#1}}
\newcommand{\arc}[1]{\wideparen{#1}}

%From M275 "Topology" at SJSU
\newcommand{\id}{\mathrm{id}}
\newcommand{\taking}[1]{\xrightarrow{#1}}
\newcommand{\inv}{^{-1}}

%From M170 "Introduction to Graph Theory" at SJSU
\DeclareMathOperator{\diam}{diam}
\DeclareMathOperator{\ord}{ord}
%\newcommand{\defeq}{\overset{\mathrm{def}}{=}}
\newcommand{\defeq}{\coloneqq}

%From the USAMO .tex filse
\newcommand{\st}{^{\text{st}}}
\newcommand{\nd}{^{\text{nd}}}
\renewcommand{\th}{^{\text{th}}}
\newcommand{\dg}{^\circ}
\newcommand{\ii}{\item}

% From Math 55 and Math 145 at Harvard
\newenvironment{subproof}[1][Proof]{%
\begin{proof}[#1] \renewcommand{\qedsymbol}{$\blacksquare$}}%
{\end{proof}}

\newcommand{\liff}{\leftrightarrow}
\newcommand{\lthen}{\rightarrow}
\newcommand{\opname}{\operatorname}
\newcommand{\surjto}{\twoheadrightarrow}
\newcommand{\injto}{\hookrightarrow}
\newcommand{\On}{\mathrm{On}} % ordinals
\DeclareMathOperator{\img}{im} % Image
\DeclareMathOperator{\Img}{Im} % Image
\DeclareMathOperator{\coker}{coker} % Cokernel
\DeclareMathOperator{\Coker}{Coker} % Cokernel
\DeclareMathOperator{\Ker}{Ker} % Kernel
\DeclareMathOperator{\rank}{rank}
\DeclareMathOperator{\Spec}{Spec} % spectrum
\DeclareMathOperator{\Tr}{Tr} % trace
\DeclareMathOperator{\pr}{pr} % projection
\DeclareMathOperator{\ext}{ext} % extension
\DeclareMathOperator{\pred}{pred} % predecessor
\DeclareMathOperator{\dom}{dom} % domain
\DeclareMathOperator{\ran}{ran} % range
\DeclareMathOperator{\Hom}{Hom} % homomorphism
\DeclareMathOperator{\End}{End} % endomorphism

% Things Lie
\newcommand{\kg}{\mathfrak g}
\newcommand{\kh}{\mathfrak h}
\newcommand{\kn}{\mathfrak n}
\newcommand{\ku}{\mathfrak u}
\newcommand{\kz}{\mathfrak z}
\DeclareMathOperator{\Ext}{Ext} % Ext functor
\DeclareMathOperator{\Tor}{Tor} % Tor functor

% More script letters etc.
\newcommand{\SA}{\mathscr A}
\newcommand{\SB}{\mathscr B}
\newcommand{\SC}{\mathscr C}
\newcommand{\SD}{\mathscr D}
\newcommand{\SF}{\mathscr F}
\newcommand{\SG}{\mathscr G}
\newcommand{\SH}{\mathscr H}
\newcommand{\OO}{\mathcal O}

%% Napkin commands
\newcommand{\prototype}[1]{
	\emph{{\color{red} Prototypical example for this section:} #1} \par\medskip
}
\newenvironment{moral}{%
	\begin{mdframed}[linecolor=green!70!black]%
	\bfseries\color{green!50!black}}%
	{\end{mdframed}}

%%% END EXCERPT OF EVAN.STY
%%fakesection Commutative diagrams
\usepackage{diagrams}
\newarrow{Inj}C--->
\newarrow{Surj}----{>>}
\newarrow{Proj}----{>>}
\newarrow{Embed}>--->
\newarrow{Incl}C--->
\newarrow{Mapsto}|--->
\newarrow{Isom}<--->
\newarrow{Dotted}....>
\newarrow{Dashed}{}{dash}{}{dash}>
\newarrow{Line}-----
\usepackage{tikz-cd}
\usetikzlibrary{decorations.pathmorphing}
\tikzcdset{row sep/normal=3.14em,column sep/normal=3.14em}

%%fakesection Page layout
\usepackage[headsepline]{scrlayer-scrpage}
\renewcommand{\headfont}{}
\addtolength{\textheight}{3.14cm}
\setlength{\footskip}{0.5in}
\setlength{\headsep}{10pt}

\def\shortdate{\leavevmode\hbox{\the\year-\twodigits\month-\twodigits\day}}
\def\twodigits#1{\ifnum#1<10 0\fi\the#1}
\automark[chapter]{chapter}

\newcommand{\napkinversion}{\shortdate}

\rohead{\footnotesize\thepage}
\rehead{\footnotesize \textbf{\sffamily Napkin}, by \emph{Evan Chen} (\napkinversion)}
\lehead{\footnotesize\thepage}
\lohead{\footnotesize \leftmark}
\chead{}
\rofoot{}
\refoot{}
\lefoot{}
\lofoot{}
%\cfoot{\pagemark}

%%fakesection Fancy section and chapter heads
\renewcommand*{\sectionformat}{\color{purple}\S\thesection\autodot\enskip}
\renewcommand*{\subsectionformat}{\color{purple}\S\thesubsection\autodot\enskip}
\newcommand{\problemhead}{A few harder problems to think about}
\renewcommand{\thesubsection}{\thesection.\roman{subsection}}

\addtokomafont{chapterprefix}{\raggedleft}
\RedeclareSectionCommand[beforeskip=0.5em]{chapter}
\renewcommand*{\chapterformat}{%
\mbox{\scalebox{1.5}{\chapappifchapterprefix{\nobreakspace}}%
\scalebox{2.718}{\color{purple}\thechapter\autodot}\enskip}}

\addtokomafont{partprefix}{\rmfamily}
\renewcommand*{\partformat}{\color{purple}\scalebox{2.5}{\thepart}}

%%fakesection Theorems
\usepackage{thmtools}
\usepackage[framemethod=TikZ]{mdframed}

\theoremstyle{definition}
\mdfdefinestyle{mdbluebox}{%
	roundcorner = 10pt,
	linewidth=1pt,
	skipabove=12pt,
	innerbottommargin=9pt,
	skipbelow=2pt,
	nobreak=true,
	linecolor=blue,
	backgroundcolor=TealBlue!5,
}
\declaretheoremstyle[
	headfont=\sffamily\bfseries\color{MidnightBlue},
	mdframed={style=mdbluebox},
	headpunct={\\[3pt]},
	postheadspace={0pt}
]{thmbluebox}

\mdfdefinestyle{mdredbox}{%
	linewidth=0.5pt,
	skipabove=12pt,
	frametitleaboveskip=5pt,
	frametitlebelowskip=0pt,
	skipbelow=2pt,
	frametitlefont=\bfseries,
	innertopmargin=4pt,
	innerbottommargin=8pt,
	nobreak=true,
	linecolor=RawSienna,
	backgroundcolor=Salmon!5,
}
\declaretheoremstyle[
	headfont=\bfseries\color{RawSienna},
	mdframed={style=mdredbox},
	headpunct={\\[3pt]},
	postheadspace={0pt},
]{thmredbox}

\declaretheorem[%
style=thmbluebox,name=Theorem,numberwithin=section]{theorem}
\declaretheorem[style=thmbluebox,name=Lemma,sibling=theorem]{lemma}
\declaretheorem[style=thmbluebox,name=Proposition,sibling=theorem]{proposition}
\declaretheorem[style=thmbluebox,name=Corollary,sibling=theorem]{corollary}
\declaretheorem[style=thmredbox,name=Example,sibling=theorem]{example}

\mdfdefinestyle{mdgreenbox}{%
	skipabove=8pt,
	linewidth=2pt,
	rightline=false,
	leftline=true,
	topline=false,
	bottomline=false,
	linecolor=ForestGreen,
	backgroundcolor=ForestGreen!5,
}
\declaretheoremstyle[
	headfont=\bfseries\sffamily\color{ForestGreen!70!black},
	bodyfont=\normalfont,
	spaceabove=2pt,
	spacebelow=1pt,
	mdframed={style=mdgreenbox},
	headpunct={ --- },
]{thmgreenbox}

\mdfdefinestyle{mdblackbox}{%
	skipabove=8pt,
	linewidth=3pt,
	rightline=false,
	leftline=true,
	topline=false,
	bottomline=false,
	linecolor=black,
	backgroundcolor=RedViolet!5!gray!5,
}
\declaretheoremstyle[
	headfont=\bfseries,
	bodyfont=\normalfont\small,
	spaceabove=0pt,
	spacebelow=0pt,
	mdframed={style=mdblackbox}
]{thmblackbox}

\theoremstyle{theorem}
\declaretheorem[name=Question,sibling=theorem,style=thmblackbox]{ques}
\declaretheorem[name=Exercise,sibling=theorem,style=thmblackbox]{exercise}
\declaretheorem[name=Remark,sibling=theorem,style=thmgreenbox]{remark}
\declaretheorem[name=Step,style=thmgreenbox]{step} % only used in Lebesgue int

\theoremstyle{definition}
\newtheorem{claim}[theorem]{Claim}
\newtheorem{definition}[theorem]{Definition}
\newtheorem{fact}[theorem]{Fact}
\newtheorem{abuse}[theorem]{Abuse of Notation}

\newtheorem{problem}{Problem}[chapter]
\renewcommand{\theproblem}{\thechapter\Alph{problem}}
\newtheorem{sproblem}[problem]{Problem}
\newtheorem{dproblem}[problem]{Problem}
\renewcommand{\thesproblem}{\theproblem$^{\star}$}
\renewcommand{\thedproblem}{\theproblem$^{\dagger}$}
\newcommand{\listhack}{$\empty$\vspace{-2em}}

%%fakesection Answers
\usepackage{answers}
\Newassociation{hint}{answeritem}{tex/backmatter/all-hints}
\Newassociation{sol}{answeritem}{tex/backmatter/all-solns}
\renewcommand{\solutionextension}{out}
\renewenvironment{answeritem}[1]{\item[\bfseries #1.]}{}

%%fakesection Hot chili peppers
\reversemarginpar
\newcommand{\prechili}{\vspace*{0.3em}\hspace*{1.5em}}
\newcommand{\nochili}{\hspace*{1.5em}}
\newcommand{\chili}{\includegraphics[width=1.5em]{media/chili.png}}
\newcommand{\gim}{\marginpar{\prechili\nochili\nochili\chili}}
\newcommand{\yod}{\marginpar{\prechili\nochili\chili\chili}}
\newcommand{\kurumi}{\marginpar{\prechili\chili\chili\chili}}

%%fakesection Table of contents
% First add ToC to ToC
\makeatletter
\usepackage{etoolbox}
\pretocmd{\tableofcontents}{%
  \if@openright\cleardoublepage\else\clearpage\fi
  \pdfbookmark[0]{\contentsname}{toc}%
}{}{}%
\makeatother
\setcounter{tocdepth}{1}
% Fix spacing issues in ToC
\usepackage[tocindentauto]{tocstyle}
\usetocstyle{KOMAlike}

%%fakesection Asymptote definitions
\usepackage{patch-asy}
\numberwithin{asy}{chapter}
\renewcommand{\theasy}{\thechapter\Alph{asy}}
\begin{asydef}
	import olympiad;
	import cse5;
	import extras;

	size(6cm);
	usepackage("amsmath");
	usepackage("amssymb");
	defaultpen(fontsize(11pt));
	pointfontsize=11;
	settings.tex="latex";
	settings.outformat="pdf";
\end{asydef}
\def\asydir{asy}

%%fakesection Bibliography
\usepackage[backend=biber,style=alphabetic]{biblatex}
\DeclareLabelalphaTemplate{
	\labelelement{
		\field[final]{shorthand}
		\field{label}
		\field[strwidth=2,strside=left]{labelname}
	}
	\labelelement{
		\field[strwidth=2,strside=right]{year}    
	}
}
\DeclareFieldFormat{labelalpha}{\textbf{\scriptsize #1}}
\addbibresource{references.bib}
\addbibresource{images.bib}

%%fakesection Mini ToC
\usepackage[tight]{minitoc}
\mtcsetfont{parttoc}{chapter}{\sffamily\bfseries}
\mtcsetfont{parttoc}{section}{\footnotesize\rmfamily\upshape\mdseries}
\mtcsetfont{parttoc}{subsection}{\footnotesize\rmfamily\upshape\mdseries}
%\mtcsetdepth{parttoc}{1}
\setcounter{parttocdepth}{1}
\renewcommand*{\partheadstartvskip}{\vspace*{20em}}
\renewcommand*{\partheadendvskip}{}
%\noptcrule
\renewcommand\beforeparttoc{\noindent{\bfseries \Large Part \thepart: Contents}}
%\hspace{\fill}\rule{0.95\linewidth}{2pt}\hspace{\fill}
\doparttoc[n]

%%fakesection Napkin Macros
\newcommand{\Zc}[1]{\ZZ/#1\ZZ}
\newcommand{\Zcc}[1]{\ZZ/(#1)\ZZ}
\newcommand{\Zm}[1]{(\ZZ/#1\ZZ)^\times}
\newcommand{\EE}{\mathbb E}
\newcommand{\TT}{\mathbb T}
\newcommand{\im}{^{\text{img}}} % or ``?
\newcommand{\pre}{^{\text{pre}}}
\newcommand{\normalin}{\trianglelefteq}
\newcommand{\triv}{\mathrm{triv}}
\newcommand{\largeotimes}[1]{\mathop{\otimes}\limits_{#1}}
\newcommand{\Mat}{\mathrm{Mat}}
\newcommand{\PGL}[1]{\mathbf{PGL}_{#1}(\CC)}

\DeclareMathOperator{\Stab}{Stab}
\DeclareMathOperator{\FixPt}{FixPt}
\DeclareMathOperator{\refl}{refl}
\DeclareMathOperator{\Fun}{Fun}
\DeclareMathOperator{\Irrep}{Irrep}
\DeclareMathOperator{\Res}{Res}
\DeclareMathOperator{\Reg}{Reg}
\DeclareMathOperator{\Classes}{Classes}
\DeclareMathOperator{\Frac}{Frac}
\newcommand{\FunCl}{\Fun_{\mathrm{class}}}
\newcommand{\Homrep}{\Hom_{\mathrm{rep}}}
\newcommand{\ab}{^\text{ab}} % abelianization
\DeclareMathOperator{\ev}{ev}
\DeclareMathOperator{\Wind}{\mathbf I}
\newcommand{\Ctriv}{\CC_{\mathrm{triv}}}
\newcommand{\Csign}{\CC_{\mathrm{sign}}}

% Alg geom macros
\newcommand{\VV}{\mathcal V}
\newcommand{\Vp}{\mathcal V_{\text{pr}}}
\newcommand{\Aff}{\mathbb A}
\newcommand{\II}{\mathcal I_{\text{rad}}}
\newcommand{\RP}{\mathbb{RP}}
\newcommand{\CP}{\mathbb{CP}}
\newcommand{\Cl}{\mathrm{Cl}}
\newcommand{\restrict}[1]{\bgroup\restriction_{#1}\egroup}
\DeclareMathOperator{\Opens}{OpenSets}
\DeclareMathOperator{\Proj}{Proj}
\DeclareMathOperator{\res}{res}
\newcommand{\km}{\mathfrak m}
\newcommand{\sh}{^\mathrm{sh}}

\renewcommand{\Re}{\opname{Re}}
\renewcommand{\Im}{\opname{Im}}

%% Category Theory macros
\renewcommand{\AA}{\mathcal A}
\newcommand{\BB}{\mathcal B}
\newcommand{\obj}{\operatorname{obj}}
\newcommand{\op}{^{\mathrm{op}}}


\usepackage{tkz-graph}
\pgfarrowsdeclare{biggertip}{biggertip}{%
  \setlength{\arrowsize}{1pt}  
  \addtolength{\arrowsize}{.1\pgflinewidth}  
  \pgfarrowsrightextend{0}  
  \pgfarrowsleftextend{-5\arrowsize}  
}{%
  \setlength{\arrowsize}{1pt}  
  \addtolength{\arrowsize}{.1\pgflinewidth}  
  \pgfpathmoveto{\pgfpoint{-5\arrowsize}{4\arrowsize}}  
  \pgfpathlineto{\pgfpointorigin}  
  \pgfpathlineto{\pgfpoint{-5\arrowsize}{-4\arrowsize}}  
  \pgfusepathqstroke  
}  
\tikzset{
	EdgeStyle/.style = {>=biggertip}
}

\usepackage[all,cmtip,2cell]{xy}
\UseTwocells
\newcommand\nattfm[5]{\xymatrix@C+2pc{#1 \rtwocell<4>^{#2}_{#4}{\; #3} & #5}}

%% Alg Top macros
\DeclareMathOperator{\Cells}{Cells}
\newcommand{\HdR}{H_{\mathrm{dR}}}

%% Alg NT macros
\newcommand{\ka}{\mathfrak a}
\newcommand{\kb}{\mathfrak b}
\newcommand{\kp}{\mathfrak p}
\newcommand{\kq}{\mathfrak q}

\newcommand{\Frob}{\mathrm{Frob}}
\DeclareMathOperator{\Norm}{N}
\DeclareMathOperator{\Ram}{Ram}
\newcommand{\TrK}{\Tr_{K/\QQ}}
\newcommand{\NK}{\Norm_{K/\QQ}}
\newcommand{\kf}{\mathfrak f}
\newcommand{\kP}{\mathfrak P}
\newcommand{\kQ}{\mathfrak Q}

%% Diff geo macros
\newcommand{\ee}{\mathbf e}
\newcommand{\fpartial}[2]{\frac{\partial #1}{\partial #2}}

%% ST macros
\newcommand{\CH}{\mathsf{CH}}
\newcommand{\ZFC}{\mathsf{ZFC}}

\newcommand{\Name}{\text{Name}}
\newcommand{\Po}{\mathbb P}
\newcommand{\nrank}{\opname{n-rank}} % ranks
\newcommand{\PP}{\mathcal P}

\newcommand{\EmptySet}{\mathrm{EmptySet}}
\newcommand{\PowerSet}{\mathrm{PowerSet}}
\newcommand{\Pairing}{\mathrm{Pairing}}
\newcommand{\Infinity}{\mathrm{Infinity}}
\newcommand{\Extensionality}{\mathrm{Extensionality}}
\newcommand{\Foundation}{\mathrm{Foundation}}
\newcommand{\Union}{\mathrm{Union}}
\newcommand{\Comprehension}{\mathrm{Comprehension}}
\newcommand{\Replacement}{\mathrm{Replacement}}

\usepackage{mathrsfs}

\newcommand\MM{\mathscr M}
\newcommand\llex{<_{\text{lex}}}

\DeclareMathOperator{\cof}{cof}

%% Quantum macros
\usepackage{braket}
\newcommand{\cvec}[1]{\begin{bmatrix} #1 \end{bmatrix}}
\newcommand{\pair}[2]{\begin{bmatrix} #1 \\ #2 \end{bmatrix}}
\newcommand{\zup}{\ket\uparrow}
\newcommand{\zdown}{\ket\downarrow}
\newcommand{\xup}{\ket\rightarrow}
\newcommand{\xdown}{\ket\leftarrow}
\newcommand{\yup}{\ket\otimes}
\newcommand{\ydown}{\ket\odot}
\newcommand{\UCNOT}{U_{\mathrm{CNOT}}}
\newcommand{\UQFT}{U_{\mathrm{QFT}}}

%% Measure
\DeclareMathOperator{\Var}{Var}
\newcommand{\cme}{^\text{cm}}
%%fakesection Misc haxx
\pdfstringdefDisableCommands{\def\Spec{\text{Spec }}}
