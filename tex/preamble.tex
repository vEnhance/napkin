\usepackage[nothm,diagrams,patchasy]{evan}
\usepackage[normalem]{ulem}
\usepackage{wasysym}
\usepackage{multirow}
\usepackage{prerex}

%% Environments, meta-level stuff, etwc.
\newcommand{\vocab}[1]{\textbf{\color{blue} #1}}
\newcommand{\problemhead}{Problems to Think About}
\newcommand{\prototype}[1]{
	\emph{{\color{red} Prototypical example for this section:} #1}\par
}
\newenvironment{moral}{\begin{mdframed}\bfseries\color{green!70!black}}%
	{\end{mdframed}}


%% Theorems
\usepackage{thmtools}
\usepackage[framemethod=TikZ]{mdframed}

\theoremstyle{definition}
\mdfdefinestyle{mdbluebox}{%
	roundcorner = 10pt,
	linewidth=1pt,
	skipabove=12pt,
	innerbottommargin=9pt,
	skipbelow=2pt,
	linecolor=blue,
	nobreak=true
}
\declaretheoremstyle[
	headfont=\sffamily\bfseries,
	mdframed={style=mdbluebox},
	headpunct={\\[3pt]},
	postheadspace={0pt}
]{thmbluebox}

\mdfdefinestyle{mdrecbox}{%
	linewidth=0.5pt,
	skipabove=12pt,
	frametitleaboveskip=5pt,
	frametitlebelowskip=0pt,
	skipbelow=2pt,
	frametitlefont=\bfseries,
	innertopmargin=4pt,
	innerbottommargin=8pt,
	nobreak=true,
}
\declaretheoremstyle[
	headfont=\bfseries,
	mdframed={style=mdrecbox},
	headpunct={\\[3pt]},
	postheadspace={0pt},
]{thmrecbox}

\declaretheorem[%
style=thmbluebox,name=Theorem,numberwithin=section]{theorem}
\declaretheorem[style=thmbluebox,name=Lemma,sibling=theorem]{lemma}
\declaretheorem[style=thmbluebox,name=Proposition,sibling=theorem]{proposition}
\declaretheorem[style=thmbluebox,name=Corollary,sibling=theorem]{corollary}
\declaretheorem[style=thmrecbox,name=Example,sibling=theorem]{example}

\theoremstyle{theorem}
\newtheorem{ques}[theorem]{Question}
\newtheorem{exercise}[theorem]{Exercise}

\theoremstyle{definition}
\newtheorem{claim}[theorem]{Claim}
\newtheorem{definition}[theorem]{Definition}
\newtheorem{fact}[theorem]{Fact}
\newtheorem{remark}[theorem]{Remark}
\newtheorem{abuse}[theorem]{Abuse of Notation}

\newtheorem{problem}{Problem}[chapter]
\renewcommand{\theproblem}{\thechapter\Alph{problem}}
\newtheorem{sproblem}[problem]{Problem}
\newtheorem{dproblem}[problem]{Problem}
\renewcommand{\thesproblem}{\theproblem$^{\star}$}
\renewcommand{\thedproblem}{\theproblem$^{\dagger}$}
\newcommand{\listhack}{$\empty$\vspace{-2em}}

% Answers
\usepackage{answers}
\Newassociation{hint}{answeritem}{tex/backmatter/all-hints}
\Newassociation{sol}{answeritem}{tex/backmatter/all-solns}
\renewcommand{\solutionextension}{out}
\renewenvironment{answeritem}[1]{\item[\bf #1.]}{}

% Hot chili peppers
\reversemarginpar
\newcommand{\prechili}{\vspace*{0.3em}\hspace*{1.5em}}
\newcommand{\nochili}{\hspace*{1.5em}}
\newcommand{\chili}{\includegraphics[width=1.5em]{media/chili.png}}
\newcommand{\gim}{\marginpar{\prechili\nochili\nochili\chili}}
\newcommand{\yod}{\marginpar{\prechili\nochili\chili\chili}}
\newcommand{\kurumi}{\marginpar{\prechili\chili\chili\chili}}

% Add ToC to PDF bookmarks
\makeatletter
\usepackage{etoolbox}
\pretocmd{\tableofcontents}{%
  \if@openright\cleardoublepage\else\clearpage\fi
  \pdfbookmark[0]{\contentsname}{toc}%
}{}{}%
\makeatother

%% Asymptote definitions
\numberwithin{asy}{chapter}
\renewcommand{\theasy}{\thechapter\Alph{asy}}
\begin{asydef}
	defaultpen(fontsize(11pt));
	pointfontsize=11;
	void bigbox(string L) {
		filldraw((-4,3)--(4,3)--(4,-3)--(-4,-3)--cycle,
			lightblue+opacity(0.2), black);
		label(L, (4,3), dir(-45));
	}
	void bigblob(string L) {
		filldraw((-4,3)..(-5,1)..(-4.7,0)..(-4.1,-2)..(-4,-3)
			..(-1,-3.3)..(1,-3.2)..(4,-3)
			..(4.3,-1)..(4.1,0.5)..(4,3)
			..(2,3.3)..(0,3.1)..(-3,2.9)..cycle,
			lightblue+opacity(0.2), black);
		label(L, (4,3), dir(5));
	}

	real[] default_ticks = {-4,-3,-2,-1,0,1,2.3,4};
	void cplane(real theta=-45, real[] T = default_ticks, real xmin = -4.5, real xmax = 4.5, real S = 3) {
		graph.xaxis("Re", xmin, xmax, graph.Ticks(format="%", Ticks=T, Size=S), Arrows);
		graph.yaxis("Im", xmin, xmax, graph.Ticks(format="%", Ticks=T, Size=S), Arrows);
		Drawing("0", origin, dir(theta));
		markscalefactor *= xmax;
	}

	pair opendot(pair P, pen p = defaultpen) {
		dot(P, p, UnFill);
		dot(P, p, Draw);
		return P;
	}
\end{asydef}
\def\asydir{asy}

%% Bibliography
\usepackage[backend=biber,style=alphabetic]{biblatex}
\DeclareLabelalphaTemplate{
	\labelelement{
		\field[final]{shorthand}
		\field{label}
		\field[strwidth=2,strside=left]{labelname}
	}
	\labelelement{
		\field[strwidth=2,strside=right]{year}    
	}
}
\DeclareFieldFormat{labelalpha}{\textbf{\scriptsize #1}}
\addbibresource{references.bib}



%% Macros
\newcommand{\Zc}[1]{\ZZ_{#1}}
\newcommand{\Zm}[1]{\ZZ_{#1}^\times}
\newcommand{\pre}{^{\text{pre}}}
\newcommand{\normalin}{\trianglelefteq}
\newcommand{\triv}{\mathrm{triv}}
\newcommand{\largeotimes}[1]{\mathop{\otimes}\limits_{#1}}
\newcommand{\Mat}{\mathrm{Mat}}
\newcommand{\PGL}[1]{\mathbf{PGL}_{#1}(\CC)}

\DeclareMathOperator{\GL}{GL}
\DeclareMathOperator{\SL}{SL}
\DeclareMathOperator{\Stab}{Stab}
\DeclareMathOperator{\refl}{refl}
\DeclareMathOperator{\Fun}{Fun}
\DeclareMathOperator{\Irrep}{Irrep}
\DeclareMathOperator{\Res}{Res}
\DeclareMathOperator{\Reg}{Reg}
\DeclareMathOperator{\ev}{ev}
\DeclareMathOperator{\Wind}{\mathbf I}

\newcommand{\VV}{\mathcal V}
\newcommand{\Aff}{\mathbb A}
% \newcommand{\II}{\mathbb I}
\newcommand{\CP}{\mathbb{CP}}
\newcommand{\Cl}{\mathrm{Cl}}

\renewcommand{\Re}{\opname{Re}}
\renewcommand{\Im}{\opname{Im}}


%% Category Theory macros
\renewcommand{\AA}{\mathcal A}
\newcommand{\BB}{\mathcal B}
\newcommand{\obj}{\operatorname{obj}}

\usepackage{tkz-graph}
\pgfarrowsdeclare{biggertip}{biggertip}{%
  \setlength{\arrowsize}{1pt}  
  \addtolength{\arrowsize}{.1\pgflinewidth}  
  \pgfarrowsrightextend{0}  
  \pgfarrowsleftextend{-5\arrowsize}  
}{%
  \setlength{\arrowsize}{1pt}  
  \addtolength{\arrowsize}{.1\pgflinewidth}  
  \pgfpathmoveto{\pgfpoint{-5\arrowsize}{4\arrowsize}}  
  \pgfpathlineto{\pgfpointorigin}  
  \pgfpathlineto{\pgfpoint{-5\arrowsize}{-4\arrowsize}}  
  \pgfusepathqstroke  
}  
\tikzset{
	EdgeStyle/.style = {>=biggertip}
}

\newcommand\catname\mathsf

\usepackage[all,cmtip,2cell]{xy}
\UseTwocells
\newcommand\nattfm[5]{\xymatrix@C+2pc{#1 \rtwocell<4>^{#2}_{#4}{\; #3} & #5}}

%% Alg NT macros
\renewcommand{\aa}{\mathfrak a}
\newcommand{\bb}{\mathfrak b}
\newcommand{\pp}{\mathfrak p}
\newcommand{\qq}{\mathfrak q}

\newcommand{\OO}{\mathcal O}
\newcommand{\Frob}{\mathrm{Frob}}
\DeclareMathOperator{\Norm}{N}
\DeclareMathOperator{\Gal}{Gal}
\newcommand{\TrK}{\Tr_{K/\QQ}}
\newcommand{\NK}{\Norm_{K/\QQ}}


%% ST macros
\newcommand{\CH}{\mathsf{CH}}
\newcommand{\ZFC}{\mathsf{ZFC}}

\newcommand{\Name}{\text{Name}}
\newcommand{\Po}{\mathbb P}
\newcommand{\nrank}{\opname{n-rank}} % ranks

\newcommand{\EmptySet}{\mathrm{EmptySet}}
\newcommand{\PowerSet}{\mathrm{PowerSet}}
\newcommand{\Pairing}{\mathrm{Pairing}}
\newcommand{\Infinity}{\mathrm{Infinity}}
\newcommand{\Extensionality}{\mathrm{Extensionality}}
\newcommand{\Foundation}{\mathrm{Foundation}}
\newcommand{\Union}{\mathrm{Union}}
\newcommand{\Comprehension}{\mathrm{Comprehension}}
\newcommand{\Replacement}{\mathrm{Replacement}}

\usepackage{mathrsfs}

\newcommand\MM{\mathscr M}
\newcommand\llex{<_{\text{lex}}}

\DeclareMathOperator{\cof}{cof}
