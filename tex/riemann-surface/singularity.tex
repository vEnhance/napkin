\chapter{Singularities on Riemann surfaces}
\label{ch:singularity_riemann}

\section{Filling in the holes}
\prototype{The Riemann sphere is formed by filling in a single hole in the complex plane $\CC$.}
\todo{write this}

\section{Nodes of a plane curve}
\prototype{The set defined by the equation $x^2-y^2=0$ has a simple node.}

\subsection{Affine plane curve}
A common type of Riemann surface is defined as the locus of zeroes of an equation in $\CC^2$.
Intuitively speaking, $\CC^2$ has complex dimension $2$, so a single equation limits one degree of
freedom, and the remaining part has dimension $1$.
\todo{write this}

\subsection{Types of nodes}

\todo{write this}

