\chapter{General topological spaces}
\label{ch:top_more}

Here I'm just talking about some various properties of general topological spaces.
The chapter is admittedly a bit disconnected (pun not intended, but hey, why not).

\todo{rework this chapter}
I should say a little about metric spaces vs general topological spaces here.
In most of the sections we only really talk about continuous maps or open sets,
so it doesn't make a difference whether we restrict our attention
to metric spaces or not.
But when we talk about completeness, we consider \emph{sequences} of points,
and in particular the distances between them.
This notion doesn't make sense for general spaces, so for that section one must
keep in mind we are only working in metric spaces.

The most important topological notion is missing from this chapter:
that of a \emph{compact} space.
It is so important that I have dedicated a separate chapter just for it.

Note that in contrast to the warning on open/closed sets I gave earlier,
\begin{moral}
	The adjectives in this chapter will be used to describe \emph{spaces}.
\end{moral}

%As I alluded to earlier, sequences in metric spaces are super nice,
%but sequences in general topological spaces \emph{suck} (to the point where
%I didn't bother to define convergence of general sequences).

\section{Forgetting the metric}
Notice something interesting about the previous theorem --
it doesn't reference the metrics of $M$ and $N$ at all.
Instead, it refers only to the open sets.

This leads us to consider:
what if we could refer to spaces \emph{only} by their open sets,
forgetting about the fact that we had a metric to begin with?
That's exactly what we do in ``point-set topology''.

\begin{definition}
	A \vocab{topological space} is a pair $(X, \mathcal T)$,
	where $X$ is a set of points,
	and $\mathcal T$ is the \vocab{topology}, which consists of several subsets of $X$, called the \vocab{open sets} of $X$.
	The topology must obey the following axioms.
	\begin{itemize}
		\ii $\varnothing$ and $X$ are both in $\mathcal T$.
		\ii Finite intersections of open sets are also in $\mathcal T$.
		\ii Arbitrary unions (possibly infinite) of open sets are also in $\mathcal T$.
	\end{itemize}
\end{definition}
So this time, the open sets are \emph{given}.
Rather than defining a metric and getting open sets from the metric,
we instead start from just the open sets.
\begin{abuse}
	We refer to the space $(X, \mathcal T)$ by just $X$.
	(Do you see a pattern here?)
\end{abuse}

\begin{example}[Examples of topologies]
	\listhack
	\begin{enumerate}[(a)]
		\ii Given a metric space $M$, we can let $\mathcal T$ be
		the open sets in the metric sense.
		The point is that the axioms are satisfied.
		\ii In particular, \vocab{discrete space} is a topological space in which every set is open. (Why?)
		\ii Given $X$, we can let $\mathcal T = \left\{ \varnothing, X \right\}$,
		the opposite extreme of the discrete space.
	\end{enumerate}
\end{example}

Now we can port over our metric definitions.
\begin{definition}
	An \vocab{open neighborhood} of a point $x \in X$ is an
	open set $U$ which contains $x$ (see figure).
\end{definition}
\begin{center}
	\begin{asy}
		size(4cm);
		bigblob("$X$");
		pair p = Drawing("x", (0.3,0.1), dir(-90));
		real r = 1.55;
		draw(shift(p) * scale(1.6,1.2)*unitcircle, dashed);
		label("$U$", p+r*dir(45), dir(45));
	\end{asy}
\end{center}

\begin{abuse}
	Just to be perfectly clear:
	by a ``open neighborhood'' I mean \emph{any} open set containing $x$.
	But by an ``$r$-neighborhood'' I always mean the
	points with distance less than $r$ from $x$,
	and so I can only use this term if my space is a metric space.
\end{abuse}

\begin{abuse}
	There's another related term commonly used:
	a \emph{neighborhood} $V$ of $x$ is a set
	which contains some open neighborhood of $x$ (often $V$ itself).
	Think of it as ``open around $x$'', though
	not always at other points.
	However, for most purposes, you should think of neighborhoods
	as just open neighborhoods.
\end{abuse}
\begin{definition}
	A function $f : X \to Y$ of topological spaces
	is \vocab{continuous} at $p \in X$ if the pre-image of any
	open neighborhood of $fp$ is an open neighborhood of $p$.
	It is continuous if it is continuous at every point,
	meaning that the pre-image of any open set is open.
\end{definition}

You can also port over the notion of sequences and convergent sequences.
But I won't bother to do so because sequences lose most of the nice properties
they had before.

%\begin{definition}
%	A sequence $(x_n)$ of points in a topological space $X$ is said to \vocab{converge to} $x \in X$ if for every neighborhood of $x$,
%	eventually all terms of the sequence lie in that neighborhood.
%\end{definition}
%\begin{remark}
%	Unfortunately, for general topological spaces we no longer have the nice property
%	that any function which preserves sequential limits is automatically continuous.
%\end{remark}

%There's one other property of open sets that we have in a metric space that isn't implied by the above: for any two points of $X$, we can find an open set containing one but not the other.
%A space which also has this property is called a \vocab{Kolmogorov space}.
%This property is a good property to have, because if $x,y \in X$ are in the same open sets, the topology can't tell them apart.

Finally, what are the homeomorphisms?
The same definition carries over: a bijection which is continuous in both directions.
\begin{definition}
	A \vocab{homeomorphism} of topological spaces $(X, \tau_X)$ and $(Y, \tau_Y)$
	is a bijection from $X$ to $Y$
	which induces a bijection from $\tau_X$ to $\tau_Y$:
	i.e.\ the bijection preserves open sets.
\end{definition}
Therefore, any property defined only in terms of open sets is preserved by homeomorphism.
Such a property is called a \vocab{topological property}.
That's why $(0,1)$ homeomorphic to $\RR$ is not so surprising,
because the notion of being ``bounded'' is not a notion
which can be expressed in terms of open sets.

\begin{remark}
	As you might have guessed, there exist topological spaces which cannot be realized
	as metric spaces (in other words, are not \vocab{metrizable}).
	One example is just to take $X = \{a,b,c\}$ and the topology
	$\tau_X = \left\{ \varnothing, \{a,b,c\} \right\}$.
	This topology is fairly ``stupid'':
	it can't tell apart any of the points $a$, $b$, $c$!
	But any metric space can tell its points apart (because $d(x,y) > 0$ when $x \neq y$).
	We'll see less trivial examples later.
\end{remark}

\section{Closed sets}

This leads us to a definition for a general topological space.
\begin{definition}
	In a general topological space $X$, we say that $S \subseteq X$ is
	\vocab{closed} in $X$ if the complement $X \setminus S$ is open in $X$.
\end{definition}
Hence, for general topological spaces, open and closed sets carry the same information,
and it is entirely a matter of taste whether we define everything in terms
of open sets or closed sets.
In particular,
\begin{ques}
	Show that the (possibly infinite) intersection of closed sets is closed
	while the union of finitely many closed sets is closed.
	(Hint: just look at complements.)
\end{ques}
\begin{ques}
	Show that a function is continuous if and only if the pre-image
	of every closed set is closed.
\end{ques}
Mathematicians seem to have agreed that they like open sets better.


\section{Connected spaces}
It is possible for a set to be both open and closed
(or \vocab{clopen}) in a topological space $X$;
for example $\varnothing$ and the entire space are examples of clopen sets.
In fact, the presence of a nontrivial clopen set other than these two leads
to a so-called \emph{disconnected} space.

\begin{ques}
	Show that a space $X$ has a nontrivial clopen set
	(one other than $\varnothing$ and $X$)
	if and only if $X$ can be written as a disjoint union $U \sqcup V$
	where $U$ and $V$ are both open and nonempty.
	(Use the definition that closed sets are complements of open sets.)
\end{ques}

We say $X$ is \vocab{disconnected} if there are nontrivial clopen sets,
and \vocab{connected} otherwise.
To see why this should be a reasonable definition, it might help
to solve \Cref{prob:disconnected_better_def}.

\begin{example}[Disconnected and connected spaces]
	\listhack
	\begin{enumerate}[(a)]
		\ii The metric space 
		\[ \{ (x,y) \mid x^2+y^2\le 1 \} \cup \{ (x,y) \mid (x-4)^2+y^2\le1\} \subseteq \RR^2 \]
		is disconnected (it consists of two disks).

		\ii A discrete space on more than one point is disconnected,
		since \emph{every} set is clopen in the discrete space.

		\ii Convince yourself that the set
		\[ \left\{ x \in \QQ : x^2 < 2014 \right\} \]
		is a clopen subset of $\QQ$.
		Hence $\QQ$ is disconnected too -- it has \emph{gaps}.

		\ii $[0,1]$ is connected.
	\end{enumerate}
\end{example}

%\begin{remark}
%	For general topological spaces $X$, it is still true that if $S$ is closed
%	(meaning $X \setminus S$ is open), then it contains the limits of all its sequences.
%	But the converse of this statement no longer holds.
%\end{remark}

\section{Bases of spaces}
\prototype{$\RR$ has a basis of open intervals, and $\RR^2$ has a basis of open disks.}

You might have noticed that the open sets of $\RR$ are a little annoying to describe:
the prototypical example of an open set is $(0,1)$,
but there are other open sets like
\[
	(0,1)
	\cup \left( 1, \frac32 \right)
	\cup \left( 2, \frac 73 \right)
	\cup (2014, 2015). \]
\begin{ques}
	Check this is an open set.
\end{ques}

But okay, this isn't \emph{that} different.
All I've done is taken a bunch of my prototypes and threw a bunch of $\cup$ signs at it.
And that's the idea behind a basis.

\begin{definition}
	A \vocab{basis} for a topological space $X$
	is a subset $\mathcal B$ of the open sets
	such that every open set in $X$
	is a union of some (possibly infinite) number of elements in
	$\mathcal B$.
\end{definition}

And all we're doing is saying
\begin{example}[Basis of $\RR$]
	The open intervals form a basis of $\RR$.
\end{example}
In fact, more generally we have:
\begin{theorem}[Basis of metric spaces]
	The $r$-neighborhoods form a basis of any metric space $M$.
\end{theorem}
\begin{proof}
	Kind of silly -- given an open set $U$
	draw an $r_p$-neighborhood $U_p$ contained entirely inside $U$.
	Then $\bigcup_p U_p$ is contained in $U$ and covers
	every point inside it.
\end{proof}

Hence, an open set in $\RR^2$ is nothing more than a union
of a bunch of open disks, and so on.
The point is that in a metric space, the only open sets you really
ever have to worry too much about are the $r$-neighborhoods.


\section{Subspacing}
Earlier in this chapter,
I declared that all the adjectives we defined were used to describe spaces.
However, I now want to be more flexible and describe
how they can be used to define subsets of spaces as well.

We've talked about some properties that a space can have;
say, a space is \emph{path-connected}, or \emph{simply connected}, or \emph{complete}.
If you think about it, it would make sense to use the same adjectives on sets.
You'd say a set $S$ is path-connected if there's a path between
every pair of points in $S$ through points in $S$. And so on.

The reason you can do this is that for a metric space $M$,
\begin{moral}
	Every subset $S \subseteq M$ is a metric space in its own right.
\end{moral}
To see this, we just use the same distance function;
this is called the \vocab{subspace topology}.
For example, you can think of a circle $S^1$ as a connected set
by viewing it as a subspace of $\RR^2$.
Thus,
\begin{abuse}
	Any adjective used to describe a space can equally be used for a subset of a space,
	and I'll thus be letting these mix pretty promiscuously.
	So in what follows, I might refer to subsets being complete,
	even when I only defined completeness for spaces.
	This is what I mean.
\end{abuse}
So to be perfectly clear:
\begin{itemize}
	\ii The statement ``$[0,1]$ is complete'' makes sense (and is true);
	it says that $[0,1]$ is a complete metric space.
	\ii The statement ``$[0,1]$ is a complete subset of $\RR$'' is valid;
	it says that the subspace $[0,1]$ of $\RR$ is a complete metric space,
	which is of course true.
	\ii The statement ``$[0,1]$ is a closed subset of $\RR$'' makes sense;
	it says that the set of points $[0,1]$ form a closed subset of a parent space $\RR$.
	\ii The statement ``$[0,1]$ is closed'' does \emph{not} make sense.
	Closed sets are only defined relative to parent spaces.
\end{itemize}

To make sure you understand this:
\begin{exercise}
	Let $M$ be a complete metric space and let $S \subseteq M$.
	Prove that $S$ is complete if and only if it is closed in $M$.
	(This is obvious once you figure out what the question is asking.)
	In particular, $[0,1]$ is complete.
\end{exercise}

One can also define the subspace topology in a general topological space $X$.
Given a subset $S \subseteq X$, the open sets of the subspace $S$ are those of the form $S \cap U$,
where $U$ is an open set of $X$.
So for example, if we view $S^1$ as a subspace of $\RR^2$,
then any open arc is an open set, because you can view it as the intersection of an open disk with $S^1$.
\begin{center}
	\begin{asy}
		size(3cm);
		draw(unitcircle, black+1);
		MP("S^1", dir(60), dir(60));
		MP("\mathbb R^2", dir(-45)*1.2, dir(-45));
		pair A = dir(-30);
		pair B = dir(50);
		draw(CP(dir(10), A), dotted);
		draw(arc(origin,A,B), blue+2);
		dotfactor *= 2;
		opendot(A, blue);
		opendot(B, blue);
	\end{asy}
\end{center}

Needless to say, for metric spaces it doesn't matter which of these definitions I choose.
(Proving this turns out to be surprisingly annoying, so I won't do so.)

\section{\problemhead}

\begin{dproblem}
	Let $X$ be a topological space.
	Show that there exists a nonconstant continuous function $X \to \{0,1\}$ if and
	only if $X$ is disconnected (here $\{0,1\}$ is given the discrete topology).
	\label{prob:disconnected_better_def}
\end{dproblem}

\begin{sproblem}
	Let $f \colon X \to Y$ be a continuous function.
	Show that if $X$ is connected then so is $f\im(X)$.
\end{sproblem}

\begin{problem}[Furstenberg]
	We declare a subset of $\ZZ$ to be open
	if it's the union (possibly empty or infinite)
	of arithmetic sequences $\left\{ a + nd \mid n \in \ZZ \right\}$,
	where $a$ and $d$ are positive integers.
	\begin{enumerate}[(a)]
		\ii Verify this forms a topology on $\ZZ$,
		called the \vocab{evenly spaced integer topology}.
		\ii Prove there are infinitely many primes by considering $\bigcup_p p\ZZ$.
	\end{enumerate}
	\begin{hint}
		Note that $p\ZZ$ is closed for each $p$.
		If there were finitely many primes, then
		$\bigcup p\ZZ = \ZZ \setminus \{-1,1\}$ would have to be closed;
		i.e.\ $\{-1,1\}$ would be open, but all open sets here are infinite.
	\end{hint}
\end{problem}

\begin{problem}
	\gim
	Prove that the evenly spaced integer topology on $\ZZ$ is metrizable.
	In other words, show that one can impose a metric $d : \ZZ^2 \to \RR$
	which makes $\ZZ$ into a metric space whose open sets are those described above.
	% https://teratologicmuseum.wordpress.com/2009/05/05/a-metric-for-the-evenly-spaced-integer-topology/
	\begin{hint}
		The balls at $0$ should be of the form $n! \cdot \ZZ$.
	\end{hint}
	\begin{sol}
		Let $d(x,y) = 2017^{-n}$,
		where $n$ is the largest integer
		such that $n!$ divides $\left\lvert x-y \right\rvert$.
	\end{sol}
\end{problem}

%\begin{problem}
%	A topological space $X$ is called \vocab{locally path-connected}
%	if for every point $x \in X$, some neighborhood $U$ of $x$ is path-connected.
%	Prove that $X$ is path-connected if and only if it is connected
%	and locally path-connected.
%	\label{prob:local_path_connected}
%\end{problem}

\begin{problem}
	\gim
	We know that any open set $U \subseteq \RR$
	is a union of open intervals (allowing $\pm\infty$ as endpoints).
	One can show that it's actually possible to write $U$ as the
	union of \emph{pairwise disjoint} open intervals.\footnote{You are invited to try
	and prove this, but I personally found the proof quite boring.}
	Prove that there exists such a disjoint union with at most \emph{countably many}
	intervals in it.
	\begin{hint}
		Appeal to $\QQ$.
	\end{hint}
	\begin{sol}
		You can pick a rational number in each interval and
		there are only countably many rational numbers. Done!
	\end{sol}
\end{problem}

