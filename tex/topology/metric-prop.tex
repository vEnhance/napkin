\chapter{Properties of metric spaces}
At the end of the last chapter on metric spaces,
we introduced two adjectives ``open'' and ``closed''.
These are important because they'll grow up
to be the \emph{definition} for a general topological space,
once we graduate from metric spaces.

To move forward, we provide a couple niceness adjectives
that applies to \emph{entire metric spaces},
rather than just a set relative to a parent space.
They are ``(totally) bounded'' and ``complete''.
These adjectives are specific to metric spaces,
but will grow up to become the notion of \emph{compactness},
which is, in the words of \cite{ref:pugh},
``the single most important concept in real analysis''.
At the end of the chapter,
we will know enough to realize that something is amiss
with our definition of homeomorphism,
and this will serve as the starting point for the next chapter,
when we define fully general topological spaces.

\section{Boundedness}
\prototype{$[0,1]$ is bounded but $\RR$ is not.}
Here is one notion of how to prevent a metric space
from being a bit too large.

\begin{definition}
	A metric space $M$ is \vocab{bounded}
	if there is a constant $D$ such that
	$d(p,q) \le D$ for all $p,q \in M$.
\end{definition}
You can change the order of the quantifiers:
\begin{proposition}
	[Boundedness with radii instead of diameters]
	A metric space $M$ is bounded if and only if
	for every point $p \in M$, there is a radius $R$
	(possibly depending on $p$) such that $d(p,q) \le R$ for all $q \in M$.
\end{proposition}
\begin{exercise}
	Use the triangle inequality to show these are equivalent.
	(The names ``radius'' and ``diameter'' are a big hint!)
\end{exercise}

\begin{example}
	[Examples of bounded spaces]
	\listhack
	\begin{enumerate}[(a)]
		\ii Finite intervals like $[0,1]$ and $(a,b)$ are bounded.
		\ii The unit square $[0,1]^2$ is bounded.
		\ii $\RR^n$ is not bounded for any $n \ge 1$.
		\ii A discrete space on an infinite set is bounded.
		\ii $\NN$ is not bounded, despite being
		homeomorphic to the discrete space!
	\end{enumerate}
\end{example}

The fact that a discrete space on an infinite set
is ``bounded'' might be upsetting to you, so
here is a somewhat stronger condition you can use:

\begin{definition}
	A metric space is \vocab{totally bounded}
	if for any $\eps > 0$,
	we can cover $M$ with finitely many $\eps$-neighborhoods.
\end{definition}
For example, if $\eps = 1/2$, you can cover $[0,1]^2$
by $\eps$-neighborhoods.
\begin{center}
	\begin{asy}
		size(4cm);
		draw(shift( (-0.5,-0.5) )*unitsquare, black+1);
		real d = 0.4;
		real r = 0.5;
		draw(CR(dir( 45)*d, r), dotted);
		draw(CR(dir(135)*d, r), dotted);
		draw(CR(dir(225)*d, r), dotted);
		draw(CR(dir(315)*d, r), dotted);
	\end{asy}
\end{center}
\begin{exercise}
	Show that ``totally bounded'' implies ``bounded''.
\end{exercise}
\begin{example}
	[Examples of totally bounded spaces]
	\listhack
	\begin{enumerate}[(a)]
		\ii A subset of $\RR^n$ is bounded if and only if
		it is totally bounded.

		This is for Euclidean geometry reasons:
		for example in $\RR^2$ if I can cover a set
		by a single disk of radius $2$,
		then I can certainly cover it by finitely many
		disks of radius $1/2$.
		(We won't prove this rigorously.)

		\ii So for example $[0,1]$ or $[0,2] \times [0,3]$
		is totally bounded.

		\ii In contrast, a discrete space on
		an infinite set is not totally bounded.
	\end{enumerate}
\end{example}

\section{Completeness}
\prototype{$\RR$ is complete, but $\QQ$ and $(0,1)$ are not.}

So far we can only talk about sequences converging if they have a limit.
But consider the sequence
\[ x_1 = 1, \; x_2 = 1.4, \; x_3 = 1.41, \; x_4 = 1.414, \dots. \]
It converges to $\sqrt 2$ in $\RR$, of course.
But it fails to converge in $\QQ$;
there is no \emph{rational} number this sequence converges to.
And so somehow, if we didn't know about the existence of $\RR$, we would
have \emph{no idea} that the sequence $(x_n)$ is ``approaching'' something.

That seems to be a shame.
Let's set up a new definition to describe these sequences whose terms
\textbf{get close to each other},
even if they don't approach any particular point in the space.
Thus, we only want to mention the given points in the definition.

\begin{definition}
	Let $x_1, x_2, \dots$ be a sequence which lives in a metric space $M = (M,d_M)$.
	We say the sequence is \vocab{Cauchy} if for any $\eps > 0$, we have
	\[ d_M(x_m, x_n) < \eps \]
	for all sufficiently large $m$ and $n$.
\end{definition}

\begin{ques}
	Show that a sequence which converges is automatically Cauchy.
	(Draw a picture.)
\end{ques}

Now we can define:
\begin{definition}
	A metric space $M$ is \vocab{complete} if every
	Cauchy sequence converges.
\end{definition}

\begin{example}
	[Examples of complete spaces]
	\listhack
	\begin{enumerate}[(a)]
		\ii $\RR$ is complete.
		(Depending on your definition of $\RR$,
		this either follows by definition, or requires some work.
		We won't go through this here.)
		\ii The discrete space is complete,
		as the only Cauchy sequences are eventually constant.
		\ii The closed interval $[0,1]$ is complete.
		\ii $\RR^n$ is complete as well.
		(You're welcome to prove this by induction on $n$.)
	\end{enumerate}
\end{example}

\begin{example}
	[Non-examples of complete spaces]
	\listhack
	\begin{enumerate}[(a)]
		\ii The rationals $\QQ$ are not complete.
		\ii The open interval $(0,1)$ is not complete,
		as the sequence $0.9$, $0.99$, $0.999$, $0.9999$, \dots
		is Cauchy but does not converge.
	\end{enumerate}
\end{example}

So, metric spaces need not be complete, like $\QQ$.
But we certainly would like them to be complete,
and in light of the following theorem this is not unreasonable.
\begin{theorem}
	[Completion]
	Every metric space can be ``completed'',
	i.e.\ made into a complete space by adding in some points.
\end{theorem}
We won't need this construction at all,
so it's left as \Cref{prob:completion}.
\begin{example}
	[$\QQ$ completes to $\RR$]
	The completion of $\QQ$ is $\RR$.
\end{example}
(In fact, by using a modified definition of completion
not depending on the real numbers,
other authors often use this as the definition of $\RR$.)

\section{Let the buyer beware}
There is something suspicious about both these notions:
neither are preserved under homeomorphism!

\begin{example}
	[Something fishy is going on here]
	\label{ex:fishy}
	Let $M = (0,1)$ and $N = \RR$.
	As we saw much earlier $M$ and $N$ are homeomorphic.
	However:
	\begin{itemize}
		\ii $(0,1)$ is totally bounded, but not complete.
		\ii $\RR$ is complete, but not bounded.
	\end{itemize}
\end{example}

This is the first hint of something going awry with the metric.
As we progress further into our study of topology,
we will see that in fact \emph{open and closed sets}
(which we motivated by using the metric)
are the notion that will really shine later on.
I insist on introducing the metric first so that
the standard pictures of open and closed sets make sense,
but eventually it becomes time to remove the training wheels.

%It's a theorem that $\RR$ is complete.
%To prove this I'd have to define $\RR$ rigorously, which I won't do here.
%In fact, there are some competing definitions of $\RR$.
%It is sometimes defined as the completion of the space $\QQ$.
%Other times it is defined using something called \emph{Dedekind cuts}.
%For now, let's just accept that $\RR$ behaves as we expect and is complete.

%And thus, I can now tell you exactly what $\RR$ is.
%You might notice that it's actually not that easy to define:
%we can define $\QQ = \left\{ a/b : a,b \in \NN \right\}$, but
%what do we say for $\RR$?
%Here's the answer:
%\begin{definition}
%	$\RR$ is the completion of the metric space $\QQ$.
%\end{definition}

\section{Subspaces, and (inb4) a confusing linguistic point}
\prototype{A circle is obtained as a subspace of $\RR^2$.}

As we've already been doing implicitly in examples, we'll now say:
\begin{definition}
	Every subset $S \subseteq M$ is a metric space in its own right,
	by re-using the distance function on $M$.
	We say that $S$ is a \vocab{subspace} of $M$.
\end{definition}
For example, we saw that the circle $S^1$
is just a subspace of $\RR^2$.

It thus becomes important to distinguish between
\begin{enumerate}[(i)]
	\ii \textbf{``absolute'' adjectives} like ``complete'' or ``bounded'',
	which can be applied to both spaces,
	and hence even to subsets of spaces (by taking a subspace),
	and
	\ii \textbf{``relative'' adjectives}
	like ``open (in $M$)'' and ``closed (in $M$)'',
	which make sense only relative to a space,
	even though people are often sloppy and omit them.
\end{enumerate}
So ``$[0,1]$ is complete'' makes sense,
as does ``$[0,1]$ is a complete subset of $\RR$'',
which we take to mean ``$[0,1]$ is a complete subspace of $\RR$''.
This is since ``complete'' is an absolute adjective.

But here are some examples of ways in which relative adjectives
require a little more care:
\begin{itemize}
	\ii Consider the sequence $1$, $1.4$, $1.41$, $1.414$, \dots.
	Viewed as a sequence in $\RR$, it converges to $\sqrt 2$.
	But if viewed as a sequence in $\QQ$,
	this sequence does \emph{not} converge!
	Similarly, the sequence $0.9$, $0.99$, $0.999$, $0.9999$
	does not converge in the space $(0,1)$,
	although it does converge in $[0,1]$.

	The fact that these sequences fail to converge
	even though they ``ought to'' is weird and bad,
	and was why we defined complete spaces to begin with.

	\ii In general, it makes no sense to ask a question like ``is $[0,1]$ open?''.
	The questions ``is $[0,1]$ open in $\RR$?''
	and ``is $[0,1]$ open in $[0,1]$?'' do make sense, however.
	The answer to the first question is ``no''
	but the answer to the second question is ``yes'';
	indeed, every space is open in itself.
	Similarly, $[0, \half)$ is an open set % chktex 9
	in the space $M = [0,1]$
	because it is the ball \emph{in $M$}
	of radius $\half$ centered at $0$.

	\ii Dually, it doesn't make sense to ask ``is $[0,1]$ closed''?
	It is closed \emph{in $\RR$} and \emph{in itself}
	(but every space is closed in itself, anyways).
\end{itemize}

%Thus to list out all four cases:
%\begin{itemize}
%	\ii The statement ``$[0,1]$ is complete'' makes sense (and is true);
%	it says that $[0,1]$ is a complete metric space.
%	\ii The statement ``$[0,1]$ is a complete subset of $\RR$'' is valid;
%	it says that the subspace $[0,1]$ of $\RR$ is a complete metric space.
%	\ii The statement ``$[0,1]$ is a closed subset of $\RR$'' makes sense;
%	it says that the set of points $[0,1]$ form a closed subset of a parent space $\RR$.
%	\ii The statement ``$[0,1]$ is closed'' does \emph{not} make sense officially.
%	Closed sets are only defined relative to parent spaces.
%\end{itemize}

To make sure you understand the above,
here are two exercises to help you practice relative adjectives.

\begin{exercise}
	Let $M$ be a complete metric space and let $S \subseteq M$.
	Prove that $S$ is complete if and only if it is closed in $M$.
	% (This is obvious once you figure out what the question is asking.)
	In particular, $[0,1]$ is complete.
\end{exercise}
\begin{exercise}
	Let $M = [0,1] \cup (2,3)$.
	Show that $[0,1]$ and $(2,3)$ are both open and closed in $M$.
\end{exercise}

This illustrates a third point:
a nontrivial set can be both open and closed\footnote{Which
	always gets made fun of.}
As we'll see in \Cref{ch:top_more}, this implies the space is disconnected;
i.e.\ the only examples look quite like the one we've given above.

\section{\problemhead}
\begin{dproblem}[Banach fixed point theorem]
	Let $M = (M,d)$ be a complete metric space.
	Suppose $T \colon M \to M$ is a continuous map
	such that for any $p \neq q \in M$,
	\[ d\left( T(p), T(q) \right) < 0.999 d(p,q). \]
	(We call $T$ a \vocab{contraction}.)
	Show that $T$ has a unique fixed point.
\end{dproblem}

\begin{problem}
	[Henning Makholm,
	on \href{https://math.stackexchange.com/a/3051746/229197}{math.SE}]
	We let $M$ and $N$ denote the metric spaces obtained
	by equipping $\RR$ with the following two metrics:
	\begin{align*}
		d_M(x,y) &= \min \left\{ 1, \left\lvert x-y \right\rvert \right\} \\
		d_N(x,y) &= \left\lvert e^x - e^y \right\rvert.
	\end{align*}
	\begin{enumerate}[(a)]
		\ii Fill in the following $2 \times 3$ table
		with ``yes'' or ``no'' for each cell.
		\begin{center}
		\begin{tabular}[h]{l|ccc}
			& Complete? & Bounded? & Totally bounded? \\ \hline
			$M$ \\
			$N$ \\
		\end{tabular}
		\end{center}
		\ii Are $M$ and $N$ homeomorphic?
	\end{enumerate}
	\begin{hint}
		(a): $M$ is complete and bounded
		but not totally bounded. $N$ is all no.
		For (b) show that $M \cong \RR \cong N$.
	\end{hint}
	\begin{sol}
		Part (a) is essentially by definition.
		The space $M$ is bounded since no distances exceed $1$,
		but not totally bounded since we can't cover $M$
		with finitely many $\half$-neighborhoods.
		The space $M$ is complete since a sequence of real
		numbers converges in $M$ if it converges in the usual sense.
		As for $N$, the sequence $-1$, $-2$, \dots is Cauchy
		but fails to converge; and it is obviously not bounded.

		To show (b), the identity map (!) is an homeomorphism $M \cong \RR$
		and $\RR \cong N$, since it is continuous.

		This illustrates that $M \cong N$ despite the fact
		that $M$ is both complete and bounded
		but $N$ is neither complete nor bounded.
		On the other hand, we will later see that complete
		and totally bounded implies \emph{compact},
		which is a very strong property preserved under homeomorphism.
	\end{sol}
\end{problem}


\begin{dproblem}[Completion of a metric space]
	\gim
	\label{prob:completion}
	Let $M$ be a metric space.
	Construct a complete metric space $\ol M$
	such that $M$ is a subspace of $\ol M$,
	and every open set of $\ol M$ contains a point of $M$
	(meaning $M$ is \vocab{dense} in $\ol M$).
	\begin{hint}
		As a set, we let $\ol M$ be the set of Cauchy
		sequences $(x_n)$ in $M$, modulo the relation that
		$(x_n) \sim (y_n)$ if $\lim_n d(x_n, y_n) = 0$.
	\end{hint}
\end{dproblem}

\begin{problem}
	Show that a metric space is totally bounded
	if and only if any sequence has a Cauchy subsequence.
	\begin{sol}
		See \url{https://math.stackexchange.com/q/556150/229197}.
	\end{sol}
\end{problem}

\begin{problem}
	\kurumi
	Prove that $\QQ$ is not homeomorphic to any complete metric space.
	\begin{hint}
		The standard solution seems to be
		via the so-called ``Baire category theorem''.
	\end{hint}
\end{problem}
