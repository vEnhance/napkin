\chapter{Covering projections}
\label{ch:covering_projections}
A few chapters ago we talked about what a fundamental group was,
but we didn't actually show how to compute any of them
except for the most trivial case of a simply connected space.
In this chapter we'll introduce the notion of a \emph{covering projection},
which will let us see how some of these groups can be found.

\section{Even coverings and covering projections}
\prototype{$\RR$ covers $S^1$.}
What we want now is a notion where a big space $E$, a ``covering space'',
can be projected down onto a base space $B$ in a nice way.
Here is the notion of ``nice'':
\begin{definition}
	Let $p \colon E \to B$ be a continuous function.
	Let $U$ be an open set of $B$.
	We call $U$ \vocab{evenly covered} (by $p$) if $p\pre(U)$ is a disjoint union of open sets (possibly infinite) such that $p$ restricted to any of these sets is a homeomorphism.
\end{definition}
Picture:
\begin{center}
	\includegraphics[width=4cm]{media/even-covering.png}
	\\ \scriptsize Image from \cite{img:even_covering}
\end{center}
All we're saying is that $U$ is evenly covered if its pre-image
is a bunch of copies of it. (Actually, a little more: each of the pancakes is homeomorphic to $U$, but we also require that $p$ is the homeomorphism.)

\begin{definition}
	A \vocab{covering projection} $p \colon E \to B$
	is a surjective continuous map such that every base point $b \in B$
	has an open neighborhood $U \ni b$ which is evenly covered by $p$.
\end{definition}
\begin{exercise}
	[On requiring surjectivity of $p$]
	Let $p \colon E \to B$ be satisfying this definition,
	except that $p$ need not be surjective.
	Show that the image of $p$ is a disjoint union of connected components of $B$.
	Thus if $B$ is connected and $E$ is nonempty,
	then $p \colon E \to B$ is already surjective.
	For this reason, some authors omit the surjectivity hypothesis
	as usually $B$ is path-connected.
\end{exercise}

Here is the most stupid example of a covering projection.
\begin{example}[Tautological covering projection]
	Let's take $n$ disconnected copies of any space $B$:
	formally, $E = B \times \{1, \dots, n\}$ with the discrete topology
	on $\{1, \dots, n\}$.
	Then there exists a tautological covering projection
	$E \to B$ by $(x,m) \mapsto x$;
	we just project all $n$ copies.

	This is a covering projection because \emph{every} open set in $B$
	is evenly covered.
\end{example}
This is not really that interesting because $B \times [n]$ is not path-connected.

A much more interesting example is that of $\RR$ and $S^1$.

\begin{example}[Covering projection of $S^1$]
	Take $p \colon \RR \to S^1$ by $\theta \mapsto e^{2\pi i \theta}$.
	This is essentially wrapping the real line
	into a single helix and projecting it down.
\end{example}
\begin{center}
	\begin{asy}
		var flatten = yscale(0.4);
		guide g;
		for (real a=90+360*5; a>=90; a-=20){
			g = g .. flatten*dir(a)+(0, a/360/3);
		}
		draw(g, Arrows);
		draw((0, -0.4)--(0, -1.4), Arrow, L=Label("$p$"));
		draw(shift(0, -2)*flatten*unitcircle, L=Label("$S^1$", Relative(0.1)));
	\end{asy}
\end{center}


We claim this is a covering projection.
Indeed, consider the point $1 \in S^1$
(where we view $S^1$ as the unit circle in the complex plane).
We can draw a small open neighborhood of it
whose pre-image is a bunch of copies in $\RR$.
\begin{center}
	\begin{asy}
		size(12cm);

		real[] t = {-2,-1,0,1,2};
		xaxis(-3.5,3.5, graph.LeftTicks(Ticks=t), Arrows);

		pen bloo = blue+1.5;

		dotfactor *= 2;
		pair A,B;
		for (real x = -2; x <= 2; ++x) {
			A = (x-0.2, 0); B = (x+0.2, 0);
			draw(A--B, bloo); opendot(A, blue); opendot(B, blue);
		}
		MP("\mathbb R", (3,0), dir(90));

		add(shift( (0,3) ) * CC());

		path darrow = (0,2.5)--(0,1.5);
		MP("p", midpoint(darrow), dir(0));
		draw(darrow, EndArrow);

		real r = 1.4;
		draw(scale(r)*unitcircle);
		MP("S^1", r*dir(45), dir(45));
		A = r*dir(-20);
		B = r*dir(20);
		draw(arc(origin, A, B), bloo);
		opendot(A, blue); opendot(B, blue);
		dot("$1$", r*dir(0), dir(0));
	\end{asy}
\end{center}

Note that not all open neighborhoods work this time:
notably, $U = S^1$ does not work because the pre-image
would be the entire $\RR$.

\begin{example}[Covering of $S^1$ by itself]
	The map $S^1 \to S^1$ by
	$z \mapsto z^{3}$ is also a covering projection.
	Can you see why?
\end{example}

\begin{example}
	[Covering projections of $\CC \setminus \{0\}$]
	For those comfortable with complex arithmetic,
	\begin{enumerate}[(a)]
		\ii The exponential map $\exp \colon \CC \to \CC \setminus \{0\}$
		is a covering projection.
		\ii For each $n$, the $n$th power map
		$-^n: \CC \setminus \{0\} \to \CC \setminus \{0\}$
		is a covering projection.
	\end{enumerate}
\end{example}

\section{Lifting theorem}
\label{sec:lifting_thm}
\prototype{$\RR$ covers $S^1$.}
Now here's the key idea: we are going to try to interpret
loops in $B$ as paths in $\RR$.
This is often much simpler.
For example, we had no idea how to compute the fundamental group of $S^1$,
but the fundamental group of $\RR$ is just the trivial group.
So if we can interpret loops in $S^1$ as paths in $\RR$,
that might (and indeed it does!)\ make computing $\pi_1(S^1)$ tractable.

\begin{definition}
	Let $\gamma \colon [0,1] \to B$ be a path and $p \colon E \to B$ a covering projection.
	A \vocab{lifting} of $\gamma$ is a path $\tilde\gamma \colon [0,1] \to E$
	such that $p \circ \tilde\gamma = \gamma$.
\end{definition}
Picture:
\begin{center}
\begin{tikzcd}
	& E \ar[d, "p"] \\
	{[0,1]} \ar[r, "\gamma"'] \ar[ru, "\tilde{\gamma}"] & B
\end{tikzcd}
\end{center}


\begin{example}[Typical example of lifting]
	Take $p \colon \RR \to S^1 \subseteq \CC$ by $\theta \mapsto e^{2 \pi i \theta}$
	(so $S^1$ is considered again as the unit circle).
	Consider the path $\gamma$ in $S^1$ which starts at $1 \in \CC$
	and wraps around $S^1$ once, counterclockwise, ending at $1$ again.
	In symbols, $\gamma \colon [0,1] \to S^1$ by $t \mapsto e^{2\pi i t}$.

	Then one lifting $\tilde\gamma$ is the path which walks from $0$ to $1$.
	In fact, \emph{for any integer $n$}, walking from $n$ to $n+1$ works.

	\begin{center}
		\begin{asy}
			size(6cm);

			real[] t = {-1,0,1,2};
			xaxis(-2,3, graph.LeftTicks(Ticks=t), Arrows);
			MP("\mathbb R", (2.5,0), dir(90));
			path gt = (0,0.3)--(1,0.3);
			draw(gt, blue, EndArrow);
			label("$\tilde\gamma$", midpoint(gt), dir(90), blue);
			add(shift( (0,3) ) * CC());

			path darrow = (0,2.5)--(0,1.5);
			MP("p", midpoint(darrow), dir(0));
			draw(darrow, EndArrow);

			real r = 1.2;
			draw(scale(r)*unitcircle);
			MP("S^1", r*dir(45), dir(45));
			dot("$1$", r*dir(0), dir(0));
			path g = dir(20)..dir(100)..dir(180)..dir(260)..dir(340);
			draw(g, red, EndArrow);
			label("$\gamma$", midpoint(g), -dir(midpoint(g)), red);

			MP("p(0) = 1", (2.5,0.5));
			MP("p(1) = 1", (2.5,0));
		\end{asy}
	\end{center}

	Similarly, the counterclockwise path from $1 \in S^1$ to $-1 \in S^1$
	has a lifting: for some integer $n$, the path from $n$ to $n+\half$.
	\label{ex:lifting_circle}
\end{example}

The above is the primary example of a lifting.
It seems like we have the following structure: given a path $\gamma$
in $B$ starting at $b_0$, we start at any point in the fiber $p\pre(b_0)$.
(In our prototypical example, $B = S^1$, $b_0 = 1 \in \CC$
and that's why we start at any integer $n$.)
After that we just trace along the path in $B$, and we get
a corresponding path in $E$.
\begin{ques}
	Take a path $\gamma$ in $S^1$ with $\gamma(0) = 1 \in \CC$.
	Convince yourself that once we select an integer $n \in \RR$,
	then there is exactly one lifting starting at $n$.
\end{ques}

It turns out this is true more generally.
\begin{theorem}[Lifting paths]
	Suppose $\gamma \colon [0,1] \to B$ is a path with $\gamma(0) = b_0$, and
	$ p \colon (E,e_0) \to (B,b_0) $
	is a covering projection.
	Then there exists a \emph{unique} lifting $\tilde\gamma \colon [0,1] \to E$
	such that $\tilde\gamma(0) = e_0$.
\end{theorem}
\begin{proof}
	For every point $b \in B$, consider an evenly covered
	open neighborhood $U_b$ in $B$.
	Then the family of open sets
	\[ \left\{ \gamma\pre(U_b) \mid b \in B \right\} \]
	is an open cover of $[0,1]$.
	As $[0,1]$ is compact we can take a finite subcover.
	Thus we can chop $[0,1]$ into finitely many interior-disjoint closed intervals
	$[0,1] = I_1 \sqcup I_2 \sqcup \dots \sqcup I_N$ in that order,
	such that for every $I_k$, $\gamma\im(I_k)$ is contained
	in some $U_b$.

	We'll construct $\tilde\gamma$ interval by interval now,
	starting at $I_1$.
	Initially, place a robot at $e_0 \in E$ and a mouse at $b_0 \in B$.
	For each interval $I_k$, the mouse moves around according
	to however $\gamma$ behaves on $I_k$.
	But the whole time it's in some evenly covered $U_k$;
	the fact that $p$ is a covering projection tells us that
	there are several copies of $U_k$ living in $E$.
	Exactly one of them, say $V_k$, contains our robot.
	So the robot just mimics the mouse until it gets to the end of $I_k$.
	Then the mouse is in some new evenly covered $U_{k+1}$,
	and we can repeat.
\end{proof}

The theorem can be generalized to a diagram
\begin{center}
\begin{tikzcd}
	& (E, e_0) \ar[d, "p"] \\
	(Y, y_0)  \ar[ru, "\tilde{f}"] \ar[r, "f"'] & (B, b_0)
\end{tikzcd}
\end{center}
where $Y$ is some general path-connected space, as follows.
\begin{theorem}[General lifting criterion]
	\label{thm:lifting}
	Let $f \colon (Y,y_0) \to (B, b_0)$ be continuous
	and consider a covering projection $p \colon (E, e_0) \to (B, b_0)$.
	(As usual, $Y$, $B$, $E$ are path-connected.)
	Then a lifting $\tilde f$ with $\tilde f(y_0) = e_0$ exists if and only if
	\[ f_\sharp\im(\pi_1(Y, y_0)) \subseteq p_\sharp\im(\pi_1(E, e_0)), \]
	i.e.\ the image of $\pi_1(Y, y_0)$ under $f$ is contained in
	the image of $\pi_1(E, e_0)$ under $p$ (both viewed as subgroups of $\pi_1(B, b_0)$).
	If this lifting exists, it is unique.
\end{theorem}
As $p_\sharp$ is injective,
we actually have $p_\sharp\im(\pi_1(E, e_0)) \cong \pi_1(E, e_0)$.
But in this case we are interested in the actual elements, not just the isomorphism classes of the groups.
\begin{ques}
	What happens if we put $Y= [0,1]$?
\end{ques}

\begin{remark}[Lifting homotopies]
	Here's another cool special case:
	Recall that a homotopy can be encoded as a continuous function $[0,1] \times [0,1] \to X$.
	But $[0,1] \times [0,1]$ is also simply connected.
	Hence given a homotopy $\gamma_1 \simeq \gamma_2$ in the base space $B$, we can lift it to get
	a homotopy $\tilde\gamma_1 \simeq \tilde\gamma_2$ in $E$.
\end{remark}
Another nice application of this result is \Cref{ch:complex_log}.

\section{Lifting correspondence}
\prototype{$(\RR,0)$ covers $(S^1,1)$.}
Let's return to the task of computing fundamental groups.
Consider a covering projection $p \colon (E, e_0) \to (B, b_0)$.

A loop $\gamma$ can be lifted uniquely to $\tilde\gamma$ in $E$
which starts at $e_0$ and ends at some point $e$ in the fiber $p\pre(b_0)$.
You can easily check that this $e \in E$ does not change if we
pick a different path $\gamma'$ homotopic to $\gamma$.
\begin{ques}
	Look at the picture in \Cref{ex:lifting_circle}.

	Put one finger at $1 \in S^1$, and one finger on $0 \in \RR$.
	Trace a loop homotopic to $\gamma$ in $S^1$ (meaning, you can
	go backwards and forwards but you must end with exactly one full
	counterclockwise rotation)
	and follow along with the other finger in $\RR$.

	Convince yourself that you have to end at the point $1 \in \RR$.
\end{ques}

Thus every homotopy class of a loop at $b_0$ (i.e.\ an element of $\pi_1(B, b_0)$) can be associated with some $e$ in the fiber of $b_0$.
The below proposition summarizes this and more.
\begin{proposition}
	Let $p \colon (E,e_0) \to (B,b_0)$ be a covering projection.
	Then we have a function of sets
	\[ \Phi \colon \pi_1(B, b_0) \to p\pre(b_0) \]
	by $[\gamma] \mapsto \tilde\gamma(1)$, where $\tilde\gamma$
	is the unique lifting starting at $e_0$.
	Furthermore,
	\begin{itemize}
		\ii If $E$ is path-connected, then $\Phi$ is surjective.
		\ii If $E$ is simply connected, then $\Phi$ is injective.
	\end{itemize}
\end{proposition}
\begin{ques}
	Prove that $E$ path-connected implies $\Phi$ is surjective.
	(This is really offensively easy.)
\end{ques}
\begin{proof}
	To prove the proposition, we've done everything except show
	that $E$ simply connected implies $\Phi$ injective.
	To do this suppose that $\gamma_1$ and $\gamma_2$ are loops
	such that $\Phi([\gamma_1]) = \Phi([\gamma_2])$.

	Applying lifting, we get paths $\tilde\gamma_1$ and $\tilde\gamma_2$
	both starting at some point $e_0 \in E$ and ending at some point $e_1 \in E$.
	Since $E$ is simply connected that means they are \emph{homotopic},
	and we can write a homotopy $F \colon [0,1] \times [0,1] \to E$
	which unites them.
	But then consider the composition of maps
	\[ [0,1] \times [0,1] \taking{F} E \taking{p} B. \]
	You can check this is a homotopy from $\gamma_1$ to $\gamma_2$.
	Hence $[\gamma_1] = [\gamma_2]$, done.
\end{proof}

This motivates:
\begin{definition}
	A \vocab{universal cover} of a space $B$ is a covering projection
	$p \colon E \to B$ where $E$ is simply connected (and in particular path-connected).
\end{definition}
\begin{abuse}
	When $p$ is understood, we sometimes just say $E$ is the universal cover.
\end{abuse}

\begin{example}[Fundamental group of $S^1$]
	Let's return to our standard $p \colon \RR \to S^1$.
	Since $\RR$ is simply connected, this is a universal cover of $S^1$.
	And indeed, the fiber of any point in $S^1$
	is a copy of the integers: naturally in bijection with loops in $S^1$.

	You can show (and it's intuitively obvious) that the bijection
	\[ \Phi \colon \pi_1(S^1) \leftrightarrow \ZZ \]
	is in fact a group homomorphism if we equip $\ZZ$ with its
	additive group structure $\ZZ$.
	Since it's a bijection, this leads us to conclude $\pi_1(S^1) \cong \ZZ$.
\end{example}

\section{Regular coverings}
\prototype{$\RR \to S^1$ comes from $n \cdot x = n + x$}

Here's another way to generate some coverings.
Let $X$ be a topological space and $G$ a group acting on its points.
Thus for every $g$, we get a map $X \to X$ by
\[ x \mapsto g \cdot x. \]
We require that this map is continuous\footnote{%
	Another way of phrasing this: the action,
	interpreted as a map $G \times X \to X$, should be continuous,
	where $G$ on the left-hand side is interpreted as a set with
	the discrete topology.}
for every $g \in G$, and that the stabilizer of each point in $X$ is trivial.
Then we can consider a quotient space $X/G$ defined by fusing any points
in the same orbit of this action.
Thus the points of $X/G$ are identified with the orbits of the action.
Then we get a natural ``projection''
\[ X \to X/G \]
by simply sending every point to the orbit it lives in.
\begin{definition}
	Such a projection is called \vocab{regular}.
	(Terrible, I know.)
\end{definition}

\begin{example}[$\RR \to S^1$ is regular]
	Let $G = \ZZ$, $X = \RR$
	and define the group action of $G$ on $X$ by
	\[ n \cdot x = n + x \]
	You can then think of $X/G$ as ``real numbers modulo $1$'',
	with $[0,1)$ a complete set of representatives and $0 \sim 1$. % chktex 9
	\begin{center}
		\begin{asy}
			size(9cm);
			dotfactor *= 2;
			pair A = MP("0", (-5.1,0), 1.4*dir(90));
			pair B = MP("1", (-3,0), 1.4*dir(90));
			draw(A--B);
			Drawing("\frac13", (-4.4,0), 1.4*dir(90));
			Drawing("\frac23", (-3.7,0), 1.4*dir(90));
			MP("\mathbb R / G", (-4,-0.6), dir(-90));
			dot(A); opendot(B);
			draw(unitcircle);
			draw( (-2.4,0)--(-1.6,0), EndArrow);
			dot("$0=1$", dir(0), dir(0));
			dot("$\frac13$", dir(120), dir(120));
			dot("$\frac23$", dir(240), dir(240));
			label("$S^1$", origin, origin);
		\end{asy}
	\end{center}
	So we can identify $X/G$ with $S^1$
	and the associated regular projection
	is just our usual $\exp \colon \theta \mapsto e^{2i\pi \theta}$.
\end{example}

\begin{example}[The torus]
	Let $G = \ZZ \times \ZZ$ and $X = \RR^2$,
	and define the group action of $G$ on $X$ by $(m,n) \cdot (x,y)
	= (m+x, n+y)$.
	As $[0,1)^2$ is a complete set of representatives, % chktex 9
	you can think of it as a unit square with the edges identified.
	We obtain the torus $S^1 \times S^1$
	and a covering projection $\RR^2 \to S^1 \times S^1$.
\end{example}

\begin{example}[$\mathbb {RP}^2$]
	Let $G = \Zc 2 = \left<T \mid T^2 = 1\right>$ and
	let $X = S^2$ be the surface of the sphere,
	viewed as a subset of $\RR^3$.
	We'll let $G$ act on $X$ by sending $T \cdot \vec x = - \vec x$;
	hence the orbits are pairs of opposite points (e.g.\ North and South pole).

	Let's draw a picture of a space.
	All the orbits have size two:
	every point below the equator gets fused with a point above the equator.
	As for the points on the equator, we can take half of them; the other half
	gets fused with the corresponding antipodes.

	Now if we flatten everything,
	you can think of the result as a disk with half its boundary:
	this is $\RP^2$ from before.
	The resulting space has a name: \emph{real projective $2$-space},
	denoted $\mathbb{RP}^2$.
	\begin{center}
		\begin{asy}
			size(3cm);
			dotfactor *= 2;
			draw(dir(-90)..dir(0)..dir(90));
			draw(dir(90)..dir(180)..dir(-90), dashed);
			fill(unitcircle, yellow+opacity(0.2));
			dot(dir(90));
			opendot(dir(-90));
			label("$\mathbb{RP}^2$", origin, origin);
		\end{asy}
	\end{center}

	This gives us a covering projection $S^2 \to \mathbb{RP}^2$
	(note that the pre-image of a sufficiently small patch is just two copies
	of it on $S^2$.)
\end{example}
\begin{example}
	[Fundamental group of $\mathbb{RP}^2$]
	As above, we saw that there was a covering projection
	$S^2 \to \mathbb{RP}^2$.
	Moreover the fiber of any point has size two.
	Since $S^2$ is simply connected, we have a natural bijection
	$\pi_1(\mathbb{RP}^2)$ to a set of size two; that is,
	\[ \left\lvert \pi_1(\mathbb{RP}^2) \right\rvert = 2. \]
	This can only occur if $\pi_1(\mathbb{RP}^2) \cong \Zc 2$,
	as there is only one group of order two!
\end{example}

\begin{ques}
	Show each of the continuous maps $x \mapsto g \cdot x$ is in fact a homeomorphism.
	(Name its continuous inverse).
\end{ques}
% WOW I thought this was always a covering projection gg

\section{The algebra of fundamental groups}
\prototype{$S^1$, with fundamental group $\ZZ$.}
Next up, we're going to turn functions between spaces into homomorphisms of fundamental groups.

Let $X$ and $Y$ be topological spaces and $f \colon (X, x_0) \to (Y, y_0)$.
Recall that we defined a group homomorphism
\[ f_\sharp \colon \pi_1(X, x_0) \to \pi_1(Y_0, y_0)
	\quad\text{by}\quad
	[\gamma] \mapsto [f \circ \gamma]. \]
% which gave us a functor $\catname{Top}_\ast \to \catname{Grp}$.

More importantly, we have:
\begin{proposition}
	Let $p \colon (E,e_0) \to (B,b_0)$ be a covering projection of path-connected spaces.
	Then the homomorphism $p_\sharp \colon \pi_1(E, e_0) \to \pi_1(B, b_0)$ is \emph{injective}.
	Hence $p_\sharp \im(\pi_1(E, e_0))$ is an isomorphic copy of $\pi_1(E, e_0)$
	as a subgroup of $\pi_1(B, b_0)$.
\end{proposition}
\begin{proof}
	We'll show $\ker p_\sharp$ is trivial.
	It suffices to show if $\gamma$ is a nulhomotopic loop in $B$
	then its lift is nulhomotopic.

	By definition, there's a homotopy $F \colon [0,1] \times [0,1] \to B$
	taking $\gamma$ to the constant loop $1_B$.
	We can lift it to a homotopy $\tilde F \colon [0,1] \times [0,1] \to E$
	that establishes $\tilde\gamma \simeq \tilde 1_B$.
	But $1_E$ is a lift of $1_B$ (duh) and lifts are unique.
\end{proof}

\begin{example}[Subgroups of $\ZZ$]
	Let's look at the space $S^1$ with fundamental group $\ZZ$.
	The group $\ZZ$ has two types of subgroups:
	\begin{itemize}
		\ii The trivial subgroup.
		This corresponds to the canonical projection $\RR \to S^1$,
		since $\pi_1(\RR)$ is the trivial group ($\RR$ is simply connected)
		and hence its image in $\ZZ$ is the trivial group.
		\ii $n\ZZ$ for $n \ge 1$.
		This is given by the covering projection $S^1 \to S^1$
		by $z \mapsto z^n$.
		The image of a loop in the covering $S^1$ is a ``multiple of $n$''
		in the base $S^1$.
	\end{itemize}
\end{example}

It turns out that these are the \emph{only} covering projections of $S^n$ by path-connected spaces: there's one for each subgroup of $\ZZ$.
(We don't care about disconnected spaces because, again, a covering projection
via disconnected spaces is just a bunch of unrelated ``good'' coverings.)
For this statement to make sense I need to tell you what it means for
two covering projections to be equivalent.

\begin{definition}
	Fix a space $B$.
	Given two covering projections $p_1 \colon E_1 \to B$ and $p_2 \colon E_2 \to B$
	a \vocab{map of covering projections} is a continuous function $f \colon E_1 \to E_2$
	such that $p_2 \circ f = p_1$.
	\begin{center}
	\begin{tikzcd}
		E_1 \ar[r, "f"] \ar[rd, "p_1"'] & E_2 \ar[d, "p_2"] \\
		& B
	\end{tikzcd}
	\end{center}
	Then two covering projections $p_1$ and $p_2$ are isomorphic if there are
	$f \colon E_1 \to E_2$ and $g \colon E_2 \to E_1$
	such that $f \circ g = \id_{E_1}$ and $g \circ f = \id_{E_2}$.
\end{definition}
\begin{remark}
	[For category theorists]
	The set of covering projections forms a category in this way.
\end{remark}

It's an absolute miracle that this is true more generally:
the greatest triumph of covering spaces is the following result.
Suppose a space $X$ satisfies some nice conditions, like:
\begin{definition}
	A space $X$ is called \vocab{locally connected}
	if for each point $x \in X$ and open neighborhood $V$ of it,
	there is a connected open set $U$ with $x \in U \subseteq V$.
\end{definition}
\begin{definition}
	A space $X$ is \vocab{semi-locally simply connected}
	if for every point $x \in X$
	there is an open neighborhood $U$
	such that all loops in $U$ are nulhomotopic.
	(But the contraction need not take place in $U$.)
\end{definition}
\begin{example}[These conditions are weak]
	Pretty much every space I've shown you has these two properties.
	In other words, they are rather mild conditions, and you can think of them as just
	saying ``the space is not too pathological''.
\end{example}
Then we get:
\begin{theorem}[Group theory via covering spaces]
	Suppose $B$ is a locally connected, semi-locally simply connected space.
	Then:
	\begin{itemize}
		\ii Every subgroup $H \subseteq \pi_1(B)$ corresponds
		to exactly one covering projection $p \colon E \to B$
		with $E$ path-connected (up to isomorphism).

		(Specifically, $H$ is the image of $\pi_1(E)$ in $\pi_1(B)$ through $p_\sharp$.)
		\ii Moreover, the \emph{normal} subgroups of $\pi_1(B)$
		correspond exactly to the regular covering projections.
	\end{itemize}
\end{theorem}
Hence it's possible to understand the group theory of $\pi_1(B)$ completely
in terms of the covering projections.

Moreover, this is how the ``universal cover'' gets its name:
it is the one corresponding to the trivial subgroup of $\pi_1(B)$.
Actually, you can show that it really is universal in the sense
that if $p \colon E \to B$ is another covering projection,
then $E$ is in turn covered by the universal space.
More generally, if $H_1 \subseteq H_2 \subseteq G$ are subgroups,
then the space corresponding to $H_2$ can be covered by the space
corresponding to $H_1$.

% According to \cite{ref:covering_all_we_know}, this statement and
% its extension to group actions are ``pretty much all there is to know
% about covering projections''.

\section{\problemhead}
\todo{problems}
