\addtocounter{chapter}{-1}
\chapter{Read This First}
\section{Graph of Chapter Dependencies}
There is no need to read this book in linear order.
Here is a plot of where the various chapters are relative to each other.
In my opinion, the really cool stuff is the later chapters,
which have been bolded below.
Dependencies are indicated by arrows;
dotted lines are optional dependencies.
I suggest that you simply pick a chapter you find interesting,
and then find the shortest path.
With that in mind, I hope the length of the entire PDF is not intimidating.

Many of the later parts are subdivided into ``Part I'' and ``Part II''.
In general, the second part is substantially harder than the first part.
If you intend to read a large portion of the text,
one possible thing to do is to read all the Part I's
before reading all the Part II's.

I have tried to keep the chapters short, and no more than 15 pages each.
The original intention was that each chapter could be digested
(albeit with some effort) in a couple of hours,
true to the subtitle of ``bedtime stories''.
Unfortunately, I think in many of the Part II's this is no longer true.

%The oddball in this graph is the category theory.
%Basically, the more examples from other chapters you know,
%the easier this chapter will be to digest.
%But \emph{strictly} speaking, you don't need any prerequisites at all;
%the subject is very ``combinatorial''.

%\chapter*{Graph of Chapter Dependencies}
%\addcontentsline{toc}{chapter}{Graph of Chapter Dependencies}

\setcounter{diagheight}{50}
\begin{chart}
\halfcourse 10,45:{Chapters 1,3}{Groups}{}
\halfcourse 48,45:{Chapter 2}{Spaces}{}
\halfcourse 48,35:{Chapter 4,5}{Topology}{}
\halfcourse 63,45:{Chapter 6}{Modules}{}
\halfcourse 63,35:{Chapters 7-9}{Lin Alg}{}
\halfcourse 65,25:{Chapters 10-11}{Diff Geom}{}
\halfcourse 5,35:{Chapter 12}{Grp Actions}{}
\halfcourse 5,20:{Chapter 13}{Grp Classif}{}
\halfcourse 30,35:{Chapter 14-15}{Ideals}{}
\reqhalfcourse 55,10:{Chapters 16-18}{Cmplx Ana}{}
\reqhalfcourse 30,10:{Chapter 19-21}{Alg Top 1}{}
\reqhalfcourse 30,22:{Chapter 22-24}{Cat Th}{}
\reqhalfcourse 20,0:{Chapters 25-29}{Alg Top 2}{}
\reqhalfcourse 42,10:{Chapter 30-34}{Alg NT 1}{}
\reqhalfcourse 42,0:{Chapter 35-38}{Alg NT 2}{}
\reqhalfcourse 48,25:{Chapter 39-42}{Rep Theory}{}
\reqhalfcourse 16,10:{Chapters 43-46}{Alg Geom 1}{}
\reqhalfcourse 6,0:{Chapters 47-48}{Alg Geom 2}{}
\reqhalfcourse 58,19:{Chapters 51-53}{Quantum}{}
\reqhalfcourse 30,45:{Chapters 54-59}{Set Theory}{}

\prereqc 30,35,48,25;20:  % Ideals -> Rep Th
\prereqc 63,35,48,25;0:   % Lin Alg -> Rep Th
\coreqc  30,22,48,25;0:   % Cats -> Rep Th
\prereqc 63,35,58,19;30:  % Lin Alg -> Quantum
\prereqc 48,45,65,25;-20: % Spaces -> Diff Geom

\prereqc 10,45,30,22;0:   % Gp -> Cat Th
\coreqc  30,35,30,22;0:   % Ideals -> Cat Th
\coreqc  48,45,30,22;0:   % Space -> Cat Th
\coreqc  63,35,30,22;10:  % Lin Alg -> Cat Th
\prereqc 63,35,65,25;0:   % Lin Alg -> Diff Geo
\coreqc  48,35,30,22;-10: % Top -> Cat Th
\coreqc  30,22,16,10;0:   % Cat Th -> AG2
\prereqc 63,45,30,35;0:   % Module -> Ideal

\prereqc 30,10,20,0;0:    % AT1 -> AT2
\prereqc 30,22,20,0;20:   % Cat -> AT2
\coreqc  30,22,30,10;0:   % Cat -> AT1

\prereqc 10,45,5,35;0:    % Grp -> Grp Act
\prereq   5,35,5,20:      % Grp Act -> Grp Class
\prereq  48,45,48,35:     % Space -> Top
\prereqc 48,35,30,10;0:   % Top -> AT1
\prereqc  5,35,30,10;0:   % Grp Act -> AT1
\prereqc 48,35,55,10;-30: % Top -> Cmplx Ana
\coreqc  10,45,30,35;0:   % Grp -> Ideals
\prereqc 30,35,42,10;0:   % Ideals -> ANT1
\prereq  63,45,63,35:     % Modules -> Linalg
\prereq  63,45,42,10:     % Modules -> ANT1
\coreqc  63,35,42,10;0:   % Lin Alg -> ANT1
\prereqc 30,35,16,10;0:   % Ideals -> AG1
\prereqc 48,35,16,10;-30: % Top -> AG1
\prereqc 42,10,42,0;0:    % ANT1 -> ANT2
\prereqc 16,10,6,0;0:     % AG1 -> AG2
\prereqc 30,22,6,0;80:     % Cat -> AG2
\prereqc  5,35,42,0;-30:  % Grp Act -> ANT2
\end{chart}

\eject

\section{Questions, Exercises, and Problems}
%Most textbooks have large numbers of exercises, some easy and some hard.
In this book, there are three hierarchies:
\begin{itemize}
	\ii A \emph{question} is intended to be offensively easy.
	It is usually just a chance to help you internalize definitions.
	You really ought to answer almost all of these.
	\ii An \emph{exercise} is marginally harder.
	The difficulty is like something you might see on an AIME.
	Often I leave proofs of theorems and propositions as exercises
	if they are instructive and at least somewhat interesting.
	\ii Each chapter features several \emph{problems} at the end.
	Some of these are easy, while others are legitimately difficulty olympiad-style problems.
	There are three types:
	\begin{itemize}
		\ii \textbf{Normal problems}, which are hopefully fun but non-central.
		\ii \textbf{Daggered problems}, which are (usually interesting) results that one should know,
		but won't be used directly later.
		\ii \textbf{Starred problems}, indicating results which will be used later on.\footnote{%
			A big nuisance in college for me was that, every so often, the professor
			would invoke some result from the homework.
			In fact, sometimes crucial results were placed only on the homework.
			% Thus every time during lecture or whatever, I would have to refer back through my
			% mound of problem sets to dig out whatever said result was, and often I would
			% then have to remember the proof of it (which I had long forgotten).
			I wish they could have told me in advance ``please take note of this, we'll use it later on''.
		}
	\end{itemize}
\end{itemize}

\begin{center}
	\includegraphics[width=14cm]{media/abstruse-goose-exercise.png}
\end{center}

I personally find most exercises to not be that interesting, and I've tried to keep boring ones to a minimum.
Regardless, I've tried hard to pick problems that are fun to think about and, when possible, to give them
the kind of flavor you might find on the IMO or Putnam (even when the underlying background is different).

\gim
Harder problems are marked with \chili's, like this paragraph.
For problems that have three chili's you should probably read the hint.

\section{Paper}
At the risk of being blunt,
\begin{moral}
Read this book with pencil and paper.
\end{moral}
Here's why:

\begin{center}
	\includegraphics[width=0.5\textwidth]{media/read-with-pencil.jpg} \\
	{\footnotesize Source: \url{http://mathwithbaddrawings.com/2015/03/17/the-math-major-who-never-reads-math/}}
\end{center}
\textbf{You are not God.}
You cannot keep everything in your head.\footnote{
	See also \url{https://usamo.wordpress.com/2015/03/14/writing/} and the source above.
}
If you've printed out a hard copy, then write in the margins.
If you're trying to save paper, grab a notebook or something along with the ride.
Somehow, someway, make sure you can write. Thanks.


\section{Examples}
I am pathologically obsessed with examples.
In this book, I place all examples in large boxes to draw emphasis to them, which leads to some pages of the book simply consisting of sequences of boxes one after another. I hope the reader doesn't mind.

I also often highlight a ``prototypical example'' for some sections,
and reserve the color red for such a note.
The philosophy is that any time the reader sees a definition or a theorem about such an object, they should test it against the prototypical example.
If the example is a good prototype, it should be immediately clear why this definition is intuitive, or why the theorem should be true, or why the theorem is interesting, et cetera.

Let me tell you a secret.  When I wrote a definition or a theorem in this book, I would have to recall the exact statement from my memory, and my memory is very poor. So instead, I have to consider the prototypical example, and then only after that do I remember what the definition or the theorem is.
Incidentally, this is also how I learned all the definitions in the first place.
I hope you'll find it useful as well.

\section{SNSD}
Where appropriate, I have added math jokes from the
Tumblr \emph{Topological Girl's Generation}.
I hope you find them enjoyable.

\section{Topic Choices}
The appendix contains a list of resources I like and thoughts about
particular chapters and subjects and the like.
I encourage you to check it out.

The most important part of that is the first section.
For example, I learned category theory from \cite{ref:msci} and loved it,
and I don't think I could do a better job than it,
which is why I only barely touch on category theory here!

\section{Conventions and Notations}
This part describes some of the less familiar notations and definitions
and settles for once and for all some annoying issues (``is zero a natural number?'').

\subsection*{Sets and Equivalence Relations}
%\begin{definition}
%We denote by $\ZZ$, $\QQ$, $\RR$, and $\CC$ the set of integers,
%rational numbers, real numbers, and complex numbers, respectively.
%\end{definition}
$\NN$ is the set of \emph{positive} integers, not including $0$.

%\begin{definition}
%For sets $S$ and $T$, $S \setminus T$ is set subtraction.
%Also, $S \times T$ is the Cartesian product.
%\end{definition}

An \vocab{equivalence relation} on a set $X$ is a relation $\sim$
which is symmetric, reflexive, and transitive.
A equivalence relation partitions $X$ into several \vocab{equivalence classes}.
We will denote this by $X / {\sim}$.
An element of such an equivalence class is a \vocab{representative} of that equivalence class.


\subsection*{Functions}
Let $X \taking f Y$ be a function.

\begin{itemize}
\ii By $f\pre(T)$ I mean the \vocab{pre-image}
\[ f\pre(T) \defeq \left\{ x \in X \mid f(x) \in T \right\} \]
in contrast to the usual $f\inv(T)$; I only use $f\inv$ for an inverse \emph{function}.

By abuse of notation, we may abbreviate $f\pre(\{y\})$ to $f\pre(y)$.
We call $f\pre(y)$ a \vocab{fiber}.

\ii By $f``(S)$ I mean the \vocab{image}
\[ f``(S) \defeq \left\{ f(x) \mid x \in S \right\}. \]
The notation {``} is from set theory, and is meant to indicate ``point-wise''.
Most authors use $f(S)$, but this is abuse of notation,
and I prefer $f``(S)$ for emphasis.

\ii Sometimes functions $f : X \to Y$ are \emph{injective} or \emph{surjective};
I may emphasize this sometimes by writing $f : X \injto Y$ or $f : X \surjto Y$, respectively.
\end{itemize}

\subsection*{Rings and Vector Spaces}
All rings are commutative with a multiplicative identity $1$ unless otherwise specified.
We allow $0=1$ in general rings but not in integral domains.

Vector spaces are assumed to be finite-dimensional unless otherwise specified.

\subsection*{Choice}
We accept the Axiom of Choice.
