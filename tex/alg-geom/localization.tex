\chapter{Localization}
\todo{change $R$ to $A$}
Before we proceed on to defining an affine scheme,
we will take the time to properly cover one more algebraic construction
that of a \emph{localization}.
This is mandatory because when we define a scheme,
we will find that all the sections and stalks
are actually obtained using this construction.

You may remember that when we were working with affine varieties,
there were constantly expressions of the form
\[ \left\{ \frac{f}{g} \mid g(p) \ne 0 \right\} \]
and the like.
The point is that we introduced a lot of denominators.
This is exactly the idea behind the construction we are about to see.

\section{The definition}
\begin{definition}
	Let $S \subseteq R$, where $R$ is a ring,
	and assume $S$ is closed under multiplication.
	Then the \vocab{localization of $R$ at $S$}, denoted $S\inv R$,
	is defined as the set of fractions
	\[ \left\{ r/s \mid r \in R, s \in S \right\} \]
	where we declare two fractions $r_1 / s_1 = r_2 / s_2$ 
	to be equal if
	\[ \exists s \in S : \quad s(r_1s_2 - r_2s_1) = 0. \]
\end{definition}

\begin{example}
	
\end{example}
\todo{give examples}

In particular, if $0 \in S$ then $S\inv R$ is the trivial ring.
So we usually only take situations where $0 \notin S$.
\begin{ques}
	Assume $R$ is an integral domain and $S = R \setminus \{0\}$.
	Show that $S\inv R$ is just the field of fractions.
\end{ques}
\begin{example}
	[Why the extra $s$?]
	The reason we need the condition $s(r_1s_2 - r_2s_1) = 0$
	rather than the simpler $r_1s_2 - r_2s_1 = 0$ is that
	otherwise the equivalence relation may fail to be transitive.
	Here is a counterexample: take
	\[ R = \Zc{12} \qquad S = \{ 2, 4, 8 \}. \]
	Then we have for example
	\[ \frac12 = \frac24 = \frac64 = \frac32. \]
	So we need to have $\frac12=\frac32$ which is only true
	with the first definition.

	Of course, if $R$ is an integral domain (and $0\notin S$)
	then this is a moot point.
\end{example}

We will now look at two special cases,
which have a concrete interpretation.

\todo{ideals bijection}

\section{First important special case: localization at a prime}
\begin{definition}
	Let $R$ be a ring and $\pp$ a prime ideal.
	Then $R_\pp$ is defined to be $S\inv R$ for $S = R\setminus\pp$.
	We call this \vocab{localization at $\pp$}.
	Addition is defined in the obvious way.
\end{definition}

\begin{ques}
	Why is $S$ multiplicative closed in the above definition?
\end{ques}

\todo{example: localizing the origin of $\Aff^1$}
\todo{example: localizing the origin of $\Aff^2$}
\todo{example: localizing at a hyperbola}
\todo{example: localizing at $(0)$}

\section{Second important special case: localization away from an element}
\begin{definition}
	For $g \in R$, we define the \vocab{localization of $R$ at $g$},
	denoted $R_g$ to be $\{1, g, g^2, g^3, \dots\}\inv R$.
	(Note that $\left\{ 1, g, g^2, \dots \right\}$ is multiplicatively closed.)
\end{definition}
Warning: this is admittedly somewhat terrible notation, since $R_\pp$
is the localization with multiplicative set $R \setminus \pp$,
while $R_g$ is the localization with multiplicative set $\{1,g,g^2,\dots\}$;
these two are quite different beasts!
In $R_\pp$ the subscript denotes the \emph{forbidden} denominators;
in $R_g$ the subscript denotes the \emph{allowed} denominators.

\begin{example}
	[Localization at an element]
	Let $R = \CC[x,y,z]$ and let $g = x$.
	Then
	\[ R_g = \left\{ \frac{P(x,y,z)}{x^n} \mid
		P \in \CC[x,y,z], \; n \ge 0 \right\}. \]
\end{example}

\todo{example: punctured line}
\todo{example: line minus a parabola}

\section{Local rings, and locally ringed spaces}

The stalks of the examples we produced above are special types
of rings, called \emph{local rings}.
Algebraically, the definition of these is:
\begin{definition}
	A \vocab{local ring} $R$ is a ring with exactly one maximal ideal.
\end{definition}
\begin{exercise}
	Show that a ring $R$ is a local ring if there exists
	a proper ideal $\mm \subsetneq R$ such that
	all elements of $R \setminus \mm$ are units.
	(Bonus: prove the converse.)
\end{exercise}

% Wikipedia has a good explanation here
To see why this definition applies to the stalks above,
we need to identify what the maximal ideal is.
Let's go back to the example of $X = \RR$ and $\SF(U)$ the smooth functions,
and consider the stalk $\SF_{p}$, where $p \in X$.
Define the ideal $\mm_p$ to be the set of germs $(s,U)$ for which $s(p) = 0$.

Then $\mm_p$ is maximal: we have an exact sequence
\[ 0 \to \mm_p \to \SF_p \taking{(s,U) \mapsto s(p)} \RR \to 0 \]
and so $\SF_p / \mm_p \cong \RR$, which is a field.

It remains to check there are no nonzero maximal ideals.
Now note that if $s \notin \mm_p$,
then $s$ is nonzero in some neighborhood of $p$,
then one can construct the function $1/s$ in a neighborhood of $p$.
So \textbf{every element of $\SF_p \setminus \mm_p$ is a unit};
$\mm_p$ is in fact the only maximal ideal!

More generally,
\begin{moral}
	If $\SF$ consists of ``field-valued functions'',
	the stalk $\SF_p$ probably has a maximal ideal
	consisting of the germs vanishing at $p$.
\end{moral}
The discussion above implies, for example:
\begin{proposition}
	[Stalks are often local rings]
	The stalks of each of the following types of sheaves are local rings:
	\begin{enumerate}[(a)]
		\ii Sheaves of continuous real/complex functions on a topological space
		\ii Sheaves of smooth functions on any manifold
		\ii Regular functions on an algebraic variety $V$.
	\end{enumerate}
\end{proposition}

We can now define:
\begin{definition}
	A \vocab{ringed space} is a topological space $X$ equipped
	with a sheaf $\OO_X$ of rings.
	Suppose that for every point $p$, the stalk $\OO_{X,p}$
	is a local ring.
	Then we say that $(X, \OO_X)$ is a \vocab{locally ringed space}.
	We denote the maximal ideals by $\mm_{X,p}$.
\end{definition}

In particular, in the previous chapter we showed that every
affine variety could be built into a locally ringed space. Hooray!
\begin{abuse}
	A ringed space $(X, \OO_X)$ is abbreviated to just $X$,
	while $p \in X$ means ``$p$ is in the topological space $X$''.
\end{abuse}



\section{A valuable square}

\section{\problemhead}
