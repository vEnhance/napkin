\chapter{Affine Varieties as Ringed Spaces}
As in the previous chapter, we are working only over affine varieties in $\CC$ for simplicity.

\section{Synopsis}
Group theory was a strange creature in the early 19th century.
During the 19th century, a group was literally defined
as a subset of $\GL(n)$ or of $S_n$.
Indeed, the word ``group'' hadn't been invented yet.
This may sound ludicrous, but it was true -- Sylow developed his theorems without this notion.
Only much later did the abstract definition of a group was given,
an abstract set $G$ which was independent of any \emph{embedding} into $S_n$,
and an object in its own right.

We are about to make the same type of change for our affine varieties.
Rather than thinking of them as an object locked into an ambient space $\Aff^n$
we are instead going to try to make them into an object in their own right.
Specifically, for us an affine variety will become a \emph{topological space}
equipped with a \emph{ring of functions} for each of its open sets:
this is why we call it a \vocab{ringed space}.

The bit about the topological space is not too drastic.
The key insight is the addition of the ring of functions.
For example, consider the double point from last chapter.

\begin{center}
	\begin{asy}
		import graph;
		size(5cm);

		real f(real x) { return x*x; }
		graph.xaxis("$x$", red);
		graph.yaxis("$y$");
		draw(graph(f,-2,2,operator ..), blue, Arrows);
		label("$\mathcal V(y-x^2)$", (0.8, f(0.8)), dir(-45), blue);
		label("$\mathbb A^2$", (2,3), dir(45));
		dotfactor *= 1.5;
		dot(origin, heavygreen);
	\end{asy}
\end{center}

As a set, it is a single point,
and thus it can have only one possible topology.
But the addition of the function ring will let us tell it apart
from just a single point.

This construction is quite involved, so we'll proceed as follows:
we'll define the structure bit by bit onto our existing affine varieties in $\Aff^n$,
until we have all the data of a ringed space.
In later chapters, these ideas will grow up to
become the core of modern algebraic geometry: the \emph{scheme}.

\section{The Zariski Topology on $\Aff^n$}
\prototype{In $\Aff^1$, closed sets are finite collections of points.
In $\Aff^2$, a nonempty open set is the whole space minus some finite collection of curves/points.}

We begin by endowing a topological structure on every variety $V$.
Since our affine varieties (for now) all live in $\Aff^n$, all we have to do
is put a suitable topology on $\Aff^n$, and then just view $V$ as a subspace.

However, rather than putting the standard Euclidean topology on $\Aff^n$,
we put a much more bizarre topology.
\begin{definition}
	In the \vocab{Zariski topology} on $\Aff^n$,
	the \emph{closed sets} are those of the form 
	\[ \VV(I) \qquad\text{where}\quad I \subset \CC[x_1, \dots, x_n]. \]
	Of course, the open sets are complements of such sets.
\end{definition}

\begin{example}
	[Zariski Topology on $\Aff^1$]
	Let us determine the open sets of $\Aff^1$,
	which as usual we picture as a straight line
	(ignoring the fact that $\CC$ is two-dimensional).

	Since $\CC[x]$ is a principal ideal domain, rather than looking at $\VV(I)$
	for every $I \subseteq \CC[x]$, we just have to look at $\VV(f)$ for a single $f$.
	There are a few flavors of polynomials $f$:
	\begin{itemize}
		\ii The zero polynomial $0$ which vanishes everywhere:
		this implies that the entire space $\Aff^1$ is a closed set.
		\ii The constant polynomial $1$ which vanishes nowhere.
		This implies that $\varnothing$ is a closed set.
		\ii A polynomial $c(x-t_1)(x-t_2)\dots(x-t_n)$ of degree $n$.
		It has $n$ roots, and so $\{t_1, \dots, t_n\}$ is a closed set.
	\end{itemize}
	Hence the closed sets of $\Aff^1$ are exactly all of $\Aff^1$
	and finite sets of points (including $\varnothing$).
	Consequently, the \emph{open} sets of $\Aff^1$ are
	\begin{itemize}
		\ii $\varnothing$, and
		\ii $\Aff^1$ minus a finite collection (possibly empty) of points.
	\end{itemize}
\end{example}

Thus, the picture of a ``typical'' open set $\Aff^1$ might be 
\begin{center}
	\begin{asy}
		size(6cm);
		pair A = (-9,0); pair B = (9,0);
		pen bloo = blue+1.5;
		draw(A--B, blue, Arrows);
		draw(A--B, bloo);
		// label("$\mathbb V()$", (0,0), 2*dir(-90));
		opendot((-3,0), bloo);
		opendot((-1,0), bloo);
		opendot((4,0), bloo);
		label("$\mathbb A^1$", B-(2,0), dir(90));
	\end{asy}
\end{center}
It's everything except a few marked points!

\begin{example}[Zariski Topology on $\Aff^2$]
	Similarly, in $\Aff^2$, the interesting closed sets are going to consist of finite unions
	(possibly empty) of the following:
	\begin{itemize}
		\ii Closed curves, like $\VV(y-x^2)$ (which is a parabola).
		\ii Single points, like $\VV(x-3,y-4)$ (which is the point $(3,4)$).
	\end{itemize}
	Of course, the entire space $\Aff^2 = \VV(0)$ and the empty set $\varnothing = \VV(1)$
	are closed sets.

	Thus the nonempty open sets in $\Aff^2$ consist of the \emph{entire} plane,
	minus a finite collection of points and one-dimensional curves.
\end{example}
\begin{ques}
	Draw a picture (to the best of your artistic ability) of a ``typical''
	open set in $\Aff^2$.
\end{ques}

All this is to say
\begin{moral}
	The nonempty Zariski open sets are \emph{huge}.
\end{moral}
This is an important difference than what you're used to in topology.
To be very clear:
\begin{itemize}
	\ii In the past, if I said something like ``has so-and-so property in a neighborhood of point $p$'',
	one thought of this as saying ``is true in a small region around $p$''.
	\ii In the Zariski topology, ``has so-and-so property in a neighborhood of point $p$'' should be thought
	of as saying ``is true for virtually all points, other than those on certain curves''.
\end{itemize}
Indeed, ``neighborhood'' is no longer a very accurate description.

It remains to verify that as I've stated it, the closed sets actually form a topology.
That is, I need to verify briefly that
\begin{itemize}
	\ii $\varnothing$ and $\Aff^n$ are both closed.
	\ii Intersections of closed sets (even infinite) are still closed.
	\ii Finite unions of closed sets are still closed.
\end{itemize}
Well, closed sets are the same as varieties, so we already know this!

\section{The Zariski Topology on Affine Varieties}
\prototype{If $V = \VV(y-x^2)$ is a parabola, then $V$ minus $(1,1)$ is open in $V$.
Also, the plane minus the origin is $D(x) \cup D(y)$.}

As we said before, by considering a variety $V$ as a subspace of $\Aff^n$
it inherits the Zariski topology.
One should think of an open subset of $V$ as ``$V$ minus a few Zariski-closed sets''.
For example:
\begin{example}[Open Set of a Variety]
	Let $V = \VV(y-x^2) \subseteq \Aff^2$ be a parabola,
	and let $U = V \setminus \{(1,1)\}$. We claim $U$ is open in $V$.
	\begin{center}
		\begin{asy}
		import graph;
		size(5cm);

		real f(real x) { return x*x; }
		graph.xaxis("$x$");
		graph.yaxis("$y$");
		draw(graph(f,-2,2,operator ..), blue, Arrows);
		label("$\mathcal V(y-x^2)$", (0.8, f(0.8)), dir(-45));

		opendot( (1,1), blue+1);
		\end{asy}
	\end{center}
	Indeed, $\tilde U = \Aff^2 \setminus \{(1,1)\}$ is open in $\Aff^2$
	(since it is the complement of the closed set $\VV(x-1,y-1)$),
	so $U = \tilde U \cap V$ is open in $V$.
	Note that on the other hand the set $U$ is \emph{not} open in $\Aff^2$.
\end{example}

We'll finish by giving a nice basis of the Zariski topology.
\begin{definition}
	Given $V \subseteq \Aff^n$ an affine variety and $f \in \CC[x_1, \dots, x_n]$,
	we define the \vocab{distinguished open set} $D(f)$ to be the open set in $V$
	of points not vanishing on $f$:
	\[ D(f) = \left\{ p \in V \mid f(p) \neq 0 \right\} = V \setminus \VV(f). \]
\end{definition}
\cite{ref:vakil} suggests remembering the notation $D(f)$ as ``doesn't-vanish set''.
\begin{example}
	[Examples of (Unions of) Distinguished Open Sets]
	\listhack
	\begin{enumerate}[(a)]
		\ii If $V = \Aff^1$ then $D(x)$ corresponds to a line minus a point.
		\ii If $V = \VV(y-x^2)$, then $D(x)$ corresponds to the parabola minus $(1,1)$.
		\ii If $V = \Aff^2$, then $D(x) \cup D(y) = \Aff^2 \setminus \{ (0,0) \}$ is the punctured plane.
	\end{enumerate}
\end{example}
\begin{ques}
	Show that every open set of a $V \subset \Aff^n$
	can be written as a finite union of distinguished open sets.
\end{ques}
\begin{exercise}
	Give an example of an open set of $\Aff^2$ which is not a distinguished open set.
\end{exercise}

\section{Coordinate Rings}
\prototype{If $V = \VV(y-x^2)$ then $\CC[V] = \CC[x,y]/(y-x^2)$.}

The next thing we do is consider the functions from $V$ to the base field $\CC$.
We restrict our attention to algebraic (polynomial) functions on a variety $V$:
they should take every point $(a_1, \dots, a_n)$ on $V$ to some complex number $P(a_1, \dots, a_n) \in \CC$.
For example, a valid function on a three-dimensional affine variety might be $(a,b,c) \mapsto a$;
we just call this projection ``$x$''.
Similarly we have a canonical projection $y$ and $z$,
and we can create polynomials by combining them,
say $x^2y + 2xyz$.

\begin{definition}
	The \vocab{coordinate ring} $\CC[V]$ of a variety $V$
	is the ring of polynomial functions on $V$.
	(Notation explained next section.)
\end{definition}

At first glance, we might think this is just $\CC[x_1, \dots, x_n]$.
But on closer inspection we realize that \emph{on a given variety},
some of these functions are the same.
For example, consider in $\Aff^2$ the parabola $V = \VV(y-x^2)$.
Then the two functions
\begin{align*}
	V & \to \CC \\
	(x,y) & \mapsto x^2 \\
	(x,y) & \mapsto y
\end{align*}
are actually the same function!
We have to ``mod out'' by the ideal $I$ which generates $V$.
This leads us naturally to the following:
\begin{theorem}[Coordinate Rings Correspond to Ideal]
	Let $I$ be a semiprime ideal, and $V = \VV(I) \subseteq \Aff^n$.
	Then \[ \CC[V] \cong \CC[x_1, \dots, x_n] / I.  \]
\end{theorem}
\begin{proof}
	There's a natural surjection as above
	\[ \CC[x_1, \dots, x_n] \surjto \CC[V] \]
	and the kernel is $I$.
\end{proof}
Thus properties of a variety $V$ correspond to properties of the ring $\CC[V]$.

\section{The Sheaf of Regular Functions}
\prototype{Let $V = \Aff^1$, $U = V \setminus \{0\}$. Then $1/x \in \OO_V(U)$ is regular on $U$.}

Let $V$ be an affine variety and let $\CC[V]$ be its coordinate ring.
We want to define a notion of $\OO_V(U)$ for any open set $U$:
the ``nice'' functions on any open subset.
Obviously, any function is the $\CC[V]$ will work as a function on $\OO_V(U)$.
However, to capture more of the structure we want to
loosen our definition slightly of ``nice'' function slightly
by allowing \emph{rational} functions.

The chief example is that $1/x$ should be a regular function
on $\Aff^1 \setminus \{0\}$.
The first natural guess is:
\begin{definition}
	Let $U \subseteq V$ be an open set of the variety $V$.
	A \vocab{rational function} on $U$
	is a quotient $g(x) / h(x)$ of two elements $g$ and $h$ in $\CC[V]$,
	where we require that $h(x) \neq 0$ for $x \in U$.
\end{definition}
However, the definition is slightly too restrictive;
we have to allow for multiple representations in the following way.
\begin{definition}
	Let $U \subseteq V$ be open.
	We say a function $\phi : U \to \CC$ is a \vocab{regular function} if
	for every point $p$, we can find an open set $U_p$ containing $p$
	and a rational function $g_p/h_p$ on $U_p$ such that
	\[ \phi(x) = \frac{g_p(x)}{h_p(x)} \qquad \forall x \in U_p. \]
	% In particular, we require $h_p(x) \neq 0$ on the set $U_p$.
	We denote the set of all regular functions on $U$ by $\OO_V(U)$.
\end{definition}

Thus,
\begin{moral}
	$\phi$ is regular on $U$ if it is locally a rational function.
\end{moral}

This definition is misleadingly complicated,
and the examples should illuminate it significantly.
Firstly, in practice, most of the time we will be able to find
a ``global'' representation of a regular function as a quotient,
and we will not need to fuss with the $p$'s.
For example:
\begin{example}
	[Regular Functions]
	\listhack
	\begin{enumerate}[(a)]
		\ii Any function in $f \in \CC[V]$ is clearly regular,
		since we can take $h_p = 1$, $g_p = f$ for every $p$.
		So $R \subseteq \OO_V(U)$ for any open set $U$.
		\ii Let $V = \Aff^1$, $U_0 = V \setminus \{0\}$. Then $1/x \in \OO_V(U_0)$ is regular on $U$.
		\ii Let $V = \Aff^1$, $U_{12} = V \setminus \{1,2\}$. Then 
		\[ \frac{1}{(x-1)(x-2)} \in \OO_V(U_{12}) \]
		is regular on $U$.
	\end{enumerate}
\end{example}
The ``local'' clause with $p$'s is still necessary, though.
\begin{example}
	[Requiring Local Representations]
	\label{ex:local_rep}
	Consider the variety
	\[ V = \VV(ab-cd) \subseteq \Aff^4 \]
	and the open set $U = V \setminus \VV(b,d)$.
	There is a regular function on $U$ given by
	\[
		(a,b,c,d)
		\mapsto
		\begin{cases}
			a/d & d \neq 0 \\
			c/b & b \neq 0.
		\end{cases}
	\]
	Clearly these are the ``same function'' (since $ab=cd$),
	but we cannot write ``$a/d$'' or ``$c/b$''
	to express it because we run into divide-by-zero issues.
	That's why in the definition of a regular function,
	we have to allow multiple representations.
\end{example}

The division-by-zero, as one would expect,
essentially prohibits regular functions on the entire space $V$;
i.e.\ there are no regular functions in $\OO_V(V)$
that were not already in $\CC[V]$.
Actually, we have the following more general result which computes the
regular functions on distinguished open sets.

\begin{theorem}
	[Regular Functions on Distinguished Open Sets]
	Let $V \subset \Aff^n$ be an affine variety and $D(f)$ a distinguished open subset of it.
	Then
	\[
		\OO_V( D(f) )
		=
		\left\{ \frac{g}{f^n} \mid g \in \CC[V] \text{ and } n \in \ZZ \right\}.
	\]
	In particular, $\OO_V(V) = \OO_V(D(1)) \cong \CC[V]$.
\end{theorem}
The proof of this theorem requires the Nullstellensatz,
so it relies on $\CC$ being algebraically closed.
In fact, a counter-example is easy to find if we replace $\CC$ by $\RR$:
consider $\frac{1}{x^2+1}$.
\begin{proof}
	This is long, so you may want to omit this on a first reading.
	\todo{write this out}
\end{proof}
This means that the \emph{global} regular functions
are just the same as those in the coordinate ring:
you don't gain anything new by allowing it to be locally a quotient.
(The same goes for distinguished open sets.)

\begin{example}[Regular Functions on Distinguished Open Sets]
	\listhack
	\begin{enumerate}[(a)]
		\ii Taking $f=1$ we recover $\OO_V(V) \cong \CC[V]$ for any affine variety $V$.
		\ii Let $V = \Aff^1$, $U_0 = V \setminus \{0\}$. Then
		\[ \OO_V(U_0)
			= \left\{ \frac{P(x)}{x^n} \mid P \in \CC[x], \quad n \in \ZZ \right\}. \]
		So more examples are $1/x$ and $(x+1)/x^3$.
	\end{enumerate}
\end{example}

\begin{ques}
	Why doesn't our theorem on regular functions apply to \Cref{ex:local_rep}?
\end{ques}

The regular functions will become of crucial importance
once we begin defining a scheme in the next chapter.

\section{A Glimpse of Ringed Spaces}
In summary, given an affine variety $V$ we have added the following structure to it:
\begin{itemize}
	\ii It has a structure of a set of points,
	\ii It has a structure of a topological space $V$ on these points, and
	\ii For every open set $U \subseteq V$ we have associated a ring $\OO_V(U)$ to it.
\end{itemize}
It turns out that the function $\OO_V$ taking open sets and returning rings is something called a \emph{sheaf}.
In the next chapter we'll define an abstract \emph{ringed space},
which is \emph{any} topological space $X$ together with a sheaf $\OO_X$.
In this way, we will have made our affine varieties into abstract ringed spaces,
independent of any particular embedding in $\Aff^n$.

\section\problemhead
\begin{problem}
	Show that an affine variety $V$ is always compact (in the Zariski topology).
\end{problem}

\begin{dproblem}
	Show that for any $n \ge 1$ the Zariski topology of $\Aff^n$
	is \emph{not} Hausdorff.
\end{dproblem}

\begin{sproblem}
	Let $V = \Aff^2$ and let $U = \Aff^2 \setminus \{(0,0)\}$ be the punctured plane.
	Compute $\OO_V(U)$.
\end{sproblem}
