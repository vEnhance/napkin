\chapter{Affine varieties}
In this chapter we introduce affine varieties.
We introduce them in the context of coordinates,
but over the course of the other chapters
we'll gradually move away from this perspective to
viewing varieties as ``intrinsic objects'',
rather than embedded in coordinates.

For simplicity, we'll do almost everything over the field of complex numbers,
but the discussion generalizes to any algebraically closed field.

\section{Affine varieties}
\prototype{$\VV(y-x^2)$ is a parabola in $\Aff^2$.}
%An \vocab{affine variety} is just the zero locus of a set of polynomials.
%We think of it as living in the $n$-dimensional space $\Aff^n$.

\begin{definition}
	Given a set of polynomials $S \subseteq \CC[x_1, \dots, x_n]$
	(not necessarily finite or even countable),
	we let $\VV(S)$ denote the set of points vanishing on \emph{all}
	the polynomials in $S$.
	Such a set is called an \vocab{affine variety}.
	It lives in \vocab{$n$-dimensional affine space}, denoted $\Aff^n$
	(to distinguish it from projective space later).
\end{definition}
For example, a parabola is the zero locus of the polynomial $\VV(y-x^2)$. Picture:
\begin{center}
	\begin{asy}
		import graph;
		size(5cm);

		real f(real x) { return x*x; }
		graph.xaxis("$x$");
		graph.yaxis("$y$");
		draw(graph(f,-2,2,operator ..), blue, Arrows);
		label("$\mathcal V(y-x^2)$", (0.8, f(0.8)), dir(-45), blue);
		label("$\mathbb A^2$", (2,3), dir(45));
	\end{asy}
\end{center}

\begin{example}[Examples of affine varieties]
	These examples are in two-dimensional space $\Aff^2$,
	whose points are pairs $(x,y)$.
	\begin{enumerate}[(a)]
	\ii A straight line can be thought of as $\VV(Ax + By + C)$.
	\ii A parabola as above can be pictured as $\VV(y-x^2)$.
	\ii A hyperbola might be the zero locus of the polynomial $\VV(xy-1)$.
	\ii The two axes can be thought of as $\VV(xy)$; this is the set of points
	such that $x=0$ \emph{or} $y=0$.
	\ii A point $(x_0, y_0)$ can be thought of as $\VV(x-x_0, y-y_0)$.
	\ii The entire space $\Aff^2$ can be thought of as $\VV(0)$.
	\ii The empty set is the zero locus of the constant polynomial $1$, that is $\VV(1)$.
	\end{enumerate}
\end{example}

\section{Naming affine varieties via ideals}
\prototype{$\VV(I)$ is a parabola, where $I=(y-x^2)$.}
As you might have already noticed, a variety can be named by $\VV(-)$ in multiple ways.
For example, the set of solutions to
\[ x=3 \text{ and } y=4 \]
is just the point $(3,4)$.
But this is also the set of solutions to
\[ x=3 \text{ and } y=x+1. \]
So, for example
\[ \{(3,4)\}
	= \VV(x-3, y-4)
	= \VV(x-3, y-x-1).
	\]
That's a little annoying, because in an ideal\footnote{Pun not intended
	but left for amusement value.}
world we would have \emph{one} name
for every variety.
Let's see if we can achieve this.

A partial solution is to use \emph{ideals} rather than small sets.
That is, consider the ideal
\[
	I = \left( x-3, y-4 \right)
	= \left\{ p(x,y) \cdot (x-3) + q(x,y) \cdot (y-4)
	\mid p,q \in \CC[x,y] \right\}
\]
and look at $\VV(I)$.
\begin{ques}
	Convince yourself that $\VV(I) = \{(3,4)\}$.
\end{ques}
So rather than writing $\VV(x-3, y-4)$ it makes sense to
think about this as $\VV\left( I \right)$, where $I = (x-3,y-4)$ is the \emph{ideal}
generated by the two polynomials $x-3$ and $y-4$.
This is an improvement because
\begin{ques}
	Check that $(x-3, y-x-1) = (x-3, y-4)$.
\end{ques}

Needless to say, this pattern holds in general.
\begin{ques}
	Let $\{f_i\}$ be a set of polynomials, and consider
	the ideal $I$ generated by these $\{f_i\}$.
	Show that $\VV(\{f_i\}) = \VV(I)$.
\end{ques}

Thus we will only consider $\VV(I)$ when $I$ is an ideal.
Of course, frequently our ideals are generated by one or two polynomials,
which leads to:
\begin{abuse}
	Given a set of polynomials $f_1, \dots, f_m$
	we let $\VV(f_1, \dots, f_m)$ be shorthand for
	$\VV\left( \left( f_1, \dots, f_m \right) \right)$.
	In other words we let $\VV(f_1, \dots, f_m)$
	abbreviate $\VV(I)$, where $I$ is the \emph{ideal} $I=(f_1, \dots, f_m)$.
\end{abuse}

This is where the Noetherian condition really shines:
it guarantees that every ideal $I \subseteq \CC[x_1, \dots, x_n]$
can be written in the form above with \emph{finitely} many polynomials,
because it is \emph{finitely generated}.
(The fact that $\CC[x_1, \dots, x_n]$ is Noetherian follows from the Hilbert basis theorem,
which is \Cref{thm:hilbert_basis}).
This is a relief, because dealing with infinite sets of polynomials is not much fun.

\section{Radical ideals and Hilbert's Nullstellensatz}
\prototype{$\sqrt{(x^2)} = (x)$ in $\CC[x]$, $\sqrt{(12)} = (6)$ in $\ZZ$.}
You might ask whether the name is unique now:
that is, if $\VV(I) = \VV(J)$, does it follow that $I=J$?
The answer is unfortunately no: the counterexample can be found in just $\Aff^1$.
It is
\[ \VV(x) = \VV(x^2). \]
In other words, the set of solutions to $x=0$
is the same as the set of solutions to $x^2=0$.

Well, that's stupid.
We want an operation which takes the ideal $(x^2)$ and makes it into the ideal $(x)$.
The way to do so is using the radical of an ideal.

\begin{definition}
	Let $R$ be a ring.
	The \vocab{radical} of an ideal $I \subseteq R$, denoted $\sqrt I$,
	is defined by
	\[ \sqrt I = \left\{ r \in R
			\mid r^m \in I \text{ for some integer $m \ge 1$} \right\}. \]
	If $I = \sqrt I$, we say the ideal $I$ itself is \vocab{radical}.
\end{definition}
For example, $\sqrt{(x^2)} = (x)$.
You may like to take the time to verify that $\sqrt I$ is actually an ideal.

\begin{remark}
	[Number theoretic motivation]
	This is actually the same as the notion of ``radical'' in number theory.
	In $\ZZ$, the radical of an ideal $(n)$ corresponds to just
	removing all the duplicate prime factors, so for example
	\[ \sqrt{(12)} = (6). \]
	In particular, if you try to take $\sqrt{(6)}$,
	you just get $(6)$ back;
	you don't squeeze out any new prime factors.

	This is actually true more generally,
	and there is a nice corresponding alternate definition:
	for any ideal $I$, we have
	\[ \sqrt I = \bigcap_{I \subseteq \kp \text{ prime}} \kp. \]
	Although we could prove this now,
	it will be proved later in \Cref{thm:radical_intersect_prime},
	when we first need it.
\end{remark}

Here are the immediate properties you should know.
\begin{proposition}
	[Properties of radical]
	\label{prop:radical}
	In any ring:
	\begin{itemize}
		\ii If $I$ is an ideal, then $\sqrt I$ is always a radical ideal.
		\ii Prime ideals are radical.
		\ii For $I \subseteq \CC[x_1, \dots, x_n]$
		we have $\VV(I) = \VV(\sqrt I)$.
	\end{itemize}
\end{proposition}
\begin{proof}
	These are all obvious.
	\begin{itemize}
		\ii If $f^m \in \sqrt I$ then $f^{mn} \in I$, so $f \in \sqrt I$.
		\ii If $f^n \in \kp$ for a prime $\kp$,
		then either $f \in \kp$ or $f^{n-1} \in \kp$,
		and in the latter case we may continue by induction.
		\ii We have $f(x_1, \dots, x_n) = 0$
		if and only if $f(x_1, \dots, x_n)^m = 0$ for some integer $m$.
		\qedhere
	\end{itemize}
\end{proof}

The last bit makes sense: you would never refer to $x=0$ as $x^2=0$,
and hence we would always want to call $\VV(x^2)$ just $\VV(x)$.
With this, we obtain a theorem called Hilbert's Nullstellensatz.
\begin{theorem}[Hilbert's Nullstellensatz]
	\label{thm:hilbert_null}
	Given an affine variety $V = \VV(I)$,
	the set of polynomials which vanish
	on all points of $V$ is precisely $\sqrt I$.
	Thus if $I$ and $J$ are ideals in $\CC[x_1, \dots, x_n]$, then
	\[ \VV(I) = \VV(J) \text{ if and only if $\sqrt I = \sqrt J$}. \]
\end{theorem}
In other words
\begin{moral}
	Radical ideals in $\CC[x_1, \dots, x_n]$ correspond
	exactly to $n$-dimensional affine varieties.
\end{moral}
The proof of Hilbert's Nullstellensatz will be given in
\Cref{prob:hilbert_from_weak}; for now it is worth remarking that
it relies essentially on the fact that $\CC$ is
\emph{algebraically closed}.
For example, it is false in $\RR[x]$,
with $(x^2+1)$ being a maximal ideal with empty vanishing set.

\section{Pictures of varieties in $\Aff^1$}
\prototype{Finite sets of points (in fact these are the only nontrivial examples).}
Let's first draw some pictures.
In what follows I'll draw $\CC$ as a straight line\dots sorry.

First of all, let's look at just the complex line $\Aff^1$.
What are the various varieties on it?
For starters, we have a single point $9 \in \CC$,
generated by $(x-9)$.

\begin{center}
	\begin{asy}
		size(6cm);
		pair A = (-9,0); pair B = (9,0);
		draw(A--B, Arrows);
		label("$\mathcal V(x-9)$", (0,0), 2*dir(-90), blue);
		dot("$9$", (3,0), dir(90), blue);
		label("$\mathbb A^1$", A+(2,0), dir(90));
	\end{asy}
\end{center}

Another example is the point $4$.
And in fact, if we like we can get an ideal consisting of just these two points;
consider $\VV\left( (x-4)(x-9) \right)$.

\begin{center}
	\begin{asy}
		size(6cm);
		pair A = (-9,0); pair B = (9,0);
		draw(A--B, Arrows);
		label("$\mathcal V( (x-4)(x-9) )$", (0,0), 2*dir(-90), blue);
		dot("$4$", (-1,0), dir(90), blue);
		dot("$9$", (3,0), dir(90), blue);
		label("$\mathbb A^1$", A+(2,0), dir(90));
	\end{asy}
\end{center}

In general, in $\Aff^1$ you can get finitely
many points $\left\{ a_1, \dots, a_n \right\}$ by
just taking \[ \VV\left( (x-a_1)(x-a_2)\dots(x-a_n) \right). \]
On the other hand, you can't get the set $\{0,1,2,\dots\}$ as an affine variety;
the only polynomial vanishing
on all those points is the zero polynomial.
In fact, you can convince yourself that these
are the only affine varieties, with two exceptions:
\begin{itemize}
	\ii The entire line $\Aff^1$ is given by $\VV(0)$, and
	\ii The empty set is given by $\VV(1)$.
\end{itemize}
\begin{exercise}
	Show that these are the only varieties of $\Aff^1$.
	(Let $\VV(I)$ be the variety and pick a $0 \neq f \in I$.)
\end{exercise}

As you might correctly guess, we have:
\begin{theorem}[Intersections and unions of varieties]
	\label{thm:many_aff_variety}
	\listhack
	\begin{enumerate}[(a)]
		\ii The intersection of affine varieties
		(even infinitely many) is an affine variety.
		\ii The union of finitely many affine varieties
		is an affine variety.
	\end{enumerate}
	In fact we have
	\[ \bigcap_\alpha \VV(I_\alpha)
		= \VV\left( \sum_\alpha I_\alpha \right)
		\qquad\text{and}\qquad
		\bigcup_{k=1}^n \VV(I_k)
		= \VV\left( \bigcap_{k=1}^n I_k \right). \]
\end{theorem}
You are welcome to prove this easy result yourself.
\begin{remark}
	Part (a) is a little misleading in that the sum $I+J$ need not be radical:
	take for example $I = (y-x^2)$ and $J = (y)$ in $\CC[x,y]$,
	where $x \in \sqrt{I+J}$ and $x \notin I+J$.
	But in part (b) for radical ideals $I$ and $J$,
	the intersection $I \cap J$ is radical.
\end{remark}

\section{Prime ideals correspond to irreducible affine varieties}
\prototype{$(xy)$ corresponds to the union of two lines in $\Aff^2$.}

Note that most of the affine varieties of $\Aff^1$, like $\{4,9\}$,
are just unions of the simplest ``one-point'' ideals.
To ease our classification,
we can restrict our attention to the case of \emph{irreducible} varieties:
\begin{definition}
	A variety $V$ is \vocab{irreducible} if it cannot be written
	as the union of two proper sub-varieties $V = V_1 \cup V_2$.
\end{definition}
\begin{abuse}
	Warning: in other literature,
	irreducible is part of the definition of variety.
\end{abuse}

\begin{example}
	[Irreducible varieties of $\Aff^1$]
	The irreducible varieties of $\Aff^1$ are:
	\begin{itemize}
		\ii the empty set $\VV(1)$,
		\ii a single point $\VV(x-a)$, and
		\ii the entire line $\Aff^1 = \VV(0)$.
	\end{itemize}
\end{example}
\begin{example}
	[The union of two axes]
	Let's take a non-prime ideal in $\CC[x,y]$, such as $I = (xy)$.
	Its vanishing set $\VV(I)$ is the union of two lines $x=0$ and $y=0$.
	So $\VV(I)$ is reducible.
\end{example}

%We have already seen that the radical ideals
%are in one-to-one correspondence with affine varieties.
%In the next sections we answer the two questions:
%\begin{itemize}
%	\ii What property of $\VV(I)$ corresponds to $I$ being prime?
%	\ii What property of $\VV(I)$ corresponds to $I$ being maximal?
%\end{itemize}
%The first question is easier to answer.

In general:
\begin{theorem}[Prime $\iff$ irreducible]
	Let $I$ be a radical ideal, and $V = \VV(I)$ a nonempty variety.
	Then $I$ is prime if and only if $V$ is irreducible.
\end{theorem}
\begin{proof}
	First, assume $V$ is irreducible; we'll show $I$ is prime.
	Let $f,g \in \CC[x_1, \dots, x_n]$ so that $fg \in I$.
	Then $V$ is a subset of the union $\VV(f) \cup \VV(g)$;
	actually, $V = \left( V \cap \VV(f) \right) \cup \left( V \cap \VV(g) \right)$.
	Since $V$ is irreducible, we may assume $V = V \cap \VV(f)$,
	hence $f$ vanishes on all of $V$. So $f \in I$.

	The reverse direction is similar.
\end{proof}

%\begin{remark}
%	The above proof illustrates the following principle:
%	Let $V$ be an irreducible variety.
%	Suppose that $V \subseteq V_1 \cup V_2$;
%	this implies $V = (V_1 \cap V) \cup (V_2 \cap V)$.
%	Recall that the intersection of two varieties is a variety.
%	Thus an irreducible variety can't even be \emph{contained}
%	in a nontrivial union of two varieties.
%\end{remark}

\section{Pictures in $\Aff^2$ and $\Aff^3$}
\prototype{Various curves and hypersurfaces.}

With this notion, we can now draw pictures in
``complex affine plane'', $\Aff^2$.
What are the irreducible affine varieties in it?

As we saw in the previous discussion,
naming irreducible affine varieties in $\Aff^2$
amounts to naming the prime ideals of $\CC[x,y]$.
Here are a few.
\begin{itemize}
	\ii The ideal $(0)$ is prime. $\VV(0)$ as usual corresponds to the entire plane.
	\ii The ideal $(x-a, y-b)$ is prime,
	since $\CC[x,y] / (x-a, y-b) \cong \CC$ is an integral domain.
	(In fact, since $\CC$ is a field, the ideal $(x-a,y-b)$ is \emph{maximal}).
	The vanishing set of this is $\VV(x-a, y-b) = \{ (a,b) \} \in \CC^2$,
	so these ideals correspond to a single point.
	\ii Let $f(x,y)$ be an irreducible polynomial, like $y-x^2$.
	Then $(f)$ is a prime ideal! Here $\VV(I)$ is a ``degree one curve''.
\end{itemize}

By using some polynomial algebra
(again you're welcome to check this; Euclidean algorithm),
these are in fact the only prime ideals of $\CC[x,y]$.
Here's a picture.

\begin{center}
	\begin{asy}
		import graph;
		graph.xaxis("$x$", -4, 4);
		graph.yaxis("$y$", -4, 4);

		real f (real x) { return x*x; }
		draw(graph(f,-2,2,operator ..), blue);
		label("$\mathcal V(y-x^2)$", (1,1), dir(-45), blue);
		dot("$\mathcal V(x-1,y+2)$", (1,-2), dir(-45), red);
	\end{asy}
\end{center}


As usual, you can make varieties which are just unions of these irreducible ones.
For example, if you wanted the variety consisting of a parabola $y=x^2$
plus the point $(20,15)$ you would write
\[ \VV \left( (y-x^2)(x-20), (y-x^2)(y-15) \right). \]

The picture in $\Aff^3$ is harder to describe.
Again, you have points $\VV(x-a, y-b, z-c)$ corresponding to
be zero-dimensional points $(a,b,c)$, and two-dimensional surfaces
$\VV(f)$ for each irreducible polynomial $f$ (for example, $x+y+z=0$ is a plane).
But there are more prime ideals, like $\VV(x,y)$, which corresponds to the
intersection of the planes $x=0$ and $y=0$: this is the one-dimensional $z$-axis.
It turns out there is no reasonable way to classify the ``one-dimensional'' varieties;
they correspond to ``irreducible curves''.

Thus, as Ravi Vakil \cite{ref:vakil} says:
the purely algebraic question
of determining the prime ideals of $\CC[x,y,z]$
has a fundamentally geometric answer.

\section{Maximal ideals}
\prototype{All maximal ideals are $(x_1-a_1, \dots, x_n-a_n)$.}
We begin by noting:
\begin{proposition}
	[$\VV(-)$ is inclusion reversing]
	If $I \subseteq J$ then $\VV(I) \supseteq \VV(J)$.
	Thus $\VV(-)$ is \emph{inclusion-reversing}.
\end{proposition}
\begin{ques}
	Verify this.
\end{ques}
Thus, bigger ideals correspond to smaller varieties.
As the above pictures might have indicated,
the smallest varieties are \emph{single points}.
Moreover, as you might guess from the name,
the biggest ideals are the \emph{maximal ideals}.
As an example, all ideals of the form
\[ \left( x_1-a_1, \dots, x_n-a_n \right) \]
are maximal, since the quotient
\[ \CC[x_1, \dots, x_n] / \left( x_1-a_1, \dots, x_n-a_n \right) \cong \CC \]
is a field.
The question is: are all maximal ideals of this form?

The answer is in the affirmative.
%It's equivalent to:
%\begin{theorem}
%	[Weak Nullstellensatz, phrased as nonempty varieties]
%	Let $I \subsetneq \CC[x_1, \dots, x_n]$ be a proper ideal.
%	Then the variety $\VV(I) \neq \varnothing$.
%\end{theorem}
% From this we can deduce that all maximal ideals are of the above form.
\begin{theorem}
	[Weak Nullstellensatz, phrased with maximal ideals]
	Every maximal ideal of $\CC[x_1, \dots, x_n]$
	is of the form $(x_1-a_1, \dots, x_n-a_n)$.
\end{theorem}
The proof of this is surprisingly nontrivial,
so we won't include it here yet; see \cite[\S7.4.3]{ref:vakil}.
%% TODO we might include this eventually
%\begin{proof}
%	[WN implies MI]
%	Let $J$ be a maximal ideal, and consider the corresponding variety $V = \VV(J)$.
%	By WN, it contains some point $p=(a_1, \dots, a_n)$.
%	Now, define $I = (x_1-a_1, \dots, x_n-a_n)$; this ideal contains all polynomials
%	vanishing at $p$, so necessarily $J \subseteq I \subsetneq \CC[x_1, \dots, x_n]$.
%	Then by maximality of $J$ we have $J=I$.
%\end{proof}
Again this uses the fact that $\CC$ is algebraically closed.
(For example $(x^2+1)$ is a maximal ideal of $\RR[x]$.)
Thus:
\begin{moral}
	Over $\CC$, maximal ideals correspond to single points.
\end{moral}

Consequently, our various ideals over $\CC$ correspond to various flavors
of affine varieties:
\begin{center}
	\begin{tabular}[h]{|cc|}
		\hline
		Algebraic flavor & Geometric flavor \\ \hline
		radical ideal & affine variety \\
		prime ideal & irreducible variety \\
		maximal ideal & single point \\
		any ideal & (scheme?) \\ \hline
	\end{tabular}
\end{center}
There's one thing I haven't talked about: what's the last entry?

\section{Motivating schemes with non-radical ideals}
One of the most elementary motivations for the scheme
is that we would like to use them to count multiplicity.
That is, consider the intersection
\[ \VV(y-x^2) \cap \VV(y) \subseteq \Aff^2 \]
This is the intersection of the parabola with the tangent $x$-axis,
this is the green dot below.

\begin{center}
	\begin{asy}
		import graph;
		size(5cm);

		real f(real x) { return x*x; }
		graph.xaxis("$x$", red);
		graph.yaxis("$y$");
		draw(graph(f,-2,2,operator ..), blue, Arrows);
		label("$\mathcal V(y-x^2)$", (0.8, f(0.8)), dir(-45), blue);
		label("$\mathbb A^2$", (2,3), dir(45));
		dotfactor *= 1.5;
		dot(origin, heavygreen);
	\end{asy}
\end{center}

Unfortunately, as a variety, it is just a single point!
However, we want to think of this as a ``double point'':
after all, in some sense it has multiplicity $2$.
You can detect this when you look at the ideals:
\[ (y-x^2) + (y) = (x^2,y) \]
and thus, if we blithely ignore taking the radical, we get
\[ \CC[x,y] / (x^2,y) \cong \CC[\eps] / (\eps^2). \]
So the ideals in question are noticing the presence of a double point.

In order to encapsulate this, we need a more refined object than
a variety, which (at the end of the day) is just a set of points;
it's not possible using topology along to encode more information
(there is only one topology on a single point!).
This refined object is the \emph{scheme}.

\section\problemhead
\todo{some actual computation here would be good}

\begin{problem}
	Show that a \emph{real} affine variety $V \subseteq \Aff_\RR^n$
	can always be written in the form $\VV(f)$.
	\begin{hint}
		Squares are nonnegative.
	\end{hint}
	\begin{sol}
		If $V = \VV(I)$ with $I = (f_1, \dots, f_m)$
		(as usual there are finitely many polynomials since $\RR[x_1, \dots, x_n]$ is Noetherian)
		then we can take $f = f_1^2 + \dots + f_m^2$.
	\end{sol}
\end{problem}

\begin{problem}
	[Complex varieties can't be empty]
	\label{prob:complex_variety_nonempty}
	Prove that if $I$ is a proper ideal in $\CC[x_1, \dots, x_n]$
	then $\VV(I) \neq \varnothing$.
	\begin{hint}
		This is actually an equivalent formulation
		of the Weak Nullstellensatz.
	\end{hint}
	\begin{sol}
		Let $I$ be an ideal, and let $\km$ be a maximal ideal contained in it.
		(If you are worried about the existence of $\km$,
		it follows from Krull's Theorem, \Cref{prob:krull_max_ideal}).
		Then $\km = (x_1 - a_1, \dots, x_n - a_n)$ by Weak Nullstellensatz.
		Consequently, $(a_1, \dots, a_n)$ is the unique point of $\VV(\km)$,
		and hence this point is also in $\VV(I)$.
	\end{sol}
\end{problem}

\begin{problem}
	\yod
	\label{prob:hilbert_from_weak}
	Show that Hilbert's Nullstellensatz in $n$ dimensions
	follows from the Weak Nullstellensatz.
	(This solution is called the \vocab{Rabinowitsch Trick}.)
	\begin{hint}
		Use the weak Nullstellensatz on $n+1$ dimensions.
		Given $f$ vanishing on everything,
		consider $x_{n+1}f-1$.
	\end{hint}
	\begin{sol}
		The point is is to check that if $f$ vanishes on all of $\VV(I)$,
		then $f \in \sqrt I$.

		Take a set of generators $f_1, \dots, f_m$,
		in the original ring $\CC[x_1, \dots, x_n]$;
		we may assume it's finite by the Hilbert basis theorem.

		We're going to do a trick now:
		consider $S = \CC[x_1, \dots, x_n, x_{n+1}]$ instead.
		Consider the ideal $I' \subseteq S$ in the bigger ring
		generated by $\{f_1, \dots, f_m\}$ and the polynomial $x_{n+1} f - 1$.
		The point of the last guy is that its zero locus
		does not touch our copy $x_{n+1}=0$ of $\Aff^n$
		nor any point in the ``projection'' of $f$ through $\Aff^{n+1}$
		(one can think of this as $\VV(I)$ in the smaller ring
		direct multiplied with $\CC$).
		Thus $\VV(I') = \varnothing$, and by the weak Nullstellensatz
		we in fact have $I' = \CC[x_1, \dots, x_{n+1}]$.
		So
		\[ 1 = g_1f_1 + \dots + g_mf_m + g_{m+1} \left( x_{n+1}f-1 \right). \]
		Now the hack: \textbf{replace every instance of $x_{n+1}$ by $\frac 1f$},
		and then clear all denominators.
		Thus for some large enough integer $N$ we can get
		\[ f^N = f^N(g_1f_1 + \dots + g_mf_m) \]
		which eliminates any fractional powers of $f$ in the right-hand side.
		It follows that $f^N \in I$.
	\end{sol}
\end{problem}
