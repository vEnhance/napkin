\chapter{Affine schemes: the sheaf}
\label{ch:spec_sheaf}

We now complete our definition of $X = \Spec A$ by
defining the sheaf $\OO_X$ on it, making it into a ringed space.
This is done quickly in the first section.

However, we will then spend the next several chapters
trying to convince the reader to \emph{forget}
the definition we gave, in practice.
This is because practically,
the sections of the sheaves are best computed by not using
the definition directly, but by using some other results.

Along the way we'll develop some related theory:
in computing the stalks we'll find out the definition of a local ring,
and in computing the sections we'll find out about distinguished open sets.

A reminder once again: \Cref{ch:spec_examples} has
many more concrete examples.
It's not a bad idea to look through there for more examples
if anything in this chapter trips you up.

\section{A useless definition of the structure sheaf}
\prototype{Still $\CC[x_1, \dots, x_n] / I$.}

We have now endowed $\Spec A$ with the Zariski topology,
and so all that remains is to put a sheaf $\OO_{\Spec A}$ on it.
To do this we want a notion of ``regular functions'' as before.

This is easy to do since we have localizations on hand.
\begin{definition}
	First, let $\SF$ be the pre-sheaf of ``globally rational'' functions:
	i.e.\ we define $\SF(U)$ to be the localization
	\[
		\SF(U) = \left\{
			\frac fg \mid f, g \in A
			\text{ and } g(\kp) \neq 0 \; \forall \kp \in U
		\right\}
		= \left(A \setminus \bigcup_{\kp \in U} \kp \right)\inv A.
	\]
	We now define the structure sheaf on $\Spec A$.
	It is
	\[ \OO_{\Spec A} = \SF\sh \]
	i.e.\ the sheafification of the $\SF$ we just defined.
\end{definition}
\begin{exercise}
	Compare this with the definition for $\OO_V$
	with $V$ a complex variety, and check that they essentially match.
\end{exercise}
And thus, we have completed the transition to adulthood,
with a complete definition of the affine scheme.

If you really like compatible germs,
you can write out the definition:
\begin{definition}
	Let $A$ be a ring.
	Then $\Spec A$ is made into a ringed space by setting
	\[ \OO_{\Spec A}(U)
		= \left\{ (f_\kp \in A_\kp)_{\kp \in U}
		\text{ which are locally quotients} \right\}. \]
	That is, it consists of sequence $(f_\kp)_{\kp \in U}$, with
	each $f_\kp \in A_\kp$, such that for every point $\kp$ there
	is an open neighborhood $U_\kp$ and an $f,g \in A$ such that
	$f_\kq = \frac fg \in A_\kq$ for all $\kq \in U_\kp$.
\end{definition}

We will now \textbf{basically forget about this definition},
because we will never use it in practice.
In the next two sections, we will show you:
\begin{itemize}
	\ii that the stalks $\OO_{\Spec A, \kp}$ are just $A_\kp$, and
	\ii that the sections $\OO_{\Spec A}(U)$
	can be computed, for any open set $U$,
	by focusing only on the special case where $U = D(f)$
	is a distinguished open set.
\end{itemize}
These two results will be good enough for all of our purposes,
so we will be able to not use this definition.
(Hence the lack of examples in this section.)

\section{The value of distinguished open sets (or: how to actually compute sections)}
\prototype{$D(x)$ in $\Spec \CC[x]$ is the punctured line.}

We will now really hammer in the importance of
the distinguished open sets.
The definition is analogous to before:
\begin{definition}
	Let $f \in \Spec A$.
	Then $D(f)$ is the set of $\kp$ such that $f(\kp) \neq 0$,
	a \vocab{distinguished open set}.
\end{definition}
Distinguished open sets will have three absolutely crucial properties,
which build on each other.

\subsection{A basis of the Zariski topology}
The first is a topological observation:
\begin{theorem}
	[Distinguished open sets form a base]
	\label{thm:distinguished_base}
	The distinguished open sets $D(f)$
	form a basis for the Zariski topology:
	any open set $U$ is a union of distinguished open sets.
\end{theorem}
\begin{proof}
	Let $U$ be an open set;
	suppose it is the complement of closed set $V(I)$.
	Then verify that \[ U = \bigcup_{f \in I} D(f). \qedhere \]
\end{proof}

\subsection{Sections are computable}
The second critical fact is that the sections
on distinguished open sets can be computed explicitly.
\begin{theorem}
	[Sections of $D(f)$ are localizations away from $f$]
	Let $A$ be a ring and $f \in A$.
	Then \[ \OO_{\Spec A}(D(f)) \cong A[1/f]. \]
\end{theorem}
\begin{proof}
	Omitted, but similar to
	\Cref{thm:reg_func_distinguish_open}.
\end{proof}

\begin{example}
	[The punctured line is isomorphic to a hyperbola]
	The ``hyperbola effect'' appears again:
	\[ \OO_{\Spec \CC[x]} (D(x))
		= \CC[x, x\inv]
		\cong \CC[x,y] / (xy-1). \]
\end{example}

On a tangential note,
we had better also note somewhere that $\Spec A = D(1)$
is itself distinguished open, so the global sections can be recovered.
\begin{corollary}
	[$A$ is the ring of global sections]
	The ring of global sections of $\Spec A$ is $A$.
\end{corollary}
\begin{proof}
	By previous theorem, $\OO_{\Spec A}(\Spec A)
	= \OO_{\Spec A}(D(1)) = A[1/1] = A$.
\end{proof}

\subsection{They are affine}
\label{subsec:distinguished_open_affine}
We know $\OO_X(D(f)) = A[1/f]$.
In fact, if you draw $\Spec A[1/f]$,
you will find that it looks exactly like $D(f)$.
So the third final important fact is that
$D(f)$ will actually be \emph{isomorphic} to $\Spec A[1/f]$
(just like the line minus the origin is isomorphic to the hyperbola).
We can't make this precise yet,
because we have not yet discussed morphisms of schemes,
but it will be handy later (though not right away).
%However, you can already see this at the level of topological spaces;
%see \Cref{prob:homeomorphism}.
%Since distinguished open sets form a base,
%though, this means that open sets of affine schemes
%are, at least locally, themselves affine schemes:
%given any open set $U \subseteq \Spec A$, and point $\kp \in U$,
%there is some open neighborhood $V \ni p$ contained in $U$
%which is itself affine.

\subsection{Classic example: the punctured plane}
\label{subsec:punctured_plane}
We now give the classical example of a computation which shows
how you can forget about sheafification,
if you never liked it.\footnote{This perspective is
	so useful that some sources, like Vakil \cite[\S4.1]{ref:vakil}
	will \emph{define} $\OO_{\Spec A}$
	by requiring $\OO_{\Spec A}(D(f)) = A[1/f]$,
	rather than use sheafification as we did.}
The idea is that:
\begin{moral}
	We can compute any section $\OO_X(U)$ in practice
	by using distinguished open sets and sheaf axioms.
\end{moral}

Let $X = \Spec \CC[x,y]$,
and consider the origin, i.e.\ the point $\km = (x,y)$.
This ideal is maximal, so it corresponds to a closed point,
and we can consider the open set $U$
consisting of all the points other than $\km$.
We wish to compute $\OO_X(U)$.

\begin{center}
\begin{asy}
	graph.xaxis("$\mathcal{V}(y)$", red);
	graph.yaxis("$\mathcal{V}(x)$", red);
	fill(box( (-3,-3), (3,3) ), opacity(0.2)+lightcyan);
	opendot(origin, blue+1.5);
	label("$\mathfrak m = (x,y)$", origin, dir(45), blue);
\end{asy}
\end{center}

Unfortunately, $U$ is not distinguished open.
But, we can compute it anyways by writing $U = D(x) \cup D(y)$:
conveniently, $D(x) \cap D(y) = D(xy)$.
By the sheaf axioms,
we have a pullback square
\begin{center}
\begin{tikzcd}
	\OO_X(U) \ar[r] \ar[d] & \OO_X(D(x)) = \CC[x,y,x\inv] \ar[d] \\
	\OO_X(D(y)) = \CC[x,y,y\inv] \ar[r] & \OO_X(D(xy)) = \CC[x,y,x\inv, y\inv].
\end{tikzcd}
\end{center}
In other words, $\OO_X(U)$ consists of pairs
\begin{align*}
	f &\in \CC[x,y,x\inv] \\
	g &\in \CC[x,y,y\inv]
\end{align*}
which agree on the overlap:
$f = g$ on $D(x) \cap D(y)$.
Well, we can describe
$f$ as a polynomial with some $x$'s in the denominator, and
$g$ as a polynomial with some $y$'s in the denominator.
If they match, the denominator is actually constant.
Put crudely,
\[ \CC[x,y,x\inv] \cap \CC[x,y,y\inv] = \CC[x,y]. \]
In conclusion,
\[ \OO_X(U) = \CC[x,y]. \]
That is, we get no additional functions.



\section{The stalks of the structure sheaf}
\prototype{The stalk of $\Spec \CC[x,y]$ at $\km = (x,y)$
are rational functions defined at the origin.}

Don't worry, this one is easier than last section.

\subsection{They are localizations}
\begin{theorem}
	[Stalks of $\Spec A$ are $A_\kp$]
	Let $A$ be a ring and let $\kp \in \Spec A$.
	Then \[ \OO_{\Spec A, \kp} \cong A_\kp. \]
	In particular $X$ is a locally ringed space.
\end{theorem}
\begin{proof}
	Since sheafification preserved stalks,
	it's enough to check it for $\SF$ the pre-sheaf
	of globally rational functions in our definition.
	The proof is basically the same as \Cref{thm:stalks_affine_var}:
	there is an obvious map $\SF_\kp \to A_\kp$ on germs by
	\[ \left(U, f/g \in \SF(U) \right)
		\mapsto f/g \in A_\kp . \]
	(Note the $f/g$ on the left lives in $\SF(U)$
	but the one on the right lives in $A_\kp$).
	We show injectivity and surjectivity:
	\begin{itemize}
		\ii Injective: suppose $(U_1, f_1 / g_1)$ and $(U_2, f_2 / g_2)$
		are two germs with $f_1/g_1 = f_2/g_2 \in A_\kp$.
		This means $h(g_1 f_2 - f_2 g_1) = 0$ in $A$, for some nonzero $h$.
		Then both germs identify with
		the germ $(U_1 \cap U_2 \cap D(h), f_1 / g_1)$.
		\ii Surjective: let $U = D(g)$. \qedhere
	\end{itemize}
\end{proof}

\begin{example}
	[Denominators not divisible by $x$]
	We have seen this example so many times
	that I will only write it in the new notation,
	and make no further comment:
	if $X = \Spec \CC[x]$ then
	\[ \OO_{\Spec X, (x)} = \CC[x]_{(x)}
		= \left\{  \frac fg \mid g(0) \neq 0 \right\}. \]
\end{example}
\begin{example}
	[Denominators not divisible by $x$]
	Let $X = \Spec \CC[x,y]$ and let $\km = (x,y)$ be the origin.
	Then
	\[ \CC[x,y]_{(x,y)}
		= \left\{ \frac{f(x,y)}{g(x,y)} \mid g(0,0) \neq 0 \right\}. \]
\end{example}

If you want more examples,
take any of the ones from \Cref{sec:localize_prime_ideal},
and try to think about what they mean geometrically.

\subsection{Motivating local rings: germs should package values}
Let's return to our well-worn example $X = \Spec \CC[x,y]$
and consider $\km = (x,y)$ the origin.
The stalk was
\[ \OO_{X, \km} = \CC[x,y]_{(x,y)}
	= \left\{ \frac{f(x,y)}{g(x,y)} \mid g(0,0) \neq 0 \right\}. \]
So let's take some section like $f = \frac{1}{xy + 4}$,
which is a section of $U = D(xy+4)$ (or some smaller open set,
but we'll just use this one for simplicity).
We also have $U \ni m$, and so $f$ gives a germ at $\km$.

On the other hand, $f$ also has a value at $\km$:
it is $f \pmod{\km} = \frac 14$.
And in general, the ring of possible values of a section
at the origin $\km$ is $\CC[x,y] / \km \cong \CC$.

Now, you might recall that I pressed the point of view
that a germ might be thought of as an ``enriched value''.
Then it makes sense that if you know the germ of a section $f$ at
a point $\km$ --- i.e., you know the ``enriched value'' ---
then you should be able to value as well.
What this means is that we ought to have some map
\[ A_\km \to A/\km \]
sending germs to their associated values.

Indeed you can, and this leads us to\dots

\section{Local rings and residue fields:
linking germs to values}
\prototype{The residue field of $\Spec \CC[x,y]$ at $\km = (x,y)$
is $\CC$.}

\subsection{Localizations give local rings}
This notation is about to get really terrible, but bear with me.
\begin{theorem}
	[Stalks are local rings]
	\label{thm:stalks_local_ring}
	Let $A$ be a ring and $\kp$ any prime ideal.
	Then the localization $A_\kp$ has exactly one maximal ideal,
	given explicitly by
	\[ \kp A_\kp =
		\left\{ \frac fg \mid f \in \kp \; g \notin \kp \right\}. \]
\end{theorem}
The ideal $\kp A_\kp$ thus captures the idea
of ``germs vanishing at $\kp$''.\footnote{The notation $\kp A_\kp$ really means
the set of $f \cdot h$ where $f \in \kp$
(viewed as a subset of $A_\kp$ by $f \mapsto \frac f1$) and $h \in A_{\kp}$.
I personally find this is more confusing than helpful,
so I'm footnoting it.}

Proof in a moment;
for now let's introduce some words so we can
give our examples in the proper language.
\begin{definition}
	A ring $R$ with exactly one maximal ideal $\km$
	will be called a \vocab{local ring}.
	The \vocab{residue field} is the quotient $A / \km$.
\end{definition}
\begin{ques}
	Are fields local rings?
\end{ques}

Thus what we find is that:
\begin{moral}
	The stalks consist of the possible enriched values (germs);
	the residue field is the set of (un-enriched) values.
\end{moral}

\begin{example}
	[The stalk at the origin of {$\Spec \CC[x,y]$}]
	Again set $A = \CC[x,y]$, $X = \Spec A$ and $\kp = (x,y)$
	so that $\OO_{X,\kp} = A_\kp$.
	(I switched to $\kp$ for the origin,
	to avoid confusion with the maximal ideal $\kp A_{\kp}$
	of the local ring $A_\kp$.)
	As we said many times already,
	$A_{\kp}$ consists of rational functions not vanishing at the origin,
	such as $f = \frac{1}{xy+4}$.

	What is the unique maximal ideal $\kp A_\kp$?
	Answer: it consists of the rational functions
	which \emph{vanish} at the origin:
	for example, $\frac{x}{x^2+3y}$, or $\frac{3x+5y}{2}$,
	or $\frac{-xy}{4(xy+4)}$.
	If we allow ourselves to mod out by such functions,
	we get the residue field $\CC$,
	and $f$ will have the value $\frac14$, since
	\[ \frac{1}{xy+4} -
		{\underbrace{\frac{-xy}{4(xy+4)}}_{\text{vanishes at origin}}}
		= \frac14. \]

	More generally, suppose $f$ is any section of some open
	set containing $\kp$.
	Let $c \in \CC$ be the value $f(\kp)$, that is, $f \pmod \kp$.
	Then $f - c$ is going to be another section
	which vanishes at the origin $\kp$,
	so as promised, $f \equiv c \pmod{\kp A_{\kp}}$.
\end{example}

Okay, we can write down a proof of the theorem now.
\begin{proof}
	[Proof of \Cref{thm:stalks_local_ring}]
	One may check set $I = \kp A_\kp$ is an ideal of $A_\kp$.
	Moreover, $1 \notin I$, so $I$ is proper.

	To prove it is maximal and unique,
	it suffices to prove that any $f \in A_\kp$ with $f \notin I$
	is a \emph{unit} of $A_\kp$.
	This will imply $I$ is maximal: there are no more non-units to add.
	It will also imply $I$ is the only maximal ideal:
	because any proper ideal can't contain units, so is contained in $I$.

	This is actually easy.
	An element of $A_\kp$ not in $I$ must be $x = \frac fg$
	for $f,g \in A$ and $f,g \notin \kp$.
	For such an element, $x\inv = \frac gf \notin \kp$ too.
	So $x$ is a unit. End proof.
\end{proof}

Even more generally:
\begin{moral}
	If a sheaf $\SF$ consists of ``field-valued functions'',
	the stalk $\SF_p$ probably has a maximal ideal
	consisting of the germs vanishing at $p$.
\end{moral}

\begin{example}[Local rings in non-algebraic geometry sheaves]
Let's go back to the example of $X = \RR$ and $\SF(U)$ the smooth functions,
and consider the stalk $\SF_{p}$, where $p \in X$.
Define the ideal $\km_p$ to be the set of germs $(s,U)$ for which $s(p) = 0$.

Then $\km_p$ is maximal: we have an exact sequence
\[ 0 \to \km_p \to \SF_p \taking{(s,U) \mapsto s(p)} \RR \to 0 \]
and so $\SF_p / \km_p \cong \RR$, which is a field.

It remains to check there are no nonzero maximal ideals.
Now note that if $s \notin \km_p$,
then $s$ is nonzero in some open neighborhood of $p$,
and one can construct the function $1/s$ on it.
So \textbf{every element of $\SF_p \setminus \km_p$ is a unit};
and again $\km_p$ is in fact the only maximal ideal!

Thus the stalks of each of the following types of sheaves
are local rings, too.
\begin{itemize}
	\ii Sheaves of continuous real/complex functions on a topological space
	\ii Sheaves of smooth functions on any manifold
	\ii etc.
\end{itemize}
\end{example}

\subsection{Computing values: a convenient square}
\label{sec:convenient_square}
Very careful readers might have noticed something
a little uncomfortable in our extended example with $\Spec A$
with $A = \CC[x,y]$ and $\kp = (x,y)$ the origin.
Let's consider $f = \frac{1}{xy+4}$.
We took $f \pmod{x,y}$ in the original ring $A$ in order
to decide the value ``should'' be $\frac14$.
However, all our calculations actually
took place not in the ring $A$, but instead in the ring $A_\kp$.
Does this cause issues?

Thankfully, no, nothing goes wrong, even in a general ring $A$.

\begin{definition}
	We let the quotient $A_\kp / \kp A_\kp$,
	i.e.\ the \vocab{residue field} of the stalk of $\Spec A$ at $\kp$,
	be denoted by $\kappa(\kp)$.
\end{definition}

Then the following is a special case of
\Cref{thm:localization_commute_quotient}
(localization commutes with quotients):
\begin{theorem}
	[The germ-to-value square]
	Let $A$ be a ring and $\kp$ a prime ideal.
	The following diagram commutes:
	\begin{center}
	\begin{tikzcd}
		A \ar[r, "\text{localize}"] \ar[d, "\bmod \kp"']
			& A_\kp \ar[d, "\bmod \kp"] \\
		A/\kp \ar[r, "\Frac(-)"] & \kappa (\kp)
	\end{tikzcd}
	\end{center}
	In particular, $\kappa(\kp)$
	can also be described as $\Frac(A/\kp)$.
\end{theorem}
So for example, if $A = \CC[x,y]$ and $\kp = (x,y)$,
then $A/\kp = \CC$ and $\Frac(A_\kp) = \Frac(\CC) = \CC$, as we expected.
In practice, $\Frac(A/\kp)$ is probably the easier way
to compute $\kappa(\kp)$ for any prime ideal $\kp$.



\section{Recap}
To recap the last two chapters, let $A$ be a ring.
\begin{itemize}
	\ii We define $X = \Spec A$ to be the set of prime ideals of $A$.
	\begin{itemize}
		\ii The maximal ideals are the ``closed points'' we are used to,
		but the prime ideals are ``generic points''.
	\end{itemize}

	\ii We equip $\Spec A$ with the Zariski topology by declaring
	$\VV(I)$ to be the closed sets, for ideals $I \subseteq A$.
	\begin{itemize}
		\ii The distinguished open sets $D(f)$,
		form a topological basis.
		\ii The irreducible closed sets are exactly the closures of points.
	\end{itemize}

	\ii Finally, we defined a sheaf $\OO_X$.
	We set up the definition such that
	\begin{itemize}
		\ii $\OO_{X}(D(f)) = A[1/f]$:
		at distinguished open sets $D(f)$,
		we get localizations too.
		\ii $\OO_{X,\kp} = A_\kp$:
		the stalks are localizations at a prime.
	\end{itemize}
	Since $D(f)$ is a basis,
	these two properties lets us explicitly compute $\OO_X(U)$
	for any open set $U$,
	so we don't have to resort to the definition using sheafification.
\end{itemize}

\section{Functions are determined by germs, not values}
\prototype{The functions $0$ and $x$ on $\Spec \CC[x]/(x^2)$.}

We close the chapter with a word of warning.
In any ringed space, a section is determined by its germs;
so that on $\Spec A$ a function $f \in A$ is determined
by its germ in each stalk $A_\kp$.
However, we now will mention that an $f \in A$ is \emph{not}
determined by its value $f(\kp) = f \pmod \kp$ at each point.

The famous example is:
\begin{example}
	[On the double point, all multiples of $x$ are zero at all points]
	The space $\Spec \CC[x] / (x^2)$ has only one point, $(x)$.
	The functions $0$ and $x$ (and for that matter $2x$, $3x$, \dots)
	all vanish on it.
	This shows that functions are not determined uniquely
	by values in general.
\end{example}

Fortunately, we can explicitly characterize
when this sort of ``bad'' behavior happens.
Indeed, we want to see when $f(\kp) = g(\kp)$ for every $\kp$,
or equivalently, $h = f-g$ vanishes on every prime ideal $\kp$.
This is equivalent to having
\[ h \in \bigcap_{\kp} \kp = \sqrt{(0)} \]
the radical of the \emph{zero} ideal.
Thus in the prototype, the failure was caused by the fact that $x^n = 0$
for some large $n$.

\begin{definition}
	For a ring $A$, the radical of the zero ideal, $\sqrt{(0)}$,
	is called the \vocab{nilradical} of $A$.
	Elements of the nilradical are called \vocab{nilpotents}.
	We say $A$ is \vocab{reduced} if $0$ is the only nilpotent,
	i.e.\ $\sqrt{(0)} = (0)$.
\end{definition}
\begin{ques}
	Are integral domains reduced?
\end{ques}

Then our above discussion gives:
\begin{theorem}
	[Nilpotents are the only issue]
	Two functions $f$ and $g$ have the same value
	on all points of $\Spec A$ if and only if $f-g$ is nilpotent.
\end{theorem}
In particular, when $A$ is a reduced ring,
even the values $f(\kp)$ as $\kp \in \Spec A$
are enough to determine $f \in A$.

\section{\problemhead}
As \Cref{ch:spec_examples} contains many
examples of affine schemes to train your intuition;
it's likely to be worth reading even before attempting these problems.

\begin{dproblem}
	[Spectrums are quasicompact]
	\gim
	Show that $\Spec A$ is quasicompact for any ring $A$.
\end{dproblem}

\begin{problem}
	[Punctured gyrotop, communicated by Aaron Pixton]
	The gyrotop is the scheme $X = \Spec \CC[x,y,z] / (xy,z)$.
	We let $U$ denote the open subset obtained
	by deleting the closed point $\km = (x,y,z)$.
	Compute $\OO_X(U)$.
	\begin{hint}
		$k[x,y] \times k[z,z\inv]$.
	\end{hint}
	\begin{soln}
		Let $V = D(x) \cup D(y) \subset U$ denote the punctured plane,
		so its complement $D(z)$ looks like a punctured line.
		Then $V \cap D(z) = \varnothing$ and the following diagram
		of restriction maps commutes
		\begin{center}
		\begin{tikzcd}
			& \OO_X(X) = A \ar[d] \ar[ldd] \ar[rdd] & \\
			& \OO_X(U) \ar[ld] \ar[rd] & \\
			\OO_X(D(z)) \ar[rd] & & \OO_X(V) \ar[ld] \\
			& \OO_X(\varnothing) = 0 &
		\end{tikzcd}
		\end{center}
		By sheaf axioms we should actually have
		\[ \OO_X(U) = \OO_X(D(z)) \times \OO_X(V). \]
		We have $\OO_X(D(z)) = A_z = k[x,y,z,z^{-1}]/(xz,yz) \cong k[z,z^{-1}]$.
		On the other hand $\OO_X(V) = k[x,y]$ as shown in \S4.4.1 of Vakil.
		So
		\[ \OO_X(U) = k[x,y] \times k[z,z\inv]. \]
	\end{soln}
\end{problem}

\begin{problem}
	Show that a ring $R$ is a local ring
	if and only of the following property is true:
	for any $x \in R$,
	either $x$ or $1-x$ is a unit.
\end{problem}

\begin{problem}
	Let $R$ be a local ring, and $\km$ be its maximal ideal.
	Describe $R_\km$.
	\begin{hint}
		It's isomorphic to $R$!
	\end{hint}
\end{problem}

\begin{problem}
	Let $A$ be a ring, and $\km$ a maximal ideal.
	Consider $\km$ as point of $\Spec A$
	Show that $\kappa(\km) \cong A/\km$.
\end{problem}
