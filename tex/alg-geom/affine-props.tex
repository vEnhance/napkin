\chapter{More on affine schemes}
\section{Topological interpretation}

\section{Generic points and irreducible closed sets}
biject them to generic points

\section{Local rings}
The stalks of the examples we produced above are special types
of rings, called \emph{local rings}.
Algebraically, the definition of these is:
\begin{definition}
	A \vocab{local ring} $R$ is a ring with exactly one maximal ideal.
\end{definition}
\begin{exercise}
	Show that a ring $R$ is a local ring if there exists
	a proper ideal $\mm \subsetneq R$ such that
	all elements of $R \setminus \mm$ are units.
	(Bonus: prove the converse.)
\end{exercise}

% Wikipedia has a good explanation here
To see why this definition applies to the stalks above,
we need to identify what the maximal ideal is.
Let's go back to the example of $X = \RR$ and $\SF(U)$ the smooth functions,
and consider the stalk $\SF_{p}$, where $p \in X$.
Define the ideal $\mm_p$ to be the set of germs $(s,U)$ for which $s(p) = 0$.

Then $\mm_p$ is maximal: we have an exact sequence
\[ 0 \to \mm_p \to \SF_p \taking{(s,U) \mapsto s(p)} \RR \to 0 \]
and so $\SF_p / \mm_p \cong \RR$, which is a field.

It remains to check there are no nonzero maximal ideals.
Now note that if $s \notin \mm_p$,
then $s$ is nonzero in some neighborhood of $p$,
then one can construct the function $1/s$ in a neighborhood of $p$.
So \textbf{every element of $\SF_p \setminus \mm_p$ is a unit};
$\mm_p$ is in fact the only maximal ideal!

More generally,
\begin{moral}
	If $\SF$ consists of ``field-valued functions'',
	the stalk $\SF_p$ probably has a maximal ideal
	consisting of the germs vanishing at $p$.
\end{moral}
The discussion above implies, for example:
\begin{proposition}
	[Stalks are often local rings]
	The stalks of each of the following types of sheaves are local rings:
	\begin{enumerate}[(a)]
		\ii Sheaves of continuous real/complex functions on a topological space
		\ii Sheaves of smooth functions on any manifold
		\ii Regular functions on an algebraic variety $V$.
	\end{enumerate}
\end{proposition}
Even better:
\begin{theorem}
	[Localization at prime ideal gives local ring]
	If $A$ is a ring and $\kp$ is a prime ideal
	then $A_\kp$ is a local ring.
\end{theorem}
\begin{exercise}
	Prove this.
\end{exercise}

We can now define:
\begin{definition}
	A \vocab{ringed space} is a topological space $X$ equipped
	with a sheaf $\OO_X$ of rings.
	Suppose that for every point $p$, the stalk $\OO_{X,p}$
	is a local ring.
	Then we say that $(X, \OO_X)$ is a \vocab{locally ringed space}.
	We denote the maximal ideals by $\mm_{X,p}$.
\end{definition}

In particular, in the previous chapter we showed that every
affine variety could be built into a locally ringed space. Hooray!
\begin{abuse}
	A ringed space $(X, \OO_X)$ is abbreviated to just $X$,
	while $p \in X$ means ``$p$ is in the topological space $X$''.
\end{abuse}


\section{A valuable square}
\todo{commute with quotient}

