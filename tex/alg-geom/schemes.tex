\chapter{Schemes}
Now that we understand sheaves well, we can readily define the scheme.
It will be a locally ringed space, so we need to define
\begin{itemize}
	\ii The set of points,
	\ii The topology on it, and
	\ii The structure sheaf on it.
\end{itemize}

In the case of $\Aff^n$, we used $\CC^N$ as the set of points
and $\CC[x_1, \dots, x_n]$ but then remarked that the set of points
of $\CC^n$ corresponded to the maximal ideals of $\CC[x_1, \dots, x_n]$.

In an \emph{affine scheme}, we will take an \emph{arbitrary} ring $R$,
and generate the entire structure from just $R$ itself.
The final result is called $\Spec R$, the \vocab{spectrum} of $R$.
The affine varieties $\VV(I)$ we met earlier will just be
$\CC[x_1, \dots, x_n] / I$, but now we will be able to take
\emph{any} ideal $I$, thus finally completing the table at the end
of the ``affine variety'' chapter.

In particular, $\Spec \CC[x] / (x^2)$ will be the double we sought for so long.

\section{The set of points}
\prototype{$\Spec \CC[x_1, \dots, x_n] / I$.}

First surprise, for a ring $R$:
\begin{moral}
	$\Spec R$ is defined as the set of prime ideals of $R$.
\end{moral}

This might be a little surprising, since we might have guessed
that $\Spec R$ should just have the maximal ideals.
What do the remaining ideals correspond to?
The answer is that they will be so-called \emph{generic points}
points which are ``somewhere'' in the space, but nowhere in particular.

\begin{remark}
	As usual $R$ itself is not a prime ideal, but $(0)$
	does if $R$ is an integral domain.
\end{remark}

\begin{example}
	[Examples of spectrums]
	\listhack
	\begin{enumerate}[(a)]
		\ii $\Spec \CC[x]$ consists of a point $(x-a)$ for every $a \in \CC$,
		which correspond to what we geometrically think of as $\Aff^1$.
		In additionally consists of a point $(0)$,
		which we think of as a ``generic point'', nowhere in particular.

		\ii $\Spec \CC[x,y]$ consists of points $(x-a,y-b)$
		(which are the maximal ideals) as well as $(0)$ again, a generic
		point that is thought of as ``somewhere in $\CC^2$,
		but nowhere in particular''.
		It also consists of generic points corresponding to irreducible
		polynomials $f(x,y)$, for example $(y-x^2)$,
		which is a ``generic point on the parabola''.

		\ii If $k$ is a field, $\Spec k$ is a single point,
		since the only maximal ideal of $k$ is $(0)$.
	\end{enumerate}
\end{example}
\begin{example}
	[Complex affine varieties]
	Let $I \subseteq \CC[x_1, \dots, x_n]$ be an ideal.
	Then \[ \Spec \CC[x_1, \dots, x_n] /I \] contains a
	point for every closed irreducible subvariety of $\VV(I)$.
	So in addition to the ``geometric points'' we have
	``generic points'' along each of the varieties.
\end{example}
\begin{example}
	[More examples of spectrums]
	\listhack
	\begin{enumerate}[(a)]
		\ii $\Spec \ZZ$ consists of a point for every prime $p$,
		plus a generic point that is somewhere, but no where in particular.

		\ii $\Spec \CC[x] / (x^2)$ has only $(x)$ as a prime ideal.
		The ideal $(0)$ is not prime since $0 = x \cdot x$.
		Thus as a \emph{topological space},
		$\Spec \CC[x] / (x^2)$ is a single point.
		
		\ii $\Spec \Zc{60}$ consists of three points.
		What are they?
	\end{enumerate}
\end{example}

\section{The Zariski topology of the spectrum}
\prototype{Still $\Spec \CC[x_1, \dots, x_n] / I$.}

Now, we endow a topology on $\Spec R$.
Since the points on $\Spec R$ are the prime ideals, we continue
the analogy by thinking of the points $f$ as functions on $\Spec R$. That is,
\begin{definition}
	Let $f \in R$ and $\pp \in \Spec R$.
	Then the \vocab{value} of $f$ at $\pp$ is defined to be $f \pmod{\pp}$.
	We denote it $f(\pp)$.
\end{definition}
\begin{example}
	[Vanishing locii in $\Aff^n$]
	Suppose $R = \CC[x_1, \dots, x_n]$,
	and $\pp = (x_1-a_1, x_2-a_2, \dots, x_n-a_n)$ is a maximal ideal of $R$.
	Then for a polynomial $f \in \CC$,
	\[ f \pmod \pp = f(a_1, \dots, a_n) \]
	with the identification that $\CC/\pp \cong \CC$.
\end{example}
Indeed if you replace $R$ with $\CC[x_1, \dots, x_n]$
and $\Spec R$ with $\Aff^n$ in everything that follows,
then everything will be clear.

\begin{definition}
	Let $f \in R$. We define the \vocab{vanishing locus} of $f$ to be
	\[ \VV(f) = \left\{ \pp \in \Spec R \mid f(\pp) = 0 \right\}
		= \left\{ \pp \in \Spec R \mid f \in \pp \right\}. \]
	More generally, just as in the affine case,
	we define the vanishing locus for an ideal $I$ as
	\begin{align*}
		\VV(I) &= \left\{ \pp \in \Spec R \mid f(\pp)=0 \forall f \in I \right\} \\
		&= \left\{ \pp \in \Spec R \mid f \in \pp \; \forall f \in I \right\} \\
		&= \left\{ p \in \Spec R \mid I \subseteq \pp \right\}.
	\end{align*}
	Finally, we define the \vocab{Zariski topology} on $\Spec R$
	by declaring that the sets of the form $\VV(I)$ are closed.
\end{definition}

Now, the following topological notion will come in handy:
\begin{definition}
	A point $p \in X$ is a \vocab{closed point} if the set $\{p\}$ is closed.
\end{definition}
\begin{ques}
	Show that $\pp \in \Spec R$ is a closed point
	if and only if $\pp$ is a maximal ideal.
\end{ques}
Therefore the Zariski topology lets us refer back to the old ``geometric''
as just the closed points.
\begin{example}
	[Generic points, continued]
	Let $R = \CC[x,y]$ and let $\pp = (y-x^2) \in \Spec \RR$;
	this is the ``generic point'' on a parabola.
	It is not closed, but we can compute its closure:
	\[
		\ol{\{\pp\}}
		= \VV(\pp) = \left\{ \qq \in \Spec R \mid \qq \supseteq \pp \right\}.
	\]
	This closure contains the point $\pp$ as well
	as several maximal ideals $\qq$, such as $(x-2,y-4)$ and $(x-3,y-9)$.
	In other words, the closure of the ``generic point'' of the parabola
	is literally the set of all points that are actually on the parabola
	(including generic points).

	That means the way to picture $\pp$ is a point that 
	is ``somewhere on the parabola'', but nowhere in particular.
	It makes sense then that if we take the closure,
	we get the entire parabola,
	since $\pp$ ``could have been'' any of those points.
\end{example}
The previous example illustrates the following important observation:
\begin{exercise}
	Let $I \subsetneq R$ be a proper ideal.
	Construct a bijection between the maximal ideals of $R$ containing $I$,
	and the maximal ideals of $R/I$.
	(\cite{ref:vakil} labels this exercise as
	``Essential Algebra Exercise (Mandatory if you haven't done it before)''.)
\end{exercise}

\begin{example}
	[The generic point of the $y$-axis isn't on the $x$-axis]
	Let $R = \CC[x,y]$ again.
	Consider $\VV(y)$, which is the $x$-axis of $\Spec R$.
	Then consider $\pp = (x)$, which is the generic point on the $y$-axis.
	Observe that
	\[ \pp \notin \VV(y). \]
	The geometric way of saying this is that a \emph{generic point}
	on the $y$-axis does not lie on the $x$-axis.
\end{example}

Finally, as before, we can define distinguished open sets,
which form a basis of the Zariski topology.
\begin{definition}
	Let $f \in \Spec R$.
	Then $D(f)$ is the set of $\pp$ such that $f(\pp) \neq 0$,
	a \vocab{distinguished open set}.
	These open sets form a basis for the Zariski topology on $\Spec R$.
\end{definition}

\section{The structure sheaf}
\prototype{Still $\CC[x_1, \dots, x_n] / I$.}

We have now endowed $\Spec R$ with the Zariski topology,
and so all that remains is to put a sheaf $\OO_{\Spec R}$ on it.
To do this we want a notion of ``regular functions'' as before.

In order for this to make sense, we have to talk about rational
functions in a meaningful way.
If $R$ was an integral domain, then we could just use the field of fractions;
for example if $R = \CC[x_1, \dots, x_n]$ then we could just
look at rational quotients of polynomials.

Unfortunately, in the general situation $R$ may not be an integral domain:
for example, the ring $R = \CC[x] / (x^2)$ corresponding to the double point.
So we will need to define something a little different:
we will construct the \emph{localization} of $R$ at a set $S$,
which we think of as the ``set of allowed denominators''.

\begin{definition}
	Let $S \subseteq R$, where $R$ is a ring,
	and assume $S$ is closed under multiplication.
	Then the \vocab{localization of $R$ at $S$}, denoted $S\inv R$,
	is defined as the set of fractions
	\[ \left\{ r/s \mid r \in R, s \in S \right\} \]
	where we declare two fractions $r_1 / s_1 = r_2 / s_2$ 
	to be equal if 
	\[ \exists s \in S : \quad s(r_1s_2 - r_2s_1) = 0. \]
\end{definition}
In particular, if $0 \in S$ then $S\inv R$ is the trivial ring.
So we usually only take situations where $0 \notin S$.
\begin{ques}
	Assume $R$ is an integral domain and $S = R \setminus \{0\}$.
	Show that $S\inv R$ is just the field of fractions.
\end{ques}

\begin{example}
	[Why the extra $s$?]
	The reason we need the condition $s(r_1s_2 - r_2s_1) = 0$
	rather than the simpler $r_1s_2 - r_2s_1 = 0$ is that
	otherwise the equivalence relation may fail to be transitive.
	Here is a counterexample: take
	\[ R = \Zc{12} \qquad S = \{ 0, 4, 8 \}. \]
	Then we have for example
	\[ \frac12 = \frac24 = \frac64 = \frac32. \]
	So we need to have $\frac12=\frac32$ which is only true
	with the first definition.

	Of course, if $R$ is an integral domain (and $0\notin S$)
	then this is a moot point.
\end{example}

The most important special case is the localization at a prime ideal.
\begin{definition}
	Let $R$ be a ring and $\pp$ a prime ideal.
	Then $R_\pp$ is defined to be $S\inv R$ for $S = R\setminus\pp$.
	We call this \vocab{localization at $\pp$}.
	Addition is defined in the obvious way.
\end{definition}
\begin{ques}
	Why is $S$ multiplicative closed in the above definition?
\end{ques}
Thus,
\begin{moral}
	If $R$ is functions on the space $\Spec R$,
	we think of $R_\pp$ as rational quotients $f/g$ where $g(\pp) \neq 0$.
\end{moral}
In particular, if $R = \CC[x_1, \dots, x_n]$
then this is precisely the definition of rational function from before!

Now, we can define the sheaf as ``locally rational'' functions.
This is done by a sheafification.
First, let $\SF$ be the pre-sheaf of ``globally rational'' functions:
i.e.\ we define $\SF(U)$ to be the following localization of $R$
to the functions vanishing outside $U$:
\[
	\SF(U) = \left\{ 
		\frac fg \mid f, g \in R
		\text{ and } g(\pp) \neq 0 \; \forall \pp \in U
	\right\}
	= \left(R \setminus \bigcup_{\pp \in U} \pp \right)\inv R.
\]
For every $\pp \in U$ we can view $f/g$ as an element in $R_\pp$
(since $g(\pp) \ne 0$).
As one might expect this is an isomorphism
\begin{lemma}[Stalks of the ``globally rational'' pre-sheaf]
	\label{lem:global_rational_stalk}
	The stalk of $\SF$ defined above $\SF_\pp$ is isomorphic to $R_\pp$.
\end{lemma}
\begin{proof}
	There is an obvious map $\SF_\pp \to R_\pp$ on germs by
	\[
		\left(U, f/g \in \SF(U) \right)
		\mapsto f/g \in R_\pp . \]
	(Note the $f/g$ on the left lives in
	$\SF(U) = \left(R \setminus \bigcup_{\pp \in U} \pp \right)\inv R$
	but the one on the right lives in $R_\pp$).
	Now suppose $(U_1, f_1 / g_1)$ and $(U_2, f_2 / g_2)$
	are germs with $f_1/g_1 = f_2/g_2 \in R_\pp$.
	\begin{exercise}
		Now, show both germs are equal to $(U_1 \cap U_2 \cap D(h), f_1 / g_1)$
		with $D(h)$ the distinguished open set.
	\end{exercise}
	It is also surjective, since given $f/g \in R_\pp$ we take $U = D(g)$
	the distinguished open set for $g$.
\end{proof}

Then, we set \[ \OO_{\Spec R} = \SF\sh. \]
In fact, we can even write out the definition of the sheafification,
by viewing the germ at each point as an element of $R_\pp$.
\begin{definition}
	Let $R$ be a ring. Then $\Spec R$ is made into a ringed space by setting
	\[ \OO_{\Spec R}(U) 
		= \left\{ (f_\pp \in R_\pp)_{\pp \in U}
		\text{ which are locally $f/g$} \right\}. \]
	That is, it consists of sequence $(f_\pp)_{\pp \in U}$, with
	each $f_\pp \in R_\pp$, such that for every point $\pp$ there
	is a neighborhood $U_\pp$ and an $f,g \in R$ such that
	$f_\qq = \frac fg \in R_\qq$ for all $\qq \in U_\pp$.
\end{definition}

\section{Properties of affine schemes}
Now that we're done defining an affine scheme,
we state some important results about them.

\begin{definition}
	For $g \in R$, we define the \vocab{localization of $R$ at $g$},
	denoted $R_g$ to be $\{1, g, g^2, g^3, \dots\}\inv R$.
	(Note that $\left\{ 1, g, g^2, \dots \right\}$ is multiplicatively closed.)
\end{definition}
This is admittedly somewhat terrible notation, since $R_\pp$
is the localization with multiplicative set $R \setminus \pp$,
while $R_g$ is the localization with multiplicative set $\{1,g,g^2,\dots\}$;
these two are quite different beasts!

\begin{example}
	[Localization at an element]
	Let $R = \CC[x,y,z]$ and let $g = x$.
	Then
	\[ R_g = \left\{ \frac{P(x,y,z)}{x^n} \mid
		P \in \CC[x,y,z], \; n \ge 0 \right\}. \]
\end{example}
\begin{theorem}
	[On the affine structure sheaf]
	Let $R$ be a ring and $\Spec R$ the associated affine scheme.
	\begin{enumerate}[(a)]
		\ii Let $\pp$ be a prime ideal.
		Then $\OO_{\Spec R, \pp} \cong R_\pp$.
		\ii Let $D(g)$ be a distinguished open set.
		Then $\OO_{\Spec S}(D(g)) \cong R_g$.
	\end{enumerate}
	\label{thm:affine_struct_master}
\end{theorem}
This matches the results that we've seen when $\Spec R$ is an affine variety.
\begin{proof}
	Part (a) follows by \Cref{lem:global_rational_stalk}
	and \Cref{lem:pre_sheaf_stalk}.
	For part (b), we need the following:
	\begin{ques}
		If $I$ and $J$ are ideals of a ring $R$,
		then $\VV(I) \subseteq \VV(J)$ if and only if
		$\sqrt{J} \subseteq \sqrt{I}$.
		(Use the fact that $\sqrt I = \cap_{I \subseteq \pp} \pp$.)
	\end{ques}
	Then we can repeat the proof of \Cref{thm:reg_func_distinguish_open}.
\end{proof}

\begin{example}
	[Examples of structure sheaves]
	\listhack
	\begin{enumerate}[(a)]
		\ii Let $X = \Spec \CC[x]$.
		Then $\OO_X(U)$ is the set of regular functions on $U$
		in the sense that we've seen before.
		\ii Let $X = \Spec \ZZ$.
		Let $U = X \setminus \{ (7), (11) \}$.
		Then $\OO_X(U)$ can be identified with the set of rational numbers
		which have denominator not divisible by $7$ or $11$.
	\end{enumerate}
\end{example}

We now state the most important theorem.
In fact, this theorem is the justification that our definition
of a scheme is the correct one.
\begin{theorem}
	[Affine schemes and commutative rings are the same category]
	Let $R$ and $S$ be rings.
	There is a natural bijection between maps of schemes $\Spec R \to \Spec S$	
	and ring homomorphisms $\psi : S \to R$.
\end{theorem}
\begin{proof}
	First, we need to do the construction:
	given a map of schemes $f : \Spec R \to \Spec S$ 
	we need to construct a homomorphism $S \to R$.
	This should be the easy part, because $f$ has a lot of data.
	Indeed, recall that for every $U \subseteq \Spec S$
	there is supposed to be a map
	\[ f^\ast_U : \OO_{\Spec S} (U) \to \OO_{\Spec R}(f\pre(U)). \]
	If we take $U$ to be the entire space $\Spec S$ we now have a ring homomorphism
	\[ S = \OO_{\Spec S} (\Spec S) \to \OO_{\Spec R} (\Spec R) = R. \]
	which is one part of the construction.

	The more involved direction is building from a homomorphism
	$\psi : S \to R$ a map $f : \Spec R \to \Spec S$.
	We define it by:
	\begin{itemize}
		\ii On points, $f(\pp) = \psi\pre(\pp)$ for each prime ideal $\pp$.
		One can check $\psi\pre(\pp)$ is indeed prime
		and that $f$ is continuous in the Zariski topology.
		\ii We first construct a map on the stalks of the sheaf as follows:
		for every prime ideal $\qq$ in $R$, let $\psi``(\qq) = \pp$ be its image
		(so that $f(\qq) = \pp$) and consider the map
		\[ \psi_\qq : \OO_{\Spec S, \qq} = S_{\qq}
			\to R_{\psi``(\qq)} = \OO_{\Spec R, \psi``(\qq)}
			\quad\text{by}\quad f/g \mapsto \psi(f)/\psi(g). \]
		Again, one can check this is well-defined and a map of local rings.
		This is called the localization of $\psi$ at $\qq$.
		\ii Finally, we use the above map to construct an
		$f^\ast_U : \OO_{\Spec S}(U) \to \OO_{\Spec R}(f\pre(U))$
		for every open set  $U \subseteq \Spec S$.
		Since $\OO_{\Spec S}$ is a sheafification, we think of its
		sections as sequences of compatible germs $(g_\qq)_{\qq \in U}$
		and then map it via the map $\psi_\qq$ above.
	\end{itemize}
	One then has to check that everything is well-defined.
	This is left as an exercise to the diligent reader;
	for an actual proof see \cite[Proposition 6.3.2]{ref:vakil}.
\end{proof}
\begin{remark}
	Categorically, this says that the ``global section functor''
	$\Spec R \mapsto \OO_{\Spec R}(R) = R$ is a fully faithful functor
	from the category $\catname{AffSch}$ to the category $\catname{CRing}$.
	(It is also essentially surjective on objects.)
\end{remark}
To be more philosophical,
\begin{moral}
	The category of affine schemes and the
	category of commutative rings are \emph{exactly the same},
	down to the morphisms between two pairs of objects.
\end{moral}

\section{Schemes}
\begin{definition}
	A \vocab{scheme} is a locally ringed space $(X, \OO_X)$
	with an open cover $\{U_\alpha\}$ of $X$
	such that each pair $(U_\alpha, \OO_X \restrict{U_\alpha})$
	is isomorphic to an affine scheme.
\end{definition}
Hooray!

\section{Projective scheme}
\prototype{Projective varieties, in the same way.}
The most important class of schemes which are not affine are
\emph{projective} schemes.
The complete the obvious analogy:
\[
	\frac{\text{Affine variety}}{\text{Projective variety}}
	= 
	\frac{\text{Affine scheme}}{\text{Projective scheme}}.
\]
Let $S$ be \emph{any} (commutative) graded ring, like $\CC[x_0, \dots, x_n]$.
\begin{definition}
	We define $\Proj S$, the \vocab{projective scheme over $S$}:
	\begin{itemize}
		\ii As a set, $\Proj S$ consists of \emph{homogeneous prime ideals}
		$\pp$ which do not contain $S^+$.
		\ii If $I \subseteq S$ is homogeneous, then
		we let $\Vp(I) = \{ \pp \in \Proj S \mid \pp \subseteq I \}$.
		Then the \vocab{Zariski topology} is imposed by declaring 
		sets of the form $\Vp(I)$ to be closed.
		\ii We now define a pre-sheaf $\SF$ on $\Proj S$ by
		\[ \SF(U) = 
			\left\{ \frac{f}{g} \mid 
			g(\pp) \neq 0 \; \forall \pp \in U \text{ and }
			\deg f = \deg g \right\}.
		\]
		In other words, the rational functions are quotients $f/g$
		where $f$ and $g$ are \emph{homogeneous of the same degree}.
		Then we let \[ \OO_{\Proj S} = \SF\sh \] be the sheafification.
	\end{itemize}
\end{definition}
\begin{definition}
	The \vocab{distinguished open sets} $D(f)$ of the $\Proj S$
	are defined as $\left\{ \pp \in \Proj S : f(\pp) \neq 0 \right\}$,
	as before; these form a basis for the Zariski topology of $\Proj S$.
\end{definition}
Now, we want analogous results as we did for affine structure sheaf.
So, we define a slightly modified localization:
\begin{definition}
	Let $S$ be a graded ring.
	\begin{enumerate}[(i)]
		\ii For a prime ideal $\pp$, let
		\[ S_{(\pp)} = \left\{ \frac fg \mid g(\pp) \neq 0 \text{ and }
			\deg f = \deg g \right\} \]
		denote the elements of $S_\pp$ with ``degree zero''.
		\ii For any homogeneous $g \in S$ of degree $d$, let
		\[ S_{(g)} = \left\{ \frac{f}{g^r} \mid 
			\deg f = r \deg g \right\} \]
		denote the elements of $S_g$ with ``degree zero''.
	\end{enumerate}
\end{definition}

\begin{theorem}
	[On the projective structure sheaf]
	Let $S$ be a graded ring and let $\Proj S$ the associated projective scheme.
	\begin{enumerate}[(a)]
		\ii Let $\pp \in \Proj S$.
		Then $\OO_{\Proj S, \pp} \cong S_{(\pp)}$.
		\ii Suppose $g$ is homogeneous with $\deg g > 0$. Then
		\[ D(g) \cong \Spec S_{(g)} \]
		as locally ringed spaces.
		In particular, $\OO_{\Proj S}(D(g)) \cong S_{(g)}$.
	\end{enumerate}
\end{theorem}
\begin{ques}
	Conclude that $\Proj S$ is a scheme.
\end{ques}

Of course, the archetypal example is that
\[ \Proj \CC[x_0, x_1, \dots, x_n] / I \]
corresponds to the projective subvariety of $\CP^n$
cut out by $I$ (when $I$ is radical).
In the general case of an arbitrary ideal $I$, we
call such schemes \vocab{projective subscheme} of $\CP^n$
For example, the ``double point'' is given by $\Proj[x_0,x_1]/(x_0^2)$.

\begin{remark}
	No comment yet on what the global sections of $\OO_{\Proj S}(\Proj S)$ are.
	(The theorem above requires $\deg g > 0$, so we cannot just take $g=1$.)
	One might hope that in general $\OO_{\Proj S}(\Proj S) \cong S^0$
	in analogy to our complex projective varieties, but
	one needs some additional assumptions on $S$ for this to hold.
\end{remark}

\section{Where to go from here}
This chapter concludes the long setup for the definition of a scheme.
Unfortunately, with respect to algebraic geometry this is as much as I have
the patience or knowledge to write about.
So, if you want to actually see how schemes are used in ``real life'',
you'll have to turn elsewhere.

A good introduction I happily recommend is \cite{ref:gathmann}.
See \Cref{ch:refs} for further remarks.

\section\problemhead
\begin{problem}
	Describe the points of $\Spec \RR[x,y]$.
	\begin{hint}
		Galois conjugates.
	\end{hint}
\end{problem}

\begin{dproblem}
	[Chinese remainder theorem]
	Consider $X = \Spec \Zc{60}$, which as a topological space has three points.
	By considering $\OO_X(X)$ prove the Chinese theorem
	\[ \Zc{60} \cong \Zc{4} \times \Zc{3} \times \Zc{5}. \]
	\begin{hint}
		Appeal to \Cref{prob:finite_sheaf}.
	\end{hint}
\end{dproblem}

\begin{problem}
	Given an affine scheme $X = \Spec R$,
	show that there is a unique morphism of schemes $X \to \Spec \ZZ$,
	and describe where it sends points of $X$.
	\begin{hint}
		Use the proof that $\catname{AffSch} \simeq \catname{CRing}$.
	\end{hint}
	\begin{sol}
		$\pp$ gets sent to the characteristic of the field $\OO_{X,\pp} / \mm_{X,\pp}$.
	\end{sol}
\end{problem}

\todo{Critch stalk-local detection}
