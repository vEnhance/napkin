\chapter{Affine schemes: the Zariski topology}
\label{ch:spec_zariski}
Now that we understand sheaves well,
we can define an affine scheme.
It will be a ringed space, so we need to define
\begin{itemize}
	\ii The set of points,
	\ii The topology on it, and
	\ii The structure sheaf on it.
\end{itemize}
In this chapter, we handle the first two parts;
\Cref{ch:spec_sheaf} does the last one.

Quick note: \Cref{ch:spec_examples}
contains a long list of examples of affine schemes.
So if something written in this chapter is not making sense,
one thing worth trying is skimming through \Cref{ch:spec_examples}
to see if any of the examples there are more helpful.

\section{Some more advertising}
Let me describe what the construction of $\Spec A$ is going to do.

In the case of $\Aff^n$, we used $\CC^n$ as the set of points
and $\CC[x_1, \dots, x_n]$ as the ring of functions
but then remarked that the set of points
of $\CC^n$ corresponded to the maximal ideals of $\CC[x_1, \dots, x_n]$.
In an \emph{affine scheme}, we will take an \emph{arbitrary} ring $A$,
and generate the entire structure from just $A$ itself.
The final result is called $\Spec A$, the \vocab{spectrum} of $A$.
The affine varieties $\VV(I)$ we met earlier will just be
$\Spec \CC[\VV(I)] = \Spec \CC[x_1, \dots, x_n] / I$
, but now we will be able to take
\emph{any} ideal $I$, thus finally completing the table at the end
of the ``affine variety'' chapter.

To emphasize the point:
\begin{moral}
	For affine varieties $V$, the spectrum of the coordinate ring $\CC[V]$ is $V$.
\end{moral}
Thus, we may also think of $\Spec$ as the opposite operation of taking the
ring of global sections, defined purely-algebraically in order to depend only on
the intrinsic properties of the affine variety itself (the ring $\OO_V$) and not
the embedding.

The construction of the affine scheme in this way
will have three big generalizations:
\begin{enumerate}
	\ii We no longer have to work over an algebraically
	closed field $\CC$, or even a field at all.
	This will be the most painless generalization:
	you won't have to adjust your current picture much for this to work.

	\ii We allow non-radical ideals:
	$\Spec \CC[x] / (x^2)$ will be the double point
	we sought for so long.
	This will let us formalize the notion of a ``fat'' or ``fuzzy'' point.

	\ii Our affine schemes will have so-called \emph{non-closed points}:
	points which you can visualize as floating around,
	somewhere in the space but nowhere in particular.
	(They'll correspond to prime non-maximal ideals.)
	These will take the longest to get used to,
	but as we progress we will begin to see that these non-closed points
	actually make life \emph{easier},
	once you get a sense of what they look like.
\end{enumerate}

\section{The set of points}
\prototype{$\Spec \CC[x_1, \dots, x_n] / I$.}

First surprise, for a ring $A$:
\begin{definition}
	The set $\Spec A$ is defined as the set of prime ideals of $A$.
\end{definition}

This might be a little surprising, since we might have guessed
that $\Spec A$ should just have the maximal ideals.
What do the remaining ideals correspond to?
The answer is that they will be so-called \emph{non-closed points}
or \emph{generic points} which are ``somewhere'' in the space,
but nowhere in particular.
(The name ``non-closed'' is explained next chapter.)

\begin{remark}
	As usual $A$ itself is not a prime ideal, but $(0)$
	is prime if and only if $A$ is an integral domain.
\end{remark}

\begin{example}
	[Examples of spectrums]
	\listhack
	\begin{enumerate}[(a)]
		\ii $\Spec \CC[x]$ consists of a point $(x-a)$ for every $a \in \CC$,
		which correspond to what we geometrically think of as $\Aff^1$.
		It additionally consists of a point $(0)$,
		which we think of as a ``non-closed point'', nowhere in particular.

		\ii $\Spec \CC[x,y]$ consists of points $(x-a,y-b)$
		(which are the maximal ideals) as well as $(0)$ again,
		a non-closed point that is thought of as ``somewhere in $\CC^2$,
		but nowhere in particular''.
		It also consists of non-closed points corresponding to irreducible
		polynomials $f(x,y)$, for example $(y-x^2)$,
		which is a ``generic point on the parabola''.

		\ii If $k$ is a field, $\Spec k$ is a single point,
		since the only maximal ideal of $k$ is $(0)$.
	\end{enumerate}
\end{example}

\begin{example}
	[Complex affine varieties]
	Let $I \subseteq \CC[x_1, \dots, x_n]$ be an ideal.
	By \Cref{prop:prime_quotient},
	the set \[ \Spec \CC[x_1, \dots, x_n] /I \]
	consists of those prime ideals of $\CC[x_1, \dots, x_n]$
	which contain $I$: in other words, it has a
	point for every closed irreducible subvariety of $\VV(I)$.
	So in addition to the ``geometric points''
	(corresponding to the maximal ideals $(x_1-a_1, \dots, x_n-a_n)$
	we have non-closed points along each of the varieties).
\end{example}

The non-closed points are the ones you are not used to:
there is one for each non-maximal prime ideal
(visualized as ``irreducible subvariety'').
I like to visualize them in my head like a fly:
you can hear it, so you know it is floating \emph{somewhere} in the room,
but as it always moving, you never know exactly where.
So the generic point of $\Spec \CC[x,y]$ corresponding to the prime
ideal $(0)$ is floating everywhere in the plane,
the one for the ideal $(y-x^2)$ floats along the parabola, etc.
\begin{center}
	\includegraphics[scale=0.4]{media/calvin-hobbes-fly.png} \\
	\footnotesize Image from \cite{img:calvin_hobbes_fly}.
\end{center}

\begin{remark}
	[Why don't the prime non-maximal ideals correspond to the whole parabola?]
	We have already seen a geometric reason in \Cref{sec:localize_prime_ideal} earlier:
	localizing a ring at a prime non-maximal ideal gives the functions that may blow up somewhere in the
	parabola, but not \emph{generically}.
\end{remark}

\begin{example}
	[More examples of spectrums]
	\listhack
	\begin{enumerate}[(a)]
		\ii $\Spec \ZZ$ consists of a point for every prime $p$,
		plus a generic point that is somewhere, but nowhere in particular.

		\ii $\Spec \CC[x] / (x^2)$ has only $(x)$ as a prime ideal.
		The ideal $(0)$ is not prime since $0 = x \cdot x$.
		Thus as a \emph{topological space},
		$\Spec \CC[x] / (x^2)$ is a single point.

		\ii $\Spec \Zc{60}$ consists of three points.
		What are they?
	\end{enumerate}
\end{example}

\section{The Zariski topology on the spectrum}
\prototype{Still $\Spec \CC[x_1, \dots, x_n] / I$.}

Now, we endow a topology on $\Spec A$.
Since the points on $\Spec A$ are the prime ideals, we continue
the analogy by thinking of the points $f$ as functions on $\Spec A$. That is:
\begin{definition}
	Let $f \in A$ and $\kp \in \Spec A$.
	Then the \vocab{value} of $f$ at $\kp$ is defined to be $f \pmod{\kp}$,
	an element of $A/\kp$.
	We denote it $f(\kp)$.
\end{definition}
\begin{example}
	[Vanishing locii in $\Aff^n$]
	Suppose $A = \CC[x_1, \dots, x_n]$,
	and $\km = (x_1-a_1, x_2-a_2, \dots, x_n-a_n)$ is a maximal ideal of $A$.
	Then for a polynomial $f \in \CC$,
	\[ f \pmod \km = f(a_1, \dots, a_n) \]
	with the identification that $A/\km \cong \CC$.
\end{example}
\begin{example}
	[Functions on $\Spec \ZZ$]
	Consider $A = \Spec \ZZ$.
	Then $2019$ is a function on $A$.
	Its value at the point $(5)$ is $4 \pmod 5$;
	its value at the point $(7)$ is $3 \pmod 7$.
\end{example}

Indeed if you replace $A$ with $\CC[x_1, \dots, x_n]$
and $\Spec A$ with $\Aff^n$ in everything that follows,
then everything will become quite familiar.

\begin{definition}
	Let $f \in A$. We define the \vocab{vanishing locus} of $f$ to be
	\[ \VV(f) = \left\{ \kp \in \Spec A \mid f(\kp) = 0 \right\}
		= \left\{ \kp \in \Spec A \mid f \in \kp \right\}. \]
	More generally, just as in the affine case,
	we define the vanishing locus for an ideal $I$ as
	\begin{align*}
		\VV(I) &= \left\{ \kp \in \Spec A \mid f(\kp)=0 \; \forall f \in I \right\} \\
		&= \left\{ \kp \in \Spec A \mid f \in \kp \; \forall f \in I \right\} \\
		&= \left\{ \kp \in \Spec A \mid I \subseteq \kp \right\}.
	\end{align*}
	Finally, we define the \vocab{Zariski topology} on $\Spec A$
	by declaring that the sets of the form $\VV(I)$ are closed.
\end{definition}

We now define a few useful topological notions:
\begin{definition}
	Let $X$ be a topological space.
	A point $p \in X$ is a \vocab{closed point}
	if the set $\{p\}$ is closed.
\end{definition}
\begin{ques}
	[Mandatory]
	Show that a point (i.e.\ prime ideal)
	$\km \in \Spec A$ is a closed point
	if and only if $\km$ is a maximal ideal.
\end{ques}
Recall also in \Cref{def:closure} we denote by $\ol S$
the closure of a set $S$ (i.e.\ the smallest closed set containing $S$);
so you can think of a closed point $p$ also
as one whose closure is just $\{p\}$.
Therefore the Zariski topology lets us refer back to the old ``geometric''
as just the closed points.

\begin{example}
	[Non-closed points, continued]
	Let $A = \CC[x,y]$ and let $\kp = (y-x^2) \in \Spec A$;
	this is the ``generic point'' on a parabola.
	It is not closed, but we can compute its closure:
	\[
		\ol{\{\kp\}}
		= \VV(\kp) = \left\{ \kq \in \Spec A \mid \kq \supseteq \kp \right\}.
	\]
	This closure contains the point $\kp$ as well
	as several maximal ideals $\kq$, such as $(x-2,y-4)$ and $(x-3,y-9)$.
	In other words, the closure of the ``generic point'' of the parabola
	is literally the set of all points that are actually on the parabola
	(including generic points).

	That means the way to picture $\kp$ is a point that
	is floating ``somewhere on the parabola'', but nowhere in particular.
	It makes sense then that if we take the closure,
	we get the entire parabola,
	since $\kp$ ``could have been'' any of those points.
\end{example}

\begin{center}
\begin{asy}
	graph.xaxis();
	graph.yaxis();
	real f(real x) { return x*x; }
	graph.xaxis("$x$");
	graph.yaxis("$y$");
	draw(graph(f,-2,2,operator ..), red+dotted, Arrows(TeXHead));
	dot("$(y-x^2)$", (1.4, f(1.4)), dir(-45), red);
	dot("$(x+1,y-1)$", (-1,1), dir(225), blue);
\end{asy}
\end{center}

\begin{example}
	[The generic point of the $y$-axis isn't on the $x$-axis]
	Let $A = \CC[x,y]$ again.
	Consider $\VV(y)$, which is the $x$-axis of $\Spec A$.
	Then consider $\kp = (x)$, which is the generic point on the $y$-axis.
	Observe that
	\[ \kp \notin \VV(y). \]
	The geometric way of saying this is that a \emph{generic point}
	on the $y$-axis does not lie on the $x$-axis.
\end{example}

We now also introduce one more word:
\begin{definition}
	A topological space $X$ is \vocab{irreducible}
	if either of the following two conditions hold:
	\begin{itemize}
		\ii The space $X$ cannot be written as the
		union of two proper closed subsets.
		\ii Any two nonempty open sets of $X$ intersect.
	\end{itemize}
	A subset $Z$ of $X$ (usually closed) is irreducible
	if it is irreducible as a subspace.
\end{definition}
\begin{exercise}
	Show that the two conditions above are indeed equivalent.
	Also, show that the closure of a point is always irreducible.
\end{exercise}

This is the analog of the ``irreducible'' we defined for affine varieties,
but it is now a topological definition,
although in practice this definition is only useful for spaces with the Zariski topology.
Indeed, if any two nonempty open sets intersect (and there is more than one point),
the space is certainly not Hausdorff!
As with our old affine varieties,
the intuition is that $\VV(xy)$ (the union of two lines)
should not be irreducible.

\begin{example}
	[Reducible and irreducible spaces]
	\listhack
	\begin{enumerate}[(a)]
		\ii The closed set $\VV(xy) = \VV(x) \cup \VV(y)$ is reducible.
		\ii The entire plane $\Spec \CC[x,y]$ is irreducible.
		There is actually a simple (but counter-intuitive,
		since you are just getting used to generic points)
		reason why this is true:
		the generic point $(0)$ is in \emph{every} open set,
		ergo, any two open sets intersect.
	\end{enumerate}
\end{example}

So actually, the generic points
kind of let us cheat our way through the following bit:
\begin{proposition}
	[Spectrums of integral domains are irreducible]
	If $A$ is an integral domain, then $\Spec A$ is irreducible.
\end{proposition}
\begin{proof}
	Just note $(0)$ is a prime ideal, and is in every open set.
\end{proof}
You should compare this with our old classical result that
$\CC[x_1, \dots, x_n]/I$
was irreducible as an affine variety exactly when $I$ was prime.
This time, the generic point actually takes care of the work for us:
the fact that it is \emph{allowed} to float
anywhere in the plane lets us capture the idea that
$\Aff^2$ should be irreducible
without having to expend any additional effort.
\begin{remark}
	Surprisingly, the converse of this proposition is false:
	we have seen $\Spec \CC[x]/(x^2)$ has only one point,
	so is certainly irreducible.
	But $A = \CC[x]/(x^2)$ is not an integral domain.
	So this is one weird-ness introduced by allowing ``non-radical'' behavior.
\end{remark}

At this point you might notice something:
\begin{theorem}
	[Points are in bijection with irreducible closed sets]
	Consider $X = \Spec A$.
	For every irreducible closed set $Z$,
	there is exactly one point $\kp$ such that $Z = \ol{\{\kp\}}$.
	(In particular points of $X$ are in bijection
	with closed subsets of $X$.)
\end{theorem}
\begin{proof}
	[Idea of proof]
	The point $\kp$ corresponds to the closed set $\VV(\kp)$,
	which one can show is irreducible.
	% Maybe I really should prove this here,
	% but I don't really want to draw to much attention to radicals yet;
	% there's too much going on already.
\end{proof}
This gives you a better way to draw non-closed points:
they are the generic points lying along any irreducible closed set
(consisting of more than just one point).

At this point,\footnote{Pun not intended}
I may as well give you the real definition of generic point.
\begin{definition}
	Given a topological space $X$, a \vocab{generic point} $\eta$
	is a point whose closure is the entire space $X$.
\end{definition}
So for us, when $A$ is an integral domain,
$\Spec A$ has generic point $(0)$.
\begin{abuse}
	Very careful readers might note I am being a little careless
	with referring to $(y-x^2)$ as
	``the generic point along the parabola''
	in $\Spec \CC[x,y]$.
	What's happening is that $\VV(y-x^2)$ is a closed set,
	and as a topological subspace, it has generic point $(y-x^2)$.
\end{abuse}

\section{Krull dimension}
Surprisingly, the topology is good enough to give dimension.
For affine schemes:
\begin{definition}
	Let $A$ be a commutative ring.
	Consider chains of prime ideals
	$\kp_0 \subsetneq \kp_1 \subsetneq \dots \subsetneq \kp_n \subseteq A$,
	where $n$ is called the \emph{length}.
	The supremum of all possible $n$ is the called \vocab{Krull dimension} of $A$.

	The Krull dimension is always nonnegative unless $A$ is the zero ring,
	in which case either $-1$ or $-\infty$ are used conventionally.
\end{definition}

This definition should match your intuition.
\begin{example}
	[Examples of Krull dimension]
	\listhack
	\begin{enumerate}[(a)]
		\ii $\CC[x_1, \dots, x_n]$ has Krull dimension $n$, with the chain
		\[ (0) \subsetneq (x_1) \subsetneq (x_1, x_2) \subsetneq \dots \subsetneq (x_1, \dots, x_n) \]
		having length $n$.
		This matches our expectation that $\Spec \CC[x_1, \dots, x_n]$ corresponds to $\Aff^n$.
		\ii $\CC[x,y] / (y-x^2)$ has Krull dimension $1$, with the chain
		\[ (x,y) \subsetneq (y-x^2) \]
		having length $1$.
		Geometrically, we think of $(x,y)$ as the origin and $(y-x^2)$ as the parabola itself.
		\ii $\ZZ$ has Krull dimension $1$.
		\ii $\ZZ/(60)$ has Krull dimension $0$; it's just three points.
	\end{enumerate}
\end{example}

You can do this more generally with a topological space $X$:
the Krull dimension of a space is the supremum of chains of irreducible closed subspaces
$Z_0 \subsetneq Z_1 \subsetneq \dots \subsetneq Z_n$.
You'd only want to use this definition in situations
where $X$ had a Zariski topology:
in particular, if $X = \Spec A$, this is just the Krull dimension of the ring $A$ itself.

\section{On radicals}
Back when we studied classical algebraic geometry in $\CC^n$,
we saw Hilbert's Nullstellensatz (\Cref{thm:hilbert_null}) show up
to give bijections between radical ideals and affine varieties;
we omitted the proof, because it was nontrivial.

However, for a \emph{scheme}, where the points \emph{are} prime ideals
(rather than tuples in $\CC^n$),
the corresponding results will actually be \emph{easy}:
even in the case where $A = \CC[x_1, \dots, x_n]$,
the addition of prime ideals (instead of just maximal ideals)
will actually \emph{simplify} the proof,
because radicals play well with prime ideals.

We still have the following result.
\begin{proposition}
	[$\VV(\sqrt I) = \VV(I)$]
	For any ideal $I$ of a ring $A$
	we have $\VV(\sqrt I) = \VV(I)$.
\end{proposition}
\begin{proof}
	We have $\sqrt I \supseteq I$.
	Hence automatically $\VV(\sqrt I) \subseteq \VV(I)$.

	Conversely, if $\kp \in \VV(I)$, then $I \subseteq \kp$,
	so $\sqrt I \subseteq \sqrt{\kp} = \kp$
	(by \Cref{prop:radical}).
\end{proof}

We hinted the key result in an earlier remark,
and we now prove it.
\begin{theorem}
	[Radical is intersection of primes]
	\label{thm:radical_intersect_prime}
	Let $I$ be an ideal of a ring $A$.
	Then \[ \sqrt I = \bigcap_{\kp \supseteq I} \kp. \]
\end{theorem}
\begin{proof}
	This is a famous statement from commutative algebra,
	and we prove it here only for completeness.
	It is ``doing most of the work''.

	Note that if $I \subseteq \kp$,
	then $\sqrt I \subseteq \sqrt{\kp} = \kp$;
	thus $\sqrt I \subseteq \bigcap_{\kp \supseteq I} \kp$.

	Conversely, suppose $x \notin \sqrt I$,
	meaning $1, x, x^2, x^3, \dots \notin I$.
	Then, consider the localization $(A/I)[1/x]$, which is not the zero ring.
	Like any ring, it has some maximal ideal (Krull's theorem).
	This means our usual bijection between prime ideals of $(A/I)[1/x]$,
	prime ideals of $A/I$ and prime ideals of $A$
	gives some prime ideal $\kp$ of $A$ containing $I$ but not containing $x$.
	Thus $x \notin \bigcap_{\kp \supseteq I} \kp$,
	as desired.

	The key idea here is, for $x \in A$,
	$x^n = 0$ for some positive \emph{finite} integer $n$ if and only if $A[1/x] = 0$.

	So, in other words,
	\begin{align*}
		& x \in \sqrt{(0)} \\
		& \iff \text{$x^n = 0$ for some positive integer $n$} \\
		& \iff A[1/x] = 0 \\
		& \iff \text{for all prime ideals $\kp$, $x \in \kp$} \\
		& \iff x \in \bigcap_\kp \kp.
	\end{align*}

	When $I$ is not $(0)$, consider the ring $A/I$ instead.
\end{proof}

Geometrically speaking, this theorem states:
\begin{quote}
	For any $f$ a regular function on $\Spec A/I$, then
	\[
		\text{$f^n = 0$ for some positive integer $n$}
		\iff \text{$f$ vanishes at all points in $\Spec A/I$}.
	\]
\end{quote}
To which, the proof above reads:
\begin{align*}
	& f \in \sqrt{(I)} \\
	& \iff \text{$f^n \in I$ for some positive integer $n$} \\
	& \iff (A/I)[1/f] = 0 \\
	& \iff \text{$\Spec (A/I)[1/f]$ is empty} \\
	& \iff \text{for all $\kp \in \Spec A/I$, $f$ vanishes at $\kp$} \\
	& \iff f \in \bigcap_{\kp \in \Spec A/I} \kp.
\end{align*}

You may want to run through the proof with the example $A = k[x]$, $I = (x^2)$ and $f = x$
in \Cref{sec:spec_double_point},
keeping in mind the image of $\Spec A/I$ as a ``fuzzy'' point and $f$ being a nonzero function that
takes value zero at every point.

\begin{remark}
	[A variant of Krull's theorem]
	The longer direction of this proof is essentially
	saying that for any $x \in A$,
	there is a maximal ideal of $A$ not containing $x$.
	The ``short'' proof is to use Krull's theorem on $(A/I)[1/x]$ as above,
	but one can also still prove it directly using Zorn's lemma
	(by copying the proof of the original Krull's theorem).
\end{remark}

\begin{example}
	[$\sqrt{(2016)} = (42)$ in $\ZZ$]
	In the ring $\ZZ$, we see that $\sqrt{(2016)} = (42)$,
	since the distinct primes containing $(2016)$
	are $(2)$, $(3)$, $(7)$.
\end{example}

Geometrically, this gives us a good way to describe $\sqrt I$:
it is the \emph{set of all functions vanishing on all of $\VV(I)$}.
Indeed, we may write
\[ \sqrt I = \bigcap_{\kp \supseteq I} \kp
	= \bigcap_{\kp \in \VV(I)} \kp
	= \bigcap_{\kp \in \VV(I)} \left\{ f \in A \mid f(\kp) = 0 \right\}. \]

We can now state:
\begin{theorem}
	[Radical ideals correspond to closed sets]
	Let $I$ and $J$ be ideals of $A$,
	and considering the space $\Spec A$.
	Then
	\[ \VV(I) = \VV(J) \iff \sqrt I = \sqrt J. \]
	In particular, radical ideals exactly
	correspond to closed subsets of $\Spec A$.
\end{theorem}
\begin{proof}
	If $\VV(I) = \VV(J)$,
	then $\sqrt I = \bigcap_{\kp \in \VV(I)} \kp =
	\bigcap_{\kp \in \VV(J)} \kp = \sqrt J$ as needed.

	Conversely, suppose $\sqrt I = \sqrt J$.
	Then $\VV(I) = \VV(\sqrt I) = \VV(\sqrt J) = \VV(J)$.
\end{proof}

Compare this to the theorem we had earlier
that the \emph{irreducible} closed subsets correspond to \emph{prime} ideals!

\section{\problemhead}
As \Cref{ch:spec_examples} contains many
examples of affine schemes to train your intuition,
it's possibly worth reading even before attempting these problems,
even though there will be some parts that won't make sense yet.

\begin{problem}
	[{$\Spec \QQ[x]$}]
	Describe the points and topology of $\Spec \QQ[x]$.
	\begin{hint}
		Galois conjugates.
	\end{hint}
\end{problem}

\begin{problem}
	[Product rings]
	Describe the points and topology of $\Spec A \times B$
	in terms of $\Spec A$ and $\Spec B$.
\end{problem}

\begin{sproblem}
	[How to actually think about Artinian rings]
	\label{prob:artinian}
	Let $A$ be a Noetherian ring.
	Prove that all of the following are equivalent:
	\begin{enumerate}[(i)]
		\ii $A$ is Artinian, i.e.\ satisfies the descending chain condition
		(as defined in \Cref{prob:akizuki}).
		\ii $A$ has Krull dimension $0$ or $A$ is the zero ring.
		\ii $\Spec A$ is finite and discrete.
		\ii $\Spec A$ is finite.
	\end{enumerate}
\end{sproblem}

\endinput

\begin{problem}
	[From Andrew Critch]
	\onechili
	Let $A$ be a Noetherian ring.
	Show that $A$ is an integral domain if and only if it has no idempotents,
	and $A_\kp$ is an integral domain for every prime $\kp$.
	\begin{hint}
		Show that if $\Spec A$ is connected and its stalks are irreducible,
		then $\Spec A$ is itself irreducible.
		Consider nilradical $N = \sqrt{(0)}$.
	\end{hint}
	\begin{sol}
		This is the proposition on the second page of
		\url{https://www.acritch.com/media/math/Stalk-local_detection_of_irreducibility.pdf}
	\end{sol}
\end{problem}
