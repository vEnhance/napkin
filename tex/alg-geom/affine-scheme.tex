\chapter{Defining an affine scheme}
Now that we understand sheaves well,
we can define an affine scheme.
It will be a ringed space, so we need to define
\begin{itemize}
	\ii The set of points,
	\ii The topology on it, and
	\ii The structure sheaf on it.
\end{itemize}
\todo{change $R$ to $A$ everywhere}

\section{Some more advertising}
Let me describe what the construction of $\Spec A$ is going to do.

In the case of $\Aff^n$, we used $\CC^n$ as the set of points
and $\CC[x_1, \dots, x_n]$ as the ring of functions
but then remarked that the set of points
of $\CC^n$ corresponded to the maximal ideals of $\CC[x_1, \dots, x_n]$.
In an \emph{affine scheme}, we will take an \emph{arbitrary} ring $R$,
and generate the entire structure from just $R$ itself.
The final result is called $\Spec R$, the \vocab{spectrum} of $R$.
The affine varieties $\VV(I)$ we met earlier will just be
$\CC[x_1, \dots, x_n] / I$, but now we will be able to take
\emph{any} ideal $I$, thus finally completing the table at the end
of the ``affine variety'' chapter.

The construction of the affine scheme in this way
will have three big generalizations:
\begin{enumerate}
	\ii We no longer have to work over an algebraically
	closed field $\CC$, or even a field at all.
	This will be the most painless generalization:
	you won't have to adjust your current picture much for this to work.

	\ii We allow non-radical ideals:
	$\Spec \CC[x] / (x^2)$ will be the double point
	we sought for so long.
	This will let us formalize the notion of a ``fat'' or ``fuzzy'' point.

	\ii Our affine schemes will have so-called \emph{generic points}:
	points which you can visualize as floating around,
	somewhere in the space but nowhere in particular.
	(They'll correspond to prime non-maximal ideals.)
	These will take the longest to get used to,
	but once we discuss morphisms of schemes
	we will begin to see that these generic points actually make life
	\emph{easier}, once you get a sense of what they look like.
\end{enumerate}

\section{The set of points}
\prototype{$\Spec \CC[x_1, \dots, x_n] / I$.}

First surprise, for a ring $R$:
\begin{moral}
	$\Spec R$ is defined as the set of prime ideals of $R$.
\end{moral}

This might be a little surprising, since we might have guessed
that $\Spec R$ should just have the maximal ideals.
What do the remaining ideals correspond to?
The answer is that they will be so-called \emph{generic points}
points which are ``somewhere'' in the space, but nowhere in particular.

\begin{remark}
	As usual $R$ itself is not a prime ideal, but $(0)$
	is prime if $R$ is an integral domain.
\end{remark}

\begin{example}
	[Examples of spectrums]
	\listhack
	\begin{enumerate}[(a)]
		\ii $\Spec \CC[x]$ consists of a point $(x-a)$ for every $a \in \CC$,
		which correspond to what we geometrically think of as $\Aff^1$.
		In additionally consists of a point $(0)$,
		which we think of as a ``generic point'', nowhere in particular.

		\ii $\Spec \CC[x,y]$ consists of points $(x-a,y-b)$
		(which are the maximal ideals) as well as $(0)$ again, a generic
		point that is thought of as ``somewhere in $\CC^2$,
		but nowhere in particular''.
		It also consists of generic points corresponding to irreducible
		polynomials $f(x,y)$, for example $(y-x^2)$,
		which is a ``generic point on the parabola''.

		\ii If $k$ is a field, $\Spec k$ is a single point,
		since the only maximal ideal of $k$ is $(0)$.
	\end{enumerate}
\end{example}
\begin{example}
	[Complex affine varieties]
	Let $I \subseteq \CC[x_1, \dots, x_n]$ be an ideal.
	Then \[ \Spec \CC[x_1, \dots, x_n] /I \] contains a
	point for every closed irreducible subvariety of $\VV(I)$.
	So in addition to the ``geometric points'' we have
	``generic points'' along each of the varieties.
\end{example}
\begin{example}
	[More examples of spectrums]
	\listhack
	\begin{enumerate}[(a)]
		\ii $\Spec \ZZ$ consists of a point for every prime $p$,
		plus a generic point that is somewhere, but no where in particular.

		\ii $\Spec \CC[x] / (x^2)$ has only $(x)$ as a prime ideal.
		The ideal $(0)$ is not prime since $0 = x \cdot x$.
		Thus as a \emph{topological space},
		$\Spec \CC[x] / (x^2)$ is a single point.
		
		\ii $\Spec \Zc{60}$ consists of three points.
		What are they?
	\end{enumerate}
\end{example}

\section{The Zariski topology of the spectrum}
\prototype{Still $\Spec \CC[x_1, \dots, x_n] / I$.}

Now, we endow a topology on $\Spec R$.
Since the points on $\Spec R$ are the prime ideals, we continue
the analogy by thinking of the points $f$ as functions on $\Spec R$. That is,
\begin{definition}
	Let $f \in R$ and $\pp \in \Spec R$.
	Then the \vocab{value} of $f$ at $\pp$ is defined to be $f \pmod{\pp}$.
	We denote it $f(\pp)$.
\end{definition}
\begin{example}
	[Vanishing locii in $\Aff^n$]
	Suppose $R = \CC[x_1, \dots, x_n]$,
	and $\pp = (x_1-a_1, x_2-a_2, \dots, x_n-a_n)$ is a maximal ideal of $R$.
	Then for a polynomial $f \in \CC$,
	\[ f \pmod \pp = f(a_1, \dots, a_n) \]
	with the identification that $\CC/\pp \cong \CC$.
\end{example}
Indeed if you replace $R$ with $\CC[x_1, \dots, x_n]$
and $\Spec R$ with $\Aff^n$ in everything that follows,
then everything will be clear.

\begin{definition}
	Let $f \in R$. We define the \vocab{vanishing locus} of $f$ to be
	\[ \VV(f) = \left\{ \pp \in \Spec R \mid f(\pp) = 0 \right\}
		= \left\{ \pp \in \Spec R \mid f \in \pp \right\}. \]
	More generally, just as in the affine case,
	we define the vanishing locus for an ideal $I$ as
	\begin{align*}
		\VV(I) &= \left\{ \pp \in \Spec R \mid f(\pp)=0 \forall f \in I \right\} \\
		&= \left\{ \pp \in \Spec R \mid f \in \pp \; \forall f \in I \right\} \\
		&= \left\{ \pp \in \Spec R \mid I \subseteq \pp \right\}.
	\end{align*}
	Finally, we define the \vocab{Zariski topology} on $\Spec R$
	by declaring that the sets of the form $\VV(I)$ are closed.
\end{definition}

We now define a useful topological notion:
\begin{definition}
	A point $p \in X$ is a \vocab{closed point} if the set $\{p\}$ is closed.
\end{definition}
\begin{ques}
	[Important]
	Show that $\pp \in \Spec R$ is a closed point
	if and only if $\pp$ is a maximal ideal.
\end{ques}
Therefore the Zariski topology lets us refer back to the old ``geometric''
as just the closed points.
\begin{example}
	[Generic points, continued]
	Let $R = \CC[x,y]$ and let $\pp = (y-x^2) \in \Spec \RR$;
	this is the ``generic point'' on a parabola.
	It is not closed, but we can compute its closure:
	\[
		\ol{\{\pp\}}
		= \VV(\pp) = \left\{ \qq \in \Spec R \mid \qq \supseteq \pp \right\}.
	\]
	This closure contains the point $\pp$ as well
	as several maximal ideals $\qq$, such as $(x-2,y-4)$ and $(x-3,y-9)$.
	In other words, the closure of the ``generic point'' of the parabola
	is literally the set of all points that are actually on the parabola
	(including generic points).

	That means the way to picture $\pp$ is a point that 
	is ``somewhere on the parabola'', but nowhere in particular.
	It makes sense then that if we take the closure,
	we get the entire parabola,
	since $\pp$ ``could have been'' any of those points.
\end{example}

\begin{example}
	[The generic point of the $y$-axis isn't on the $x$-axis]
	Let $R = \CC[x,y]$ again.
	Consider $\VV(y)$, which is the $x$-axis of $\Spec R$.
	Then consider $\pp = (x)$, which is the generic point on the $y$-axis.
	Observe that
	\[ \pp \notin \VV(y). \]
	The geometric way of saying this is that a \emph{generic point}
	on the $y$-axis does not lie on the $x$-axis.
\end{example}

\section{The structure sheaf}
\prototype{Still $\CC[x_1, \dots, x_n] / I$.}

We have now endowed $\Spec R$ with the Zariski topology,
and so all that remains is to put a sheaf $\OO_{\Spec R}$ on it.
To do this we want a notion of ``regular functions'' as before.

In order for this to make sense, we have to talk about rational
functions in a meaningful way.
If $R$ was an integral domain, then we could just use the field of fractions;
for example if $R = \CC[x_1, \dots, x_n]$ then we could just
look at rational quotients of polynomials.

Unfortunately, in the general situation $R$ may not be an integral domain:
for example, the ring $R = \CC[x] / (x^2)$ corresponding to the double point.
So we will need to define something a little different:
we will construct the \emph{localization} of $R$ at a set $S$,
which we think of as the ``set of allowed denominators''.
The most important special case is the localization at a prime ideal.

Now, we can define the sheaf as ``locally rational'' functions.
This is done by a sheafification.
First, let $\SF$ be the pre-sheaf of ``globally rational'' functions:
i.e.\ we define $\SF(U)$ to be
\[
	\SF(U) = \left\{
		\frac fg \mid f, g \in R
		\text{ and } g(\pp) \neq 0 \; \forall \pp \in U
	\right\}
	= \left(R \setminus \bigcup_{\pp \in U} \pp \right)\inv R.
\]
This is the localization of $R$ to the functions vanishing outside $U$.
For every $\pp \in U$ we can view $f/g$ as an element in $R_\pp$
(since $g(\pp) \ne 0$).

As promised:
\begin{lemma}[Stalks of the ``globally rational'' pre-sheaf]
	\label{lem:global_rational_stalk}
	The stalk of $\SF$ defined above $\SF_\pp$ is isomorphic to
	the localization $R_\pp$.
\end{lemma}
\begin{proof}
	There is an obvious map $\SF_\pp \to R_\pp$ on germs by
	\[
		\left(U, f/g \in \SF(U) \right)
		\mapsto f/g \in R_\pp . \]
	(Note the $f/g$ on the left lives in
	$\SF(U) = \left(R \setminus \bigcup_{\pp \in U} \pp \right)\inv R$
	but the one on the right lives in $R_\pp$).
	Now suppose $(U_1, f_1 / g_1)$ and $(U_2, f_2 / g_2)$
	are germs with $f_1/g_1 = f_2/g_2 \in R_\pp$.
	\begin{exercise}
		Now, show both germs are equal to $(U_1 \cap U_2 \cap D(h), f_1 / g_1)$
		with $D(h)$ the distinguished open set.
	\end{exercise}
	It is also surjective, since given $f/g \in R_\pp$ we take $U = D(g)$
	the distinguished open set for $g$.
\end{proof}

\begin{definition}
	We now define the structure sheaf on $\Spec A$.
	It is
	\[ \OO_{\Spec R} = \SF\sh \]
	i.e.\ the sheafification of the $\SF$ we just defined.
\end{definition}
\begin{theorem}
	[Stalks are localizations]
	We have \[ \OO_{\Spec A, \pp} \cong A_\pp. \]
	In particular $\Spec A$ is a locally ringed space.
\end{theorem}
\begin{proof}
	Stalks are preserved by sheafification,
	sot his follows from the earlier lemma.
\end{proof}

In fact, we can even write out the definition of the sheafification,
by viewing the germ at each point as an element of $R_\pp$.
\begin{definition}
	Let $R$ be a ring. Then $\Spec R$ is made into a ringed space by setting
	\[ \OO_{\Spec R}(U)
		= \left\{ (f_\pp \in R_\pp)_{\pp \in U}
		\text{ which are locally $f/g$} \right\}. \]
	That is, it consists of sequence $(f_\pp)_{\pp \in U}$, with
	each $f_\pp \in R_\pp$, such that for every point $\pp$ there
	is a neighborhood $U_\pp$ and an $f,g \in R$ such that
	$f_\qq = \frac fg \in R_\qq$ for all $\qq \in U_\pp$.
\end{definition}

\begin{example}
	[Examples of structure sheaves]
	\listhack
	\begin{enumerate}[(a)]
		\ii Let $X = \Spec \CC[x]$.
		Then $\OO_X(U)$ is the set of regular functions on $U$
		in the sense that we've seen before.
		\ii Let $X = \Spec \ZZ$.
		Let $U = X \setminus \{ (7), (11) \} = D(77)$.
		Then $\OO_X(U)$ can be identified with the set of rational numbers
		which have denominator of the form $7^x 11^y$.
	\end{enumerate}
\end{example}


\section{Example: fat points}
So, let me elaborate a little on the ``double point'' scheme
\[ X_2 = \Spec \CC[x] / (x^2) \]
since it is such an important motivating example.
How it does differ from the ``one-point'' scheme $X_1 = \Spec \CC[x] / (x)$?

The difference can only be seen on the level of the structure sheaves.
Indeed,
\begin{itemize}
	\ii As a set of points, $X_2$ has only one point, call it $p_2 = (x)$.
	Similarly, $X_1$ has only one point, call it $p_1 = (0)$.
	\ii The two Zariski topologies are the same
	(at any rate there is only one topology on a one-point space).
	\ii But at the structure sheaf level, $X_2$ has a ``bigger'' sheaf:
	the ring of functions on the single point $p_2$ is instead
	a \emph{two-dimensional} $\CC$-vector space $\CC[x]/(x^2)$.
	This formalizes the notion that this point is ``fat'':
	specifying a function from $p_2$ to $\CC$ now gives you
	\emph{two} degrees of freedom instead of just one.
\end{itemize}

Another way to think about is in terms of functions.
Consider polynomials $f = a_0 + a_1x + a_2x^2 + \dots$ on $\CC[x]$.
Then we have a sequence of maps
\begin{diagram}
	\CC[x] & \rTo & \CC[x]/(x^2) & \rTo & \CC[x] / (x) \\
	f & \rMapsto & a_0 + a_1x & \rMapsto & a_0
\end{diagram}
So $p_1$ only remembers the value of $f$, i.e.\ it remembers $f(0)$.
But the point $p_2$ remembers more than just the value of $f$:
it also remembers the first derivative of $f$.
In \cite{ref:vakil} one draws a picture of this by taking $0 \in \CC$
and adding a little bit of ``infinitesimal fuzz'' around it.

One can play the analogy more.
There's a ``triple point'' $X_3 = \Spec \CC[x] / (x^3) = \{p_3\}$
whose ring of functions has three degrees of freedom:
specifying a ``function'' on $p_3$ gives you three degrees of freedom.
Analogously, it remembers both the first and second derivatives 
of any polynomial in $\CC[x]$.
In \cite{ref:vakil}, this is ``the point $0$ with even more fuzz''.
Going even further,
$\CC[x,y] / (x^2,y)$ is ``the origin with fuzz in the $x$-direction'',
$\CC[x,y] / (x,y)^2$ is ``the origin with fuzz in all directions'',
and so on and so forth.

\section{The importance of distinguished open sets}
\prototype{Let $X = \Spec \CC[x] = \Aff^1$ and $U = D(x)$,
the line minus the origin.}

We will now really hammer in the importance of
the distinguished open sets.
First the definition, analogous to before:
\begin{definition}
	Let $f \in \Spec R$.
	Then $D(f)$ is the set of $\pp$ such that $f(\pp) \neq 0$,
	a \vocab{distinguished open set}.
\end{definition}
Open sets will have three absolutely crucial properties.

The first is topological:
\begin{theorem}
	[Distinguished open sets form a base]
	The distinguished open sets $D(f)$
	form a basis for the Zariski topology:
\end{theorem}
\begin{proof}
	\todo{to be written}
\end{proof}

The second critical fact is that we can compute the relevant sections.
\begin{theorem}
	Let $X = \Spec A$ and $f \in A$.
	Then \[ \OO_X(D(f)) \cong A_f. \]
\end{theorem}
\begin{proof}
	Part (a) follows by \Cref{lem:global_rational_stalk}
	and \Cref{lem:pre_sheaf_stalk}.
	For part (b), we need:
	\begin{ques}
		If $I$ and $J$ are ideals of a ring $R$,
		then $\VV(I) \subseteq \VV(J)$ if and only if
		$\sqrt{J} \subseteq \sqrt{I}$.
		(Use the fact that $\sqrt I = \cap_{I \subseteq \pp} \pp$.)
	\end{ques}
	Then we can repeat the proof of \Cref{thm:reg_func_distinguish_open}.
\end{proof}

\begin{example}
	[The punctured line is isomorphic to a hyperbola]
\end{example}
\todo{write}

The previous two results answer an important prayer:
how we can forget about sheafification.
That's because the distinguished open sets form a base,
with $D(f) \cap D(g) = D(fg)$,
and because we know $\OO_X(D(f))$ for every $f$,
we have ourselves a sheaf on a base.
Therefore:
\begin{moral}
	We can compute any section $\OO_X(U)$ in practice
	by using distinguished open sets.
\end{moral}
And thus we can forget about the original definition of $\OO_X$
which involved sheafification.
Here is a classical example.
\begin{example}
	[Punctured plane]
\end{example}
\todo{write me}


The third final important fact is that
$D(f)$ will actually be \emph{isomorphic} to $\Spec A_f$
as a locally ringed space,
just like the line minus the origin is isomorphic to the hyperbola.
We can't make this precise yet,
because we have not yet discussed morphisms of locally ringed space.
However, you can already see this at the level of topological spaces;
see \Cref{prob:homeomorphism}.

\section{Recap, and a gazillion examples}
To recap, let $A$ be a ring.
\begin{itemize}
	\ii We define $X = \Spec A$ to be the set of prime ideals of $A$.
	\begin{itemize}
		\ii The maximal ideals are the ``closed points'' we are used to,
		but the prime ideals are ``generic points''.
	\end{itemize}

	\ii We equip $\Spec A$ with the Zariski topology by declaring
	$\VV(I)$ to be the closed sets, for ideals $I \subseteq A$.
	\begin{itemize}
		\ii The distinguished open sets $D(f)$,
		the complements of $\VV(f)$, form a topological basis.
	\end{itemize}

	\ii Finally, we defined a sheaf $\OO_X$.
	We set up the definition such that
	\begin{itemize}
		\ii $\OO_{X,\pp} = A_\pp$:
		the stalks are localizations at a prime.
		\ii $\OO_{X}(D(f)) = A_f$:
		at distinguished open sets $D(f)$,
		we get localizations too.
	\end{itemize}
	Since $D(f)$ is a basis,
	these two properties lets us explicitly compute $\OO_X(U)$
	for any open set $U$,
	so we don't have to resort to the definition using sheafification.
\end{itemize}

We now give an enormous list of examples.
Each example gets its own subsection,
rather than having page-long orange boxes.

\subsection{$\Spec \CC[x]$}
\subsection{$\Spec \RR$}
\subsection{$\Spec \ZZ$}
\subsection{$\Spec \Zc{60}$}
\subsection{$\Spec k[x]/(x^2)$}
\subsection{$\Spec k[x]_{(x)}$}


\section{Two mandatory algebra exercises}

\todo{write this}
The previous example illustrates an important observation:
\begin{exercise}
	Let $I \subsetneq R$ be a proper ideal.
	Construct a bijection between the maximal ideals of $R$ containing $I$,
	and the maximal ideals of $R/I$.
\end{exercise}

(\cite{ref:vakil} labels this exercise as
``Essential Algebra Exercise (Mandatory if you haven't done it before)''.)


\section{Irreducible closed sets}
biject them to generic points

\section\problemhead

\begin{problem}
	\label{prob:homeomorphism}
	Let $A$ be a ring, $X = \Spec A$ and $f \in A$.
	Show that $\Spec A_f$ and $D(f) \subset X$ are homeomorphic,
	as topological spaces.
\end{problem}

\begin{problem}
	Describe the points of $\Spec \RR[x]$.
	\begin{hint}
		Galois conjugates.
	\end{hint}
\end{problem}

\begin{dproblem}
	[Chinese remainder theorem]
	Consider $X = \Spec \Zc{60}$, which as a topological space has three points.
	By considering $\OO_X(X)$ prove the Chinese theorem
	\[ \Zc{60} \cong \Zc{4} \times \Zc{3} \times \Zc{5}. \]
	\begin{hint}
		Appeal to \Cref{prob:finite_sheaf}.
	\end{hint}
\end{dproblem}

\begin{problem}
	[From Andrew Critch]
	\gim
	Let $R$ be a Noetherian ring.
	Show that $R$ is an integral domain if and only if it has no idempotents,
	and $R_\pp$ is an integral domain for every prime $\pp$.
	\begin{hint}
		Show that if $\Spec R$ is connected and its stalks are irreducible,
		then $\Spec R$ is itself irreducible.
		Consider nilradical $N = \sqrt{(0)}$.
	\end{hint}
	\begin{sol}
		This is the proposition on the second page of
		\url{http://www.acritch.com/media/math/Stalk-local_detection_of_irreducibility.pdf}
	\end{sol}
\end{problem}
