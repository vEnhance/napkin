\chapter{Defining an affine scheme}
Now that we understand sheaves well,
we can define an affine scheme.
It will be a ringed space, so we need to define
\begin{itemize}
	\ii The set of points,
	\ii The topology on it, and
	\ii The structure sheaf on it.
\end{itemize}

\section{Some more advertising}
Let me describe what the construction of $\Spec A$ is going to do.

In the case of $\Aff^n$, we used $\CC^n$ as the set of points
and $\CC[x_1, \dots, x_n]$ as the ring of functions
but then remarked that the set of points
of $\CC^n$ corresponded to the maximal ideals of $\CC[x_1, \dots, x_n]$.
In an \emph{affine scheme}, we will take an \emph{arbitrary} ring $A$,
and generate the entire structure from just $A$ itself.
The final result is called $\Spec A$, the \vocab{spectrum} of $A$.
The affine varieties $\VV(I)$ we met earlier will just be
$\CC[x_1, \dots, x_n] / I$, but now we will be able to take
\emph{any} ideal $I$, thus finally completing the table at the end
of the ``affine variety'' chapter.

The construction of the affine scheme in this way
will have three big generalizations:
\begin{enumerate}
	\ii We no longer have to work over an algebraically
	closed field $\CC$, or even a field at all.
	This will be the most painless generalization:
	you won't have to adjust your current picture much for this to work.

	\ii We allow non-radical ideals:
	$\Spec \CC[x] / (x^2)$ will be the double point
	we sought for so long.
	This will let us formalize the notion of a ``fat'' or ``fuzzy'' point.

	\ii Our affine schemes will have so-called \emph{non-closed points}:
	points which you can visualize as floating around,
	somewhere in the space but nowhere in particular.
	(They'll correspond to prime non-maximal ideals.)
	These will take the longest to get used to,
	but as we progress we will begin to see that these non-closed points
	actually make life \emph{easier},
	once you get a sense of what they look like.
\end{enumerate}

\section{The set of points}
\prototype{$\Spec \CC[x_1, \dots, x_n] / I$.}

First surprise, for a ring $A$:
\begin{definition}
	The set $\Spec A$ is defined as the set of prime ideals of $A$.
\end{definition}

This might be a little surprising, since we might have guessed
that $\Spec A$ should just have the maximal ideals.
What do the remaining ideals correspond to?
The answer is that they will be so-called \emph{non-closed points}
or \emph{generic points} which are ``somewhere'' in the space,
but nowhere in particular.
(The name ``non-closed'' is explained next chapter.)

\begin{remark}
	As usual $A$ itself is not a prime ideal, but $(0)$
	is prime if $A$ is an integral domain.
\end{remark}

\begin{example}
	[Examples of spectrums]
	\listhack
	\begin{enumerate}[(a)]
		\ii $\Spec \CC[x]$ consists of a point $(x-a)$ for every $a \in \CC$,
		which correspond to what we geometrically think of as $\Aff^1$.
		In additionally consists of a point $(0)$,
		which we think of as a ``non-closed point'', nowhere in particular.

		\ii $\Spec \CC[x,y]$ consists of points $(x-a,y-b)$
		(which are the maximal ideals) as well as $(0)$ again,
		a non-closed point that is thought of as ``somewhere in $\CC^2$,
		but nowhere in particular''.
		It also consists of non-closed points corresponding to irreducible
		polynomials $f(x,y)$, for example $(y-x^2)$,
		which is a ``generic point on the parabola''.

		\ii If $k$ is a field, $\Spec k$ is a single point,
		since the only maximal ideal of $k$ is $(0)$.
	\end{enumerate}
\end{example}

\begin{example}
	[Complex affine varieties]
	Let $I \subseteq \CC[x_1, \dots, x_n]$ be an ideal.
	By \Cref{prop:prime_quotient},
	the set \[ \Spec \CC[x_1, \dots, x_n] /I \]
	consists of those prime ideals of $\CC[x_1, \dots, x_n]$
	which contain $I$: in other words, it has a
	point for every closed irreducible subvariety of $\VV(I)$.
	So in addition to the ``geometric points'' 
	(corresponding to the maximal ideals $(x_1-a_1, \dots, x_n-a_n)$
	we have non-closed points along each of the varieties).
\end{example}

\begin{example}
	[More examples of spectrums]
	\listhack
	\begin{enumerate}[(a)]
		\ii $\Spec \ZZ$ consists of a point for every prime $p$,
		plus a generic point that is somewhere, but no where in particular.

		\ii $\Spec \CC[x] / (x^2)$ has only $(x)$ as a prime ideal.
		The ideal $(0)$ is not prime since $0 = x \cdot x$.
		Thus as a \emph{topological space},
		$\Spec \CC[x] / (x^2)$ is a single point.
		
		\ii $\Spec \Zc{60}$ consists of three points.
		What are they?
	\end{enumerate}
\end{example}

The non-closed points are the ones you are not used to:
there is one for each non-maximal prime ideal
(visualized as ``irreducible subvariety'').
I like to visualize them in my head like a fly:
you can hear it, so you know it is floating \emph{somewhere} in the room,
but as it always moving, you never know exactly where.
So the generic point of $\Spec \CC[x,y]$ corresponding to the prime
ideal $(0)$ is floating everywhere in the plane,
the one for the ideal $(y-x^2)$ floats along the parabola, etc.
\begin{center}
	\includegraphics[scale=0.4]{media/calvin-hobbes-fly.png} \\
	\footnotesize Image from \cite{img:calvin_hobbes_fly}.
\end{center}

\section{The Zariski topology on the spectrum}
\prototype{Still $\Spec \CC[x_1, \dots, x_n] / I$.}

Now, we endow a topology on $\Spec A$.
Since the points on $\Spec A$ are the prime ideals, we continue
the analogy by thinking of the points $f$ as functions on $\Spec A$. That is:
\begin{definition}
	Let $f \in A$ and $\kp \in \Spec A$.
	Then the \vocab{value} of $f$ at $\kp$ is defined to be $f \pmod{\kp}$.
	We denote it $f(\kp)$.
\end{definition}
\begin{example}
	[Vanishing locii in $\Aff^n$]
	Suppose $A = \CC[x_1, \dots, x_n]$,
	and $\km = (x_1-a_1, x_2-a_2, \dots, x_n-a_n)$ is a maximal ideal of $A$.
	Then for a polynomial $f \in \CC$,
	\[ f \pmod \kp = f(a_1, \dots, a_n) \]
	with the identification that $\CC/\km \cong \CC$.
\end{example}
\begin{example}
	[Functions on $\Spec \ZZ$]
	Consider $A = \Spec \ZZ$.
	Then $2019$ is a function on $A$.
	Its value at the point $(5)$ is $4 \pmod 5$;
	its value at the point $(7)$ is $3 \pmod 7$.
\end{example}

Indeed if you replace $A$ with $\CC[x_1, \dots, x_n]$
and $\Spec A$ with $\Aff^n$ in everything that follows,
then everything will become quite familiar.

\begin{definition}
	Let $f \in A$. We define the \vocab{vanishing locus} of $f$ to be
	\[ \VV(f) = \left\{ \kp \in \Spec A \mid f(\kp) = 0 \right\}
		= \left\{ \kp \in \Spec A \mid f \in \kp \right\}. \]
	More generally, just as in the affine case,
	we define the vanishing locus for an ideal $I$ as
	\begin{align*}
		\VV(I) &= \left\{ \kp \in \Spec A \mid f(\kp)=0 \; \forall f \in I \right\} \\
		&= \left\{ \kp \in \Spec A \mid f \in \kp \; \forall f \in I \right\} \\
		&= \left\{ \kp \in \Spec A \mid I \subseteq \kp \right\}.
	\end{align*}
	Finally, we define the \vocab{Zariski topology} on $\Spec A$
	by declaring that the sets of the form $\VV(I)$ are closed.
\end{definition}

We now define a few useful topological notions:
\begin{definition}
	Let $X$ be a topological space.
	A point $p \in X$ is a \vocab{closed point}
	if the set $\{p\}$ is closed.
\end{definition}
\begin{ques}
	[Mandatory]
	Show that a point (i.e.\ prime ideal)
	$\km \in \Spec A$ is a closed point
	if and only if $\km$ is a maximal ideal.
\end{ques}
Recall also in \Cref{def:closure} we denote by $\ol S$
the closure of a set $S$ (i.e.\ the smallest closed set containing $S$);
so you can think of a closed point $p$ also
as one whose closure is just $\{p\}$,
while with a generic point
Therefore the Zariski topology lets us refer back to the old ``geometric''
as just the closed points.

\begin{example}
	[Non-closed points, continued]
	Let $A = \CC[x,y]$ and let $\kp = (y-x^2) \in \Spec A$;
	this is the ``generic point'' on a parabola.
	It is not closed, but we can compute its closure:
	\[
		\ol{\{\kp\}}
		= \VV(\kp) = \left\{ \kq \in \Spec A \mid \kq \supseteq \kp \right\}.
	\]
	This closure contains the point $\kp$ as well
	as several maximal ideals $\kq$, such as $(x-2,y-4)$ and $(x-3,y-9)$.
	In other words, the closure of the ``generic point'' of the parabola
	is literally the set of all points that are actually on the parabola
	(including generic points).

	That means the way to picture $\kp$ is a point that
	is floating ``somewhere on the parabola'', but nowhere in particular.
	It makes sense then that if we take the closure,
	we get the entire parabola,
	since $\kp$ ``could have been'' any of those points.
\end{example}

\missingfigure{draw it}

\begin{example}
	[The generic point of the $y$-axis isn't on the $x$-axis]
	Let $A = \CC[x,y]$ again.
	Consider $\VV(y)$, which is the $x$-axis of $\Spec A$.
	Then consider $\kp = (x)$, which is the generic point on the $y$-axis.
	Observe that
	\[ \kp \notin \VV(y). \]
	The geometric way of saying this is that a \emph{generic point}
	on the $y$-axis does not lie on the $x$-axis.
\end{example}

We now also introduce one more word:
\begin{definition}
	A topological space $X$ is \vocab{irreducible}
	if either of the following two conditions hold:
	\begin{itemize}
		\ii The space $X$ cannot be written as the
		union of two proper closed subsets.
		\ii Any two nonempty open sets of $X$ intersect.
	\end{itemize}
	A subset $Z$ of $X$ (usually closed) is irreducible
	if it is irreducible as a subspace.
\end{definition}
\begin{exercise}
	Show that the two conditions above are indeed equivalent.
	Also, show that the closure of a set is always irreducible.
\end{exercise}

This is the analog of the ``irreducible''
we defined for affine varieties,
but it is now a topological definition,
although in practice this definition is only
useful for spaces with the Zariski topology.
Indeed, if any two nonempty open sets intersect
(and there is more than one point),
the space is certainly not Hausdorff!
As with our old affine varieties,
the intuition is that $\VV(xy)$ (the union of two lines)
should not be irreducible.

\begin{example}
	[Reducible and irreducible spaces]
	\listhack
	\begin{enumerate}[(a)]
		\ii The closed set $\VV(xy) = \VV(x) \cup \VV(y)$ is reducible.
		\ii The entire plane $\Spec \CC[x,y]$ is irreducible.
		There is actually a simple (but counter-intuitive,
		since you are just getting used to generic points)
		reason why this is true:
		the generic point $(0)$ is in \emph{every} open set,
		ergo, any two open sets intersect.
	\end{enumerate}
\end{example}

So actually, the generic points
kind of let us cheat our way through the following bit:
\begin{proposition}
	[Spectrums of integral domains are irreducible]
	If $A$ is an integral domain,
	then $\Spec A$ is irreducible.
\end{proposition}
\begin{proof}
	Just note $(0)$ is a prime ideal,
	and in every open set.
\end{proof}
You should compare this with the result that $\CC[x_1, \dots, x_n]/I$
was irreducible exactly when $I$ was prime.
But this time, the generic point actually takes care
of the work for us:
the fact that it is \emph{allowed} to float
anywhere in the plane lets us capture the idea that
$\Aff^2$ should be irreducible
without having to expend any additional effort.
\begin{remark}
	Surprisingly, the converse of this proposition is false:
	we have seen $\Spec \CC[x]/(x^2)$ has only one point,
	so is certainly irreducible.
	But $A = \CC[x]/(x^2)$ is not an integral domain.
	So this is one weird-ness introduced by allowing ``non-radical'' behavior.
\end{remark}

At this point you might notice something:
\begin{theorem}
	[Points are in bijection with irreducible closed sets]
	Consider $X = \Spec A$.
	For every irreducible closed set $Z$,
	there is exactly one point $\kp$ such that $Z = \ol{\{\kp\}}$.
	(In particular points of $X$ are in bijection
	with closed subsets of $X$.)
\end{theorem}
\begin{proof}
	[Idea of proof]
	The point $\kp$ corresponds to the closed set $\VV(\kp)$,
	which one can show is irreducible.
	% Maybe I really should prove this here,
	% but I don't really want to draw to much attention to radicals yet;
	% there's too much going on already.
\end{proof}
This gives you a better way to draw non-closed points:
they are the generic points lying along any irreducible closed set
(consisting of more than just one point).

\section{A useless definition of the structure sheaf}
\prototype{Still $\CC[x_1, \dots, x_n] / I$.}

We have now endowed $\Spec A$ with the Zariski topology,
and so all that remains is to put a sheaf $\OO_{\Spec A}$ on it.
To do this we want a notion of ``regular functions'' as before.

This is easy to do since we have localizations on hand.
\begin{definition}
	First, let $\SF$ be the pre-sheaf of ``globally rational'' functions:
	i.e.\ we define $\SF(U)$ to be the localization
	\[
		\SF(U) = \left\{
			\frac fg \mid f, g \in A
			\text{ and } g(\kp) \neq 0 \; \forall \kp \in U
		\right\}
		= \left(A \setminus \bigcup_{\kp \in U} \kp \right)\inv A.
	\]
	We now define the structure sheaf on $\Spec A$.
	It is
	\[ \OO_{\Spec A} = \SF\sh \]
	i.e.\ the sheafification of the $\SF$ we just defined.
\end{definition}
\begin{exercise}
	Compare this with the definition for $\OO_V$
	with $V$ a complex variety, and check that they essentially match.
\end{exercise}
And thus, we have completed the transition to adulthood,
with a complete definition of the affine scheme.

If you really like compatible germs,
you can write out the definition:
\begin{definition}
	Let $A$ be a ring.
	Then $\Spec A$ is made into a ringed space by setting
	\[ \OO_{\Spec A}(U)
		= \left\{ (f_\kp \in A_\kp)_{\kp \in U}
		\text{ which are locally quotients} \right\}. \]
	That is, it consists of sequence $(f_\kp)_{\kp \in U}$, with
	each $f_\kp \in A_\kp$, such that for every point $\kp$ there
	is a neighborhood $U_\kp$ and an $f,g \in A$ such that
	$f_\kq = \frac fg \in A_\kq$ for all $\kq \in U_\kp$.
\end{definition}

We will now \textbf{basically forget about this definition},
because we will never use it in practice.
In the next two sections, we will show you how to compute
the stalks and sections of any affine scheme, without having
to think about sheafification.
(Hence the lack of examples in this section.)

\section{The stalks of the structure sheaf}
The stalks are the easy case,
and you probably can guess what's happening.

\begin{theorem}
	[Stalks of $\Spec A$ are $A_\kp$]
	Let $A$ be a ring and let $\kp \in \Spec A$.
	Then \[ \OO_{\Spec A, \kp} \cong A_\kp. \]
	In particular $X$ is a locally ringed space.
\end{theorem}
\begin{proof}
	Since sheafification preserved stalks,
	it's enough to check it for $\SF$ the pre-sheaf
	of globally rational functions in our definition.
	The proof is basically the same as \Cref{thm:stalks_affine_var}:
	there is an obvious map $\SF_\kp \to A_\kp$ on germs by
	\[ \left(U, f/g \in \SF(U) \right)
		\mapsto f/g \in A_\kp . \]
	(Note the $f/g$ on the left lives in $\SF(U)$
	but the one on the right lives in $A_\kp$).
	We show injectivity and surjectivity:
	\begin{itemize}
		\ii Injective: suppose $(U_1, f_1 / g_1)$ and $(U_2, f_2 / g_2)$
		are two germs with $f_1/g_1 = f_2/g_2 \in A_\kp$.
		This means $h(g_1 f_2 - f_2 g_1) = 0$ in $A$, for some nonzero $h$.
		Then both germs identify with
		the germ $(U_1 \cap U_2 \cap D(h), f_1 / g_1)$.
		\ii Surjective: let $U = D(g)$. \qedhere
	\end{itemize}
\end{proof}

\begin{example}
	[Denominators not divisible by $x$]
	We have seen this example so many times
	that I will only write it in the new notation,
	and make no further comment:
	if $X = \Spec \CC[x]$ then
	\[ \OO_{\Spec X, (x)} = \CC[x]_{(x)}. \]
\end{example}

If you want more examples,
take any of the ones from \Cref{sec:localize_prime_ideal},
and try to think about what they mean geometrically.

\section{The value of distinguished open sets}
\prototype{$D(x)$ in $\Spec \CC[x]$ is the punctured line.}

We will now really hammer in the importance of
the distinguished open sets.
The definition is analogous to before:
\begin{definition}
	Let $f \in \Spec A$.
	Then $D(f)$ is the set of $\kp$ such that $f(\kp) \neq 0$,
	a \vocab{distinguished open set}.
\end{definition}
Distinguished open sets will have three absolutely crucial properties,
which build on each other.

\subsection{A basis of the Zariski topology}
The first is a topological observation:
\begin{theorem}
	[Distinguished open sets form a base]
	The distinguished open sets $D(f)$
	form a basis for the Zariski topology:
\end{theorem}
\begin{proof}
	Let $U$ be an open set;
	suppose it is the complement of closed set $V(I)$.
	Then verify that \[ U = \bigcup_{f \in I} D(f). \qedhere \]
\end{proof}

\subsection{Sections are computable}
The second critical fact is that we can compute the relevant sections.
\begin{theorem}
	[Sections of $D(f)$ are localizations away from $f$]
	Let $A$ be a ring and $f \in A$.
	Then \[ \OO_{\Spec A}(D(f)) \cong A_f. \]
\end{theorem}
\begin{proof}
	Omitted, but similar to
	\Cref{thm:reg_func_distinguish_open}.
\end{proof}

\begin{example}
	[The punctured line is isomorphic to a hyperbola]
	The hyperbola effect now has a new notation too:
	\[ \OO_{\Spec \CC[x]} (D(x))
		= \CC[x]_{x} = \CC[x, x\inv]
		\cong \CC[x,y] / (xy-1). \]
\end{example}

On a tangential note,
we had better also note somewhere that $\Spec A = D(1)$
is itself distinguished open, so the global sections can be recovered.
\begin{corollary}
	[$A$ is the ring of global sections]
	The ring of global sections of $\Spec A$ is $A$.
\end{corollary}
\begin{proof}
	By previous theorem, $\OO_{\Spec A}(\Spec A)
	= \OO_{\Spec A}(D(1)) = A_1 = A$.
\end{proof}

\subsection{They are affine}
We know $\OO_X(D(f)) = A_f$.
In fact, if you draw $\Spec A_f$,
you will find that it looks exactly like $D(f)$.
So the third final important fact is that
$D(f)$ will actually be \emph{isomorphic} to $\Spec A_f$
(just like the line minus the origin is isomorphic to the hyperbola).
We can't make this precise yet,
because we have not yet discussed morphisms of schemes.
%However, you can already see this at the level of topological spaces;
%see \Cref{prob:homeomorphism}.
%Since distinguished open sets form a base,
%though, this means that open sets of affine schemes
%are, at least locally, themselves affine schemes:
%given any open set $U \subseteq \Spec A$, and point $\kp \in U$,
%there is some open neighborhood $V \ni p$ contained in $U$
%which is itself affine.

\section{Classic example: the punctured plane}
To drive the point home with the previous section,
we now show you how you can forget about sheafification,
if you never liked it.\footnote{This perspective is
	so useful that some sources, like Vakil \cite[\S4.1]{ref:vakil}
	will \emph{define} $\OO_{\Spec A}$
	by requiring $\OO_{\Spec A}(D(f)) = A_f$,
	rather than use sheafification as we did.}
The idea is that:
\begin{moral}
	We can compute any section $\OO_X(U)$ in practice
	by using distinguished open sets and sheaf axioms.
\end{moral}

Let $X = \Spec \CC[x,y]$,
and consider the origin, i.e.\ the point $\km = (x,y)$.
This ideal is maximal, so it corresponds to a closed point,
and we can consider the open set $U$
consisting of all the points other than $\km$.
We wish to compute $\OO_X(U)$.

\begin{center}
\begin{asy}
	graph.xaxis("$\mathcal{V}(y)$", red);
	graph.yaxis("$\mathcal{V}(x)$", red);
	fill(box( (-3,-3), (3,3) ), opacity(0.2)+lightcyan);
	opendot(origin, blue+1.5);
	label("$\mathfrak m = (x,y)$", origin, dir(45), blue);
\end{asy}
\end{center}

Unfortunately, $U$ is not distinguished open.
But, we can compute it anyways by writing $U = D(x) \cup D(y)$:
conveniently, $D(x) \cap D(y) = D(xy)$.
By the sheaf axioms,
we have a pullback square
\begin{center}
\begin{tikzcd}
	\OO_X(U) \ar[r] \ar[d] & \OO_X(D(x)) = \CC[x,y]_{x} \ar[d] \\
	\OO_X(D(x)) = \CC[x,y]_{y} \ar[r] & \OO_X(D(xy)) = \CC[x,y]_{xy}
\end{tikzcd}
\end{center}
or equivalently
\begin{center}
\begin{tikzcd}
	\OO_X(U) \ar[r] \ar[d] & \CC[x,x\inv,y] \ar[d] \\
	\CC[x,y,y\inv] \ar[r] & \CC[x,y,x\inv,y\inv].
\end{tikzcd}
\end{center}
In other words, $\OO_X(U)$ consists of pairs
\begin{align*}
	f &\in \CC[x,y,x\inv] \\
	g &\in \CC[x,y,y\inv]
\end{align*}
which agree on the overlap:
$f = g$ on $D(x) \cap D(y)$.
Well, we can describe
$f$ as a polynomial with some $x$'s in the denominator, and
$g$ as a polynomial with some $y$'s in the denominator.
If they match, the denominator is actually constant.
Put crudely,
\[ \CC[x,y,x\inv] \cap \CC[x,y,y\inv] = \CC[x,y]. \]
In conclusion,
\[ \OO_X(U) = \CC[x,y]. \]
That is, we get no additional functions.

\section{Recap}
To recap, let $A$ be a ring.
\begin{itemize}
	\ii We define $X = \Spec A$ to be the set of prime ideals of $A$.
	\begin{itemize}
		\ii The maximal ideals are the ``closed points'' we are used to,
		but the prime ideals are ``generic points''.
	\end{itemize}

	\ii We equip $\Spec A$ with the Zariski topology by declaring
	$\VV(I)$ to be the closed sets, for ideals $I \subseteq A$.
	\begin{itemize}
		\ii The distinguished open sets $D(f)$,
		form a topological basis.
		\ii In this basis, the irreducible closed sets
		are exactly the closures of points.
	\end{itemize}

	\ii Finally, we defined a sheaf $\OO_X$.
	We set up the definition such that
	\begin{itemize}
		\ii $\OO_{X,\kp} = A_\kp$:
		the stalks are localizations at a prime.
		\ii $\OO_{X}(D(f)) = A_f$:
		at distinguished open sets $D(f)$,
		we get localizations too.
	\end{itemize}
	Since $D(f)$ is a basis,
	these two properties lets us explicitly compute $\OO_X(U)$
	for any open set $U$,
	so we don't have to resort to the definition using sheafification.
\end{itemize}

\section\problemhead
The next chapter contains many, many
detailed examples of affine schemes to train your intuition,
so it might actually work better if you read
the next chapter before attempting these problems.

\begin{problem}
	[Punctured gyrotop, communicated by Aaron Pixton]
	The gyrotop is the scheme $X = \Spec \CC[x,y,z] / (xy,z)$.
	We let $U$ denote the open subset obtained
	by deleting the closed point $\km = (x,y,z)$.
	Compute $\OO_X(U)$.
	\begin{hint}
		$k[x,y] \times k[z,z\inv]$.
	\end{hint}
\end{problem}

\begin{dproblem}
	[Spectrums are quasicompact]
	\gim
	Show that $\Spec A$ is quasicompact for any ring $A$.
\end{dproblem}

\endinput

\begin{problem}
	[From Andrew Critch]
	\gim
	Let $A$ be a Noetherian ring.
	Show that $A$ is an integral domain if and only if it has no idempotents,
	and $A_\kp$ is an integral domain for every prime $\kp$.
	\begin{hint}
		Show that if $\Spec A$ is connected and its stalks are irreducible,
		then $\Spec A$ is itself irreducible.
		Consider nilradical $N = \sqrt{(0)}$.
	\end{hint}
	\begin{sol}
		This is the proposition on the second page of
		\url{http://www.acritch.com/media/math/Stalk-local_detection_of_irreducibility.pdf}
	\end{sol}
\end{problem}
