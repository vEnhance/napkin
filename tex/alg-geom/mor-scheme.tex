\chapter{Morphisms of locally ringed spaces}
Having set up the definition of a locally ringed space,
we are almost ready to define morphisms between them.
Throughout this chapter, you should imagine your ringed spaces
are the affine schemes we have so painstakingly defined;
but it will not change anything to work in the generality
of arbitrary locally ringed spaces.

\section{Morphisms of ringed spaces via sections}
Let $(X, \OO_X)$ and $(Y, \OO_Y)$ be ringed spaces.
We want to give a define what it means
to have a function $\pi \colon X \to Y$ between them.\footnote{Notational
	connotations: for ringed spaces, $\pi$ will be used for maps,
	since $f$ is often used for sections.}
We start by requiring the map to be continuous,
but this is not enough: there is a sheaf on it!

Well, you might remember what we did for baby ringed spaces:
any time we had a function on an open set of $U \subseteq Y$,
we wanted there to be an analogous function on $\pi\pre(U) \subseteq X$.
For baby ringed spaces, this was done by composition,
since the elements of the sheaf \emph{were} really complex valued functions:
\[ \pi^\ast\phi \text{ was defined as } \phi \circ \pi. \]
The upshot was that we got a map $\OO_Y(U) \to \OO_X(\pi\pre(U))$
for every open set $U$.
\begin{center}
	\begin{asy}
		size(13cm);
		bigblob("$X$");
		pair p = (0.5,0);
		filldraw(CR(p, 1), opacity(0.2)+lightgreen, deepgreen+dashed);
		label("$\pi^{\text{pre}}(U)$", p+dir(135), dir(135), deepgreen);

		transform t = scale(0.8) * shift(14*dir(180));
		add(t * CC());
		p = t*p;

		bigblob("$Y$");
		pair q = (0,0.5);
		filldraw(CR(q, 1.2), opacity(0.2)+lightred, red+dashed);
		label("$U$", q+1.2*dir(45), dir(45), red);
		
		draw( (-9,0.5)--(-3,0.5), EndArrow);
		label("$\pi$", (-6,0.5), dir(90));
		label("$\boxed{f \in \mathcal O_Y(U)}$", (0,-5), red);
		label("$\boxed{\pi^\ast f \in \mathcal O_X(f^{\text{pre}}(U))}$",
			t*(0,-6), deepgreen);
	\end{asy}
\end{center}

Now, for general locally ringed spaces,
the sections are just random rings,
which may not be so well-behaved.
So the solution is that we \emph{include} the data of $f^\ast$
as part of the definition of a morphism.
\begin{definition}
	A \vocab{morphism of ringed spaces}
	is a continuous map $\pi \colon X \to Y$
	together with a choice of ring homomorphism
	\[ \pi^\ast \colon \OO_Y(U) \to \OO_X(\pi\pre(U)) \]
	for every open set $U \subseteq Y$, such that the restriction diagram
	\begin{center}
	\begin{tikzcd}
		\OO_Y(U) \ar[r] \ar[d] & \OO_Y(\pi\pre(U)) \ar[d] \\
		\OO_Y(V) \ar[r] & \OO_X(\pi\pre(V))
	\end{tikzcd}
	\end{center}
	commutes for $V \subseteq U$.

	There is an obvious identity map,
	and so we can also define \vocab{isomorphism} in the categorical way.
\end{definition}
% Note that the ring homomorphism has its direction flipped.
Unsurprisingly, the sections are clumsier to work with
than the stalks, now that we have grown to love localization.
Fortunately, the good news is that we do get morphisms of stalks.
\begin{proposition}
	[Induced stalk morphisms]
	If $\pi \colon X \to Y$ is a map of ringed spaces,
	then we get a map
	\[ \pi^\ast \colon \OO_{Y,q} \to \OO_{X,p} \]
	whenever $\pi(p) = q$.
\end{proposition}
Again, compare this to the pullback picture:
this is roughly saying that if a function $f$ has some enriched value at $q$,
then $\pi^\ast(f)$ should be assigned a corresponding enriched value at $p$.
The analogy is not perfect since the stalks at $q$ and $p$
may be different rings in general,
but there should at least be a ring homomorphism (the assignment).
\begin{proof}
	If $(s,U)$ is a germ at $q$,
	then $(\pi^\ast(s), \pi\pre(U))$ is a germ at $p$,
	and this is a well-defined morphism because
	of compatibility with restrictions.
\end{proof}

\begin{remark}
	Two additional facts that we mention,
	but won't prove here:
	\begin{itemize}
		\ii $\pi$ is uniquely determined by the collection of stalk maps.
		\ii $\pi$ is an isomorphism iff each stalk map is an isomorphism.
	\end{itemize}
\end{remark}
This means you can draw a morphism of locally ringed spaces
as a continuous map on the topological space,
plus for each $\pi(p) = q$,
an assign of each germ at $q$ to a germ at $p$.

\begin{center}
\begin{asy}
	size(10cm);
	pair p = (-4,0);
	pair q = ( 4,0);
	filldraw(ellipse(p, 2, 1 ), opacity(0.2)+lightcyan, black);
	filldraw(ellipse(q, 2, 1 ), opacity(0.2)+lightcyan, black);
	label("$X$", p+dir(270), dir(270), blue);
	label("$Y$", q+dir(270), dir(270), blue);
	pair pt = (-4, 3.5);
	pair qt = ( 4, 4.3);
	draw( p--pt, red );
	draw( q--qt, red );
	label("$\mathcal O_{X,p}$", pt, dir(90), red);
	label("$\mathcal O_{Y,q}$", qt, dir(90), red);
	dot("$p$", p, dir(-90), blue);
	dot("$q$", q, dir(-90), blue);
	pair pg = (-4, 1.6);
	pair qg = (4, 2.8);
	dot("$[s]_q$", qg, dir(0), red+6);
	dot("$[\pi^\ast(s)]_p$", pg, dir(180), red+6);
	draw(qg--pg, deepgreen, EndArrow, Margin(4,4));
	label("$\pi^\ast$", midpoint(qg--pg), dir(90), deepgreen);
	draw( (-1.5,0)--(1.5,0), deepgreen, EndArrow );
	label("$\pi$", origin, dir(-90), deepgreen);
\end{asy}
\end{center}

\section{Morphisms of locally ringed spaces}
On the other hand, we've seen that our stalks are local rings,
which enable us to actually talk about \emph{values}.
And so we want to add one more compatibility condition
to ensure that our notion of value is preserved.
Now the stalks at $p$ and $q$ in the previous picture might be different,
so $\kappa(p)$ and $\kappa(q)$ might even be different fields.
\begin{definition}
	A \vocab{morphism of locally ringed spaces}
	is a morphism of ringed spaces $\pi \colon X \to Y$
	with the following additional property:
	whenever $\pi(p) = q$,
	the map at the stalks induces a well-defined ring homomorphism
	\[ \pi^\ast \colon \kappa(q) \to \kappa(p). \]
\end{definition}
So we require $\pi^\ast$ induces a field homomorphisms\footnote{Which
means it is automatically injective, by \Cref{prob:field_hom}.}
on the \emph{residue fields}.
In particular, since $\pi^\ast(0) = 0$,
this means something very important:
\begin{moral}
	In a morphism of locally ringed spaces,
	a germ vanishes at $q$ if and only if
	the corresponding germ vanishes at $p$.
\end{moral}
\begin{exercise}
	[So-called ``local ring homomorphism'']
	Show that this is equivalent to requiring
	\[ (\pi^\ast)\im(\km_{Y,q}) \subseteq \km_{X,p} \]
	or in English, a germ at $q$ has value zero
	iff the corresponding germ at $p$ has value zero.
\end{exercise}
I don't like this formulation
$(\pi^\ast)\im(\km_{Y,q}) \subseteq \km_{X,p}$
as much since it hides the geometric intuition behind a lot of symbols:
that we want the notion of ``value at a point''
to be preserved in some way.

\section{A bunch of examples of morphisms between affine schemes}
\todo{WRITE ME}

\section{The big theorem}
\begin{quote}
	``He [Hartshorne] never mentions that the category of affine schemes
	is dual to the category of rings, as far as I can see.
	I'd expect to see that in huge letters
	near the definition of scheme.
	How could you miss that out?''
\end{quote}

\todo{Give definition}

\begin{theorem}
	[$\catname{CRing}$ and $\catname{AffSch}$]
	There is a bijection between ring homomorphisms $B \to A$
	and $\Spec A \to \Spec B$.
\end{theorem}

\todo{Show equivalence}

\section{Finishing up}
Recalling that $\Spec A$ is an example of a locally ringed space,
we're actually done with the definitions once we write:
\begin{definition}
	A \vocab{scheme} is a locally ringed space
	for which every point has an open neighborhood
	isomorphic to an affine scheme.

	A morphism of schemes is just a morphism of locally ringed spaces.
\end{definition}

% Notice that: $\Spec A$ is of course a scheme.
We can finally state the isomorphism that we wanted for a long time
(first mentioned in \Cref{subsec:distinguished_open_affine}):
\begin{theorem}
	[Distinguished open sets are isomorphic to affine schemes]
	Let $A$ be a ring and $f$ an element.
	Then
	\[ \Spec A_f \cong D(f) \subseteq \Spec A. \]
\end{theorem}
\begin{proof}
	Annoying check, not included yet.
	(We have already seen the bijection of prime ideals, at the level of points.)
\end{proof}

\begin{corollary}
	[Open subsets are schemes]
	\listhack
	\begin{enumerate}[(a)]
		\ii Any nonempty open subset of an affine scheme is itself a scheme.
		\ii Any nonempty open subset of any scheme (affine or not) is itself a scheme.
	\end{enumerate}
\end{corollary}
\begin{proof}
	Part (a) has essentially been done already:
	\begin{ques}
		Combine \Cref{thm:distinguished_base}
		with the previous proposition to deduce (a).
	\end{ques}
	Part (b) then follows by noting that if $U$ is an open set,
	and $p$ is a point in $U$,
	then we can take an affine open neighborhood $\Spec A$ at $p$,
	and then cover $U \cap \Spec A$ with distinguished
	open subsets of $\Spec A$ as in (a).
\end{proof}

\section{The punctured plane is not affine}
We now reprise \Cref{subsec:punctured_plane}
(except $\CC$ will be replaced by $k$).
We have seen it is an open subset $U$ of $\Spec k[xy]$, so it is affine.
\begin{ques}
	Show that in fact $U$ can be covered by
	two open sets which are both affine.
\end{ques}
However, we show now that you really do need two open sets.
\begin{proposition}
	[Famous example: punctured plane isn't affine]
	The punctured plane $U = (U, \OO_U)$,
	obtained by deleting $(x,y)$ from $\Spec k[x,y]$,
	is not isomorphic to any affine scheme $\Spec B$.
\end{proposition}

The intuition is that $\OO_U(U) = k[x,y]$, but $U$ is not the plane.
\begin{proof}
	We already know $\OO_U(U) = k[x,y]$
	and we have a good handle on it.
	For example, $y \in \OO_U(U)$ is a global section
	which vanishes on what looks like the $y$-axis.
	Similarly, $x \in \OO_X(X)$ is a global section
	which vanishes on what looks like the $y$-axis.
	In particular, no point of $U$ vanishes at both.

	Now assume for contradiction that we have an isomorphism
	\[ \psi \colon \Spec B \to U. \]
	By taking the map on global sections (part of the definition),
	\[ k[x,y] = \OO_U(U) \taking{\psi^\ast}
		\OO_{\Spec B}(\Spec B) \cong B. \]
	The global sections $x$ and $y$ in $\OO_U(U)$
	should then have images $a$ and $b$ in $B$;
	and it follows we have a ring isomorphism $B \cong k[a,b]$.

	\begin{center}
	\begin{asy}
		unitsize(0.9cm);
		draw( (0,-3)--(0,3), red );
		draw( (-3,0)--(3,0), red );
		fill(box( (-3,-3), (3,3) ), opacity(0.2)+lightcyan);
		opendot(origin, blue+1.5);
		label("$\mathcal V(x)$", (3,0), dir(135), red);
		label("$\mathcal V(y)$", (0,3), dir(-45), red);
		label("$U$", (0,3), dir(90), deepcyan);
		label("$\mathcal O_U(U) = k[x,y]$", (0,-3), dir(-90), deepcyan);
		add(shift(-8,0)*CC());
		fill(box( (-3,-3), (3,3) ), opacity(0.2)+lightblue);
		path a = (-3,-1)..(-1,0.7)..(0.2,0.5)..(2,1)..(3,0.8);
		path b = (-0.5,-3)..(0.4,-1)..(0.2,0.5)..(-0.4,2)..(-0.7,3);
		draw(a, red);
		draw(b, red);
		label("$\mathcal V(a)$", (3,0.8), dir(225), red);
		label("$\mathcal V(b)$", (-0.7,3), dir(225), red);
		dot("$(a,b)$", (0.2,0.5), dir(225), red);
		label("$\operatorname{Spec} B$", (0,3), dir(90), blue);
		label("$\mathcal O_{\operatorname{Spec} B}(\operatorname{Spec} B) \cong k[a,b]$", (0,-3), dir(-90), blue);
	\end{asy}
	\end{center}

	Now in $\Spec B$, $\VV(a) \cap \VV(b)$ is a closed
	set containing a single point, the maximal ideal $\km = (a,b)$.
	Thus in $\Spec B$ there is exactly
	one point vanishing at both $a$ and $b$.
	\emph{Because we required morphisms of schemes to preserve values}
	(hence the big fuss about locally ringed spaces),
	that means there should be a single point of $U$
	vanishing at both $x$ and $y$.
	But there isn't --- it was the origin we deleted.
\end{proof}

\section{\problemhead}
\begin{problem}
	Given an affine scheme $X = \Spec R$,
	show that there is a unique morphism of schemes $X \to \Spec \ZZ$,
	and describe where it sends points of $X$.
	\begin{hint}
		Use the proof that $\catname{AffSch} \simeq \catname{CRing}$.
	\end{hint}
	\begin{sol}
		$\kp$ gets sent to the characteristic of the field $\OO_{X,\kp} / \km_{X,\kp}$.
	\end{sol}
\end{problem}

\endinput

\section{Morphisms of (locally) ringed spaces}
Finally, it remains to define a morphism of locally ringed space.
To do this we have to build up in several steps.

\begin{remark}
	I always secretly felt that one can probably get away with just
	suspending belief and knowing ``there is a reasonable definition
	of morphisms of locally ringed spaces''.
	Daring readers are welcome to try this approach.
\end{remark}

\subsection{Morphisms of ringed spaces}
We talked about the morphisms of locally ringed spaces
when we discussed projective varieties.
We want to copy the same construction here.

The idea was as follows: to ``preserve'' the structure via $f : X \to Y$,
we took every regular function on $Y$ and made it into a function on $X$
by composing it with $f$.
This gave us a ring homomorphism $f^\ast : \OO_Y(U) \to \OO_X(f\pre(U))$
for every open set $U$ in $Y$.

Unfortunately, in the general case we don't have a notion of function,
so ``composing by $f$'' doesn't make sense.
The solution to this is to \emph{include all the ring homomorphisms $f^\ast$}
in the data for a morphism $f : X \to Y$;
we package this information using the morphism of sheaves.

Here are the full details.
Up above we only defined a morphism of sheaves on the same space,
while right now we have a sheaf on $X$ and $Y$.
So we have to first ``push'' the sheaf on $X$ into a sheaf on $Y$ by
\[ U \mapsto \OO_X\left( f\pre(U) \right). \]
This is important enough to have a name:
\begin{definition}
	Let $\SF$ be a sheaf on $X$, and $f : X \to Y$ a continuous map.
	The \vocab{pushforward sheaf} $f_\ast \SF$ on $Y$ is defined by
	\[ (f_\ast \SF)(U) = \SF(f\pre(U)) \qquad \forall U \subseteq Y. \]
	This makes sense, since $f\pre(U)$ is open in $X$.
\end{definition}
Picture:
\begin{diagram}
	X && \rTo^f & Y && X && \rTo^f & Y \\
	\Opens(X)\op && \lTo^{f\pre} & \Opens(Y)\op
	&& f\pre(U) && \lMapsto^{f\pre} & U & \\
	& \rdMapsto(3,2)_{\SF} && \dMapsto_{f_\ast \SF}
		&& & \rdMapsto(3,2)_{\SF} && \dMapsto_{f_\ast \SF} & \\
	&&& \catname{Rings}
		&& && \SF(f\pre(U)) & = (f_\ast \SF)(U) & \in \catname{Rings}
\end{diagram}
I haven't actually checked that $f_\ast \SF$ is a sheaf
(as opposed to a pre-sheaf), but this isn't hard to do.
Also, as $f\pre$ is a functor $\Opens(Y)\op \to \Opens(X)\op$,
the pushforward $f_\ast\SF$ is
just the composition of these two functors.
\begin{ques}
	Technically $f_\ast\SF$ is supposed to be a
	functor $\Opens(Y)\op \to \catname{Rings}$, so it also needs
	to come with restriction arrows. What are they?
\end{ques}

Now that we moved both sheaves onto $Y$,
we mimic the construction with quasi-projective varieties
by asking for a sheaf morphism from $\OO_Y$ to $f_\ast \OO_X$:
\begin{definition}
	A \vocab{morphism of ringed spaces} $f : (X, \OO_X) \to (Y, \OO_Y)$
	consists of a continuous map of topological spaces $f : X \to Y$,
	plus additionally a morphism of sheaves $f^\ast : \OO_Y \to f_\ast \OO_X$.
\end{definition}
\begin{exercise}
	[Mandatory]
	Check that this coincides with our work
	on baby ringed spaces if we choose $f^\ast$ to be the pullback.
\end{exercise}

\subsection{Morphisms of locally ringed spaces}
Last step!
Suppose now that $(X, \OO_X)$ and $(Y, \OO_Y)$ are locally ringed spaces.
Thus we need to deal with some information about the stalks.

Given a morphism of ringed spaces $f : (X, \OO_X) \to (Y, \OO_Y)$,
we can actually use the $f^\ast_U$ above to induce maps on the stalks,
as follows. For a point $p \in X$, construct the map
\begin{diagram}
	f^\ast_p :& \OO_{Y, f(p)} & \rTo && \OO_{X, f(p)} \\
	& (s, U) & \rMapsto && (f^\ast_U(s), f\pre(U)).
\end{diagram}
where $s \in \OO_Y(U)$.

\begin{definition}
	Let $R$ and $S$ be local rings with maximal ideals $\km_R$ and $\km_S$.
	A \vocab{morphism of local rings} is a homomorphism of rings
	$\psi : R \to S$ such that $\psi\pre(\km_S) = \km_R$.
\end{definition}
\begin{definition}
	A \vocab{morphism of locally ringed spaces}
	is a morphism of ringed spaces $f : (X, \OO_X) \to (Y, \OO_Y)$ such that
	for every point $p$ the induced map of stalks is a morphism of local rings.
\end{definition}
Recalling that $\km_{X,p}$ is the maximal ideal of $\OO_X$ at $p$,
the new condition is saying that
\[ (f^\ast_p)\pre (\km_{X,p}) = \km_{Y,f(p)}. \]
Concretely, if $g$ is a function vanishing on $f(p)$,
then its pullback $f^\ast_U(g)$ vanishes on $p$.

This completes the definition of a morphism of locally ringed spaces.
Isomorphisms of (locally) ringed spaces are defined in the usual way.


\section{Where to go from here}
This chapter concludes the long setup for the definition of a scheme.
Unfortunately, with respect to algebraic geometry this is as much as I have
the patience or knowledge to write about.
So, if you want to actually see how schemes are used in ``real life'',
you'll have to turn elsewhere.

A good reference I happily recommend is \cite{ref:gathmann}.
More intense is \cite{ref:vakil}.
See \Cref{ch:refs} for further remarks.




\endinput
\section{Projective scheme}
\prototype{Projective varieties, in the same way.}
The most important class of schemes which are not affine are
\emph{projective} schemes.
The complete the obvious analogy:
\[
	\frac{\text{Affine variety}}{\text{Projective variety}}
	= 
	\frac{\text{Affine scheme}}{\text{Projective scheme}}.
\]
Let $S$ be \emph{any} (commutative) graded ring, like $\CC[x_0, \dots, x_n]$.
\begin{definition}
	We define $\Proj S$, the \vocab{projective scheme over $S$}:
	\begin{itemize}
		\ii As a set, $\Proj S$ consists of \emph{homogeneous prime ideals}
		$\kp$ which do not contain $S^+$.
		\ii If $I \subseteq S$ is homogeneous, then
		we let $\Vp(I) = \{ \kp \in \Proj S \mid I \subseteq \kp \}$.
		Then the \vocab{Zariski topology} is imposed by declaring 
		sets of the form $\Vp(I)$ to be closed.
		\ii We now define a pre-sheaf $\SF$ on $\Proj S$ by
		\[ \SF(U) = 
			\left\{ \frac{f}{g} \mid 
			g(\kp) \neq 0 \; \forall \kp \in U \text{ and }
			\deg f = \deg g \right\}.
		\]
		In other words, the rational functions are quotients $f/g$
		where $f$ and $g$ are \emph{homogeneous of the same degree}.
		Then we let \[ \OO_{\Proj S} = \SF\sh \] be the sheafification.
	\end{itemize}
\end{definition}
\begin{definition}
	The \vocab{distinguished open sets} $D(f)$ of the $\Proj S$
	are defined as $\left\{ \kp \in \Proj S : f(\kp) \neq 0 \right\}$,
	as before; these form a basis for the Zariski topology of $\Proj S$.
\end{definition}
Now, we want analogous results as we did for affine structure sheaf.
So, we define a slightly modified localization:
\begin{definition}
	Let $S$ be a graded ring.
	\begin{enumerate}[(i)]
		\ii For a prime ideal $\kp$, let
		\[ S_{(\kp)} = \left\{ \frac fg \mid g(\kp) \neq 0 \text{ and }
			\deg f = \deg g \right\} \]
		denote the elements of $S_\kp$ with ``degree zero''.
		\ii For any homogeneous $g \in S$ of degree $d$, let
		\[ S_{(g)} = \left\{ \frac{f}{g^r} \mid 
			\deg f = r \deg g \right\} \]
		denote the elements of $S_g$ with ``degree zero''.
	\end{enumerate}
\end{definition}

\begin{theorem}
	[On the projective structure sheaf]
	Let $S$ be a graded ring and let $\Proj S$ the associated projective scheme.
	\begin{enumerate}[(a)]
		\ii Let $\kp \in \Proj S$.
		Then $\OO_{\Proj S, \kp} \cong S_{(\kp)}$.
		\ii Suppose $g$ is homogeneous with $\deg g > 0$. Then
		\[ D(g) \cong \Spec S_{(g)} \]
		as locally ringed spaces.
		In particular, $\OO_{\Proj S}(D(g)) \cong S_{(g)}$.
	\end{enumerate}
\end{theorem}
\begin{ques}
	Conclude that $\Proj S$ is a scheme.
\end{ques}

Of course, the archetypal example is that
\[ \Proj \CC[x_0, x_1, \dots, x_n] / I \]
corresponds to the projective subvariety of $\CP^n$
cut out by $I$ (when $I$ is radical).
In the general case of an arbitrary ideal $I$, we
call such schemes \vocab{projective subscheme} of $\CP^n$.
For example, the ``double point'' in  $\CP^1$
is given by $\Proj[x_0,x_1]/(x_0^2)$.

\begin{remark}
	No comment yet on what the global sections of $\OO_{\Proj S}(\Proj S)$ are.
	(The theorem above requires $\deg g > 0$, so we cannot just take $g=1$.)
	One might hope that in general $\OO_{\Proj S}(\Proj S) \cong S^0$
	in analogy to our complex projective varieties, but
	one needs some additional assumptions on $S$ for this to hold.
\end{remark}



\section{Morphisms of sheaves}
First, recall that a sheaf is a contravariant functor (pre-sheaf)
with extra conditions. In light of this, it is not hard to guess
the definition of a morphism of pre-sheaves:
\begin{definition}
	A \vocab{morphism of (pre-)sheaves} $\alpha : \SF \to \SG$ on the same
	space $X$ is a \textbf{natural transformation} of the underlying functors.
	Isomorphism of sheaves is defined in the usual way.
\end{definition}
\begin{ques}
	Show that this amounts to: for each $U \subseteq X$ we need to specify
	a morphism $\alpha_U : \SF(U) \to \SG(U)$ such that the diagram
	\begin{diagram}
		\SF(U_2) & \rTo^{\alpha_{U_2}} & \SG(U_2) \\
		\dTo^{\res_{U_1, U_2}} && \dTo_{\res_{U_1, U_2}} \\
		\SF(U_1) & \rTo_{\alpha_{U_1}} & \SG(U_1)
	\end{diagram}
	commutes any time that $U_1 \subseteq U_2$.
\end{ques}
However, in the sheaf case we like stalks more than sections because
they are theoretically easier to think about.
And in fact:
\begin{proposition}
	[Morphisms determined by stalks]
	A morphism of sheaves $\alpha : \SF \to \SG$ induces a morphism of stalks
	\[ \alpha_p : \SF_p \to \SG_p \]
	for every point $p \in X$.
	Moreover, the sequence $(\alpha_p)_{p \in X}$ determines $\alpha$ uniquely.
\end{proposition}
\begin{proof}
	The morphism $\alpha_p$ itself is just
	$(s, U) \xmapsto{\alpha_p} (\alpha_U(s), U)$.
	\begin{ques}
		Show this is well-defined.
	\end{ques}
	Now suppose $\alpha , \beta : \SF \to \SG$ satisfy $\alpha_p = \beta_p$
	for every $p$. We want to show $\alpha_U(s) = \beta_U(s)$
	for every $s \in \SF(U)$.
	\begin{ques}
		Verify this using the description of sections
		as sequences of germs. \qedhere
	\end{ques}
\end{proof}
Thus a morphism of sheaves can be instead modelled as a morphism
of all the stalks. We will see later on that this viewpoint is quite useful.

