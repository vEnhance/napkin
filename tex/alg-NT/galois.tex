\chapter{Things Galois}
%This chapter is mostly optional.
%Read the first two sections and then decide
%whether you want to read the rest of this chapter.

\section{Motivation}
\prototype{$\QQ(\sqrt2)$ and $\QQ(\cbrt{2})$.}
The key idea in Galois theory is that of \emph{embeddings},
which give us another way to get at the idea of the ``conjugate'' we described earlier.

Let $K$ be a number field.
An \vocab{embedding} $\sigma : K \injto \CC$, is an \emph{injective field homomorphism}:
it needs to preserve addition and multiplication,
and in particular it should fix $1$.
\begin{ques}
	Show that in this context, $\sigma(q) = q$ for any rational number $q$.
\end{ques}

\begin{example}
	[Examples of embeddings]
	\listhack
	\begin{enumerate}[(a)]
		\ii If $K = \QQ(i)$, the two embeddings of $K$ into $\CC$ are
		$z \mapsto z$ (the identity) and $z \mapsto \ol z$ (complex conjugation).
		\ii If $K = \QQ(\sqrt 2)$, the two embeddings of $K$ into $\CC$ are
		$a+b\sqrt 2 \mapsto a+b\sqrt 2$ (the identity) and $a+b\sqrt 2 \mapsto a-b\sqrt 2$ (conjugation).
		\ii If $K = \QQ(\cbrt 2)$, there are three embeddings:
		\begin{itemize}
			\ii The identity embedding, which sends $1 \mapsto 1$ and $\cbrt 2 \mapsto \cbrt 2$.
			\ii An embedding which sends $1 \mapsto 1$ and $\cbrt 2 \mapsto \omega \cbrt 2$,
			where $\omega$ is a cube root of unity.
			Note that this is enough to determine the rest of the embedding.
			\ii An embedding which sends $1 \mapsto 1$ and $\cbrt 2 \mapsto \omega^2 \cbrt 2$.
		\end{itemize}
	\end{enumerate}
\end{example}

I want to make several observations about these embeddings,
which will form the core ideas of Galois theory.
Pay attention here!

\begin{itemize}
\ii
First, you'll notice some duality between roots: in the first example, $i$ gets sent to $\pm i$,
$\sqrt 2$ gets sent to $\pm \sqrt 2$, and $\cbrt 2$ gets sent to the other roots of $x^3-2$.
This is no coincidence, and one can show this occurs in general.
Specifically, suppose $\alpha$ has minimal polynomial
\[ 0 = c_n \alpha^n + c_{n-1} \alpha^{n-1} + \dots + c_1\alpha + c_0 \]
where the $c_i$ are rational.
Then applying any embedding $\sigma$ to both sides gives
\begin{align*}
	0 &= \sigma(c_n \alpha^n + c_{n-1} \alpha^{n-1} + \dots + c_1\alpha + c_0) \\
	% &= \sigma(c_n \alpha^n) + \sigma(c_{n-1} \alpha^{n-1})
	% + \dots + \sigma(c_1\alpha) + \sigma(c_0) \\
	&= \sigma(c_n) \sigma(\alpha)^n + \sigma(c_{n-1}) \sigma(\alpha)^{n-1}
	+ \dots + \sigma(c_1)\sigma(\alpha) + \sigma(c_0) \\
	&= c_n \sigma(\alpha)^n + c_{n-1} \sigma(\alpha)^{n-1} + \dots + c_1\sigma(\alpha) + c_0
\end{align*}
where in the last step we have used the fact that $c_i \in \QQ$, so they are fixed by $\sigma$.
\emph{So, roots of minimal polynomials go to other roots of that polynomial.}

\ii
Next, I want to draw out a contrast between the second and third examples.
Specifically, in example (b) where we consider embeddings $K = \QQ(\sqrt 2)$
to $\CC$.  The image of these embeddings lands entirely in $K$: that is, we
could just as well have looked at $K \to K$ rather than looking at $K \to \CC$.
However, this is not true in (c): indeed $\QQ(\cbrt 2) \subset \RR$,
but the non-identity embeddings have complex outputs!

The key difference is to again think about conjugates.
Key observation:
\begin{moral}
	The field $K = \QQ(\cbrt 2)$ is ``deficient'' because the minimal polynomial $x^3-2$
	has two other roots $\omega \cbrt{2}$ and $\omega^2 \cbrt{2}$ not contained in $K$.
\end{moral}
On the other hand $K = \QQ(\sqrt 2)$ is just fine because both roots of $x^2-2$ are contained inside $K$.
Finally, one can actually fix the deficiency in $K = \QQ(\cbrt 2)$ by completing it to a field $\QQ(\cbrt 2, \omega)$.
Fields like $\QQ(i)$ or $\QQ(\sqrt 2)$ which are ``self-contained'' are called
\emph{Galois extensions}, as we'll explain shortly.

\ii
Finally, you'll notice that in the examples above, \emph{the number of embeddings from $K$ to $\CC$
happens to be the degree of $K$}.
This is an important theorem, \Cref{thm:n_embeddings}.
\end{itemize}

In this chapter we'll develop these ideas in full generality, for any field other than $\QQ$.

\section{Field extensions, algebraic closures, and splitting fields}
\prototype{$\QQ(\cbrt 2)/\QQ$ is an extension, $\CC$ is an algebraic closure of any number field.}

First, we define a notion of one field sitting inside another,
in order to generalize the notion of a number field.
\begin{definition}
	Let $K$ and $F$ be fields.
	If $F \subseteq K$, we write $K/F$ and say $K$ is a
	\vocab{field extension} of $F$.
	
	Thus $K$ is automatically an $F$-vector space
	(just like $\QQ(\sqrt 2)$ is automatically a $\QQ$-vector space).
	The \vocab{degree} is the dimension of this space, denoted $[K:F]$.
	If $[K:F]$ is finite, we say $K/F$ is a \vocab{finite (field) extension}.
\end{definition}
That's really all. There's nothing tricky at all.

\begin{ques}
	What do you call a finite extension of $\QQ$?
\end{ques}

Degrees of finite extensions are multiplicative.
\begin{theorem}[Field extensions have multiplicative degree]
	Let $F \subseteq K \subseteq L$ be fields with $L/K$, $K/F$ finite. Then
	\[ [L:K][K:F] = [L:F]. \]
\end{theorem}
\begin{proof}
	Basis bash: you can find a basis of $L$ over $K$,
	and then expand that into a basis $L$ over $F$.
	(Diligent readers can fill in details.)
\end{proof}

Next, given a field (like $\QQ(\cbrt2)$)
we want something to embed it into (in our case $\CC$).
So we just want a field that contains all the roots of all the polynomials:
\begin{theorem}[Algebraic closures]
	Let $F$ be a field.
	Then there exists a field extension $\ol F$ containing $F$, called an \vocab{algebraic closure},
	such that all polynomials in $\ol F[x]$ factor completely.
\end{theorem}
\begin{example}
	[$\CC$]
	$\CC$ is an algebraic closure of $\QQ$, $\RR$ and even itself.
\end{example}
\begin{abuse}
	Some authors also require the algebraic closure to be \emph{minimal by inclusion}:
	for example, given $\QQ$ they would want only $\ol\QQ$ (the algebraic numbers).
	It's a theorem that such a minimal algebraic closure is unique,
	and so these authors will refer to \emph{the} algebraic closure of $K$.

	I like $\CC$, so I'll use the looser definition.
\end{abuse}

\section{Embeddings into algebraic closures for number fields}
Now that I've defined all these ingredients, I can prove:
\begin{theorem}[The $n$ embeddings of a number field]
	\label{thm:n_embeddings}
	Let $K$ be a number field of degree $n$.
	Then there are exactly $n$ field homomorphisms $K \injto \CC$,
	say $\sigma_1, \dots, \sigma_n$ which fix $\QQ$.
\end{theorem}
\begin{remark}
	Note that a nontrivial homomorphism of fields is necessarily injective
	(the kernel is an ideal).
	This justifies the use of ``$\injto$'', and we call each $\sigma_i$ an
	\vocab{embedding} of $K$ into $\CC$.
\end{remark}
\begin{proof}
	This is actually kind of fun!
	Recall that any irreducible polynomial over $\QQ$ has distinct roots (\Cref{lem:irred_complex}).
	We'll adjoin elements $\alpha_1, \alpha_2, \dots, \alpha_m$ one at a time to $\QQ$,
	until we eventually get all of $K$, that is,
	\[ K = \QQ(\alpha_1, \dots, \alpha_n). \]
	Diagrammatically, this is
	\begin{diagram}
		\QQ & \rInj & \QQ(\alpha_1) & \rInj & \QQ(\alpha_1, \alpha_2) & \rInj & \dots & \rInj & K \\
		\dInj^\id && \dInj^{\tau_1} && \dInj^{\tau_2} && \dots && \dInj_{\tau_m = \sigma} \\
		\CC & \rTo & \CC & \rTo & \CC & \rTo & \dots & \rTo & \CC \\
	\end{diagram}

	First, we claim there are exactly \[ [\QQ(\alpha_1) : \QQ] \] ways to pick $\tau_1$.
	Observe that $\tau_1$ is determined by where it sends $\alpha_1$ (since it has to fix $\QQ$).
	Letting $p_1$ be the minimal polynomial of $\alpha_1$, we see that there are $\deg p_1$ choices for $\tau_1$,
	one for each (distinct) root of $p_1$. That proves the claim.

	Similarly, given a choice of $\tau_1$, there are
	\[ [\QQ(\alpha_1, \alpha_2) : \QQ(\alpha_1)] \]
	ways to pick $\tau_2$.
	(It's a little different: $\tau_1$ need not be the identity.
	But it's still true that $\tau_2$ is determined by where it sends $\alpha_2$,
	and as before there are $[\QQ(\alpha_1, \alpha_2) : \QQ(\alpha_1)]$ possible ways.)

	Multiplying these all together gives the desired $[K:\QQ]$.
\end{proof}
\begin{remark}
	The primitive element theorem actually implies that $m = 1$
	is sufficient; we don't need to build a whole tower.
	This simplifies the proof somewhat.
\end{remark}

It's common to see expressions like ``let $K$ be a number field of degree $n$,
and $\sigma_1, \dots, \sigma_n$ its $n$ embeddings'' without further explanation.
The relation between these embeddings and the Galois conjugates is given as follows.
\begin{theorem}[Embeddings are evenly distributed over conjugates]
	\label{thm:conj_distrb}
	Let $K$ be a number field of degree $n$
	with $n$ embeddings $\sigma_1$, \dots, $\sigma_n$,
	and let $\alpha \in K$ have $m$ Galois conjugates over $\QQ$. 

	Then $\sigma_j(\alpha)$ is ``evenly distributed''
	over each of these $m$ conjugates: for any Galois conjugate $\beta$,
	exactly $\frac nm$ of the embeddings send $\alpha$ to $\beta$.
\end{theorem}
\begin{proof}
	In the previous proof, adjoin $\alpha_1 = \alpha$ first.
\end{proof}

So, now we can define the trace and norm over $\QQ$ in a nice way:
given a number field $K$, we set
\[
	\Tr_{K/\QQ}(\alpha) = \sum_{i=1}^n \sigma_i(\alpha)
	\quad\text{and}\quad
	\Norm_{K/\QQ}(\alpha) = \prod_{i=1}^n \sigma_i(\alpha)
\]
where $\sigma_i$ are the $n$ embeddings of $K$ into $\CC$.

\section{Everyone hates characteristic 2: separable vs irreducible}
\prototype{$\QQ$ has characteristic zero, hence irreducible polynomials are separable.}
Now, we want a version of the above theorem for any field $F$.
If you read the proof, you'll see that the only thing that ever uses anything about the field $\QQ$
is \Cref{lem:irred_complex}, where we use the fact that
\begin{quote}
	\itshape Irreducible polynomials over $F$ have no double roots.
\end{quote}

Let's call a polynomial with no double roots \vocab{separable};
thus we want irreducible polynomials to be separable.
We did this for $\QQ$ in the last chapter by taking derivatives.
Should work for any field, right?

Nope.
Suppose we took the derivative of some polynomial like $2x^3 + 24x + 9$,
namely $6x^2 + 24$.
In $\CC$ it's obvious that the derivative of a nonconstant polynomial $f'$ isn't zero.
But suppose we considered the above as a polynomial in $\FF_3$, i.e.\ modulo $3$.
Then the derivative is zero.
Oh, no!

We have to impose a condition that prevents something like this from happening.
\begin{definition}
	For a field $F$, the \vocab{characteristic} of $F$ is the smallest
	positive integer $p$ such that,
	\[ \underbrace{1_F + \dots + 1_F}_{\text{$p$ times}} = 0 \]
	or zero if no such integer $p$ exists.
\end{definition}
\begin{example}[Field characteristics]
	Old friends $\RR$, $\QQ$, $\CC$ all have characteristic zero.
	But $\FF_p$, the integers modulo $p$, is a field of characteristic $p$.
\end{example}
\begin{exercise}
	Let $F$ be a field of characteristic $p$.
	Show that if $p > 0$ then $p$ is a prime number.
	(A proof is given next chapter.)
\end{exercise}
With the assumption of characteristic zero, our earlier proof works.
\begin{lemma}[Separability in characteristic zero]
	Any irreducible polynomial in a characteristic zero field is separable.
\end{lemma}
Unfortunately, this lemma is false if the ``characteristic zero'' condition is dropped.

\begin{remark}
	The reason it's called \emph{separable} is (I think) this picture:
	I have a polynomial and I want to break it into irreducible parts.
	Normally, if I have a double root in a polynomial, that means it's not irreducible.
	But in characteristic $p > 0$ this fails.
	So inseparable polynomials are strange when you think about them: somehow
	you have double roots that can't be separated from each other.
\end{remark}

We can get this to work for any field extension in which separability is not an issue.
\begin{definition}
	A \vocab{separable extension} $K/F$ is one in which every irreducible
	polynomial in $F$ is separable (for example, if $F$ has characteristic zero).
	A field $F$ is \vocab{perfect} if any finite field extension $K/F$ is separable.
\end{definition}
In fact, as we see in the next chapter:
\begin{theorem}
	[Finite fields are perfect]
	Suppose $F$ is a field with finitely many elements. Then it is perfect.
\end{theorem}
Thus, we will almost never have to worry about separability
since every field we see in the Napkin is either finite or characteristic $0$.
So the inclusion of the word ``separable'' is mostly a formality.

Proceeding onwards, we obtain
\begin{theorem}[The $n$ embeddings of any separable extension]
	Let $K/F$ be a separable extension of degree $n$ and let $\ol F$ be an algebraic closure of $F$.
	Then there are exactly $n$ field homomorphisms $K \injto \ol F$,
	say $\sigma_1, \dots, \sigma_n$, which fix $F$.
\end{theorem}

In any case, this lets us define the trace for \emph{any} separable normal extension.
\begin{definition}
Let $K/F$ be a separable extension of degree $n$, and let $\sigma_1$, \dots, $\sigma_n$
be the $n$ embeddings into an algebraic closure of $F$. Then we define
\[
	\Tr_{K/F}(\alpha) = \sum_{i=1}^n \sigma_i(\alpha)
	\quad\text{and}\quad
	\Norm_{K/F}(\alpha) = \prod_{i=1}^n \sigma_i(\alpha).
\]
When $F = \QQ$ and the algebraic closure is $\CC$, this coincides with our earlier definition!
\end{definition}


\section{Automorphism groups and Galois extensions}
\prototype{$\QQ(\sqrt 2)$ is Galois but $\QQ(\cbrt 2)$ is not.}
We now want to get back at the idea we stated at the beginning of
this section that $\QQ(\cbrt 2)$ is deficient in a way that $\QQ(\sqrt 2)$ is not.

First, we define the ``internal'' automorphisms.
\begin{definition}
	Suppose $K/F$ is a finite extension.
	Then $\Aut(K/F)$ is the set of \emph{field isomorphisms} $\sigma : K \to K$ which fix $F$.
	In symbols
	\[ \Aut(K/F) =
		\left\{
		\sigma : K \to K \mid
		\text{$\sigma$ is identity on $F$}
	  \right\}.
	\]
	This is a group under function composition!
\end{definition}
Note that this time, we have a condition that $F$ is fixed by $\sigma$.
(This was not there before when we considered $F = \QQ$, because we got it for free.)

\begin{example}[Old examples of automorphism groups]
	Reprising the example at the beginning of the chapter in the new notation, we have:
	\begin{enumerate}[(a)]
		\ii $\Aut(\QQ(i) / \QQ) \cong \Zc 2$, with elements $z \mapsto z$ and $z \mapsto \ol z$.
		\ii $\Aut(\QQ(\sqrt 2) / \QQ) \cong \Zc 2$ in the same way.
		\ii $\Aut(\QQ(\cbrt 2) / \QQ)$ is the trivial group, with only the identity embedding!
	\end{enumerate}
\end{example}

\begin{example}[Automorphism group of $\QQ(\sqrt2,\sqrt3)$]
	Here's a new example: let $K = \QQ(\sqrt2, \sqrt3)$.
	It turns out that $\Aut(K/\QQ) = \{1, \sigma, \tau, \sigma\tau\}$, where
	\[
		\sigma :
		\begin{cases}
			\sqrt2 &\mapsto -\sqrt2 \\
			\sqrt3 &\mapsto \sqrt3
		\end{cases}
		\quad\text{and}\quad
		\tau :
		\begin{cases}
			\sqrt2 &\mapsto \sqrt2 \\
			\sqrt3 &\mapsto -\sqrt3.
		\end{cases}
	\]
	In other words, $\Aut(K/\QQ)$ is the Klein Four Group.
\end{example}

First, let's repeat the proof of the observation that these embeddings shuffle around roots
(akin to the first observation in the introduction):
\begin{lemma}
	[Root shuffling in $\Aut(K/F)$]
	Let $f \in F[x]$, suppose $K/F$ is a finite extension,
	and assume $\alpha \in K$ is a root of $f$.
	Then for any $\sigma \in \Aut(K/F)$, $\sigma(\alpha)$ is also a root of $f$.
	\label{lem:root_shuffle}
\end{lemma}
\begin{proof}
	Let $f(x) = c_n x^n + c_{n-1}x^{n-1} + \dots + c_0$,
	where $c_i \in F$. Thus,
	\[ 0 = \sigma(f(\alpha)) = \sigma\left( c_n\alpha^n + \dots + c_0 \right)
	= c_n\sigma(\alpha)^n + \dots + c_0 = f(\sigma(\alpha)). \qedhere \]
\end{proof}
In particular, taking $f$ to be the minimal polynomial of $\alpha$ we deduce
\begin{moral}
	An embedding $\sigma \in \Aut(K/F)$ sends an $\alpha \in K$
	to one of its various Galois conjugates (over $F$).
\end{moral}

Next, let's look again at the ``deficiency'' of certain fields.
Look at $K = \QQ(\cbrt 2)$.
So, again $K / \QQ$ is deficient for two reasons.
First, while there are three maps $\QQ(\cbrt 2) \injto \CC$,
only one of them lives in $\Aut(K/\QQ)$, namely the identity.
In other words, $\left\lvert \Aut(K/\QQ) \right\rvert$ is \emph{too small}.
Secondly, $K$ is missing some Galois conjugates ($\omega \cbrt 2$ and $\omega^2 \cbrt 2$).

The way to capture the fact that there are missing Galois conjugates
is the notion of a splitting field.
\begin{definition}
	Let $F$ be a field and $p(x) \in F[x]$ a polynomial of degree $n$.
	Then $p(x)$ has roots $\alpha_1, \dots, \alpha_n$ in an algebraic closure of $F$.
	The \vocab{splitting field} of $F$ is defined as $F(\alpha_1, \dots, \alpha_n)$.
\end{definition}
In other words, the splitting field is the smallest field in which $p(x)$ splits.
\begin{example}[Examples of splitting fields]
	\listhack
	\begin{enumerate}[(a)]
		\ii The splitting field of $x^2 - 5$ over $\QQ$ is $\QQ(\sqrt 5)$.
		This is a degree $2$ extension.
		\ii The splitting field of $x^2+x+1$ over $\QQ$ is $\QQ(\omega)$,
		where $\omega$ is a cube root of unity.
		This is a degree $3$ extension.
		% In particular, the splitting field of $x^3-2$ is a degree \emph{six} extension.
		\ii The splitting field of $x^2+3x+2 = (x+1)(x+2)$ is just $\QQ$!
		There's nothing to do.
	\end{enumerate}
\end{example}
\begin{example}
	[Splitting fields: a cautionary tale]
	The splitting field of $x^3 - 2$ over $\QQ$ is in fact
	\[ \QQ( \cbrt 2, \omega ) \]
	and not just $\QQ(\cbrt 2)$!
	One must really adjoin \emph{all} the roots, and it's not necessarily the case that
	these roots will generate each other.

	To be clear:
	\begin{itemize}
	\ii For $x^2-5$, we adjoin $\sqrt 5$ and this will automatically include $-\sqrt 5$.
	\ii For $x^2+x+1$, we adjoin $\omega$ and get the other root $\omega^2$ for free.
	\ii But for $x^3-2$, if we adjoin $\cbrt 2$,
	we do NOT get $\omega\cbrt2$ and $\omega^2\cbrt2$ for free.
	Indeed, $\QQ(\cbrt 2) \subset \RR$!
	\end{itemize}
	Note that in particular, the splitting field of
	$x^3-2$ over $\QQ$ is \emph{degree six}, not just degree three.
\end{example}

In general,
\textbf{the splitting field of a polynomial can be an extension of degree up to $n!$}.
The reason is that if $p(x)$ has $n$ roots and none of them are ``related'' to each other,
then any permutation of the roots will work.

Now, we obtain:
\begin{theorem}[Galois extensions are splitting]
	For finite extensions $K/F$, 
	$\left\lvert \Aut(K/F) \right\rvert$ divides $[K:F]$,
	with equality if and only if $K$ is the \emph{splitting field}
	of some separable polynomial with coefficients in $F$.
	\label{thm:Galois_splitting}
\end{theorem}
The proof of this is deferred to an optional section at the end of the chapter.
If $K/F$ is a finite extension and $\left\lvert \Aut(K/F) \right\rvert = [K:F]$,
we say the extension $K/F$ is \vocab{Galois}.
In that case, we denote $\Aut(K/F)$ by $\Gal(K/F)$ instead
and call this the \vocab{Galois group} of $K/F$.

\begin{example}
	[Examples and non-examples of Galois extensions]
	\listhack
	\begin{enumerate}[(a)]
		\ii The extension $\QQ(\sqrt2) / \QQ$ is Galois,
		since it's the splitting field of $x^2-2$ over $\QQ$.
		The Galois group has order two, $\sqrt 2 \mapsto \pm \sqrt 2$.
		\ii The extension $\QQ(\sqrt2, \sqrt 3) / \QQ$ is Galois,
		since it's the splitting field of $(x^2-5)^2-6$ over $\QQ$.
		As discussed before, the Galois group is $\Zc 2 \times \Zc 2$.
		\ii The extension $\QQ(\cbrt{2}) / \QQ$ is \emph{not} Galois.
	\end{enumerate}
\end{example}

%Here is some more intuition on what $[K:F]$ actually measures: suppose $K$ is a splitting field
%of some $(x-\alpha_1) \dots (x-\alpha_n)$, meaning $K = F(\alpha_1, \dots, \alpha_n)$.
%Then a permutation $\sigma \in \Aut(K/F)$ is determined by where it sends each $\alpha_i$.
%The dimension of $[K:F]$ measures how much ``redundancy'' there is among the $\alpha_i$.
%For example, in the case of \[ (x-\sqrt5)(x+\sqrt5) = x^2-5  \]the $\sqrt 5$ and $-\sqrt 5$ were redundant,
%in the sense that knowing $\sigma(\sqrt 5)$ tells you $\sigma(-\sqrt 5) = -\sigma(\sqrt 5)$.
%But in the $x^3-2$ case, knowing $\sigma(\cbrt{2})$ does \emph{not} tell you where
%$\omega\cbrt{2}$ should go; this is reflected in the fact that $[K:F]$ and $\Aut(K/F)$
%are both six rather than three.

To explore $\QQ(\cbrt 2)$ one last time:
\begin{example}
	[Galois closures, and the automorphism group of $\QQ(\cbrt2, \omega)$]
	Let's return to the field $K = \QQ(\cbrt 2, \omega)$,
	which is a field with $[K:\QQ] = 6$.
	Consider the two automorphisms:
	\[
		\sigma:
		\begin{cases}
			\cbrt 2 &\mapsto \omega \cbrt 2 \\
			\omega &\mapsto \omega
		\end{cases}
		\quad\text{and}\quad
		\tau:
		\begin{cases}
			\cbrt 2 &\mapsto \cbrt 2 \\
			\omega &\mapsto \omega^2.
		\end{cases}
	\]
	Notice that $\sigma^3 = \tau^2 = \id$.
	From this one can see that the automorphism group of $K$ must have order $6$
	(it certainly has order $\le 6$; now use Lagrange's theorem).
	So, $K/\QQ$ is Galois! Actually one can check explicitly that
	\[ \Gal(K/\QQ) \cong S_3 \]
	is the symmetric group on $3$ elements, with order $3! = 6$.
\end{example}
This example illustrates the fact that
given a non-Galois field extension, 
one can ``add in'' missing conjugates to make it Galois.
This is called taking a \vocab{Galois closure}.

\section{Fundamental theorem of Galois theory}
After all this stuff about Galois Theory, I might as well tell you the fundamental theorem,
though I won't prove it.
Basically, it says that if $K/F$ is Galois with Galois group $G$, then:
\begin{moral}
	Subgroups of $G$ correspond exactly to fields $E$ with $F \subseteq E \subseteq K$.
\end{moral}

To tell you how the bijection goes, I have to define a fixed field.
\begin{definition}
	Let $K$ be a field and $H$ a subgroup of $\Aut(K/F)$.
	We define the \vocab{fixed field} of $H$, denoted $K^H$, as
	\[ K^H \defeq \left\{ x \in K : \sigma(x)=xK \; \forall \sigma \in H \right\}. \]
\end{definition}
\begin{ques}
	Verify quickly that $K^H$ is actually a field.
\end{ques}

Now let's look at examples again.
Consider $K = \QQ(\sqrt2, \sqrt3)$,
where \[ G = \Gal(K/\QQ) = \{\id, \sigma, \tau, \sigma\tau\} \]
is the Klein four group
(where $\sigma(\sqrt2) = -\sqrt 2$ but $\sigma(\sqrt 3) = \sqrt 3$; $\tau$ goes the other way).
\begin{ques}
	Let $H = \{\id, \sigma\}$. What is $K^H$?
\end{ques}
In that case, the diagram of fields between $\QQ$ and $K$
matches exactly with the subgroups of $G$, as follows:
\begin{center}
\begin{minipage}[t]{4cm}
	\begin{diagram}
		& \QQ(\sqrt2, \sqrt 3) & \\
		\ldLine(1,2) & \dLine & \rdLine(1,2) \\
		\QQ(\sqrt2) & \QQ(\sqrt 6) & \QQ(\sqrt 3) \\
		& \ldLine(1,2) \dLine \rdLine(1,2) & \\
		& \QQ & 
	\end{diagram}
\end{minipage}
\qquad
\begin{minipage}[t]{4cm}
	\begin{diagram}
		& \{\id\} & \\
		\ldLine(1,2) & \dLine & \rdLine(1,2) \\
		\{\id,\tau\} & \;\; \{\id, \sigma\tau\} \;\; & \{\id,\sigma\} \\
		& \ldLine(1,2) \dLine \rdLine(1,2) & \\
		& G & 
	\end{diagram}
\end{minipage}
\end{center}
We see that subgroups correspond to fixed fields.
That, and much more, holds in general.

\begin{theorem}[Fundamental theorem of Galois theory]
	Let $K/F$ be a Galois extension with Galois group $G = \Gal(K/F)$.
	\begin{enumerate}[(a)]
	\ii There is a bijection between field towers $F \subseteq E \subseteq K$ and subgroups $H \subseteq G$:
	\[
		\left\{
		\begin{array}{c}
			K \\ \mid \\ E \\ \mid \\ F
		\end{array}
		\right\}
		\iff
		\left\{
		\begin{array}{c}
			1 \\ \mid \\ H \\ \mid \\ G
		\end{array}
		\right\}
	\]
	The bijection sends $H$ to its fixed field $K^H$, and hence is inclusion reversing.
	\ii Under this bijection, we have $[K:E] = \left\lvert H \right\rvert$ and $[E:F] = [G:H]$.
	\ii $K/E$ is always Galois, and its Galois group is $\Gal(K/E) = H$.
	\ii $E/F$ is Galois if and only if $H$ is normal in $G$. If so, $\Gal(E/F) = G/H$.
	\end{enumerate}
\end{theorem}

\begin{exercise}
	Suppose we apply this theorem for 
	\[ K = \QQ(\cbrt2, \omega). \]
	Verify that the fact $E = \QQ(\cbrt 2)$ is not Galois
	corresponds to the fact that $S_3$ does not have normal subgroups of order $2$.
\end{exercise}


\section\problemhead
\begin{sproblem}[Galois group of the cyclotomic field]
	Let $p$ be an odd rational prime and $\zeta_p$ a primitive $p$th root of unity.
	Let $K = \QQ(\zeta_p)$.
	Show that \[ \Gal(K/\QQ) \cong (\ZZ/p\ZZ)^\ast. \]
	\begin{hint}
		Look at the image of $\zeta_p$.
	\end{hint}
	\begin{sol}
		It's just $\Zc{p-1}$, since $\zeta_p$ needs to get sent
		to one (any) of the $p-1$ primitive roots of unity.
	\end{sol}
\end{sproblem}

\begin{problem}[Greek constructions]
	Prove that the three Greek constructions
	\begin{enumerate}[(a)]
		\ii doubling the cube,
		\ii squaring the circle, and
		\ii trisecting an angle
	\end{enumerate}
	are all impossible.
	(Assume $\pi$ is transcendental.)
	\begin{hint}
		Repeated quadratic extensions have degree $2$, so one can
		only get powers of two.
	\end{hint}
\end{problem}

\begin{problem}
	[China Hong Kong Math Olympiad]
	Prove that there are no rational numbers $p$, $q$, $r$ satisfying
	\[ \cos\left( \frac{2\pi}{7} \right)
		= p + \sqrt{q} + \sqrt[3]{r}.  \]
	\begin{hint}
		Left-hand side has minimal polynomial of degree $7$,
		but the right-hand side lives in a degree six extension.
	\end{hint}
\end{problem}

\begin{problem}
	Show that the only automorphism of $\RR$ is the identity.
	Hence $\Aut(\RR)$ is the trivial group.
	\begin{hint}
		Hint: $\sigma(x^2) = \sigma(x)^2 \ge 0$ plus Cauchy's Functional Equation.
	\end{hint}
\end{problem}

\begin{problem}[Artin's primitive element theorem]
	\yod
	Let $K$ be a number field.
	Show that $K \cong \QQ(\gamma)$ for some $\gamma$.
	\label{prob:artin_primitive_elm}
	\begin{hint}
		By induction, suffices to show $\QQ(\alpha, \beta) = \QQ(\gamma)$
		for some $\gamma$ in terms of $\alpha$ and $\beta$.
		For all but finitely many rational $\lambda$,
		the choice $\gamma = \alpha + \lambda \beta$ will work.
	\end{hint}
	\begin{sol}
		\url{http://www.math.cornell.edu/~kbrown/6310/primitive.pdf}
	\end{sol}
\end{problem}

\section{(Optional) Proof that Galois extensions are splitting}
We prove \Cref{thm:Galois_splitting}.
First, we extract a useful fragment from the fundamental theorem.
\begin{theorem}[Fixed field theorem]
	\label{thm:fixed_field_theorem}
	Let $K$ be a field and $G$ a subgroup of $\Aut(K)$.
	Then $[K:K^G] = \left\lvert G \right\rvert$.
\end{theorem}

The inequality itself is not difficult:
\begin{exercise}
	Show that $[K:F] \ge \Aut(K/F)$,
	and that equality holds if and only if
	the set of elements fixed by all $\sigma \in \Aut(K/F)$
	is exactly $F$.
	(Use \Cref{thm:fixed_field_theorem}.)
\end{exercise}
The equality case is trickier.

The easier direction is when $K$ is a splitting field.
Assume $K = F(\alpha_1, \dots, \alpha_n)$ is the splitting field of some separable polynomial $p \in F[x]$
with $n$ distinct roots $\alpha_1, \dots, \alpha_n$.
Adjoin them one by one:
\begin{diagram}
	F & \rInj & F(\alpha_1) & \rInj & F(\alpha_1, \alpha_2) & \rInj & \dots & \rInj & K \\
	\dTo^\id && \dTo^{\tau_1} && \dTo^{\tau_2} && \dots && \dTo_{\tau_n = \sigma} \\
	F & \rInj & F(\alpha_1) & \rInj & F(\alpha_1, \alpha_2) & \rInj & \dots & \rInj & K \\
\end{diagram}
(Does this diagram look familiar?)
Every map $K \to K$ which fixes $F$ corresponds to an above commutative diagram.
As before, there are exactly $[F(\alpha_1) : F]$ ways to pick $\tau_1$.
(You need the fact that the minimal polynomial $p_1$ of $\alpha_1$ is separable for this:
there need to be exactly $\deg p_1 = [F(\alpha_1) : F]$ distinct roots to nail $p_1$ into.)
Similarly, given a choice of $\tau_1$, there are $[F(\alpha_1, \alpha_2) : F(\alpha_1)]$ ways to pick $\tau_2$.
Multiplying these all together gives the desired $[K:F]$.

\bigskip

Now assume $K/F$ is Galois.
First, we state:
\begin{lemma}
	Let $K/F$ be Galois, and $p \in F[x]$ irreducible.
	If any root of $p$ (in $\ol F$) lies in $K$, then all of them do,
	and in fact $p$ is separable.
\end{lemma}
\begin{proof}
	Let $\alpha \in K$ be the prescribed root.
	Consider the set
	\[ S = \left\{ \sigma(\alpha) \mid \sigma \in \Gal(K/F) \right\}. \]
	(Note that $\alpha \in S$ since $\Gal(K/F) \ni \id$.)
	By construction, any $\tau \in \Gal(K/F)$ fixes $S$.
	So if we construct
	\[ \tilde p(x) = \prod_{\beta \in S} (x - \beta), \]
	then by Vieta's Formulas, we find that all the coefficients of $\tilde p$ are fixed by elements of $\sigma$.
	By the \emph{equality case} we specified in the exercise, it follows that $\tilde p$ has coefficients in $F$!
	(This is where we use the condition.)
	Also, by \Cref{lem:root_shuffle}, $\tilde p$ divides $p$.

	Yet $p$ was irreducible, so it is the minimal polynomial of $\alpha$ in $F[x]$,
	and therefore we must have that $p$ divides $\tilde p$.
	Hence $p = \tilde p$. Since $\tilde p$ was built to be separable, so is $p$.
\end{proof}
Now we're basically done -- pick a basis $\omega_1$, \dots, $\omega_n$ of $K/F$,
and let $p_i$ be their minimal polynomials; by the above, we don't get any roots outside $K$.
Consider $P = p_1 \dots p_n$, removing any repeated factors.
The roots of $P$ are $\omega_1$, \dots, $\omega_n$ and some other guys in $K$.
So $K$ is the splitting field of $P$.
