\chapter{Dirichlet's theorem on arithmetic progressions}

\section{Introduction}

In this chapter, we will sketch a proof of the following theorem:
\begin{theorem}[Dirichlet's theorem on arithmetic progressions]
	For every positive integers $a, n > 0$ such that $\gcd(a, n) = 1$, then there are infinitely
	many primes coprime to $a$ modulo $n$.
\end{theorem}

For an even more concrete example:
\begin{corollary}
There are infinitely many primes $p$ that is congruent to $3$ modulo $4$.
\end{corollary}

You can see why this is called a theorem on arithmetic progressions:
the example above is equivalent to saying that there are infinitely many primes in the arithmetic
progression
\[ 3, 7, 11, 15, 19, 23, \dots \]

If you haven't noticed, the condition $\gcd(a, n) = 1$ is also very natural:
look at the arithmetic progression $2, 6, 10, \dots$ where $a = 2$ and $n = 4$ are not coprime,
and you will see that it obviously has only finitely many primes.

(By the way, using Chebotarev density theorem which will be mentioned in the next section,
you can even deduce something stronger: for example, a ``randomly chosen'' prime has probability
$\frac{1}{2}$ to be congruent to $3$ modulo $4$. But here we will give a simple sketch of proof.)

\section{The $L$-series}

Actually, this chapter is just an excuse to introduce you to something called the $L$-series.
Here is a quote from \cite{ref:neukirch} (emphasis mine):

\begin{quote}
One of the most astounding phenomena in number theory consists in the
fact that a great number of deep arithmetic properties of a number field are
hidden within a single analytic function, its \vocab{zeta function}. This function has
a simple shape, but it is unwilling to yield its mysteries. Each time, however,
that we succeed in stealing one of these well-guarded truths, we may expect to
be rewarded by the revelation of some \emph{surprising} and \emph{significant} relationship.
This is why zeta functions, as well as their generalizations, the $L$-series,
have increasingly moved to the foreground of the arithmetic scene, and today
are more than ever the focus of number-theoretic research.
\end{quote}

Surprisingly, you will see $L$-series in many other places as well --- an elliptic curve over $\QQ$
has a $L$-series, and a modular form also have a $L$-series!\todo{really?
And is there more examples?}

So what is a $L$-series? In one sentence:\footnote{In my (limited) understanding, at the time of
writing this.}
\begin{moral}
	A $L$-series bridges between an infinite sum (complex analysis)
	and an infinite product over primes (number theory)
	using unique factorization,
	thus using tools from complex analysis to deduce hidden facts about the primes.
\end{moral}

Okay, that is a long sentence, sorry.

\subsection{Riemann zeta function}

Let's look at the Riemann zeta function, which is the prototype for a $L$-series:

\begin{definition}[Riemann zeta function]
	When $\Re(s) > 1$, we define the \vocab{Riemann zeta function} by the following infinite sum
	\[ \zeta(s) = \frac{1}{1^s} + \frac{1}{2^s} + \frac{1}{3^s} + \cdots. \]
	We can prove that the series converges whenever $\Re(s) > 1$, and that there exists an unique
	meromorphic function on $\CC$ that is equal to the series above.
	For $\Re(s) \leq 1$ define $\zeta(s)$ using that meromorphic function.
\end{definition}

This is called an \vocab{analytic continuation} of the series to the whole complex plane.
(If you recalled:
the identity theorem in complex analysis guarantees that if such a meromorphic function exists,
it must be unique. The difficult part is to prove it exists.)

So far, so good --- we have seen the infinite sum (complex analysis) part. How about the primes
(number theory) part?

For the Riemann zeta function, the idea has been discovered centuries ago by Euler
(though it wasn't formalized until much later).
Thus such product over primes are aptly called \vocab{Euler products}.
The \href{https://en.wikipedia.org/wiki/Divergence_of_the_sum_of_the_reciprocals_of_the_primes}{%
Wikipedia page} has a good explanation.

Here are some manipulations that is not fully justified.
We can factorize each integer in the denominator as follows:
\[ \zeta(s) = 1 + \frac{1}{2^s}
	+ \frac{1}{3^s}
	+ \frac{1}{2^{2s}}
	+ \frac{1}{5^s}
	+ \frac{1}{2^s} \cdot \frac{1}{3^s}
	+ \cdots
\]
Then, group like terms:
\[ \zeta(s) =
	\left(1 + \frac{1}{2^s} + \frac{1}{2^{2s}} + \cdots \right)
	\left(1 + \frac{1}{3^s} + \frac{1}{3^{2s}} + \cdots \right)
	\left(1 + \frac{1}{5^s} + \frac{1}{5^{2s}} + \cdots \right) \cdots
\]
By unique factorization, this grouping is indeed correct.
So, our infinite product is:
\[ \zeta(s) = \prod_{p\text{ prime}} \left(1 + \frac{1}{p^s} + \frac{1}{p^{2s}} + \cdots\right). \]

\begin{theorem}
	\label{thm:riemann_zeta_product_rep}
	For $\Re(s) > 1$, this infinite product converges to the same value as $\zeta(s)$.
	(That is, our questionable manipulations can be proven correct.)
\end{theorem}
The proof is omitted.

Because the ``$\cdots$'' are verbose,
we note that the series $1 + x + x^2 + \cdots$ is equal to $\frac{1}{1-x}$.
So, for brevity we write
\[ \zeta(s) = \prod_{p\text{ prime}} \frac{1}{1-p^{-s}}. \]

\subsection{Example: Euclid's theorem}

We will use \Cref{thm:riemann_zeta_product_rep} to give a roundabout proof of the following:
\begin{theorem}[Euclid's theorem]
	There are infinitely many primes.
\end{theorem}

\begin{proof}[Sketch of Proof]
	Note that, by the infinite sum representation,
	$\zeta(1) = 1+\frac{1}{2}+\frac{1}{3}+\cdots$ is the harmonic series and diverges.

	Now look at the Euler product representation,
	$\zeta(1) = \prod_{p\text{ prime}} \frac{1}{1-p^{-1}}$.
	For this to diverge, there must be infinitely many primes $p$.
\end{proof}

\section{The actual proof}

Here comes the actual proof, or at least a sketch of it.
Just as above, the plan is to do the following:
\begin{itemize}
	\ii Write down some infinite sum.
	\ii Prove that it is equal to some Euler product.
	\ii Use the infinite sum to determine the behavior of the function at $s = 1$.
	\ii Use the Euler product to say something about the primes.
\end{itemize}

Easier said than done.
You might guess we could try something like
\[
	\frac{1}{3^s} + \frac{1}{7^s} + \frac{1}{11^s} + \cdots =
	\prod_{p\text{ prime},\ p \equiv 3 \ (\opname{mod} 4)} \frac{1}{1-p^{-s}}
\]
Unfortunately, this fails miserably. (Try to expand out a few terms of the product.)

How can we fix it? Look at this:
\[
	\frac{1}{1^s} - \frac{1}{3^s} + \frac{1}{5^s} - \frac{1}{7^s} + \frac{1}{9^s} - \cdots
\]
\begin{exercise}
	Do the ``non-rigorous'' factorization and grouping of like terms as above. It should work out!
\end{exercise}

If you did the exercise correctly, you should get the following:
\[
	\left( \frac{1}{1+3^{-s}}\right)
	\left( \frac{1}{1-5^{-s}}\right)
	\left( \frac{1}{1+7^{-s}}\right)
	\left( \frac{1}{1+11^{-s}}\right)
	\left( \frac{1}{1-13^{-s}}\right) \cdots
\]
In order to save us from writing down a lot of ``$\cdots$'', for now, define the following function
on positive integers integers $n$:
\[ \chi(n) = \begin{cases}
	1 &\text{if }n \equiv 1 \pmod{4} \\
	-1&\text{if }n \equiv 3 \pmod{4} \\
	0 &\text{otherwise.}
\end{cases} \]
Then we have
\[ \sum_{n \geq 1} \frac{\chi(n)}{n^s} = \prod_{p\text{ prime}} \frac{1}{1-\chi(p) \cdot p^{-s}}. \]

Of course, what we have seen deserves a name:
\begin{definition}[Dirichlet character, informal]
	A \vocab{Dirichlet character} is a function $\chi$ as above that we are interested in and makes
	the equality works.
\end{definition}

\begin{exercise}
	Notice that the sum and product in our broken attempt also comes from a function as follows:
	\[ \chi(n) = \begin{cases}
		\textcolor{red}{0} &\text{if }n \equiv 1 \pmod{4} \\
		\textcolor{red}{1} &\text{if }n \equiv 3 \pmod{4} \\
		0 &\text{otherwise.}
	\end{cases} \]
	What makes this function $\chi$ not work (so, it's not a Dirichlet character)?
\end{exercise}

So what's the point? Turns out there's \emph{a lot} of Dirichlet characters ---
enough for us to write down (almost) the product we want!

Look: if we consider
\[ \chi_2(n) = \begin{cases}
	1 &\text{if }n \equiv 1 \pmod{4} \\
	1 &\text{if }n \equiv 3 \pmod{4} \\
	0 &\text{otherwise.}
\end{cases} \]
then we also have
\[ \sum_{n \geq 1} \frac{\chi_2(n)}{n^s} =
\prod_{p\text{ prime}} \frac{1}{1-\chi_2(p) \cdot p^{-s}}. \]

Wait, writing down all these $\sum$ signs are boring. Let us define:
\begin{definition}[$L$-series]
	For $\chi$ a Dirichlet character, define the \vocab{$L$-series}
	\[ L(\chi, s) \sum_{n \geq 1} \frac{\chi(n)}{n^s}
	= \prod_{p\text{ prime}} \frac{1}{1-\chi(p) \cdot p^{-s}}. \]
\end{definition}

Okay, back to the proof. Writing things down concretely,
\begin{align*}
	L(\chi, s) &=
	\left( \frac{1}{1+3^{-s}} \right)
	\left( \frac{1}{1-5^{-s}} \right)
	\left( \frac{1}{1+7^{-s}} \right)
	\left( \frac{1}{1+11^{-s}} \right)
	\left( \frac{1}{1-13^{-s}} \right) \cdots
	\\
	L(\chi_2, s) &=
	\left( \frac{1}{1-3^{-s}} \right)
	\left( \frac{1}{1-5^{-s}} \right)
	\left( \frac{1}{1-7^{-s}} \right)
	\left( \frac{1}{1-11^{-s}} \right)
	\left( \frac{1}{1-13^{-s}} \right) \cdots
\end{align*}
We want to ``forget'' about the primes in the arithmetic progression $1, 5, 9, 13, \dots$
so obviously we're going to divide the two series:
\[
	\frac{L(\chi, s)}{L(\chi_2, s)} =
	\left( \frac{1-3^{-s}}{1+3^{-s}} \right)
	\left( \frac{1-7^{-s}}{1+7^{-s}} \right)
	\left( \frac{1-11^{-s}}{1+11^{-s}} \right) \cdots
\]
You can easily guess what are the remaining steps:
\begin{itemize}
	\ii The numerator $L(\chi, s)$ is finite at $s = 1$.
	\ii The denominator $L(\chi_2, s)$ diverges at $s = 1$.
	\ii So the left hand side is $0$.
	\ii This cannot happen if there are only finitely many terms on the right hand side.
\end{itemize}

\section{Generalization}

It is not a coincidence that the $\chi$ above are called \emph{characters} --- functions sure, but
characters?

Turns out, if you have read the chapter on representation theory,
it has a lot in common with characters of representations:
\begin{itemize}
	\ii they're independent,
	\ii they're orthogonal,
	\ii they span the whole space.\todo{double check. Besides, what space?}
\end{itemize}

Abstract algebra aside, how are we going to construct the product in such a way that only the primes
congruent to $a$ modulo $m$ remains?
Look:

\begin{theorem}
	Fix a modulus $m$.
	Let $\chi_1, \chi_2, \dots$ be all the Abelian group homomorphism
	$\Zm{m} \to \CC^{\times}$.
	For each of them, define a $\NN \to \CC$ function the obvious way:
	\[ \chi_i(n) = \begin{cases}
		\chi_i(n \bmod m)&\text{if }\gcd(n, m) = 1 \\
		0&\text{otherwise.}
	\end{cases} \]
	Then there are exactly $|\Zm{m}|$ characters, they're Dirichlet characters,
	and most importantly, you can extract out \emph{only} the primes congruent to $a$ modulo $m$ by:
	\[
		\prod_{i=1}^{|\Zm{m}|} L(\chi_i, s)^{1/\chi_i(a)}.
	\]
\end{theorem}
\todo{this is a lie, explain what the product expand to. See \cite[page 469]{ref:neukirch}}

And:
\begin{proposition} % Neukirch chapter 7, proposition 5.13
	\label{prop:l1_finite_nonzero}
	With notation as above, unless $\chi_i$ is the trivial character which assigns $1$ to every
	element of $\Zm{m}$, then
	\[ L(\chi_i, 1) \in \CC \setminus \{0\}. \]
\end{proposition}

\begin{remark}[Anecdote]
	$|\Zm{m}|$ is equal to $\phi(m)$, where $\phi$ is the Euler's totient function.
\end{remark}

\section{More proof}

In this section, we will prove \Cref{prop:l1_finite_nonzero}.

\begin{exercise}
	Recall the Dirichlet character we used in the example above.
	\[ \chi(n) = \begin{cases}
		1 &\text{if }n \equiv 1 \pmod{4} \\
		-1&\text{if }n \equiv 3 \pmod{4} \\
		0 &\text{otherwise.}
	\end{cases} \]
	Try to prove $L(\chi, 1)$ is finite.
\end{exercise}

That one was easy --- the sum the series is decreasing and alternating.
But what if the modulus is $5$?

Let us start with writing down all Dirichlet characters modulo $5$:
\[
	\begin{array}{|r|rrrr|}
		\hline
		    & 1 & 2 & 3 & 4 \\ \hline
		\chi_1 & 1 & 1 & 1 & 1 \\
		\chi_2 & 1 & i & -i & -1 \\
		\chi_3 & 1 & -i & i & -1 \\ \hline
	\end{array}
\]
Great, we got some complex numbers.
No worries --- complex numbers need complex tools to deal with it.

(Recall that a Dirichlet character can be defined from a group homomorphism $\Zm{5} \to
\CC^{\times}$. Of course the value $\chi(n)$ would be $0$ at values $5 \mid n$.)

Consider the cyclotomic field extension $K = \QQ(\zeta_5)$.
We are going to \emph{generalize the Riemann zeta function $\zeta$ to $\zeta_K$}!

How? To start with, there is no unique factorization of elements in arbitrary number fields,
and there is no notation of \emph{positive} or \emph{negative} values either.

We do have unique factorization of ideals though, so maybe we can make something out of ideals?
Let's see:
\begin{definition} % Neukirch chapter 7, definition 5.1
	Define the Dedekind zeta function by
	\[ \zeta_K(s) = \sum_{\ka\text{ ideal of }\OO_K} \frac{1}{\Norm(\ka)^s}. \]
	When $\Re(s) \leq 1$, do analytic continuation as in the case of Riemann zeta function.
\end{definition}
\begin{exercise}
	Verify that when $K = \QQ$ then $\zeta_\QQ = \zeta$.
\end{exercise}

Of course we have the following ``Euler product'':
\begin{proposition}
	\[ \zeta_K(s) = \prod_{\kp\text{ prime ideal of }\OO_K} \frac{1}{1-\Norm(\kp)^{-s}}. \]
\end{proposition}

So?\todo{double check (how do you numerically evaluate $\zeta_K$ for $K \neq \QQ$ anyway?)}
\begin{proposition} % Neukirch chapter 7, proposition 5.12, specialized to m = 5
	For $K = \QQ(\zeta_5)$ and $\chi_i$ the Dirichlet characters modulo $5$ defined above, then
	\[ \zeta_K(s) = \frac{1}{1-5^{-s}} \cdot \prod_{i=1}^{3} L(\chi_i, s). \]
\end{proposition}
We can prove that at $s = 1$, $\zeta_K(s)$ has a pole of order exactly $1$,
which means only $L(\chi_1, s)$ is allowed to have a pole there.
The other $L(\chi_i, s)$ are thus not allowed to have one.\todo{why can't it be that one has a pole
and the other have a root to cancel it out?}

