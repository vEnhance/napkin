\chapter{The ring of integers}
\section{Norms and traces}
\prototype{$a+b\sqrt2$ as an element of $\QQ(\sqrt2)$ has norm $a^2-2b^2$ and trace $2a$.}
Remember when you did olympiads and we had like $a^2+b^2$ was the ``norm'' of $a+bi$?
Cool, let me tell you what's actually happening.

First, let me make precise the notion of a conjugate.
\begin{definition}
	Let $\alpha$ be an algebraic number, and let $P(x)$ be its minimal polynomial,
	of degree $m$.
	Then the $m$ roots of $P$ are the (Galois) \vocab{conjugates} of $\alpha$.
\end{definition}
It's worth showing at the moment that there are no repeated conjugates.
\begin{lemma}[Irreducible polynomials have distinct roots]
	An irreducible polynomial in $\QQ[x]$ cannot have a complex double root.
	\label{lem:irred_complex}
\end{lemma}
\begin{proof}
	Let $f(x) \in \QQ[x]$ be the irreducible polynomial and assume it has a double root $\alpha$.
	\textbf{Take the derivative $f'(x)$.}
	This derivative has three interesting properties.
	\begin{itemize}
		\ii The degree of $f'$ is one less than the degree of $f$.
		\ii The polynomials $f$ and $f'$ are not relatively prime
		because they share a factor $x-\alpha$.
		\ii The coefficients of $f'$ are also in $\QQ$.
	\end{itemize}
	Consider $g = \gcd(f, f')$. We must have $g \in \QQ[x]$ by Euclidean algorithm.
	But the first two facts about $f'$ ensure that $g$ is nonconstant
	and $\deg g < \deg f$.
	Yet $g$ divides $f$,
	contradiction to the fact that $f$ should be a minimal polynomial.
\end{proof}
Hence $\alpha$ has exactly as many conjugates as the degree of $\alpha$.

Now, we would \emph{like} to define the \emph{norm} of an element $\Norm(\alpha)$
as the product of its conjugates.
For example, we want $2+i$ to have norm $(2+i)(2-i) = 5$,
and in general for $a+bi$ to have norm $a^2+b^2$.
It would be \emph{really cool} if the norm was multiplicative;
we already know this is true for complex numbers!

Unfortunately, this doesn't quite work: consider
\[ \Norm(2+i) = 5 \text{ and } \Norm(2-i) = 5. \]
But $(2+i)(2-i) = 5$, which doesn't have norm $25$ like we want,
since $5$ is degree $1$ and has no conjugates at all.
The reason this ``bad'' thing is happening is that we're
trying to define the norm of an \emph{element},
when we really ought to be defining the norm of an element
\emph{with respect to a particular $K$}.

What I'm driving at is that the norm should have
different meanings depending on which field you're in.
If we think of $5$ as an element of $\QQ$, then its norm is $5$.
But thought of as an element of $\QQ(i)$, its norm really ought to be $25$.
Let's make this happen: for $K$ a number field, we will now define $\Norm_{K/\QQ}(\alpha)$
to be the norm of $\alpha$ \emph{with respect to $K$} as follows.
\begin{definition}
	Let $\alpha \in K$ have degree $n$, so $\QQ(\alpha) \subseteq K$, and set $k = (\deg K) / n$.
	The \vocab{norm} of $\alpha$ is defined as
	\[ \Norm_{K/\QQ}(\alpha) \defeq \left( \prod \text{Galois conj of $\alpha$} \right)^k. \]
	The \vocab{trace} is defined as
	\[ \Tr_{K/\QQ}(\alpha) \defeq k \cdot \left( \sum \text{Galois conj of $\alpha$} \right). \]
\end{definition}
The exponent of $k$ is a ``correction factor'' that makes the norm of $5$ into $5^2=25$
when we view $5$ as an element of $\QQ(i)$ rather than an element of $\QQ$.
For a ``generic'' element of $K$, we expect $k = 1$.
\begin{exercise}
	Use what you know about nested vector spaces to convince
	yourself that $k$ is actually an integer.
\end{exercise}
\begin{example}[Norm of $a+b\sqrt2$]
	Let $\alpha = a+b\sqrt2 \in \QQ(\sqrt2) = K$.
	If $b \neq 0$, then $\alpha$ and $K$ have the degree $2$.
	Thus the only conjugates of $\alpha$ are $a \pm b\sqrt2$, which gives
	the norm \[ (a+b\sqrt2)(a-b\sqrt2) = a^2-2b^2, \]
	The trace is $(a-b\sqrt2) + (a+b\sqrt2) = 2a$.

	Nicely, the formula $a^2-2b^2$ and $2a$ also works when $b=0$.
\end{example}
Of importance is:
\begin{proposition}[Norms and traces are rational integers]
	If $\alpha$ is an algebraic integer, its norm and trace
	are rational integers.
\end{proposition}
\begin{ques}
	Prove it. (Vieta formula.)
\end{ques}

That's great, but it leaves a question unanswered:
why is the norm multiplicative?
To do this, I have to give a new definition of norm and trace.

\begin{theorem}[Morally correct definition of norm and trace]
	Let $K$ be a number field of degree $n$, and let $\alpha \in K$.
	Let $\mu_\alpha \colon K \to K$ denote the map \[ x \mapsto \alpha x \]
	viewed as a linear map of $\QQ$-vector spaces.
	Then,
	\begin{itemize}
		\ii the norm of $\alpha$ equals the determinant $\det \mu_\alpha$, and
		\ii the trace of $\alpha$ equals the trace $\Tr \mu_\alpha$.
	\end{itemize}
\end{theorem}
Since the trace and determinant don't depend on the choice of basis,
you can pick whatever basis you want
and use whatever definition you got in high school.
Fantastic, right?

\begin{example}[Explicit computation of matrices for $a+b\sqrt2$]
	Let $K = \QQ(\sqrt2)$, and let $1$, $\sqrt 2$ be the basis of $K$.
	Let \[ \alpha = a + b \sqrt 2 \] (possibly even $b = 0$), and notice that
	\[ \left( a+b\sqrt2 \right) \left(x+y\sqrt2 \right)
		= (ax+2yb) + (bx+ay)\sqrt2. \]
	We can rewrite this in matrix form as
	\[
		\begin{bmatrix}
			a & 2b \\
			b & a
		\end{bmatrix}
		\begin{bmatrix}
			x \\ y
		\end{bmatrix}
		=
		\begin{bmatrix}
			ax+2yb \\ bx+ay
		\end{bmatrix}.
	\]
	Consequently, we can interpret $\mu_\alpha$ as the matrix
	\[ \mu_\alpha =
		\begin{bmatrix}
			a & 2b \\ b & a
		\end{bmatrix}. \]
	Of course, the matrix will change if we pick a different basis,
	but the determinant and trace do not: they are always given by
	\[ \det \mu_\alpha = a^2-2b^2 \text{ and }
		\Tr \mu_\alpha = 2a. \]
\end{example}
This interpretation explains why the same formula should work for $a+b\sqrt 2$
even in the case $b = 0$.

\begin{proof}
I'll prove the result for just the norm; the trace falls out similarly.
Set
\[ n = \deg \alpha, \qquad kn = \deg K. \]
The proof is split into two parts, depending on whether or not $k=1$.
\begin{subproof}[Proof if $k=1$]
	Set $n = \deg \alpha = \deg K$.
	Thus the norm actually \emph{is} the product of the Galois conjugates.
	Also, \[ \{1, \alpha, \dots, \alpha^{n-1}\} \]
	is linearly independent in $K$, and hence a basis (as $\dim K = n$).
	Let's use this as the basis for $\mu_\alpha$.

	Let \[ x^n+c_{n-1}x^{n-1} + \dots + c_0  \]be the minimal polynomial of $\alpha$.
	Thus $\mu_\alpha(1) = \alpha$, $\mu_\alpha(\alpha) = \alpha^2$, and so on,
	but $\mu_\alpha(\alpha^{n-1}) = -c_{n-1}\alpha^{n-1} - \dots - c_0$.
	Therefore, $\mu_\alpha$ is given by the matrix
	\[
		M =
		\begin{bmatrix}
			0 & 0 & 0 & \dots & 0 & -c_0 \\
			1 & 0 & 0 & \dots & 0 & -c_1 \\
			0 & 1 & 0 & \dots & 0 & -c_2 \\
			\vdots & \vdots & \vdots & \ddots & 0 & -c_{n-2} \\
			0 & 0 & 0 & \dots & 1 & -c_{n-1}
		\end{bmatrix}.
	\]
	Thus \[ \det M = (-1)^n c_0 \] and we're done by Vieta's formulas.
\end{subproof}
\begin{subproof}[Proof if $k > 1$]
	We have nested vector spaces
	\[ \QQ \subseteq \QQ(\alpha) \subseteq K. \]
	Let $e_1$, \dots, $e_k$ be a $\QQ(\alpha)$-basis for $K$
	(meaning: interpret $K$ as a vector space over $\QQ(\alpha)$, and pick that basis).
	Since $\{1, \alpha, \dots, \alpha^{n-1}\}$ is a $\QQ$ basis for $\QQ(\alpha)$,
	the elements
	\[
			\begin{array}{cccc}
			e_1, & e_1\alpha, & \dots, & e_1\alpha^{n-1} \\
			e_2, & e_2\alpha, & \dots, & e_2\alpha^{n-1} \\
			\vdots & \vdots & \ddots & \vdots \\
			e_k, & e_k\alpha, & \dots, & e_k\alpha^{n-1}
			\end{array}
	\]
	constitute a $\QQ$-basis of $K$.
	Using \emph{this} basis, the map $\mu_\alpha$ looks like
	\[
			\underbrace{
			\begin{bmatrix}
					M & & & \\
					& M & & \\
					& & \ddots & \\
					& & & M
			\end{bmatrix}
			}_{\text{$k$ times}}
	\]
	where $M$ is the same matrix as above:
	we just end up with one copy of our old matrix for each $e_i$.
	Thus $\det \mu_\alpha = (\det M)^k$, as needed. \qedhere
\end{subproof}
\begin{ques}
	Verify the result for traces as well. \qedhere
\end{ques}
\end{proof}

From this it follows immediately that
\[ \Norm_{K/\QQ}(\alpha\beta) = \Norm_{K/\QQ}(\alpha)\Norm_{K/\QQ}(\beta) \]
because by definition we have
\[ \mu_{\alpha\beta} = \mu_\alpha \circ \mu_\beta, \]
and that the determinant is multiplicative.
In the same way, the trace is additive.

\section{The ring of integers}
\prototype{If $K = \QQ(\sqrt 2)$, then $\OO_K = \ZZ[\sqrt 2]$.
	But if $K = \QQ(\sqrt 5)$, then $\OO_K = \ZZ[\frac{1+\sqrt5}{2}]$.}

$\ZZ$ makes for better number theory than $\QQ$.
In the same way, focusing on the \emph{algebraic integers} of $K$
gives us some really nice structure, and we'll do that here.

\begin{definition}
	Given a number field $K$, we define
	\[ \OO_K \defeq K \cap \ol\ZZ \]
	to be the \vocab{ring of integers} of $K$;
	in other words $\OO_K$ consists of the algebraic integers of $K$.
\end{definition}

We do the classical example of a quadratic field now.
Before proceeding, I need to write a silly number theory fact.
\begin{exercise}
	[Annoying but straightforward]
	Let $a$ and $b$ be rational numbers, and $d$ a squarefree positive integer.
	\begin{itemize}
		\ii If $d \equiv 2, 3 \pmod 4$, prove that
		$2a, a^2-db^2 \in \ZZ$ if and only if $a,b \in \ZZ$.
		\ii For $d \equiv 1 \pmod 4$, prove that
		$2a, a^2-db^2 \in \ZZ$ if and only if $a,b \in \ZZ$
		OR if $a -\half, b-\half \in \ZZ$.
	\end{itemize}
	You'll need to take mod $4$.
\end{exercise}

\begin{example}
	[Ring of integers of $K = \QQ(\sqrt3)$]
	Let $K$ be as above.
	We claim that \[ \OO_K = \ZZ[\sqrt 3] = \left\{ m + n\sqrt 3 \mid m,n \in \ZZ  \right\}. \]
	We set $\alpha = a + b \sqrt 3$.
	Then $\alpha \in \OO_K$ when the minimal polynomial has integer coefficients.

	If $b = 0$, then the minimal polynomial is $x-\alpha=x-a$,
	and thus $\alpha$ works if and only if it's an integer.
	If $b \neq 0$, then the minimal polynomial is
	\[ (x-a)^2 - 3b^2 = x^2 - 2a \cdot x + (a^2-3b^2). \]
	From the exercise, this occurs exactly for $a,b \in \ZZ$.
\end{example}
\begin{example}
	[Ring of integers of $K = \QQ(\sqrt 5)$]
	We claim that in this case
	\[ \OO_K = \ZZ\left[ \frac{1+\sqrt5}{2} \right]
	= \left\{ m + n \cdot \frac{1+\sqrt5}{2} \mid m,n \in \ZZ \right\}. \]
	The proof is exactly the same, except the exercise tells us instead
	that for $b \neq 0$, we have both the possibility that $a,b \in \ZZ$
	or that $a,b \in \ZZ - \half$.
	This reflects the fact that $\frac{1+\sqrt5}{2}$ is the root of $x^2-x-1 = 0$;
	no such thing is possible with $\sqrt 3$.
\end{example}
In general, the ring of integers of $K = \QQ(\sqrt d)$ is
\[ \OO_K
	=
	\begin{cases}
		\ZZ[\sqrt d] & d\equiv 2,3 \pmod 4 \\[1em]
		\ZZ\left[ \frac{1+\sqrt d}{2} \right] & d \equiv 1 \pmod 4.
	\end{cases}
\]
What we're going to show is that $\OO_K$ behaves in $K$
a lot like the integers do in $\QQ$.
First we show $K$ consists of quotients of numbers in $\OO_K$.
In fact, we can do better:
\begin{example}[Rationalizing the denominator]
	For example, consider $K = \QQ(\sqrt3)$.
	The number $x = \frac{1}{4+\sqrt3}$ is an element of $K$, but by
	``rationalizing the denominator'' we can write
	\[ \frac{1}{4+\sqrt3} = \frac{4-\sqrt3}{13}. \]
	So we see that in fact, $x$ is $\frac{1}{13}$ of an integer in $\OO_K$.
\end{example}

The theorem holds true more generally.
\begin{theorem}[$K = \QQ \cdot \OO_K$]
	Let $K$ be a number field, and let $x \in K$ be any element.
	Then there exists an integer $n$ such that $nx \in \OO_K$;
	in other words, \[ x = \frac 1n \alpha \] for some $\alpha \in \OO_K$.
\end{theorem}
\begin{exercise}
	Prove this yourself.
	(Start by using the fact that $x$ has a minimal
	polynomial with rational coefficients.
	Alternatively, take the norm.)
\end{exercise}

Now we are going to show $\OO_K$ is a ring;
we'll check it is closed under addition and multiplication.
To do so, the easiest route is:
\begin{lemma}[$\alpha \in \ol\ZZ$ $\iff$ {$\ZZ[\alpha]$} finitely generated]
	Let $\alpha \in \ol\QQ$.
	Then $\alpha$ is an algebraic integer if and only if
	the abelian group $\ZZ[\alpha]$ is finitely generated.
\end{lemma}
\begin{proof}
	Note that $\alpha$ is an algebraic integer if and only if it's the root
	of some nonzero, monic polynomial with integer coefficients.
	Suppose first that
	\[ \alpha^N = c_{N-1} \alpha^{N-1} + c_{N-2} \alpha^{N-2} + \dots + c_0. \]
	Then the set $1, \alpha, \dots, \alpha^{N-1}$ generates $\ZZ[\alpha]$,
	since we can repeatedly replace $\alpha^N$ until all powers of $\alpha$
	are less than $N$.

	Conversely, suppose that $\ZZ[\alpha]$ is finitely generated
	by some $b_1, \dots, b_m$.
	Viewing the $b_i$ as polynomials in $\alpha$, we can select a large integer
	$N$ (say $N = \deg b_1 + \dots + \deg b_m + 2015$)
	and express $\alpha^N$ in the $b_i$'s to get
	\[ \alpha^N = c_1b_1(\alpha) + \dots + c_mb_m(\alpha). \]
	The above gives us a monic polynomial in $\alpha$,
	and the choice of $N$ guarantees it is not zero.
	So $\alpha$ is an algebraic integer.
\end{proof}
\begin{example}[$\half$ isn't an algebraic integer]
	We already know $\half$ isn't an algebraic integer.
	So we expect
	\[ \ZZ \left[ \half \right] = \left\{ \frac{a}{2^m}
		\mid a, m \in \ZZ \text{ and } m \ge 0 \right\} \]
	to not be finitely generated, and this is the case.
\end{example}
\begin{ques}
	To make the last example concrete:
	name all the elements of $\ZZ[\half]$
	that cannot be written as an integer combination of
	\[ \left\{ \frac12, \frac{7}{8}, \frac{13}{64},
		\frac{2015}{4096}, \frac{1}{1048576} \right\} \]
\end{ques}

Now we can state the theorem.
\begin{theorem}[Algebraic integers are closed under $+$ and $\times$]
	The set $\ol\ZZ$ is closed under addition and multiplication;
	i.e.\ it is a ring.
	In particular, $\OO_K$ is also a ring for any number field $K$.
\end{theorem}
\begin{proof}
	Let $\alpha, \beta \in \ol\ZZ$.
	Then $\ZZ[\alpha]$ and $\ZZ[\beta]$ are finitely generated.
	Hence so is $\ZZ[\alpha, \beta]$.
	(Details: if $\ZZ[\alpha]$ has $\ZZ$-basis $a_1, \dots, a_m$ and
	$\ZZ[\beta]$ has $\ZZ$-basis $b_1, \dots, b_n$,
	then take the $mn$ elements $a_ib_j$.)

	Now $\ZZ[\alpha \pm \beta]$ and $\ZZ[\alpha \beta]$ are subsets of $\ZZ[\alpha,\beta]$ and so they are also finitely generated.
	Hence $\alpha \pm \beta$ and $\alpha\beta$ are algebraic integers.
\end{proof}

In fact, something even better is true.
As you saw, for $\QQ(\sqrt 3)$ we had $\OO_K = \ZZ[\sqrt 3]$;
in other words, $\OO_K$ was generated by $1$ and $\sqrt 3$.
Something similar was true for $\QQ(\sqrt 5)$.
We claim that in fact, the general picture looks exactly like this.

\begin{theorem}[$\OO_K$ is a free $\ZZ$-module of rank $n$]
	Let $K$ be a number field of degree $n$.
	Then $\OO_K$ is a free $\ZZ$-module of rank $n$,
	i.e.\ $\OO_K \cong \ZZ^{\oplus n}$ as an abelian group.
	In other words, $\OO_K$ has a $\ZZ$-basis of $n$ elements as
	\[ \OO_K = \left\{ c_1\alpha_1 + \dots
			+ c_{n-1}\alpha_{n-1} + c_n\alpha_n \mid c_i \in \ZZ \right\} \]
	where $\alpha_i$ are algebraic integers in $\OO_K$.
	\label{thm:OK_free_Z_module}
\end{theorem}

\begin{proof}
	TODO: add this in.
	(Originally, there was an incorrect proof;
	the mistake was pointed out 2020-02-12 on
	\url{https://math.stackexchange.com/q/3543641/229197}
	and I hope to supply a correct one soon.)
%	This is a kind of fun proof, so it may be worth
%	trying to work out yourself before reading it.
%
%	Pick a $\QQ$-basis of $\alpha_1$, \dots, $\alpha_n$ of $K$ and WLOG
%	the $\alpha_i$ are in $\OO_K$ by scaling.
%
%	Consider $\alpha \in \OO_K$,
%	and write $\alpha = c_1\alpha_1 + \dots + c_n\alpha_n$.
%	We will try to bound the denominators of $c_i$.
%	Look at $\Norm(\alpha) = \Norm(c_1\alpha_1 + \dots + c_n\alpha_n)$.
%
%	If we do a giant norm computation, we find that $\Norm(\alpha)$
%	is a polynomial in the $c_i$ with fixed coefficients.
%	(For example, $\Norm(c_1 + c_2\sqrt 2) = c_1^2 - 2c_2^2$, say.)
%	But $\Norm(\alpha)$ is an \emph{integer}, so the denominators of the $c_i$
%	have to be bounded by some very large integer $N$.
%	Thus
%	\[ \bigoplus_i \ZZ \cdot \alpha_i \subseteq \OO_K
%		\subseteq \frac 1N \bigoplus_i \ZZ \cdot \alpha_i.  \]
%	The latter inclusion shows that $\OO_K$ is a subgroup
%	of a free group, and hence it is itself free.
%	On the other hand, the first inclusion shows it's rank $n$.
\end{proof}

This last theorem shows that in many ways $\OO_K$ is a ``lattice'' in $K$.
That is, for a number field $K$ we can find $\alpha_1$, \dots, $\alpha_n$
in $\OO_K$ such that
\begin{align*}
	\OO_K &\cong \alpha_1\ZZ \oplus \alpha_2\ZZ \oplus \dots \oplus \alpha_n\ZZ \\
	K &\cong \alpha_1\QQ \oplus \alpha_2\QQ \oplus \dots \oplus \alpha_n\QQ
\end{align*}
as abelian groups.

\section{On monogenic extensions}
Recall that it turned out number fields $K$ could all be
expressed as $\QQ(\alpha)$ for some $\alpha$.
We might hope that something similar is true of the ring of integers:
that we can write \[ \OO_K = \ZZ[\theta] \]
in which case $\{1, \theta, \dots, \theta^{n-1}\}$
serves both as a basis of $K$ and as the $\ZZ$-basis for $\OO_K$ (here $n = [K:\QQ]$).
In other words, we hope that the basis of $\OO_K$ is actually a ``power basis''.

This is true for the most common examples we use:
\begin{itemize}
	\ii the quadratic field, and
	\ii the cyclotomic field in \Cref{prob:ring_int_cyclotomic}.
\end{itemize}
Unfortunately, it is not true in general:
the first counterexample is $\QQ(\alpha)$ for $\alpha$ a root of $X^3-X^2-2X-8$.

We call an extension with this nice property \vocab{monogenic}.
As we'll later see, monogenic extensions have a really nice factoring algorithm,
\Cref{thm:factor_alg}.

\section{\problemhead}

\begin{sproblem} % trivial
	\label{prob:OK_unit_norm}
	Show that $\alpha$ is a unit of $\OO_K$ (meaning $\alpha\inv \in \OO_K$)
	if and only if $\Norm_{K/\QQ}(\alpha) = \pm 1$.
	\begin{hint}
		The norm is multiplicative and equal to product of Galois conjugates.
	\end{hint}
\end{sproblem}

\begin{sproblem}
	Let $K$ be a number field.
	What is the field of fractions of $\OO_K$?
	\begin{hint}
		It's isomorphic to $K$.
	\end{hint}
\end{sproblem}

\begin{problem}
	[Russian olympiad 1984]
	Find all integers $m$ and $n$ such that
	\[ \left( 5+3\sqrt2 \right)^m = \left( 3+5\sqrt2 \right)^n. \]
	\begin{hint}
		Taking the standard norm on $\QQ(\sqrt2)$ will destroy it.
	\end{hint}
\end{problem}


\begin{problem}
	[USA TST 2012]
	Decide whether there exist $a,b,c > 2010$ satisfying
	\[ a^3+2b^3+4c^3=6abc+1. \]
	\begin{hint}
		Norm in $\QQ(\sqrt[3]2)$.
	\end{hint}
\end{problem}

\begin{dproblem}[Cyclotomic Field]
	\yod
	\label{prob:ring_int_cyclotomic}
	Let $p$ be an odd rational prime and $\zeta_p$ a primitive $p$th root of unity.
	Let $K = \QQ(\zeta_p)$.
	Prove that $\OO_K = \ZZ[\zeta_p]$.
	(In fact, the result is true even if $p$ is not a prime.)
	\begin{hint}
		Obviously $\ZZ[\zeta_p] \subseteq \OO_K$, so our goal is to show the reverse inclusion.
		Show that for any $\alpha \in \OO_K$, the trace of $\alpha(1-\zeta_p)$ is divisible by $p$.
		Given $x = a_0 + a_1\zeta_p + \dots + a_{p-2}\zeta^{p-2} \in \OO_K$ (where $a_i \in \QQ$),
		consider $(1-\zeta_p)x$.
	\end{hint}
\end{dproblem}
