


	In fact, this is just the global section functor
	applied to $\OO_{\Spec S} \to f_\ast \OO_{\Spec R}$.

	We consider the category $\catname{ShRing}_X$ of sheaves of rings on a space $X$.
	Define the \vocab{global section functor}, denoted $\Gamma(X,-)$,
	which sends each sheaf $\SF$ to the ring $\SF(X)$.
	How does the global section functor behave on arrows?
	\begin{hint}
		Given a morphism of sheaves $\alpha : \SF \to \SG$
		read off a map $\SF(X) \to \SG(X)$.
	\end{hint}


\begin{example}[Punctured plane is not affine]
	We claim that $X = \Aff^2 \setminus \{ (0,0) \}$ is not affine.
	This is confusing, so you will have to pay attention.

	Assume for contradiction that it is affine.
	Note that affine varieties $V$ have the following property:
	points of the coordinate ring $\OO_V(V)$ correspond naturally
	to points of $V$ (by taking the vanishing locus).

	But in \Cref{prob:punctured_plane},
	we computed that \[ \OO_X(X) \cong \CC[x,y] \]
	that is the regular functions of $X$ are all the ones on $\Aff^2$.
	But consider now the maximal ideal $I = (x,y) \in \OO_X(X)$.
	It does not correspond to any point of $X$,
	which is a contradiction.
\end{example}


\section{(Aside) Which rings are coordinate rings?}
As an aside, we might ask: which rings are the coordinate ring of some affine variety?
There are some obvious requirements.
\begin{itemize}
	\ii Such ring must be $\CC$-algebras, of course.
	(A $\CC$-algebra is a ring that contains a copy of $\CC$ in it.)
	\ii Such a ring must be finitely generated, since $\CC[V]$
	is generated by $x_1$, \dots, $x_n$.
\end{itemize}
There is a third more subtle condition.
Define an element $r \in R$ to be \vocab{nilpotent} if $r^N = 0$ for some $r \in R$.
For example, $x$ is nilpotent in $\CC[x]/(x^{2015})$.
Call a ring \vocab{reduced} if it has no nonzero nilpotent elements.

\begin{exercise}
	Let $I$ be an ideal of a ring $R$.
	Show that $I$ is radical if and only if $R/I$ is reduced.
\end{exercise}
\begin{remark}
	For a ring $R$, this completes the following table:
	\begin{center}
	\begin{tabular}[h]{l|l}
		Ideal $I$ & Quotient $R/I$ \\ \hline
		maximal & field \\
		prime & integral domain \\
		semiprime & reduced
	\end{tabular}
	\end{center}
	In particular, $(0)$ is maximal, prime, or radical
	if and only if $R/(0) \cong R$ is a field, integral domain,
	or reduced, respectively.
\end{remark}

Thus, the third requirement is that a coordinate ring must be reduced.
Conveniently, it turns out that these three conditions are sufficient!
\begin{theorem}[Coordinate rings are finitely generated reduced $\CC$-algebras]
	Every finitely generated reduced $\CC$-algebra is the coordinate
	ring of some complex affine variety.
\end{theorem}
\begin{proof}
	Let $R$ be this map.
	Because it's finitely generated by some $r_1$, $r_2$, \dots, $r_n$,
	there is some surjective map
	\[ \CC[x_1, \dots, x_n] \surjto R \]
	via $x_i \mapsto r_i$.
	Let $I$ be the kernel of this map.
	Since $R$ is reduced, $I$ is radical.
	Then $R \cong \CC[V]$, where $V = \Vp(I)$.
\end{proof}
