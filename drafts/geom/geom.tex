\documentclass[11pt]{scrreprt}
%%fakesection Load packages

\usepackage{lmodern}
\usepackage[pdfusetitle]{hyperref}
\ExplSyntaxOn
\sys_if_engine_luatex:T {
	\usepackage{luatex85}
}
\sys_if_engine_pdftex:T {
	\usepackage[T1]{fontenc}
}
\ExplSyntaxOff

% These are evan.sty
\usepackage{amsmath,amssymb,amsthm}
\usepackage{mathrsfs}
\usepackage[usenames,svgnames,dvipsnames]{xcolor}
\usepackage{textcomp}
\usepackage{enumerate}
\usepackage[textsize=scriptsize,shadow]{todonotes}
\usepackage{mathtools}
\usepackage{microtype}
\usepackage[normalem]{ulem}
\usepackage{stmaryrd}
\usepackage{wasysym}
\usepackage{multirow}
\usepackage{prerex}
\usepackage[nameinlink]{cleveref}
\usepackage{derivative}

%%fakesection evan.sty macros
%Small commands
%% Napkin commands
\newcommand{\prototype}[1]{
	\emph{{\color{red} Prototypical example for this section:} #1} \par\medskip
}
\newenvironment{moral}{%
	\begin{tcolorbox}[boxrule=0.4pt,colframe=green!70!black,sharp corners,
		standard jigsaw,opacityback=0,left=10pt,right=10pt,top=3pt,bottom=3pt,
		before skip=10pt,after skip=20pt]%
	\bfseries\color{green!50!black}}%
	{\end{tcolorbox}}

%%fakesection Links (hyperref loaded earlier implicitly)
\hypersetup{
	linkcolor={red!50!black},
	citecolor={green!50!black},
	urlcolor={blue!80!black},
	pdfkeywords={napkin,math},
	pdfsubject={web.evanchen.cc},
	colorlinks,
}

%%fakesection Commutative diagrams
\usepackage{tikz-cd}
\usetikzlibrary{arrows,arrows.meta}
% make a larger hook
% https://tex.stackexchange.com/questions/514451/how-to-define-a-new-hooked-arrow
\makeatletter
\pgfdeclarearrow{
	name=xGlyph,
	cache=false,
	bending mode=none,
	parameters={\tikzcd@glyph@len,\tikzcd@glyph@shorten},
	setup code={%
		\pgfarrowssettipend{\tikzcd@glyph@len\advance\pgf@x by\tikzcd@glyph@shorten}},
	defaults={
		glyph axis=axis_height,
		glyph length=+1.55ex,
		glyph shorten=+-0.1ex},
	drawing code={%
		\pgfpathrectangle{\pgfpoint{+0pt}{+-1.5ex}}{\pgfpoint{+\tikzcd@glyph@len}{+3ex}}%
		\pgfusepathqclip%
		\pgftransformxshift{+\tikzcd@glyph@len}%
		\pgftransformyshift{+-\tikzcd@glyph@axis}%
		\pgftext[right,base]{\tikzcd@glyph}}}
\makeatother
\tikzcdset{
	arrow style=tikz,
	diagrams={>={Latex}},
	tikzcd left hook/.tip={xGlyph[glyph math command=supset, swap, glyph axis = 5.7pt]},
	tikzcd right hook/.tip={xGlyph[glyph math command=supset, glyph axis = 5.7pt]},
	surjective head arrow /.tip = {tikzcd to[sep=-1.5pt]tikzcd to},
	surjective head/.style={
		-surjective head arrow
	}
}

%%fakesection Page layout
\usepackage[headsepline]{scrlayer-scrpage}
\renewcommand{\headfont}{}
\addtolength{\textheight}{3.14cm}
\setlength{\footskip}{0.5in}
\setlength{\headsep}{10pt}

\def\shortdate{\leavevmode\hbox{\the\year-\twodigits\month-\twodigits\day}}
\def\twodigits#1{\ifnum#1<10 0\fi\the#1}
\automark[chapter]{chapter}

\rohead{\footnotesize\thepage}
\rehead{\footnotesize \textbf{\sffamily Napkin}, by \emph{Evan Chen} (\napkinversion)}
\lehead{\footnotesize\thepage}
\lohead{\footnotesize \leftmark}
\chead{}
\rofoot{}
\refoot{}
\lefoot{}
\lofoot{}
%\cfoot{\pagemark}

%%fakesection Fancy section and chapter heads
\renewcommand*{\sectionformat}{\color{purple}\S\thesection\autodot\enskip}
\renewcommand*{\subsectionformat}{\color{purple}\S\thesubsection\autodot\enskip}
\newcommand{\problemhead}{A few harder problems to think about}
\renewcommand{\thesubsection}{\thesection.\roman{subsection}}

\addtokomafont{chapterprefix}{\raggedleft}
\RedeclareSectionCommand[beforeskip=0.5em]{chapter}
\renewcommand*{\chapterformat}{%
\mbox{\scalebox{1.5}{\chapappifchapterprefix{\nobreakspace}}%
\scalebox{2.718}{\color{purple}\thechapter\autodot}\enskip}}

\addtokomafont{partprefix}{\rmfamily}
\renewcommand*{\partformat}{\color{purple}\scalebox{2.5}{\thepart}}

%%fakesection Theorems
\usepackage{tcolorbox}
\tcbuselibrary{breakable,skins,hooks}

% patch tcolorbox to support continuation of a paragraph
% after box
% from https://tex.stackexchange.com/questions/568782/
\makeatletter
\tcbset{
	after app={%
		\ifx\tcb@drawcolorbox\tcb@drawcolorbox@breakable
		\else
		% add only when not breakabel
		\@endparenv
		\fi
	}
}

% for breakable
\appto\tcb@use@after@lastbox{\@endparenv\@doendpe}
\makeatother
% END patch tcolorbox for continuation of paragraph


\usepackage{thmtools}

\theoremstyle{definition}
\declaretheoremstyle[
	headfont=\sffamily\bfseries\color{MidnightBlue},
	headpunct={\\[3pt]},
	postheadspace={0pt},
]{thmtheorem}


\declaretheoremstyle[
	headfont=\bfseries\color{RawSienna},
	headpunct={\\[3pt]},
	postheadspace={0pt},
]{thmexample}

\tcbset{
	theorem box/.style={
		enhanced,
		arc=9pt,
		outer arc=10pt,
		colframe=blue,
		colback=TealBlue!5,
		boxrule=1pt,
		before skip=12pt,
		after skip=14pt,
		left=10pt,
		right=10pt
	},
	remark box/.style={
		boxrule=0pt,
		frame hidden,
		sharp corners,
		enhanced,
		borderline west={2pt}{0pt}{ForestGreen},
		before skip=8pt,
		colback=ForestGreen!5,
		after skip=12pt,
		breakable,
		left=10pt,
		right=10pt
	},
	example box/.style={
		enhanced,
		sharp corners,
		arc=9pt,
		outer arc=10pt,
		colframe=RawSienna,
		colback=Salmon!5,
		boxrule=0.5pt,
		before skip=12pt,
		after skip=14pt,
		breakable,
		top=6pt,
		bottom=8pt,
		breakable,
		left=10pt,
		right=10pt
	},
	ques box/.style={
		boxrule=0pt,
		frame hidden,
		enhanced,
		sharp corners,
		before skip=8pt,
		after skip=12pt,
		borderline west={3pt}{0pt}{black},
		colback=RedViolet!5!gray!5,
		breakable,
		left=10pt,
		right=10pt
	}
}

\declaretheorem[style=thmtheorem,name=Theorem,numberwithin=section]{theorem}
\tcolorboxenvironment{theorem}{theorem box}
\declaretheorem[style=thmtheorem,name=Lemma,sibling=theorem]{lemma}
\tcolorboxenvironment{lemma}{theorem box}
\declaretheorem[style=thmtheorem,name=Proposition,sibling=theorem]{proposition}
\tcolorboxenvironment{proposition}{theorem box}
\declaretheorem[style=thmtheorem,name=Corollary,sibling=theorem]{corollary}
\tcolorboxenvironment{corollary}{theorem box}
\declaretheorem[style=thmexample,name=Example,sibling=theorem]{example}
\tcolorboxenvironment{example}{example box}

\declaretheoremstyle[
	headfont=\bfseries\sffamily\color{ForestGreen!70!black},
	bodyfont=\normalfont,
	headpunct={ --- },
]{thmremark}
\declaretheoremstyle[
	headfont=\bfseries\sffamily\color{ForestGreen!70!black},
	bodyfont=\normalfont,
	headpunct={},
]{thmremark*}

\declaretheoremstyle[
	headfont=\bfseries,
	bodyfont=\normalfont\small
]{thmques}

\declaretheorem[name=Question,sibling=theorem,style=thmques]{ques}
\tcolorboxenvironment{ques}{ques box}
\declaretheorem[name=Exercise,sibling=theorem,style=thmques]{exercise}
\tcolorboxenvironment{exercise}{ques box}
\declaretheorem[name=Remark,sibling=theorem,style=thmremark]{remark}
\tcolorboxenvironment{remark}{remark box}
\declaretheorem[name=Remark,sibling=theorem,style=thmremark*]{remark*}
\tcolorboxenvironment{remark*}{remark box}
\declaretheorem[name=Step,style=thmremark]{step} % only used in Lebesgue int
\tcolorboxenvironment{step}{remark box}

\theoremstyle{definition}
\newtheorem{claim}[theorem]{Claim}
\newtheorem{definition}[theorem]{Definition}
\newtheorem{fact}[theorem]{Fact}
\newtheorem{abuse}[theorem]{Abuse of Notation}

\newtheorem{problem}{Problem}[chapter]
\renewcommand{\theproblem}{\thechapter\Alph{problem}}
\newtheorem{sproblem}[problem]{Problem}
\newtheorem{dproblem}[problem]{Problem}
\renewcommand{\thesproblem}{\theproblem$^{\star}$}
\renewcommand{\thedproblem}{\theproblem$^{\dagger}$}
\newcommand{\listhack}{$\empty$\vspace{-2em}}

%%fakesection Answers
\usepackage{answers}
\Newassociation{hint}{answeritem}{tex/backmatter/all-hints}
\Newassociation{sol}{answeritem}{tex/backmatter/all-solns}
\renewcommand{\solutionextension}{out}
\renewenvironment{answeritem}[1]{\item[\bfseries #1.]}{}

%%fakesection Table of contents
% First add ToC to ToC
\makeatletter
\usepackage{etoolbox}
\pretocmd{\tableofcontents}{%
	\if@openright\cleardoublepage\else\clearpage\fi
	\pdfbookmark[0]{\contentsname}{toc}%
}{}{}%
\makeatother
\setcounter{tocdepth}{1}
\RedeclareSectionCommand[tocnumwidth=4.2em]{part}
\RedeclareSectionCommand[tocpagenumberwidth=2.2em,tocnumwidth=4.2em]{chapter}
\RedeclareSectionCommand[tocpagenumberwidth=2.2em,tocnumwidth=2.8em]{section}
% adjust tocpagenumberwidth manually for large page number: https://tex.stackexchange.com/a/502168

%%fakesection Asymptote definitions
\usepackage{patch-asy}
\numberwithin{asy}{chapter}
\renewcommand{\theasy}{\thechapter\Alph{asy}}
\begin{asydef}
	import extras;
	size(6cm);
	usepackage("amsmath");
	usepackage("amssymb");
	defaultpen(fontsize(11pt));
	settings.tex = "latex";
	settings.outformat = "pdf";
\end{asydef}
\def\asydir{asy}

%%fakesection Bibliography
\usepackage[backend=biber,backref=true,style=alphabetic]{biblatex}
\DeclareLabelalphaTemplate{
	\labelelement{
		\field[final]{shorthand}
		\field{label}
		\field[strwidth=2,strside=left]{labelname}
	}
	\labelelement{
		\field[strwidth=2,strside=right]{year}
	}
}
\DeclareFieldFormat{labelalpha}{\textbf{\scriptsize #1}}
\addbibresource{references.bib}
\addbibresource{images.bib}
%% stylistic biblatex choices
\DefineBibliographyStrings{english}{%
	backrefpage  = {cited p.}, % for single page number
	backrefpages = {cited pp.} % for multiple page numbers
}
\DeclareFieldFormat{journaltitle}{\mkbibemph{#1},} % italic journal title with comma
\DeclareFieldFormat[inbook,thesis]{title}{\mkbibemph{#1}\addperiod} % italic title with period
\DeclareFieldFormat[article]{title}{#1} % title of journal article is printed as normal text
\DeclareFieldFormat[article]{volume}{\textbf{#1}\addcolon\space}
\renewcommand{\mkbibnamegiven}[1]{\textsc{#1}}
\renewcommand{\mkbibnamefamily}[1]{\textsc{#1}}
\renewcommand{\mkbibnameprefix}[1]{\textsc{#1}}
\renewcommand{\mkbibnamesuffix}[1]{\textsc{#1}}
\renewcommand{\finentrypunct}{}

%%fakesection Mini ToC
\usepackage[tight]{minitoc}
\mtcsetfont{parttoc}{chapter}{\sffamily\bfseries}
\mtcsetfont{parttoc}{section}{\footnotesize\rmfamily\upshape\mdseries}
\mtcsetfont{parttoc}{subsection}{\footnotesize\rmfamily\upshape\mdseries}
%\mtcsetdepth{parttoc}{1}
\setcounter{parttocdepth}{1}
\renewcommand*{\partheadstartvskip}{\vspace*{20em}}
\renewcommand*{\partheadendvskip}{}
%\noptcrule
\renewcommand\beforeparttoc{\noindent{\bfseries \Large Part \thepart: Contents}}
%\hspace{\fill}\rule{0.95\linewidth}{2pt}\hspace{\fill}
\doparttoc[n]

%%fakesection Misc haxx
\pdfstringdefDisableCommands{\def\Spec{\text{Spec }}\def\sigma{σ}}

\def\asydir{}
\addbibresource{../../references.bib}

\begin{document}
\title{Algebraic Geometry}
\maketitle

\chapter{Properties of Affine Varieties}
We now do two things with affine varieties: first, we put a topology on them,
and then we study functions on them.
Many of the ideas here will later grow up to become ideas in modern algebraic geometry:
the spectrum, sheaves, and schemes.

As in the previous chapter, we are working only over affine varieties in $\CC$ for simplicity.

\section{The Zariski Topology}
\prototype{In $\Aff^1$, closed sets are finite collections of points.}

First, we'd like to put a topology on any variety $V$, so that we can think
of our variety $V$ as a space in and of itself.
Since our affine varieties (for now) all live in $\Aff^n$, all we have to do
is put a suitable topology on $\Aff^n$, and let subspacing do the rest.

However, rather than putting the standard Euclidean topology on $\Aff^n$,
we put a much more bizarre topology, the \vocab{Zariski topology}.
It's defined as follows:
\begin{moral}
	The closed sets are those of the form $\VV(I)$.
\end{moral}
Here $I \subseteq \CC[x_1, \dots, x_n]$.
We declare that sets of this form -- and only those sets -- are the \emph{closed sets}.
Naturally, the open sets are complements.

\begin{example}
	[Zariski Topology on $\Aff^1$]
	Let us determine the open sets of $\Aff^1$,
	which as usual we picture as a straight line
	(ignoring the fact that $\CC$ is two-dimensional).

	Since $\CC[x]$ is a principal ideal domain, rather than looking at $\VV(I)$
	for every $I \subseteq \CC[x]$, we just have to look at $\VV(f)$ for a single $f$.
	There are a few flavors of polynomials $f$:
	\begin{itemize}
		\ii The zero polynomial $0$ which vanishes everywhere:
		this implies that $\Aff^1$ is a closed set.
		\ii The constant polynomial $1$ which vanishes nowhere.
		This implies that $\varnothing$ is a closed set.
		\ii A polynomial $c(x-t_1)(x-t_2)\dots(x-t_n)$ of degree $n$.
		It has $n$ roots, and so $\{t_1, \dots, t_n\}$ is a closed set.
	\end{itemize}
	Hence the closed sets of $\Aff^1$ are exactly all of $\Aff^1$
	and finite sets of points (including $\varnothing$).
	Consequently, the \emph{open} sets of $\Aff^1$ are
	\begin{itemize}
		\ii $\varnothing$, and
		\ii $\Aff^1$ minus a finite collection (possibly empty) of points.
	\end{itemize}
\end{example}

Thus, the picture of a typical open set looks in $\Aff^1$ might be 
\begin{center}
	\begin{asy}
		size(6cm);
		pair A = (-9,0); pair B = (9,0);
		pen bloo = blue+1.5;
		draw(A--B, blue, Arrows);
		draw(A--B, bloo);
		// label("$\mathbb V()$", (0,0), 2*dir(-90));
		opendot((-3,0), bloo);
		opendot((-1,0), bloo);
		opendot((4,0), bloo);
		label("$\mathbb A^1$", B-(2,0), dir(90));
	\end{asy}
\end{center}
It's everything except a few marked points!

\begin{example}[Zariski Topology on $\Aff^2$]
	Similarly, in $\Aff^2$, the interesting closed sets are going to consist of finite unions
	(possibly empty) of the following:
	\begin{itemize}
		\ii Closed curves, like $\VV(y-x^2)$ (which is a parabola).
		\ii Single points, like $\VV(x-3,y-4)$ (which is the point $(3,4)$).
	\end{itemize}
	Of course, the entire space $\Aff^2 = \VV(0)$ and the empty set $\varnothing = \VV(1)$
	are closed sets.

	Thus the nonempty open sets in $\Aff^2$ consist of the \emph{entire} plane,
	minus a finite collection of points and one-dimensional curves.
\end{example}
\begin{ques}
	Draw a picture (to the best of your artistic ability) of a ``generic''
	open set in $\Aff^2$
\end{ques}

All this is to say
\begin{moral}
	The nonempty Zariski open sets are \emph{huge}.
\end{moral}
This is an important difference than what you're used to in topology.
To be very clear:
\begin{itemize}
	\ii In the past, if I said something like ``has so-and-so property in a neighborhood of point $p$'',
	one thought of this as saying ``is true in a small region around $p$''.
	\ii In the Zariski topology, ``has so-and-so property in a neighborhood of point $p$'' should be thought
	of as saying ``is true for virtually all points, other than those on certain curves''.
\end{itemize}
Indeed, ``neighborhood'' is no longer a very accurate description.

It remains to verify that as I've stated it, the closed sets actually form a topology.
That is, I need to verify briefly that
\begin{itemize}
	\ii $\varnothing$ and $\Aff^n$ are both closed.
	\ii Intersections of closed sets (even infinite) are still closed.
	\ii Finite unions of closed sets are still closed.
\end{itemize}
Well, closed sets are the same as varieties, so we already know this. Moving along\dots



\section{Coordinate Rings}
\prototype{If $V = \VV(y-x^2)$ then $\CC[V] = \CC[x,y]/(y-x^2)$.}
Let's consider functions now from a given variety $V$ to the base field $\CC$.

We restrict our attention to algebraic (polynomial) functions on a variety $V$:
they should take every point $(a_1, \dots, a_n)$ on $V$ to some complex number $P(a_1, \dots, a_n) \in \CC$.
For example, a valid function on a three-dimensional affine variety might be $(a,b,c) \mapsto a$,
which we call ``$x$''.
Similarly we have a canonical projection $y$ and $z$,
and we can create polynomials by combining them,
say $x^2y + 2xyz$.

\begin{definition}
	The \vocab{coordinate ring} $\CC[V]$ of a variety $V$
	is the ring of polynomial functions on $V$.
\end{definition}

At first glance, we might think this is just $\CC[x_1, \dots, x_n]$.
But on closer inspection we realize that \emph{on a given variety},
some of these functions are the same.
For example, consider in $\Aff^2$ the variety $V = \VV(y-x^2)$.
Then the two functions
\begin{align*}
	V & \to \CC \\
	(x,y) & \mapsto x^2 \\
	(x,y) & \mapsto y
\end{align*}
are actually the same function!
This leads us naturally to the following:
\begin{theorem}[Coordinate Rings Correspond to Ideal]
	Let $I$ be a semiprime ideal, and $V = \VV(I) \subseteq \Aff^n$.
	Then
	\[
		\CC[V] \cong \CC[x_1, \dots, x_n] / I.
	\]
\end{theorem}
\begin{proof}
	There's a natural surjection as above
	\[ \CC[x_1, \dots, x_n] \surjto \CC[V] \]
	and the kernel is $I$.
\end{proof}

Thus properties of a variety $V$ correspond to properties of $\CC[V]$.

\section{Which rings are coordinate rings?}
We now ask: which rings are the coordinate ring of some variety?
There are some obvious requirements.
\begin{itemize}
	\ii Such ring must be $\CC$-algebras, of course.
	(A $\CC$-algebra is a ring that contains a copy of $\CC$ in it.)
	\ii Such a ring must be finitely generated, since $\CC[V]$
	is generated by $x_1$, \dots, $x_n$.
\end{itemize}
There is a third more subtle condition.
Define an element $r \in R$ to be \vocab{nilpotent} if $r^N = 0$ for some $r \in R$.
For example, $x$ is nilpotent in $\CC[x]/(x^{2015})$.
Call a ring \vocab{reduced} if it has no nonzero nilpotent elements.

\begin{exercise}
	Let $I$ be an ideal of a ring $R$.
	Show that $I$ is semiprime if and only if $R/I$ is reduced.
\end{exercise}
\begin{remark}
	For a ring $R$, this completes the following table:
	\begin{center}
	\begin{tabular}[h]{l|l}
		Ideal $I$ & Quotient $R/I$ \\ \hline
		Maximal & Field \\
		Prime & Integral Domain \\
		Semiprime & Reduced
	\end{tabular}
	\end{center}
	In particular, $(0)$ is maximal, prime, or semiprime
	if and only if $R/(0) \cong R$ is a field, integral domain, or reduced, respectively.
\end{remark}

Thus, the third requirement is that a coordinate ring must be reduced.
Conveniently, it turns out that these three conditions are sufficient!
\begin{theorem}[Coordinate Rings are Finitely Generated Reduced $\CC$-Algebras]
	Every finitely generated reduced $\CC$-algebra is the coordinate
	ring of some complex affine variety.
\end{theorem}
\begin{proof}
	Let $R$ be this map.
	Because it's finitely generated by some $r_1$, $r_2$, \dots, $r_n$,
	there is some surjective map
	\[ \CC[x_1, \dots, x_n] \surjto R \]
	via $x_i \mapsto r_i$.
	Let $I$ be the kernel of this map.
	Since $R$ is reduced, $I$ is radical.
	Then $R \cong \CC[V]$, where $V = \VV(I)$.
\end{proof}


\section{Ring of Regular Functions}

\section{A Basis for the Zariski Topology}

\end{document}
