\documentclass[11pt]{scrreprt}
%%fakesection Load packages

\usepackage{lmodern}
\usepackage[pdfusetitle]{hyperref}
\ExplSyntaxOn
\sys_if_engine_luatex:T {
	\usepackage{luatex85}
}
\sys_if_engine_pdftex:T {
	\usepackage[T1]{fontenc}
}
\ExplSyntaxOff

% These are evan.sty
\usepackage{amsmath,amssymb,amsthm}
\usepackage{mathrsfs}
\usepackage[usenames,svgnames,dvipsnames]{xcolor}
\usepackage{textcomp}
\usepackage{enumerate}
\usepackage[textsize=scriptsize,shadow]{todonotes}
\usepackage{mathtools}
\usepackage{microtype}
\usepackage[normalem]{ulem}
\usepackage{stmaryrd}
\usepackage{wasysym}
\usepackage{multirow}
\usepackage{prerex}
\usepackage[nameinlink]{cleveref}
\usepackage{derivative}

%%fakesection evan.sty macros
%Small commands
%% Napkin commands
\newcommand{\prototype}[1]{
	\emph{{\color{red} Prototypical example for this section:} #1} \par\medskip
}
\newenvironment{moral}{%
	\begin{tcolorbox}[boxrule=0.4pt,colframe=green!70!black,sharp corners,
		standard jigsaw,opacityback=0,left=10pt,right=10pt,top=3pt,bottom=3pt,
		before skip=10pt,after skip=20pt]%
	\bfseries\color{green!50!black}}%
	{\end{tcolorbox}}

%%fakesection Links (hyperref loaded earlier implicitly)
\hypersetup{
	linkcolor={red!50!black},
	citecolor={green!50!black},
	urlcolor={blue!80!black},
	pdfkeywords={napkin,math},
	pdfsubject={web.evanchen.cc},
	colorlinks,
}

%%fakesection Commutative diagrams
\usepackage{tikz-cd}
\usetikzlibrary{arrows,arrows.meta}
% make a larger hook
% https://tex.stackexchange.com/questions/514451/how-to-define-a-new-hooked-arrow
\makeatletter
\pgfdeclarearrow{
	name=xGlyph,
	cache=false,
	bending mode=none,
	parameters={\tikzcd@glyph@len,\tikzcd@glyph@shorten},
	setup code={%
		\pgfarrowssettipend{\tikzcd@glyph@len\advance\pgf@x by\tikzcd@glyph@shorten}},
	defaults={
		glyph axis=axis_height,
		glyph length=+1.55ex,
		glyph shorten=+-0.1ex},
	drawing code={%
		\pgfpathrectangle{\pgfpoint{+0pt}{+-1.5ex}}{\pgfpoint{+\tikzcd@glyph@len}{+3ex}}%
		\pgfusepathqclip%
		\pgftransformxshift{+\tikzcd@glyph@len}%
		\pgftransformyshift{+-\tikzcd@glyph@axis}%
		\pgftext[right,base]{\tikzcd@glyph}}}
\makeatother
\tikzcdset{
	arrow style=tikz,
	diagrams={>={Latex}},
	tikzcd left hook/.tip={xGlyph[glyph math command=supset, swap, glyph axis = 5.7pt]},
	tikzcd right hook/.tip={xGlyph[glyph math command=supset, glyph axis = 5.7pt]},
	surjective head arrow /.tip = {tikzcd to[sep=-1.5pt]tikzcd to},
	surjective head/.style={
		-surjective head arrow
	}
}

%%fakesection Page layout
\usepackage[headsepline]{scrlayer-scrpage}
\renewcommand{\headfont}{}
\addtolength{\textheight}{3.14cm}
\setlength{\footskip}{0.5in}
\setlength{\headsep}{10pt}

\def\shortdate{\leavevmode\hbox{\the\year-\twodigits\month-\twodigits\day}}
\def\twodigits#1{\ifnum#1<10 0\fi\the#1}
\automark[chapter]{chapter}

\rohead{\footnotesize\thepage}
\rehead{\footnotesize \textbf{\sffamily Napkin}, by \emph{Evan Chen} (\napkinversion)}
\lehead{\footnotesize\thepage}
\lohead{\footnotesize \leftmark}
\chead{}
\rofoot{}
\refoot{}
\lefoot{}
\lofoot{}
%\cfoot{\pagemark}

%%fakesection Fancy section and chapter heads
\renewcommand*{\sectionformat}{\color{purple}\S\thesection\autodot\enskip}
\renewcommand*{\subsectionformat}{\color{purple}\S\thesubsection\autodot\enskip}
\newcommand{\problemhead}{A few harder problems to think about}
\renewcommand{\thesubsection}{\thesection.\roman{subsection}}

\addtokomafont{chapterprefix}{\raggedleft}
\RedeclareSectionCommand[beforeskip=0.5em]{chapter}
\renewcommand*{\chapterformat}{%
\mbox{\scalebox{1.5}{\chapappifchapterprefix{\nobreakspace}}%
\scalebox{2.718}{\color{purple}\thechapter\autodot}\enskip}}

\addtokomafont{partprefix}{\rmfamily}
\renewcommand*{\partformat}{\color{purple}\scalebox{2.5}{\thepart}}

%%fakesection Theorems
\usepackage{tcolorbox}
\tcbuselibrary{breakable,skins,hooks}

% patch tcolorbox to support continuation of a paragraph
% after box
% from https://tex.stackexchange.com/questions/568782/
\makeatletter
\tcbset{
	after app={%
		\ifx\tcb@drawcolorbox\tcb@drawcolorbox@breakable
		\else
		% add only when not breakabel
		\@endparenv
		\fi
	}
}

% for breakable
\appto\tcb@use@after@lastbox{\@endparenv\@doendpe}
\makeatother
% END patch tcolorbox for continuation of paragraph


\usepackage{thmtools}

\theoremstyle{definition}
\declaretheoremstyle[
	headfont=\sffamily\bfseries\color{MidnightBlue},
	headpunct={\\[3pt]},
	postheadspace={0pt},
]{thmtheorem}


\declaretheoremstyle[
	headfont=\bfseries\color{RawSienna},
	headpunct={\\[3pt]},
	postheadspace={0pt},
]{thmexample}

\tcbset{
	theorem box/.style={
		enhanced,
		arc=9pt,
		outer arc=10pt,
		colframe=blue,
		colback=TealBlue!5,
		boxrule=1pt,
		before skip=12pt,
		after skip=14pt,
		left=10pt,
		right=10pt
	},
	remark box/.style={
		boxrule=0pt,
		frame hidden,
		sharp corners,
		enhanced,
		borderline west={2pt}{0pt}{ForestGreen},
		before skip=8pt,
		colback=ForestGreen!5,
		after skip=12pt,
		breakable,
		left=10pt,
		right=10pt
	},
	example box/.style={
		enhanced,
		sharp corners,
		arc=9pt,
		outer arc=10pt,
		colframe=RawSienna,
		colback=Salmon!5,
		boxrule=0.5pt,
		before skip=12pt,
		after skip=14pt,
		breakable,
		top=6pt,
		bottom=8pt,
		breakable,
		left=10pt,
		right=10pt
	},
	ques box/.style={
		boxrule=0pt,
		frame hidden,
		enhanced,
		sharp corners,
		before skip=8pt,
		after skip=12pt,
		borderline west={3pt}{0pt}{black},
		colback=RedViolet!5!gray!5,
		breakable,
		left=10pt,
		right=10pt
	}
}

\declaretheorem[style=thmtheorem,name=Theorem,numberwithin=section]{theorem}
\tcolorboxenvironment{theorem}{theorem box}
\declaretheorem[style=thmtheorem,name=Lemma,sibling=theorem]{lemma}
\tcolorboxenvironment{lemma}{theorem box}
\declaretheorem[style=thmtheorem,name=Proposition,sibling=theorem]{proposition}
\tcolorboxenvironment{proposition}{theorem box}
\declaretheorem[style=thmtheorem,name=Corollary,sibling=theorem]{corollary}
\tcolorboxenvironment{corollary}{theorem box}
\declaretheorem[style=thmexample,name=Example,sibling=theorem]{example}
\tcolorboxenvironment{example}{example box}

\declaretheoremstyle[
	headfont=\bfseries\sffamily\color{ForestGreen!70!black},
	bodyfont=\normalfont,
	headpunct={ --- },
]{thmremark}
\declaretheoremstyle[
	headfont=\bfseries\sffamily\color{ForestGreen!70!black},
	bodyfont=\normalfont,
	headpunct={},
]{thmremark*}

\declaretheoremstyle[
	headfont=\bfseries,
	bodyfont=\normalfont\small
]{thmques}

\declaretheorem[name=Question,sibling=theorem,style=thmques]{ques}
\tcolorboxenvironment{ques}{ques box}
\declaretheorem[name=Exercise,sibling=theorem,style=thmques]{exercise}
\tcolorboxenvironment{exercise}{ques box}
\declaretheorem[name=Remark,sibling=theorem,style=thmremark]{remark}
\tcolorboxenvironment{remark}{remark box}
\declaretheorem[name=Remark,sibling=theorem,style=thmremark*]{remark*}
\tcolorboxenvironment{remark*}{remark box}
\declaretheorem[name=Step,style=thmremark]{step} % only used in Lebesgue int
\tcolorboxenvironment{step}{remark box}

\theoremstyle{definition}
\newtheorem{claim}[theorem]{Claim}
\newtheorem{definition}[theorem]{Definition}
\newtheorem{fact}[theorem]{Fact}
\newtheorem{abuse}[theorem]{Abuse of Notation}

\newtheorem{problem}{Problem}[chapter]
\renewcommand{\theproblem}{\thechapter\Alph{problem}}
\newtheorem{sproblem}[problem]{Problem}
\newtheorem{dproblem}[problem]{Problem}
\renewcommand{\thesproblem}{\theproblem$^{\star}$}
\renewcommand{\thedproblem}{\theproblem$^{\dagger}$}
\newcommand{\listhack}{$\empty$\vspace{-2em}}

%%fakesection Answers
\usepackage{answers}
\Newassociation{hint}{answeritem}{tex/backmatter/all-hints}
\Newassociation{sol}{answeritem}{tex/backmatter/all-solns}
\renewcommand{\solutionextension}{out}
\renewenvironment{answeritem}[1]{\item[\bfseries #1.]}{}

%%fakesection Table of contents
% First add ToC to ToC
\makeatletter
\usepackage{etoolbox}
\pretocmd{\tableofcontents}{%
	\if@openright\cleardoublepage\else\clearpage\fi
	\pdfbookmark[0]{\contentsname}{toc}%
}{}{}%
\makeatother
\setcounter{tocdepth}{1}
\RedeclareSectionCommand[tocnumwidth=4.2em]{part}
\RedeclareSectionCommand[tocpagenumberwidth=2.2em,tocnumwidth=4.2em]{chapter}
\RedeclareSectionCommand[tocpagenumberwidth=2.2em,tocnumwidth=2.8em]{section}
% adjust tocpagenumberwidth manually for large page number: https://tex.stackexchange.com/a/502168

%%fakesection Asymptote definitions
\usepackage{patch-asy}
\numberwithin{asy}{chapter}
\renewcommand{\theasy}{\thechapter\Alph{asy}}
\begin{asydef}
	import extras;
	size(6cm);
	usepackage("amsmath");
	usepackage("amssymb");
	defaultpen(fontsize(11pt));
	settings.tex = "latex";
	settings.outformat = "pdf";
\end{asydef}
\def\asydir{asy}

%%fakesection Bibliography
\usepackage[backend=biber,backref=true,style=alphabetic]{biblatex}
\DeclareLabelalphaTemplate{
	\labelelement{
		\field[final]{shorthand}
		\field{label}
		\field[strwidth=2,strside=left]{labelname}
	}
	\labelelement{
		\field[strwidth=2,strside=right]{year}
	}
}
\DeclareFieldFormat{labelalpha}{\textbf{\scriptsize #1}}
\addbibresource{references.bib}
\addbibresource{images.bib}
%% stylistic biblatex choices
\DefineBibliographyStrings{english}{%
	backrefpage  = {cited p.}, % for single page number
	backrefpages = {cited pp.} % for multiple page numbers
}
\DeclareFieldFormat{journaltitle}{\mkbibemph{#1},} % italic journal title with comma
\DeclareFieldFormat[inbook,thesis]{title}{\mkbibemph{#1}\addperiod} % italic title with period
\DeclareFieldFormat[article]{title}{#1} % title of journal article is printed as normal text
\DeclareFieldFormat[article]{volume}{\textbf{#1}\addcolon\space}
\renewcommand{\mkbibnamegiven}[1]{\textsc{#1}}
\renewcommand{\mkbibnamefamily}[1]{\textsc{#1}}
\renewcommand{\mkbibnameprefix}[1]{\textsc{#1}}
\renewcommand{\mkbibnamesuffix}[1]{\textsc{#1}}
\renewcommand{\finentrypunct}{}

%%fakesection Mini ToC
\usepackage[tight]{minitoc}
\mtcsetfont{parttoc}{chapter}{\sffamily\bfseries}
\mtcsetfont{parttoc}{section}{\footnotesize\rmfamily\upshape\mdseries}
\mtcsetfont{parttoc}{subsection}{\footnotesize\rmfamily\upshape\mdseries}
%\mtcsetdepth{parttoc}{1}
\setcounter{parttocdepth}{1}
\renewcommand*{\partheadstartvskip}{\vspace*{20em}}
\renewcommand*{\partheadendvskip}{}
%\noptcrule
\renewcommand\beforeparttoc{\noindent{\bfseries \Large Part \thepart: Contents}}
%\hspace{\fill}\rule{0.95\linewidth}{2pt}\hspace{\fill}
\doparttoc[n]

%%fakesection Misc haxx
\pdfstringdefDisableCommands{\def\Spec{\text{Spec }}\def\sigma{σ}}

\def\asydir{}

\renewenvironment{hint}{\begin{gobble}}{\end{gobble}}
\renewenvironment{sol}{\begin{gobble}}{\end{gobble}}

\begin{document}
\title{Algebraic Number Theory Chapters}
\maketitle
\tableofcontents

\section{Garbage can}
\label{prob:OK_unit_norm}

\chapter{Bonus: Let's Solve Pell's Equation!}
This is an optional aside, and can be safely ignored.
(On the other hand, it's pretty short.)

\section{Units}
\prototype{$\pm 1$, roots of unity, $3-\sqrt2$ and its powers.}
Recall according to \Cref{prob:OK_unit_norm} that $\alpha \in \OO_K$ is invertible
if and only if \[ \NK(\alpha) = \pm 1. \]
We let $\OO_K^\times$ denote the set of units of in $K$.

\begin{ques}
	Show that $\OO_K^\times$ is a group under multiplication.
	Hence we name it the \vocab{unit group} of $\OO_K$
\end{ques}

What are some examples of units?
\begin{example}
	[Examples of Units in a Number Field]
	\listhack
	\begin{enumerate}
		\ii $\pm 1$ are certainly units, present in any number field.

		\ii If $\OO_K$ contains an root of unity $\omega$ (\emph{id est} $\omega^n=1$),
		then $\omega$ is a unit.
		(In fact, $\pm 1$ are special cases of this.)

		\ii However, not all units of $\OO_K$ are of this form.
		For example, if $\OO_K = \ZZ[\sqrt3]$ (from $K = \QQ(\sqrt3)$) then
		the number $2+\sqrt3$ is a unit, as its norm is
		\[ \NK(2+\sqrt3) = 2^2 - 3 \cdot 1^2 = 1. \]
		Alternatively, just note that the inverse $2-\sqrt3 \in \OO_K$ as well:
		\[ \left( 2-\sqrt3 \right)\left( 2+\sqrt3 \right) = 1. \]
		Either way, $2-\sqrt3 \in \OO_K^\times$.

		\ii Given any unit $u \in \OO_K^\times$, all its powers are also units.
		So for example, $(3-\sqrt2)^n$ is always a unit of $\ZZ[\sqrt2]$, for any $n$.
		If $u$ is not a root of unity, then this generates infinitely many new units in $\OO_K^\times$.
	\end{enumerate}
\end{example}

\begin{ques}
	Verify the claims above that
	\begin{enumerate}[(a)]
		\ii Roots of unity are units, and
		\ii Powers of units are units.
	\end{enumerate}
	One can either proceed from the definition
	or use the characterization $\NK(\alpha) = \pm 1$.
	If one definition seems more natural to you, use the other.
\end{ques}

\section{Dirichlet's Unit Theorem}
\prototype{The units of $\ZZ[\sqrt3]$ are $\pm(2+\sqrt3)^n$.}

\begin{definition}
	Let $\mu(\OO_K)$ denote the set of roots of unity
	contained in a number field $K$ (equivalently, in $\OO_K$).
\end{definition}
\begin{example}[Examples of $\mu(\OO_K)$]
	\listhack
	\begin{enumerate}[(a)]
		\ii If $K = \QQ(i)$, then $\OO_K = \ZZ[i]$. So
		\[ \mu(\OO_K) = \{\pm1, \pm i\} \quad\text{where } K = \QQ(i). \]
		For $\OO_K = \ZZ[i]$ and the only roots of unity of the form $a+bi$ are $\pm 1, \pm i$.
		\ii If $K = \QQ(\sqrt3)$, then $\OO_K = \ZZ[\sqrt 3]$.
		\[ \mu(\OO_K) = \{\pm 1\} \quad\text{where } K = \QQ(\sqrt 3). \]
		\ii If $K = \QQ(\sqrt{-3})$, then $\OO_K = \ZZ[\half(1+\sqrt{-3})]$.
		In that case,
		\[ \mu(\OO_K)
			= \left\{ \pm 1, \frac{\pm 1 \pm \sqrt{-3}}{2} \right\}
			\quad\text{where } K = \QQ(\sqrt{-3})
		\]
		where the $\pm$'s in the second term need not depend on each other;
		in other words $\mu(\OO_K) = \left\{ z \mid z^6=1 \right\}$.
	\end{enumerate}
\end{example}
\begin{exercise}
	Show that we always have that $\mu(\OO_K)$
	comprises the roots to $x^n-1$ for some integer $n$.
	(First, show it is a finite group under multiplication.)
\end{exercise}

We now quote, without proof, the so-called Dirichlet's Unit Theorem,
which gives us a much more complete picture of what the units in $\OO_K$ are.

\begin{theorem}
	[Dirichlet's Unit Theorem]
	Let $K$ be a number field with signature $(r_1, r_2)$ and set
	\[ s = r_1 + r_2 - 1. \]
	Then there exists units $u_1$, \dots, $u_s$ such that every $\alpha \in \mu(\OO_K^\times)$
	can be written \emph{uniquely} in the form
	\[ \alpha = \omega \cdot u_1^{n_1} \dots u_s^{n_s} \]
	for $\omega \in \mu(\OO_K)$, $n_1, \dots, n_s \in \ZZ$.
\end{theorem}
More succinctly:
\begin{moral}
We have $\OO_K^\times \cong \ZZ^{r_1+r_2-1} \times \mu(\OO_K)$.
\end{moral}
A choice of $u_1$, \dots, $u_s$ is called a choice of \vocab{fundamental units}.

Here are some example applications.
\begin{example}
	[Unit Groups]
	\listhack
	\begin{enumerate}[(a)]
		\ii Let $K = \QQ(i)$ with signature $(0,1)$.
		Then we obtain $s = 0$, so Dirichlet's Unit theorem says that there are no
		units other than the roots of unity.
		Thus
		\[ \OO_K^\times = \{\pm 1, \pm i\} \quad\text{where } K = \QQ(i). \]
		This is not surprising,
		since if $a+bi \in \ZZ[i]$ is a unit if and only if $a^2+b^2 = 1$.

		\ii Let $K = \QQ(\sqrt 3)$ with signature $(2,0)$.
		So, $s = 1$, so $\OO_K^\times$ should be generated by a single element.
		A fundamental unit is $2+\sqrt3$ (or $2-\sqrt3$, its inverse) with norm $1$, and so we find
		\[ \OO_K^\times = \left\{ \pm (2+\sqrt3)^n \mid n \in \ZZ \right\}.  \]

		\ii Let $K = \QQ(\sqrt 2)$ with signature $(2,0)$.
		Again $s=1$, this time with fundamental unit $1+\sqrt2$ (which has norm $-1$).
		Thus
		\[ \OO_K^\times = \left\{ \pm (1-\sqrt2)^n \mid n \in \ZZ \right\}.  \]

		\ii Let $K = \QQ(\sqrt[3]{2})$ with signature $(1,1)$.
		Then $s=1$ again, and a fundamental unit is $1 + \sqrt[3]{2} + \sqrt[3]{4}$. So
		\[ \OO_K^\times
			= \left\{ \pm \left( 1+\sqrt[3]{2}+\sqrt[3]{4} \right)^n \mid n \in \ZZ \right\}. \]
	\end{enumerate}
\end{example}

I haven't 

\section\problemhead

\begin{thebibliography}{9}
\bibitem{ref:oggier_NT} Oggier NT
\bibitem{ref:dummit_foote} Dummit/Foote
\bibitem{ref:ullery} 
\end{thebibliography}
\end{document}
