\documentclass[11pt]{scrartcl}
\usepackage[mdthm,diagrams]{evan}

\newcommand{\vocab}[1]{\textbf{\color{blue} #1}}
\newcommand{\listhack}{$\empty$\vspace{-2em}}

\newenvironment{moral}{\begin{mdframed}\bfseries\color{green!70!black}}%
	{\end{mdframed}}

\newcommand{\prototype}[1]{
	\emph{{\color{red} Prototypical example for this section:} #1}\par
}
\newtheorem*{abuse}{Abuse of Notation}

\renewcommand{\AA}{\mathcal A}
\newcommand{\BB}{\mathcal B}
\newcommand{\obj}{\operatorname{obj}}

\usepackage{tkz-graph}
\pgfarrowsdeclare{biggertip}{biggertip}{%
  \setlength{\arrowsize}{1pt}  
  \addtolength{\arrowsize}{.1\pgflinewidth}  
  \pgfarrowsrightextend{0}  
  \pgfarrowsleftextend{-5\arrowsize}  
}{%
  \setlength{\arrowsize}{1pt}  
  \addtolength{\arrowsize}{.1\pgflinewidth}  
  \pgfpathmoveto{\pgfpoint{-5\arrowsize}{4\arrowsize}}  
  \pgfpathlineto{\pgfpointorigin}  
  \pgfpathlineto{\pgfpoint{-5\arrowsize}{-4\arrowsize}}  
  \pgfusepathqstroke  
}  
\tikzset{
	EdgeStyle/.style = {>=biggertip}
}

\newcommand\catname\mathsf

\usepackage[all,cmtip,2cell]{xy}
\UseTwocells
\newcommand\nattfm[5]{\xymatrix@C+2pc{#1 \rtwocell<4>^{#2}_{#4}{\; #3} & #5}}

\newtheorem{hint}{Hint}

\begin{document}
\title{Abelian Categories}
\maketitle

In this chapter I'll translate some more familiar concepts into categorical language;
this will require some additional assumptions about our category,
culminating in the definition of a so-called ``abelian category''.
Once that's done, I'll be able to tell you what this ``diagram chasing'' thing is all about.

Much of this will seem exceedingly unmotivated until you learn about homology groups.
In light of this, it might be worth trying to read the chapter on homology groups
simultaneously with this one.

\section{Products and Coproducts}
\prototype{$X \times Y$ in any reasonable category.}
You can technically skip this section if you want, but I think it provides some nice intuition about
the so-called \emph{universal property} that will be useful later on.

Suppose I'm in a category, say $\catname{Set}$, and I have two sets $X$ and $Y$.
I want to identify (up to isomorphism) an object which corresponds to the set $X \times Y$.
The philosophy of category theory dictates that I should talk about maps only,
and avoid referring to anything about the sets themselves.
How might I do this?

Well, let's think about maps into $X \times Y$.
The key observation is that 
\begin{moral}
A function $A \taking f X \times Y$
amounts to a pair of functions $(A \taking g X, A \taking h Y)$.
\end{moral}
Put another way, there is a natural projection map $X \times Y \surjto X$ and $X \times Y \surjto Y$.
Picture:
\begin{diagram}
	&& X \\
	X \times Y & \ruSurj(2,1)^{\pi_X} & \\
	& \rdSurj(2,1)_{\pi_Y} & Y
\end{diagram}
(We have to do this in terms of projection maps rather than elements,
because category theory forces us to talk about arrows.)
Now how do I add $A$ to this diagram?
The point is that there is a bijection between functions $A \taking f X \times Y$
and pairs $(g,h)$ of functions.
Thus for every pair $A \taking g X$ and $A \taking h Y$ there is a \emph{unique} function
$A \taking f X \times Y$. Picture:

But $X \times Y$ is special in that it is ``universal'':
for any \emph{other} set $A$, if you give me a function $A \to X$ and $A \to Y$, I can use it
build a \emph{unique} function $A \to X \times Y$.
\begin{diagram}
	&& && X \\
	A & \rDotted~{\exists! f} & X \times Y & \ruTo(4,1)^g \ruSurj(2,1)_{\pi_X} & \\
	& \rdTo(4,1)_h && \rdSurj(2,1)^{\pi_Y} & Y
\end{diagram}




\section{Zero Objects, Kernels and Cokernels}
\prototype{In $\catname{Gp}$, the zero object is $\{1\}$.}
You already know for groups, rings, vector spaces, etc.\ what a kernel is:
if I have an arrow $\phi : A \to B$, I define $\ker A$ to be the set of guys which $\phi$ sends to zero.

In order to define this


\section{Exact Sequences}
\prototype{$0 \to G \to G \times H \to H \to 0$}





\end{document}
