\documentclass[11pt]{scrreprt}
%%fakesection Load packages

\usepackage{lmodern}
\usepackage[pdfusetitle]{hyperref}
\ExplSyntaxOn
\sys_if_engine_luatex:T {
	\usepackage{luatex85}
}
\sys_if_engine_pdftex:T {
	\usepackage[T1]{fontenc}
}
\ExplSyntaxOff

% These are evan.sty
\usepackage{amsmath,amssymb,amsthm}
\usepackage{mathrsfs}
\usepackage[usenames,svgnames,dvipsnames]{xcolor}
\usepackage{textcomp}
\usepackage{enumerate}
\usepackage[textsize=scriptsize,shadow]{todonotes}
\usepackage{mathtools}
\usepackage{microtype}
\usepackage[normalem]{ulem}
\usepackage{stmaryrd}
\usepackage{wasysym}
\usepackage{multirow}
\usepackage{prerex}
\usepackage[nameinlink]{cleveref}
\usepackage{derivative}

%%fakesection evan.sty macros
%Small commands
%% Napkin commands
\newcommand{\prototype}[1]{
	\emph{{\color{red} Prototypical example for this section:} #1} \par\medskip
}
\newenvironment{moral}{%
	\begin{tcolorbox}[boxrule=0.4pt,colframe=green!70!black,sharp corners,
		standard jigsaw,opacityback=0,left=10pt,right=10pt,top=3pt,bottom=3pt,
		before skip=10pt,after skip=20pt]%
	\bfseries\color{green!50!black}}%
	{\end{tcolorbox}}

%%fakesection Links (hyperref loaded earlier implicitly)
\hypersetup{
	linkcolor={red!50!black},
	citecolor={green!50!black},
	urlcolor={blue!80!black},
	pdfkeywords={napkin,math},
	pdfsubject={web.evanchen.cc},
	colorlinks,
}

%%fakesection Commutative diagrams
\usepackage{tikz-cd}
\usetikzlibrary{arrows,arrows.meta}
% make a larger hook
% https://tex.stackexchange.com/questions/514451/how-to-define-a-new-hooked-arrow
\makeatletter
\pgfdeclarearrow{
	name=xGlyph,
	cache=false,
	bending mode=none,
	parameters={\tikzcd@glyph@len,\tikzcd@glyph@shorten},
	setup code={%
		\pgfarrowssettipend{\tikzcd@glyph@len\advance\pgf@x by\tikzcd@glyph@shorten}},
	defaults={
		glyph axis=axis_height,
		glyph length=+1.55ex,
		glyph shorten=+-0.1ex},
	drawing code={%
		\pgfpathrectangle{\pgfpoint{+0pt}{+-1.5ex}}{\pgfpoint{+\tikzcd@glyph@len}{+3ex}}%
		\pgfusepathqclip%
		\pgftransformxshift{+\tikzcd@glyph@len}%
		\pgftransformyshift{+-\tikzcd@glyph@axis}%
		\pgftext[right,base]{\tikzcd@glyph}}}
\makeatother
\tikzcdset{
	arrow style=tikz,
	diagrams={>={Latex}},
	tikzcd left hook/.tip={xGlyph[glyph math command=supset, swap, glyph axis = 5.7pt]},
	tikzcd right hook/.tip={xGlyph[glyph math command=supset, glyph axis = 5.7pt]},
	surjective head arrow /.tip = {tikzcd to[sep=-1.5pt]tikzcd to},
	surjective head/.style={
		-surjective head arrow
	}
}

%%fakesection Page layout
\usepackage[headsepline]{scrlayer-scrpage}
\renewcommand{\headfont}{}
\addtolength{\textheight}{3.14cm}
\setlength{\footskip}{0.5in}
\setlength{\headsep}{10pt}

\def\shortdate{\leavevmode\hbox{\the\year-\twodigits\month-\twodigits\day}}
\def\twodigits#1{\ifnum#1<10 0\fi\the#1}
\automark[chapter]{chapter}

\rohead{\footnotesize\thepage}
\rehead{\footnotesize \textbf{\sffamily Napkin}, by \emph{Evan Chen} (\napkinversion)}
\lehead{\footnotesize\thepage}
\lohead{\footnotesize \leftmark}
\chead{}
\rofoot{}
\refoot{}
\lefoot{}
\lofoot{}
%\cfoot{\pagemark}

%%fakesection Fancy section and chapter heads
\renewcommand*{\sectionformat}{\color{purple}\S\thesection\autodot\enskip}
\renewcommand*{\subsectionformat}{\color{purple}\S\thesubsection\autodot\enskip}
\newcommand{\problemhead}{A few harder problems to think about}
\renewcommand{\thesubsection}{\thesection.\roman{subsection}}

\addtokomafont{chapterprefix}{\raggedleft}
\RedeclareSectionCommand[beforeskip=0.5em]{chapter}
\renewcommand*{\chapterformat}{%
\mbox{\scalebox{1.5}{\chapappifchapterprefix{\nobreakspace}}%
\scalebox{2.718}{\color{purple}\thechapter\autodot}\enskip}}

\addtokomafont{partprefix}{\rmfamily}
\renewcommand*{\partformat}{\color{purple}\scalebox{2.5}{\thepart}}

%%fakesection Theorems
\usepackage{tcolorbox}
\tcbuselibrary{breakable,skins,hooks}

% patch tcolorbox to support continuation of a paragraph
% after box
% from https://tex.stackexchange.com/questions/568782/
\makeatletter
\tcbset{
	after app={%
		\ifx\tcb@drawcolorbox\tcb@drawcolorbox@breakable
		\else
		% add only when not breakabel
		\@endparenv
		\fi
	}
}

% for breakable
\appto\tcb@use@after@lastbox{\@endparenv\@doendpe}
\makeatother
% END patch tcolorbox for continuation of paragraph


\usepackage{thmtools}

\theoremstyle{definition}
\declaretheoremstyle[
	headfont=\sffamily\bfseries\color{MidnightBlue},
	headpunct={\\[3pt]},
	postheadspace={0pt},
]{thmtheorem}


\declaretheoremstyle[
	headfont=\bfseries\color{RawSienna},
	headpunct={\\[3pt]},
	postheadspace={0pt},
]{thmexample}

\tcbset{
	theorem box/.style={
		enhanced,
		arc=9pt,
		outer arc=10pt,
		colframe=blue,
		colback=TealBlue!5,
		boxrule=1pt,
		before skip=12pt,
		after skip=14pt,
		left=10pt,
		right=10pt
	},
	remark box/.style={
		boxrule=0pt,
		frame hidden,
		sharp corners,
		enhanced,
		borderline west={2pt}{0pt}{ForestGreen},
		before skip=8pt,
		colback=ForestGreen!5,
		after skip=12pt,
		breakable,
		left=10pt,
		right=10pt
	},
	example box/.style={
		enhanced,
		sharp corners,
		arc=9pt,
		outer arc=10pt,
		colframe=RawSienna,
		colback=Salmon!5,
		boxrule=0.5pt,
		before skip=12pt,
		after skip=14pt,
		breakable,
		top=6pt,
		bottom=8pt,
		breakable,
		left=10pt,
		right=10pt
	},
	ques box/.style={
		boxrule=0pt,
		frame hidden,
		enhanced,
		sharp corners,
		before skip=8pt,
		after skip=12pt,
		borderline west={3pt}{0pt}{black},
		colback=RedViolet!5!gray!5,
		breakable,
		left=10pt,
		right=10pt
	}
}

\declaretheorem[style=thmtheorem,name=Theorem,numberwithin=section]{theorem}
\tcolorboxenvironment{theorem}{theorem box}
\declaretheorem[style=thmtheorem,name=Lemma,sibling=theorem]{lemma}
\tcolorboxenvironment{lemma}{theorem box}
\declaretheorem[style=thmtheorem,name=Proposition,sibling=theorem]{proposition}
\tcolorboxenvironment{proposition}{theorem box}
\declaretheorem[style=thmtheorem,name=Corollary,sibling=theorem]{corollary}
\tcolorboxenvironment{corollary}{theorem box}
\declaretheorem[style=thmexample,name=Example,sibling=theorem]{example}
\tcolorboxenvironment{example}{example box}

\declaretheoremstyle[
	headfont=\bfseries\sffamily\color{ForestGreen!70!black},
	bodyfont=\normalfont,
	headpunct={ --- },
]{thmremark}
\declaretheoremstyle[
	headfont=\bfseries\sffamily\color{ForestGreen!70!black},
	bodyfont=\normalfont,
	headpunct={},
]{thmremark*}

\declaretheoremstyle[
	headfont=\bfseries,
	bodyfont=\normalfont\small
]{thmques}

\declaretheorem[name=Question,sibling=theorem,style=thmques]{ques}
\tcolorboxenvironment{ques}{ques box}
\declaretheorem[name=Exercise,sibling=theorem,style=thmques]{exercise}
\tcolorboxenvironment{exercise}{ques box}
\declaretheorem[name=Remark,sibling=theorem,style=thmremark]{remark}
\tcolorboxenvironment{remark}{remark box}
\declaretheorem[name=Remark,sibling=theorem,style=thmremark*]{remark*}
\tcolorboxenvironment{remark*}{remark box}
\declaretheorem[name=Step,style=thmremark]{step} % only used in Lebesgue int
\tcolorboxenvironment{step}{remark box}

\theoremstyle{definition}
\newtheorem{claim}[theorem]{Claim}
\newtheorem{definition}[theorem]{Definition}
\newtheorem{fact}[theorem]{Fact}
\newtheorem{abuse}[theorem]{Abuse of Notation}

\newtheorem{problem}{Problem}[chapter]
\renewcommand{\theproblem}{\thechapter\Alph{problem}}
\newtheorem{sproblem}[problem]{Problem}
\newtheorem{dproblem}[problem]{Problem}
\renewcommand{\thesproblem}{\theproblem$^{\star}$}
\renewcommand{\thedproblem}{\theproblem$^{\dagger}$}
\newcommand{\listhack}{$\empty$\vspace{-2em}}

%%fakesection Answers
\usepackage{answers}
\Newassociation{hint}{answeritem}{tex/backmatter/all-hints}
\Newassociation{sol}{answeritem}{tex/backmatter/all-solns}
\renewcommand{\solutionextension}{out}
\renewenvironment{answeritem}[1]{\item[\bfseries #1.]}{}

%%fakesection Table of contents
% First add ToC to ToC
\makeatletter
\usepackage{etoolbox}
\pretocmd{\tableofcontents}{%
	\if@openright\cleardoublepage\else\clearpage\fi
	\pdfbookmark[0]{\contentsname}{toc}%
}{}{}%
\makeatother
\setcounter{tocdepth}{1}
\RedeclareSectionCommand[tocnumwidth=4.2em]{part}
\RedeclareSectionCommand[tocpagenumberwidth=2.2em,tocnumwidth=4.2em]{chapter}
\RedeclareSectionCommand[tocpagenumberwidth=2.2em,tocnumwidth=2.8em]{section}
% adjust tocpagenumberwidth manually for large page number: https://tex.stackexchange.com/a/502168

%%fakesection Asymptote definitions
\usepackage{patch-asy}
\numberwithin{asy}{chapter}
\renewcommand{\theasy}{\thechapter\Alph{asy}}
\begin{asydef}
	import extras;
	size(6cm);
	usepackage("amsmath");
	usepackage("amssymb");
	defaultpen(fontsize(11pt));
	settings.tex = "latex";
	settings.outformat = "pdf";
\end{asydef}
\def\asydir{asy}

%%fakesection Bibliography
\usepackage[backend=biber,backref=true,style=alphabetic]{biblatex}
\DeclareLabelalphaTemplate{
	\labelelement{
		\field[final]{shorthand}
		\field{label}
		\field[strwidth=2,strside=left]{labelname}
	}
	\labelelement{
		\field[strwidth=2,strside=right]{year}
	}
}
\DeclareFieldFormat{labelalpha}{\textbf{\scriptsize #1}}
\addbibresource{references.bib}
\addbibresource{images.bib}
%% stylistic biblatex choices
\DefineBibliographyStrings{english}{%
	backrefpage  = {cited p.}, % for single page number
	backrefpages = {cited pp.} % for multiple page numbers
}
\DeclareFieldFormat{journaltitle}{\mkbibemph{#1},} % italic journal title with comma
\DeclareFieldFormat[inbook,thesis]{title}{\mkbibemph{#1}\addperiod} % italic title with period
\DeclareFieldFormat[article]{title}{#1} % title of journal article is printed as normal text
\DeclareFieldFormat[article]{volume}{\textbf{#1}\addcolon\space}
\renewcommand{\mkbibnamegiven}[1]{\textsc{#1}}
\renewcommand{\mkbibnamefamily}[1]{\textsc{#1}}
\renewcommand{\mkbibnameprefix}[1]{\textsc{#1}}
\renewcommand{\mkbibnamesuffix}[1]{\textsc{#1}}
\renewcommand{\finentrypunct}{}

%%fakesection Mini ToC
\usepackage[tight]{minitoc}
\mtcsetfont{parttoc}{chapter}{\sffamily\bfseries}
\mtcsetfont{parttoc}{section}{\footnotesize\rmfamily\upshape\mdseries}
\mtcsetfont{parttoc}{subsection}{\footnotesize\rmfamily\upshape\mdseries}
%\mtcsetdepth{parttoc}{1}
\setcounter{parttocdepth}{1}
\renewcommand*{\partheadstartvskip}{\vspace*{20em}}
\renewcommand*{\partheadendvskip}{}
%\noptcrule
\renewcommand\beforeparttoc{\noindent{\bfseries \Large Part \thepart: Contents}}
%\hspace{\fill}\rule{0.95\linewidth}{2pt}\hspace{\fill}
\doparttoc[n]

%%fakesection Misc haxx
\pdfstringdefDisableCommands{\def\Spec{\text{Spec }}\def\sigma{σ}}

\def\asydir{}
\addbibresource{../../references.bib}
\renewcommand{\gim}{($\ast$)}


\begin{document}
\title{Singular Homology}
\maketitle

\section{Temp}
\label{prob:triple_long_exact}
\label{ex:long_exact_rel}
\label{thm:long_exact_rel}
\label{thm:open_cover_homology}


\chapter{Singular Cohomology}
Here's one way to motivate this chapter.
The following two statements hold:
\begin{itemize}
	\ii $H_n(\CP^2) \cong H_n(S^2 \vee S^4)$ for every $n$.
	\ii $H_n(\CP^3) \cong H_n(S^2 \times S^4)$ for every $n$.
\end{itemize}
This is unfortunate, because if possible we would like
to be able to tell these spaces apart (as they are
in fact not homotopy equivalent), but the homology groups 
cannot tell the difference between them.

In this chapter, we'll define a \emph{cohomology group} $H^n(X)$ and $H^n(Y)$.
In fact, the $H^n$'s are completely determined by the $H_n$'s
by the so-called \emph{Universal Coefficient Theorem}.
However, it turns out that one can take all the cohomology groups and put
them together to form a \emph{cohomology ring} $H^\bullet$.
We will then see that $H^\bullet(X) \not\cong H^\bullet(Y)$ as rings.

\section{Cochain Complexes}
\begin{definition}
A \vocab{cochain complex} $A^\bullet$ is algebraically the same as a chain complex, except that the indices increase.
So it is a sequence of abelian groups
\[ \dots \taking{\delta} A^{n-1} \taking\delta A^n \taking\delta A^{n+1} \taking\delta \dots. \]
such that $\delta^2 = 0$.
Notation-wise, we're now using subscripts, and use $\delta$ rather $\partial$.
We define the \vocab{cohomology groups} by
\[ H^n(A^\bullet) = \ker\left( A^n \taking\delta A^{n+1} \right)
	/ \img\left( A^{n-1} \taking\delta A^n \right). \]
\end{definition}

\begin{example}[de Rham Cohomology]
	We have already met one example of a cochain complex:
	let $M$ be a smooth manifold and $\Omega^k(M)$ be the
	additive group of $k$-forms on $M$.
	Then we have a cochain complex
	\[ 0 \taking d \Omega^0(M)
		\taking d \Omega^1(M) \taking d \Omega^2(M)
		\taking d \dots. \]
	The resulting cohomology is called \vocab{de Rham cohomology},
	described later.
\end{example}

Aside from de Rham's cochain complex,
\textbf{the most common way to get a cochain complex
is to \emph{dualize} a chain complex.}
Specifically, pick an abelian group $G$;
note that $\Hom(-, G)$ is a contravariant functor,
and thus takes every chain complex
\[ \dots \taking\partial A_{n+1} \taking\partial
	A_n \taking\partial A_{n-1} \taking\partial \dots \]
into a cochain complex: letting $A^n = \Hom(A_n, G)$ we obtain
\[ \dots \taking\delta A^{n-1} \taking\delta
	A^n \taking\delta A^{n+1} \taking\delta \dots. \]
where $\delta(A_n \taking{f} G) = A_{n+1} \taking\partial A \taking{f} G$.

These are the cohomology groups we study most in algebraic topology,
so we give a special notation to them.
\begin{definition}
	Given a chain complex $A_\bullet$ of abelian groups and another group $G$,
	we let \[ H^n(A_\bullet; G) \] denote the cohomology groups
	of the dual cochain complex $A^\bullet$ obtained by applying $\Hom(-,G)$.
	In other words, $H^n(A_\bullet; G) = H^n(A^\bullet)$.
\end{definition}

\section{Cohomology of Spaces}
\prototype{$C^0(X;G)$ all functions $X \to G$ while $H^0(X)$ are those functions $X \to G$
constant on path components.}

The case of interest is our usual geometric situation, with $C_\bullet(X)$.
\begin{definition}
	For a space $X$ and abelian group $G$,
	we define $C^\bullet(X;G)$ to be the dual to the
	singular chain complex $C_\bullet(X)$,
	called the \vocab{singular cochain complex} of $X$;
	its elements are called \vocab{cochains}.

	Then we define the \vocab{cohomology groups}
	of the space $X$ as 
	\[ H^n(X; G) \defeq H^n(C_\bullet(X); G) = H_n(C^\bullet(X;G)). \]
\end{definition}
\begin{remark}
	Note that if $G$ is also a ring (like $\ZZ$ or $\RR$),
	then $H^n(X; G)$ is not only an abelian group but actually a $G$-module.
\end{remark}

\begin{example}
	[$C^0(X; G)$, $C^1(X; G)$, and $H^0(X;G)$]
	Let $X$ be a topological space and consider $C^\bullet(X)$.
	\begin{itemize}
		\ii $C_0(X)$ is the free abelian group on $X$,
		and $C^0(X) = \Hom(C_0(X), G)$.
		So a $0$-cochain is a function that
		takes every point of $X$ to an element of $G$.
		\ii $C_1(X)$ is the free abelian group on $1$-simplices in $X$.
		So $C^1(X)$ needs to take every $1$-simplex to an element of $G$.
	\end{itemize}
	Let's now try to understand $\delta : C^0(X) \to C^1(X)$.
	Given a $0$-cochain $\phi \in C^0(X)$,
	i.e.\ a homomorphism $\phi : C^0(X) \to G$,
	what is $\delta\phi : C^1(X) \to G$?
	Answer: 
	\[ \delta\phi : [v_0, v_1] \mapsto \phi([v_0]) - \phi([v_1]). \]
	Hence, elements of 
	$\ker(C^0 \taking\partial C^1) \cong H^0(X;G)$
	are those cochains
	that are \emph{constant on path-connected components}.
\end{example}
In particular, much like $H_0(X)$, we have \[ H^0(X) \cong G^{\oplus r} \]
if $X$ has $r$ path-connected components (where $r$ is finite\footnote{%
	Something funny happens if $X$ has \emph{infinitely} many path-connected components:
	say $X = \bigsqcup_\alpha X_\alpha$ over an infinite indexing set.
	In this case we have
	$H_0(X) = \bigoplus_\alpha G$ while $H^0(X) = \prod_\alpha G$.
	For homology we get a \emph{direct sum} while
	for cohomology we get a \emph{direct product}.

	These are actually different for infinite indexing sets.
	For general modules $\bigoplus_\alpha M_\alpha$ is \emph{defined} to only allow
	to have \emph{finitely many} zero terms.
	(This was never mentioned earlier in the Napkin,
	since I only ever defined $M \oplus N$ and extended it to finite direct sums.)
	No such restriction holds for $\prod_\alpha G_\alpha$ a product of groups.
	This corresponds to the fact that $C_0(X)$ is formal linear sums of $0$-chains
	(which, like all formal sums, are finite)
	from the path-connected components of $G$.
	But a cochain of $C^0(X)$ is a \emph{function}
	from each path-connected component of $X$ to $G$,
	where there is no restriction.
}).

To the best of my knowledge, the higher cohomology groups $H^n(X; G)$
(or even the cochain groups $C^n(X; G) = \Hom(C_n(X), G)$) are harder to describe concretely.

\begin{abuse}
	In this chapter the only cochain complexes
	we will consider are dual complexes as above.
	So, any time we write a chain complex $A^\bullet$ it is implicitly given
	by applying $\Hom(-,G)$ to $A_\bullet$.
\end{abuse}

\section{Cohomology of Spaces is Functorial}
We now check that the cohomology groups still exhibit the same nice functorial behavior.
First, let's categorize the previous results we had:

\begin{ques}
	Define $\catname{CoCmplx}$
	the category of cochain complexes.
\end{ques}

\begin{exercise}
	Interpret $\Hom(-,G)$ as a contravariant functor
	from \[ \Hom(-,G) : \catname{Cmplx}\op \to \catname{CoCmplx}. \]
	This means in particular that given a chain map $f : A_\bullet \to B_\bullet$,
	we naturally obtain a dual map $f^\vee : B^\bullet \to A^\bullet$.
\end{exercise}

\begin{ques}
	Interpret $H^n : \catname{CoCmplx} \to \catname{Grp}$ as a functor.
	Compose these to get a contravariant functor
	$H^n(-;G) : \catname{Cmplx}\op \to \catname{Grp}$.
\end{ques}

Then in exact analog to our result that $H_n : \catname{hTop} \to \catname{Grp}$ we have:
\begin{theorem}[$H^n (-;G): \catname{hTop}\op \to \catname{Grp}$]
	For every $n$, $H^n(-;G)$ is a contravariant functor
	from $\catname{hTop}\op$ to $\catname{Grp}$.
\end{theorem}
\begin{proof}
	The idea is to leverage the work we already did in constructing
	the prism operator earlier.
	First, we construct the entire sequence of functors
	from $\catname{Top}\op \to \catname{Grp}$:
	\begin{diagram}
		\catname{Top}\op & \rTo^{C_\bullet} & \catname{Cmplx}\op & \rTo^{\Hom(-;G)}
		& \catname{CoCmplx} & \rTo^{H^n} & \catname{Grp} \\
		X && C_\bullet(X) && C^\bullet(X;G) && H^n(X;G) \\
		\dTo^f &\rMapsto& \dTo^{f_\sharp} &\rMapsto&
		\uTo^{f^\sharp} &\rMapsto& \uTo^{f^\ast} \\
		Y && C_\bullet(Y) && C^\bullet(Y;G) && H^n(Y;G).
	\end{diagram}
	Here $f^\sharp = (f_\sharp)^\vee$, and $f^\ast$
	is the resulting induced map on homology groups of the cochain complex.

	So as before all we have to show is that $f \simeq g$,
	then $f^\ast = g^\ast$.
	Recall now that there is a prism operator such that
	$f_\sharp - g_\sharp = P \partial + \partial P$.
	If we apply the entire functor $\Hom(-;G)$ we get that
	$f^\sharp - g^\sharp = \delta P^\vee + P^\vee \delta$
	where $P^\vee : C^{n+1}(Y;G) \to C^n(X;G)$.
	So $f^\sharp$ and $g^\sharp$ are chain homotopic thus $f^\ast = g^\ast$.
\end{proof}


fsection{Universal Coefficient Theorem}
We now wish to show that the cohomology groups are determined up to isomorphism
by the homology groups: given $H_n(A_\bullet)$, we can extract $H^n(A_\bullet; G)$.
This is achieved by the \emph{Universal Coefficient Theorem}.
\begin{theorem}
	[Universal Coefficient Theorem]
	Let $A_\bullet$ be a chain complex of \emph{free} abelian groups,
	and let $G$ be another abelian group.
	Then there is a natural short exact sequence
	\[
		0 \to \Ext(H_{n-1}(A_\bullet), G) \to H^n(A_\bullet; G)
		\taking{h} \Hom(H_n(A_\bullet), G) \to 0. \]
	In addition, this exact sequence is \emph{split}
	so in particular
	\[ H^n(C_\bullet; G) \cong \Ext(H_{n-1}(A_\bullet, G))
		\oplus \Hom(H_n(A_\bullet), G). \]
\end{theorem}
Fortunately, in our case of interest, $A_\bullet$ is $C_\bullet(X)$
which is by definition free.

There are two things we need to explain, what the map $h$ is and the map $\Ext$ is.

It's not too hard to guess how \[ h : H^n(A_\bullet; G) \to \Hom(H_n(A_\bullet), G) \] is defined.
An element of $H^n(A_\bullet;G)$ is represented by a function which sends a cycle
in $A_n$ to an element of $G$.
The content of the theorem is to show that $h$ is surjective with kernel $\Ext(H_{n-1}(A_\bullet), G)$.

What about $\Ext$?
It turns out that $\Ext(-,G)$ is the so-called \vocab{Ext functor}, defined as follows.
Let $H$ be an abelian group, and consider a \vocab{free resolution} of $H$,
by which we mean an exact sequence
\[ \dots \taking{f_2} F_1 \taking{f_1} F_0 \taking{f_0} H \to 0. \]
Then we can apply $\Hom(-,G)$ to get a cochain complex
\[ \dots \xleftarrow{f_2^\vee} \Hom(F_1, G) \xleftarrow{f_1^\vee}
	\Hom(F_0, G) \xleftarrow{f_0^\vee} \Hom(H,G) \to 0. \]
but \emph{this cochain complex need not be exact}
(in categorical terms, $\Hom(-,G)$ does not preserve exactness).
We define \[ \Ext(H,G) \defeq \ker(f_2^\vee) / \img(f_1^\vee) \]
and it's a theorem that this doesn't depend on the choice of the free resolution.
There's a lot of homological algebra that goes into this,
which I won't take the time to discuss;
but the upshot of the little bit that I did include is that the $\Ext$
functor is very easy to compute in practice, since
you can pick any free resolution you want and compute the above.

%By ``natural'', we mean that if $f : A_\bullet \to B_\bullet$ is a chain map,
%then we obtain a commutative diagram
%\begin{diagram}
%	0 & \rTo & \Ext(H_{n-1}(A_\bullet), G) & \rTo
%		& H^n(A_\bullet;G) & \rTo & \Hom(H_n(A_\bullet), G) & \rTo & 0 \\
%	& & \uTo^{ \Ext(f_\ast, G) } & & \uTo^{f^\ast} & & \uTo^{\Hom(f_\ast, G)} & & \\
%	0 & \rTo & \Ext(H_{n-1}(B_\bullet), G) & \rTo
%		& H^n(A_\bullet;G) & \rTo & \Hom(H_n(B_\bullet), G) & \rTo & 0 \\
%\end{diagram}
%where $f_\ast$ is the induced arrow $H_n(A_\bullet) \to H_n(B_\bullet)$.

\begin{lemma}
	[Computing the $\Ext$ Functor]
	For any abelian groups $G$, $H$, $H'$ we have
	\begin{enumerate}[(a)]
		\ii $\Ext(H \oplus H', G) = \Ext(H, G) \oplus \Ext(H', G)$.
		\ii $\Ext(H,G) = 0$ for $H$ free, and
		\ii $\Ext(\Zc n, G) = G / nG$.
	\end{enumerate}
\end{lemma}
\begin{proof}
	For (a), note that if $\dots \to F_1 \to F_0 \to H \to 0$
	and $\dots \to F_1' \to F_0' \to F_0' \to H' \to 0$ are free resolutions,
	then so is $F_1 \oplus F_1' \to F_0 \oplus F_0' \to H \oplus H' \to 0$.

	For (b), note that $0 \to H \to H \to 0$ is a free resolution.
	
	Part (c) follows by taking the free resolution
	\[ 0 \to \ZZ \taking{\times n} \ZZ \to \Zc n \to 0 \]
	and applying $\Hom(-,G)$ to it.
	\begin{ques}
		Finish the proof of (c) from here. \qedhere
	\end{ques}
\end{proof}

\begin{ques}
	Some $\Ext$ practice: compute
	$\Ext(\ZZ^{\oplus 2015}, G)$ and $\Ext(\Zc{30}, \Zc 4)$.
\end{ques}

\section{Example Computation of Cohomology Groups}
\prototype{Possibly $H^n(S^m)$.}

The Universal Coefficient Theorem gives us a direct way to compute
any cohomology groups, provided we know the homology ones.

\begin{example}
	[Cohomolgy Groups of $S^m$]
	It is straightforward to compute $H^n(S^m)$ now:
	all the $\Ext$ terms vanish since $H_n(S^m)$ is always free,
	and hence we obtain that 
	\[ H^n(S^m) \cong \Hom(H_n(S^m), G) \cong
		\begin{cases}
			G & n=m, n=0 \\
			0 & \text{otherwise}.
		\end{cases}
	\]
%	By UCT for reduced groups, we also have
%	\[ \wt H^n(S^m) \cong \Hom(\wt H_n(S^m), G) \cong
%		\begin{cases}
%			G & n=m \\
%			0 & \text{otherwise}.
%		\end{cases}
%	\]
%	since $\Hom(\ZZ, G)$.
\end{example}

\begin{example}
	[Cohomolgy Groups of Torus]
	This example has no nonzero $\Ext$ terms either,
	since this time $H^n(S^1 \times S^1)$ is always free.
	So we obtain
	\[ H^n(S^1 \times S^1) \cong \Hom(H_n(S^1 \times S^1), G). \]
	Since $H_n(S^1 \times S^1)$ is $\ZZ$, $\ZZ^{\oplus 2}$, $\ZZ$
	in dimensions $n=1,2,1$ we derive that
	\[
		H^n(S^1 \times S^1)
		\cong
		\begin{cases}
			G & n = 0,2 \\
			G^{\oplus 2} & n = 1.
		\end{cases}
	\]
\end{example}

From these examples one might notice that:
\begin{lemma}
	[$0$th Homology Groups are just Duals]
	For $n = 0$ and $n = 1$, we have
	\[ H^n(X;G) \cong \Hom(H_n(X), G). \]
\end{lemma}
\begin{proof}
	It's already been shown for $n=0$.
	For $n=1$, notice that $H_0(X)$ is free,
	so the $\Ext$ term vanishes.
\end{proof}

\begin{example}
	[Cohomolgy Groups of Klein Bottle]
	This example will actually have $\Ext$ term.
	Recall that if $K$ is a Klein Bottle then its homology groups are
	$\ZZ$ in dimension $n=0$ and $\ZZ \oplus \Zc 2$ in $n=1$, and $0$ elsewhere.

	For $n=0$, we again just have $H^0(K;G) \cong \Hom(\ZZ, G) \cong G$.
	For $n=1$, the $\Ext$ term is $\Ext(H_0(K), G) \cong \Ext(\ZZ, G) = 0$
	so \[ H^1(K;G) \cong \Hom(\ZZ \oplus \Zc2, G) \cong G \oplus \Hom(\Zc2, G). \]
	We have that $\Hom(\Zc2,G)$ is the subgroup
	of elements of order $2$ in $G$ (and $0 \in G$).

	But for $n=2$, we have our first interesting $\Ext$ group:
	the exact sequence is
	\[ 0 \to \Ext(\ZZ \oplus \Zc 2, G) \to H^2(X;G) \to \underbrace{H_2(X)}_{=0} \to 0. \]
	Thus, we have
	\[ H^2(X;G) \cong \left( \Ext(\ZZ,G) \oplus \Ext(\Zc2,G) \right) \oplus 0
		\cong G/2G. \]
	All the higher groups vanish.
	In summary:
	\[
		H^n(X;G) \cong
		\begin{cases}
			G & n = 0 \\
			G \oplus \Hom(\Zc2, G) & n = 1 \\
			G/2G & n = 2 \\
			0 & n \ge 3. 
		\end{cases}
	\]
\end{example}


\section{Relative Cohomology Groups}
One can also define relative cohomology groups in the obvious way:
dualize the chain complex
\[ \dots \taking\partial C_1(X,A) \taking\partial C_0(X,A) \to 0 \]
to obtain a cochain complex
\[
	\dots \xleftarrow\delta C^1(X,A;G) \xleftarrow\delta C^0(X,A;G)
	\leftarrow 0.
\]
We can take the cohomology groups ofthis.
\begin{definition}
	The groups thus obtained are the \vocab{relative cohomology groups}
	are denoted $H^n(X,A;G)$.
\end{definition}

In addition, we can define reduced cohomology groups as well.
One way to do it is to take the augmented singular chain complex
\[ \dots \taking\partial C_1(X) \taking\partial C_0(X) \taking\eps \ZZ \to 0 \]
and dualize it to obtain
\[
	\dots \xleftarrow\delta C^1(X;G) \xleftarrow\delta C^0(X;G)
	\xleftarrow{\eps^\vee} \underbrace{\Hom(\ZZ, G)}_{\cong G}
	\leftarrow 0.
\]
Since the $\ZZ$ we add is also free,
the Universal Coefficient Theorem still applies.
So this will give us reduced cohomology groups.

However, since we already defined the relative cohomology groups,
it is easiest to simply do the following.
\begin{definition}
	The \vocab{reduced cohomology groups} of a nonempty space $X$,
	denoted $\wt H^n(X; G)$,
	are defined to be $H^n(X, \{\ast\} ; G)$
	for some point $\ast \in X$.
\end{definition}


\section\problemhead
\begin{sproblem}
	[Wedge Product Cohomology]
	For any $G$ and $n$ we have
	\[
		\wt H^n(X \vee Y; G)
		\cong
		\wt H^n(X; G) \oplus \wt H^n(Y; G).
	\]	
\end{sproblem}

\begin{dproblem}
	Prove that for a field $F$ of characteristic zero and a space $X$
	with finitely generated homology groups:
	\[ H^k(X, F) \cong \left( H_k(X) \right)^\vee.  \]
	Thus over fields cohomology is the dual of homology.
\end{dproblem}

\begin{problem}[$\Zc2$-Cohomology of $\RP^n$]
	Prove that
	\[
		H^m(\RP^n, \Zc2)
		\cong
		\begin{cases}
			\ZZ & \text{$m=0$, or $m$ is odd and $m=n$} \\
			\Zc2 & \text{$0 < m < n$ and $m$ is odd} \\
			0 & \text{otherwise}.
		\end{cases}
	\]
\end{problem}


\chapter{Application of Cohomology}
In this final chapter on topology, I'll state (mostly without proof)
some nice properties of cohomology groups, and in particular
introduce the soc-called cup product.
For an actual treatise on the cup product,
see \cite{ref:hatcher} or \cite{ref:maxim752}.

\section{Poincar\'e Duality}
First cool result:
you may have noticed symmetry in the (co)homology groups of
``nice'' spaces like the torus or $S^n$.
In fact this is predicted by the following theorem:
\begin{theorem}
	[Poincar\'e Duality]
	If $M$ is a smooth oriented compact $n$-manifold,
	then we have a natural isomorphism
	\[ H^k(M; \ZZ) \cong H_{n-k}(M) \]
	for every $k$.
	In particular, $H^k(M) = 0$ for $k > n$.
\end{theorem}
So for smooth oriented compact manifolds,
cohomology and homology groups are not so different.

From this follows the symmetry that we mentioned
when we first defined the Betti numbers:
\begin{corollary}
	[Symmetry of Betti Numbers]
	Let $M$ be a smooth oriented compact $n$-manifold,
	and let $b_k$ denote its Betti number.
	Then \[ b_k = b_{n-k}. \]
\end{corollary}
\begin{proof}
	\Cref{prob:betti}.
\end{proof}


\section{De Rham Cohomology}
We now reveal the connection between
differential forms and singular cohomology.

Let $M$ be a smooth manifold.
We are interested in the homology and cohomology groups of $M$.
We specialize to the case $G = \RR$, the additive group of real numbers.
\begin{ques}
	Check that $\Ext(H, \RR) = 0$ for any finitely generated abelian group $H$.
\end{ques}
Thus, with real coefficients the Universal Coefficient Theorem says that
\[ H^k(M; \RR) \cong \Hom(H_k(M), \RR) = \left( H_k(M) \right)^\vee \]
where we view $H_k(X)$ as a real vector space.
So, we'd like to get a handle on either $H_k(M$) or $H^k(M; \RR)$.

Consider the cochain complex
\[
	0 \to \Omega^0(M)
	\taking d \Omega^1(M)
	\taking d \Omega^2(M)
	\taking d \Omega^3(M)
	\taking d \dots
\]
and let $\HdR^k(M)$ denote its cohomology groups.
Thus the de Rham cohomology is the closed forms modulo the exact forms.
\[
	\text{Cochain} : \text{Cocycle} : \text{Coboundary}
	= \text{$k$-form} : \text{Closed form} : \text{Exact form}. 
\]

The whole punch line is:
\begin{theorem}
	[de Rham's Theorem]
	For any smooth manifold $M$, we have a natural isomorphism
	\[ H^k(M; \RR) \cong \HdR^k(M). \]
\end{theorem}
So the theorem is that the real cohomology groups of manifolds $M$
are actually just given by the behavior of differential forms.
Thus, 
\begin{moral}
	One can metaphorically think of elements of cohomology groups
	as $G$-valued differential forms on the space.
\end{moral}

Why does this happen?
In fact, we observed already behavior of differential
forms which reflects holes in the space.
For example, let $M = S^1$ be a circle
and consider the \textbf{angle form} $\alpha$
(see \Cref{ex:angle_form}).
The from $\alpha$ is closed, but not exact,
because it is possible to run a full circle around $S^1$.
So the failure of $\alpha$ to be exact is signaling
that $H_1(S^1) \cong \ZZ$.

\section{Graded Rings}
\prototype{Polynomial rings are graded rings.}
In the de Rham cohomology, the differential forms can interact in another way:
given a $k$-form $\alpha$ and an $\ell$-form $\beta$, we can consider
a $(k+\ell)$-form
\[ \alpha \wedge \beta. \]
So we can equip the set of forms with a ``product'', satisfying
$\beta \wedge \alpha = (-1)^{k\ell} \alpha \wedge \beta$
This is a special case of a more general structure:

\begin{definition}
	A \vocab{graded ring} $R$ is an abelian group
	\[ R = \bigoplus_{d \ge 0} R_d \]
	where $R_0$, $R_1$, \dots, are abelian groups,
	with an additional associative binary operation $\times : R \to R$.
	This multiplication is required to satisfy the following property:
	if $r \in R_d$ and $s \in R_e$,
	we have $rs \in R_{d+e}$.

	Elements of an $R_d$ are called \vocab{homogeneous elements}.
	If $r \in R_d$ and $r \neq 0$, we write $|r| = d$.
\end{definition}
\begin{remark}
	Note that we do \emph{not} assume that graded rings are commutative.
	In fact, these graded rings may not even have an identity $1$.

	Other than these differences, one can also think of this as a
	``ring with a notion of degree''.
\end{remark}

\begin{example}[Examples of Graded Rings]
	\listhack
	\begin{enumerate}[(a)]
		\ii The ring $R = \ZZ[x]$ is a graded ring,
		with the $d$th component being the multiples of $x^d$.
		\ii The ring $R = \ZZ[x,y,z]$ is a graded ring,
		with the $d$th component being the abelian group
		of homogeneous degree $d$ polynomials (and $0$).
		\ii Let $V$ be a vector space, and consider
		the abelian group
		\[ \Lambda^\bullet(V) = \bigoplus_{d \ge 0} \Lambda^d(V). \]
		For example, $e_1 + (e_2 \wedge e_3) \in \Lambda^\bullet(V)$, say.
		We endow $\Lambda^\bullet(V)$ with the product $\wedge$,
		which makes it into a graded ring.
		\ii The set of differential forms of a manifold $M$,
		say \[ \Omega^\bullet(M) = \bigoplus_{d \ge 0} \Omega^d(M) \]
		endowed with formal addition and the product $\wedge$,
		forms a non-commutative graded ring.
	\end{enumerate}
\end{example}
\begin{definition}
	A graded ring $R$ is \vocab{anticommutative} if
	for any homogeneous $r$ and $s$ we have
	\[ rs = (-1)^{|r| |s|} sr \]
\end{definition}
\begin{example}[Anticommutative Graded Ring]
	Both $\Lambda^\bullet(V)$ and $\Omega^\bullet(M)$ are anticommutative.
\end{example}

Let's return to the situation of $\Omega^\bullet(M)$.
Consider again the de Rham cohomology groups $\HdR^k(M)$,
whose elements are closed forms modulo exact forms.
We claim that:
\begin{lemma}
	[Wedge Product Respects de Rham Cohomology]
	The wedge product induces a map
	\[ \wedge : \HdR^k(M) \times \HdR^\ell(M) \to \HdR^{k+\ell}(M). \]
\end{lemma}
\begin{proof}
	First, we recall that the operator $d$ satisfies
	\[
		d(\alpha \wedge \beta)
		= (d\alpha) \wedge \beta + \alpha \wedge (d\beta).
	\]
	Now suppose $\alpha$ and $\beta$ are closed forms.
	Then from the above, $\alpha \wedge \beta$ is clearly closed.
	Also if $\alpha$ is closed and $\beta = d\omega$ is exact,
	then $\alpha \wedge \beta$ is exact, from the identity
	\[ d(\alpha \wedge \omega)
		= d\alpha \wedge\omega + \alpha \wedge d\omega = \alpha \wedge \beta. \]
	Similarly if $\alpha$ is exact and $\beta$ is closed
	then $\alpha \wedge \beta$ is exact.
	Thus it makes sense to take the product modulo exact forms,
	giving the theorem above.
\end{proof}

Therefore, we can obtain a \emph{anticommutative graded ring}
\[ \HdR^\bullet(M) = \bigoplus_{k \ge 0} \HdR^k(M) \]
with $\wedge$ as a product.

\section{Cup Products}
Inspired by this, we want to see if we can construct a similar product
on $\bigoplus_{k \ge 0} H^k(X; R)$ for any topological space $X$ and ring $R$.
The way to do this is via the \emph{cup product}.

Suppose now we replace our abelian group $G$ by a commutative ring $R$ with identity.
Then this gives us a way to multiply two cochains, as follows.
\begin{definition}
	Suppose $\phi \in C^k(X;R)$ and $\psi \in C^\ell(X;R)$.
	Then we can define their \vocab{cup product}
	$\phi\smile\psi \in C^{k+\ell}(X;R)$ to be
	\[
		(\phi\smile\psi)([v_0, \dots, v_{k+\ell}])
		= 
		\phi\left( [v_0, \dots, v_k] \right)
		\cdot
		\psi\left( [v_k, \dots, v_{k+\ell}] \right)
	\]
	where the multiplication is in $R$.
\end{definition}

\begin{ques}
	Which $0$-cochain is the identity for $\smile$?
\end{ques}

First, we prove an analogous result as before:
\begin{lemma}[$\delta$ with Cup Products]
	We have
	$\delta(\phi\smile\psi) = \delta\phi\smile\psi
	+ (-1)^k\phi\smile\delta\psi$.
\end{lemma}
\begin{proof}
	Direct $\sum$ computations.
\end{proof}
Thus, by the same routine we used for de Rham cohomology, we get
an induced map
\[ \smile : H^k(X;R) \times H^\ell(X;R) \to H^{k+\ell}(X;R).  \]
We then define the \vocab{singular cohomology ring} 
whose elements are finite sums in 
\[ H^\bullet(X;R) = \bigoplus_{k \ge 0} H^k(X;R) \]
and with multiplication given by $\smile$.
In fact, it turns out that:
\begin{proposition}[Cohomology is Anticommutative]
	$H^\bullet(X; R)$ is an anticommutative graded ring.
\end{proposition}
For a proof, see \cite[Theorem 3.11, pages 210-212]{ref:hatcher}.
Moreover, we have the de Rham isomorphism
\begin{theorem}
	[de Rham Extends to Ring Isomorphism]
	For any smooth manifold $M$, the isomorphism
	of de Rham cohomology groups to singular cohomology
	groups in facts gives an isomorphism
	\[ H^\bullet(M; \RR) \cong \HdR^\bullet(M) \]
	of anticommutative graded rings.
\end{theorem}

Therefore, if ``differential forms'' are the way to visualize
the elements of a cohomology group, the wedge product is the
correct way to visualize the cup product.

We now present (mostly without proof)
the cohomology rings of some common spaces.

\begin{example}[Cohomology Ring of $S^n$]
	Consider $S^n$ for $n \ge 1$.
	The nontrivial cohomology groups are given by
	$H^0(S^n; \ZZ) \cong H^n(S^n; \ZZ) \cong \ZZ$.
	So as an abelian group
	\[ H^\bullet(S^n; \ZZ) \cong \ZZ \oplus \alpha \ZZ \]
	where $\alpha$ is the generator of $H^n(S^n, \ZZ)$.
	
	Now, observe that $|\alpha\smile\alpha| = 2n$, but
	since $H^{2n}(S^n; \ZZ) = 0$ we must have $\alpha\smile\alpha=0$.
	So even more succinctly,
	\[ H^\bullet(S^n; \ZZ) \cong \ZZ[\alpha]/(\alpha^2). \]
\end{example}

\begin{example}[Cohmology Ring of Real and Complex Projective Space]
	It turns out that
	\begin{align*}
		H^\bullet(\RP^n; \Zc2) &\cong \Zc2[\alpha]/(\alpha^{n+1}) \\
		H^\bullet(\CP^n; \ZZ) &\cong \ZZ[\beta]/(\beta^{n+1})
	\end{align*}
	where $|\alpha| = 1$ is a generator of $H^1(\RP^n; \Zc2)$
	and $|\beta| = 2$ is a generator of $H^2(\CP^n; \ZZ)$.
\end{example}

\begin{example}
	[Cohomology of Torus]
	The cohomology ring $H^\bullet(S^1 \times S^1; \ZZ)$
	of the torus is generated by elements $|\alpha| = |\beta| = 1$
	which satisfy the relations
	$\alpha \smile \alpha = \beta \smile \beta = 0$,
	and $\alpha \smile \beta = -\beta \smile \alpha$.
	(It also includes an identity $1$.)
	Thus as a $\ZZ$-module it is
	\[ H^\bullet(S^1 \times S^1; \ZZ)
		\cong \ZZ \oplus \left[ \alpha \ZZ \oplus \beta \ZZ \right]
		\oplus (\alpha \smile \beta) \ZZ. \]
	This gives the expected dimensions $1+2+1=4$.
\end{example}


\section{Relative Cohomology Rings}
For $A \subseteq X$, one can also define a relative cup product
\[ H^k(X,A;R) \times H^\ell(X,A;R) \to H^{k+\ell}(X,A;R). \]
After all, if either cochain vanishes on chains in $A$,
then so does their cup product.
This lets us define \vocab{relative cohomology ring}
and \vocab{reduced cohomology ring} (by $A = \{\ast\}$), say
\begin{align*}
\wt H^\bullet(X,A;R) &= \bigoplus_{k \ge 0} \wt H^k(X,A; R) \\
\wt H^\bullet(X;R) &= \bigoplus_{k \ge 0} \wt H^k(X;R).
\end{align*}

Once again we have functoriality:
\begin{theorem}
	[Cohomology Rings are Functorial]
	Fix a ring $R$.
	Then we have functors
	\begin{align*}
		H^\bullet(-; R) &: \catname{hTop}\op \to \catname{GradedRings} \\
		H^\bullet(-,-; R) &: \catname{hPairTop}\op \to \catname{GradedRings}.
	\end{align*}
\end{theorem}

Unfortunately, unlike with (co)homology groups,
it is a nontrivial task to determine the cup product
for even nice spaces like CW complexes.
So we will not do much in the way of computation.
However, there is a little progress we can make.

\section{Wedge Sums}

Our goal is to now compute $\wt H^\bullet(X \wedge Y)$.
To do this, we need to define the product of two rings:
\begin{definition}
	Let $R$ and $S$ be two rings.
	The product ring $R \times S$ is defined by
	taking the underlying abelian group as $R \oplus S$,
	and then declaring $r \cdot s = 0$ for $r \in R$, $s \in S$.
\end{definition}

Now, the theorem is that:
\begin{theorem}
	[Cohomology Rings of Wedge Sums]
	We have
	\[
		\wt H^\bullet(X \wedge Y; R)
		\cong \wt H^\bullet(X;R)
		\times \wt H^\bullet(Y;R).
	\]
\end{theorem}

This allows us to resolve the first question posed at the beginning.
Let $X = \CP^2$ and $Y = S^2 \vee S^4$.
We have that
\[ H^\bullet(\CP^2; \ZZ) \cong \ZZ[\alpha] / (\alpha^3). \]
Hence this ring is generated by there elements:
\begin{itemize}
	\ii $1$, in dimension $0$.
	\ii $\alpha$, in dimension $2$.
	\ii $\alpha^2$, in dimension $4$.
\end{itemize}
Next, consider the reduced ring
\[ \wt H^\bullet(S^2 \vee S^4; \ZZ) \cong
	\wt H^\bullet(S^2; \ZZ)
	\oplus \wt H^\bullet(S^4 ; \ZZ).
\]
Thus the ordinary cohomology ring $H^\bullet(S^2 \vee S^4 \; \ZZ)$
is also generated by three elements.
\begin{itemize}
	\ii $1$, in dimension $0$ (once we add back in the $0$th dimension).
	\ii $a_2$, in dimension $2$ (from $H^\bullet(S^2 ; \ZZ)$).
	\ii $a_4$, in dimension $4$ (from $H^\bullet(S^4 ; \ZZ)$).
\end{itemize}
These rings are isomorphic as abelian groups, as we expected.
However, in the first ring, the product of two degree $2$ generators is
\[ \alpha \cdot \alpha = \alpha^2. \]
In the second ring, the product of two degree $2$ generators is
\[ a_2 \cdot a_2 = a_2^2 = 0 \]
since $a_2 \smile a_2 = 0 \in H^\bullet(S^2; \ZZ)$.

Thus $S^2 \vee S^4$ and $\CP^2$ are not homotopy equivalent.

\section{K\"unneth Formula}
We now wish to tell apart the spaces $S^2 \times S^4$ and $\CP^3$.
In order to do this, we will need a formula
for $H^n(X \times Y; R)$ in terms of $H^n(X;R)$ and $H^n(Y;R)$.
Thus formulas are called \vocab{K\"unneth formulas}.
In this section we will only use a very special case,
which involves the tensor product of two rings.

\begin{definition}
	Let $S_1$ and $S_2$ be two rings which are also $R$-modules.
	We define the tensor product $S_1 \otimes_R S_2$ as follows.
	As an abelian group, it is $S_1 \otimes_R S_2$.
	The ring multiplication is given on basis elements by
	\[ (s_1 \otimes s_2)(s_1' \otimes s_2')
		= (s_1s_1') \otimes (s_2s_2').
	\]
\end{definition}

Now let $X$ and $Y$ be topological spaces, and take the product:
we have a diagram
\begin{diagram}
	&& X \times Y && \\
	X &\ldTo(2,1)^{\pi_X} && \rdTo(2,1)^{\pi_Y} & Y \\
\end{diagram}
where $\pi_X$ and $\pi_Y$ are projections.
As $H^k(-; R)$ is functiorial, this gives induced maps
\begin{align*}
	\pi_X^\ast &: H^k(X \times Y; R) \to H^k(X; R) \\
	\pi_Y^\ast &: H^k(X \times Y; R) \to H^k(Y; R)
\end{align*}
for every $k$.

By using this, we can define a so-called cross product.
\begin{definition}
	Let $R$ be a ring, and $X$ and $Y$ spaces.
	Let $\pi_X$ and $\pi_Y$ be the projections of $X \times Y$
	onto $X$ and $Y$.
	Then the \vocab{cross product} is the map
	\[
		H^\bullet(X; R) \otimes_R H^\bullet(Y;R)
		\taking{\times} H^\bullet(X \times Y; R)
	\]
	acting on cocycles as follows:
	$\phi \times \psi = \pi_X^\ast(\phi) \smile \pi_Y^\ast(\phi)$.
\end{definition}

This is just the most natural way to take a $k$-cycle
on $X$ and an $\ell$-cycle on $Y$, and create a $(k+\ell)$-cycle
on the product space $X \times Y$.


\begin{theorem}
	[K\"unneth Formula]
	Let $X$ and $Y$ be arbitrary spaces such that $H^k(Y;R)$
	is a finitely generated free $R$-module for every $k$.
	Then the cross product is an isomorphism
	\[
		H^\bullet(X;R) \otimes_R H^\bullet(Y;R)
		\to H^\bullet(X \times Y; R). 
	\]
\end{theorem}

In any case, this finally lets us resolve the question
set out at the beginning.
We saw that $H_n(\CP^3) \cong H_n(S^2 \times S^4)$ for every $n$,
and thus it follows that $H^n(\CP^3; \ZZ) \cong H^n(S^2 \times S^4; \ZZ)$ too.

But now let us look at the cohomology rings. First, we have
\[ H^\bullet(\CP^3; \RR) \cong \ZZ[\alpha] / (\alpha^3)
	\cong \ZZ \oplus \alpha\ZZ \oplus \alpha^2\ZZ \oplus \alpha^3\ZZ
\] 
where $|\alpha| = 2$; hence this group is generated by
\begin{itemize}
	\ii $1$, in degree $0$.
	\ii $\alpha$, in degree $2$.
	\ii $\alpha^2$, in degree $4$.
	\ii $\alpha^3$, in degree $6$.
\end{itemize}

Now let's analyze
\[ H^\bullet(S^2 \times S^4; \RR) \cong 
	\ZZ[\beta] / (\beta^2)
	\otimes
	\ZZ[\gamma] / (\gamma^2).
\]
It is generated thus by the following elements:
\begin{itemize}
	\ii $1 \otimes 1$, in degree $0$.
	\ii $\beta \otimes 1$, in degree $2$.
	\ii $1 \otimes \gamma$, in degree $4$.
	\ii $\beta \otimes \gamma$, in degree $6$.
\end{itemize}
Again equal as abelian groups.
But notice that if we square $\beta \otimes 1$ we get
\[ (\beta \otimes 1)(\beta \otimes 1) = \beta^2 \otimes 1 = 0. \]
Yet the degree $2$ generator of $H^\bullet(\CP^3; \ZZ)$
does not have this property.
Hence the rings are not isomorphic.

So it follows that $\CP^3$ and $S^2 \times S^4$ are not homotopy equivalent.


	
% Borsuk Ulam

\section\problemhead

\begin{dproblem}
	[Symmetry of Betti Numbers by Poincar\'e Duality]
	\label{prob:betti}
	Let $M$ be a smooth oriented compact $n$-manifold,
	and let $b_k$ denote its Betti number.
	Prove that $b_k = b_{n-k}$.
	\begin{hint}
		Write $H^k(M; \ZZ)$ in terms of $H_k(M)$
		using the UCT, and analyze the ranks.
	\end{hint}
\end{dproblem}

\begin{problem}
	Show that $\RP^n$ is not orientable for even $n$.
	\begin{hint}
		Use the previous result on Betti numbers.
	\end{hint}
\end{problem}

\begin{problem}
	Show that $\RP^3$ is not homotopy equivalent to $\RP^2 \wedge S^3$.
	\begin{hint}
		Use the $\Zc2$ cohomologies, and find the cup product.
	\end{hint}
\end{problem}

\begin{problem}
	\gim
	Show that $S^m \wedge S^n$ is not a deformation retract
	of $S^m \times S^n$ for any $m,n \ge 1$.
	\begin{hint}
		Assume that $r : S^m \times S^n \to S^m \wedge S^n$ is such a map.
		Show that the induced map
		$H^\bullet(S^m \wedge S^n; \ZZ) \to H^\bullet(S^m \times S^n; \ZZ)$
		between their cohomology rings is monic
		(since there exists an inverse map $i$).
	\end{hint}
	\begin{sol}
		See \cite[Example 3.3.14, pages 68-69]{ref:maxim752}.
	\end{sol}
\end{problem}

\end{document}
