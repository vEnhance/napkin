\documentclass[11pt]{scrreprt}
%%fakesection Load packages

\usepackage{lmodern}
\usepackage[pdfusetitle]{hyperref}
\ExplSyntaxOn
\sys_if_engine_luatex:T {
	\usepackage{luatex85}
}
\sys_if_engine_pdftex:T {
	\usepackage[T1]{fontenc}
}
\ExplSyntaxOff

% These are evan.sty
\usepackage{amsmath,amssymb,amsthm}
\usepackage{mathrsfs}
\usepackage[usenames,svgnames,dvipsnames]{xcolor}
\usepackage{textcomp}
\usepackage{enumerate}
\usepackage[textsize=scriptsize,shadow]{todonotes}
\usepackage{mathtools}
\usepackage{microtype}
\usepackage[normalem]{ulem}
\usepackage{stmaryrd}
\usepackage{wasysym}
\usepackage{multirow}
\usepackage{prerex}
\usepackage[nameinlink]{cleveref}
\usepackage{derivative}

%%fakesection evan.sty macros
%Small commands
%% Napkin commands
\newcommand{\prototype}[1]{
	\emph{{\color{red} Prototypical example for this section:} #1} \par\medskip
}
\newenvironment{moral}{%
	\begin{tcolorbox}[boxrule=0.4pt,colframe=green!70!black,sharp corners,
		standard jigsaw,opacityback=0,left=10pt,right=10pt,top=3pt,bottom=3pt,
		before skip=10pt,after skip=20pt]%
	\bfseries\color{green!50!black}}%
	{\end{tcolorbox}}

%%fakesection Links (hyperref loaded earlier implicitly)
\hypersetup{
	linkcolor={red!50!black},
	citecolor={green!50!black},
	urlcolor={blue!80!black},
	pdfkeywords={napkin,math},
	pdfsubject={web.evanchen.cc},
	colorlinks,
}

%%fakesection Commutative diagrams
\usepackage{tikz-cd}
\usetikzlibrary{arrows,arrows.meta}
% make a larger hook
% https://tex.stackexchange.com/questions/514451/how-to-define-a-new-hooked-arrow
\makeatletter
\pgfdeclarearrow{
	name=xGlyph,
	cache=false,
	bending mode=none,
	parameters={\tikzcd@glyph@len,\tikzcd@glyph@shorten},
	setup code={%
		\pgfarrowssettipend{\tikzcd@glyph@len\advance\pgf@x by\tikzcd@glyph@shorten}},
	defaults={
		glyph axis=axis_height,
		glyph length=+1.55ex,
		glyph shorten=+-0.1ex},
	drawing code={%
		\pgfpathrectangle{\pgfpoint{+0pt}{+-1.5ex}}{\pgfpoint{+\tikzcd@glyph@len}{+3ex}}%
		\pgfusepathqclip%
		\pgftransformxshift{+\tikzcd@glyph@len}%
		\pgftransformyshift{+-\tikzcd@glyph@axis}%
		\pgftext[right,base]{\tikzcd@glyph}}}
\makeatother
\tikzcdset{
	arrow style=tikz,
	diagrams={>={Latex}},
	tikzcd left hook/.tip={xGlyph[glyph math command=supset, swap, glyph axis = 5.7pt]},
	tikzcd right hook/.tip={xGlyph[glyph math command=supset, glyph axis = 5.7pt]},
	surjective head arrow /.tip = {tikzcd to[sep=-1.5pt]tikzcd to},
	surjective head/.style={
		-surjective head arrow
	}
}

%%fakesection Page layout
\usepackage[headsepline]{scrlayer-scrpage}
\renewcommand{\headfont}{}
\addtolength{\textheight}{3.14cm}
\setlength{\footskip}{0.5in}
\setlength{\headsep}{10pt}

\def\shortdate{\leavevmode\hbox{\the\year-\twodigits\month-\twodigits\day}}
\def\twodigits#1{\ifnum#1<10 0\fi\the#1}
\automark[chapter]{chapter}

\rohead{\footnotesize\thepage}
\rehead{\footnotesize \textbf{\sffamily Napkin}, by \emph{Evan Chen} (\napkinversion)}
\lehead{\footnotesize\thepage}
\lohead{\footnotesize \leftmark}
\chead{}
\rofoot{}
\refoot{}
\lefoot{}
\lofoot{}
%\cfoot{\pagemark}

%%fakesection Fancy section and chapter heads
\renewcommand*{\sectionformat}{\color{purple}\S\thesection\autodot\enskip}
\renewcommand*{\subsectionformat}{\color{purple}\S\thesubsection\autodot\enskip}
\newcommand{\problemhead}{A few harder problems to think about}
\renewcommand{\thesubsection}{\thesection.\roman{subsection}}

\addtokomafont{chapterprefix}{\raggedleft}
\RedeclareSectionCommand[beforeskip=0.5em]{chapter}
\renewcommand*{\chapterformat}{%
\mbox{\scalebox{1.5}{\chapappifchapterprefix{\nobreakspace}}%
\scalebox{2.718}{\color{purple}\thechapter\autodot}\enskip}}

\addtokomafont{partprefix}{\rmfamily}
\renewcommand*{\partformat}{\color{purple}\scalebox{2.5}{\thepart}}

%%fakesection Theorems
\usepackage{tcolorbox}
\tcbuselibrary{breakable,skins,hooks}

% patch tcolorbox to support continuation of a paragraph
% after box
% from https://tex.stackexchange.com/questions/568782/
\makeatletter
\tcbset{
	after app={%
		\ifx\tcb@drawcolorbox\tcb@drawcolorbox@breakable
		\else
		% add only when not breakabel
		\@endparenv
		\fi
	}
}

% for breakable
\appto\tcb@use@after@lastbox{\@endparenv\@doendpe}
\makeatother
% END patch tcolorbox for continuation of paragraph


\usepackage{thmtools}

\theoremstyle{definition}
\declaretheoremstyle[
	headfont=\sffamily\bfseries\color{MidnightBlue},
	headpunct={\\[3pt]},
	postheadspace={0pt},
]{thmtheorem}


\declaretheoremstyle[
	headfont=\bfseries\color{RawSienna},
	headpunct={\\[3pt]},
	postheadspace={0pt},
]{thmexample}

\tcbset{
	theorem box/.style={
		enhanced,
		arc=9pt,
		outer arc=10pt,
		colframe=blue,
		colback=TealBlue!5,
		boxrule=1pt,
		before skip=12pt,
		after skip=14pt,
		left=10pt,
		right=10pt
	},
	remark box/.style={
		boxrule=0pt,
		frame hidden,
		sharp corners,
		enhanced,
		borderline west={2pt}{0pt}{ForestGreen},
		before skip=8pt,
		colback=ForestGreen!5,
		after skip=12pt,
		breakable,
		left=10pt,
		right=10pt
	},
	example box/.style={
		enhanced,
		sharp corners,
		arc=9pt,
		outer arc=10pt,
		colframe=RawSienna,
		colback=Salmon!5,
		boxrule=0.5pt,
		before skip=12pt,
		after skip=14pt,
		breakable,
		top=6pt,
		bottom=8pt,
		breakable,
		left=10pt,
		right=10pt
	},
	ques box/.style={
		boxrule=0pt,
		frame hidden,
		enhanced,
		sharp corners,
		before skip=8pt,
		after skip=12pt,
		borderline west={3pt}{0pt}{black},
		colback=RedViolet!5!gray!5,
		breakable,
		left=10pt,
		right=10pt
	}
}

\declaretheorem[style=thmtheorem,name=Theorem,numberwithin=section]{theorem}
\tcolorboxenvironment{theorem}{theorem box}
\declaretheorem[style=thmtheorem,name=Lemma,sibling=theorem]{lemma}
\tcolorboxenvironment{lemma}{theorem box}
\declaretheorem[style=thmtheorem,name=Proposition,sibling=theorem]{proposition}
\tcolorboxenvironment{proposition}{theorem box}
\declaretheorem[style=thmtheorem,name=Corollary,sibling=theorem]{corollary}
\tcolorboxenvironment{corollary}{theorem box}
\declaretheorem[style=thmexample,name=Example,sibling=theorem]{example}
\tcolorboxenvironment{example}{example box}

\declaretheoremstyle[
	headfont=\bfseries\sffamily\color{ForestGreen!70!black},
	bodyfont=\normalfont,
	headpunct={ --- },
]{thmremark}
\declaretheoremstyle[
	headfont=\bfseries\sffamily\color{ForestGreen!70!black},
	bodyfont=\normalfont,
	headpunct={},
]{thmremark*}

\declaretheoremstyle[
	headfont=\bfseries,
	bodyfont=\normalfont\small
]{thmques}

\declaretheorem[name=Question,sibling=theorem,style=thmques]{ques}
\tcolorboxenvironment{ques}{ques box}
\declaretheorem[name=Exercise,sibling=theorem,style=thmques]{exercise}
\tcolorboxenvironment{exercise}{ques box}
\declaretheorem[name=Remark,sibling=theorem,style=thmremark]{remark}
\tcolorboxenvironment{remark}{remark box}
\declaretheorem[name=Remark,sibling=theorem,style=thmremark*]{remark*}
\tcolorboxenvironment{remark*}{remark box}
\declaretheorem[name=Step,style=thmremark]{step} % only used in Lebesgue int
\tcolorboxenvironment{step}{remark box}

\theoremstyle{definition}
\newtheorem{claim}[theorem]{Claim}
\newtheorem{definition}[theorem]{Definition}
\newtheorem{fact}[theorem]{Fact}
\newtheorem{abuse}[theorem]{Abuse of Notation}

\newtheorem{problem}{Problem}[chapter]
\renewcommand{\theproblem}{\thechapter\Alph{problem}}
\newtheorem{sproblem}[problem]{Problem}
\newtheorem{dproblem}[problem]{Problem}
\renewcommand{\thesproblem}{\theproblem$^{\star}$}
\renewcommand{\thedproblem}{\theproblem$^{\dagger}$}
\newcommand{\listhack}{$\empty$\vspace{-2em}}

%%fakesection Answers
\usepackage{answers}
\Newassociation{hint}{answeritem}{tex/backmatter/all-hints}
\Newassociation{sol}{answeritem}{tex/backmatter/all-solns}
\renewcommand{\solutionextension}{out}
\renewenvironment{answeritem}[1]{\item[\bfseries #1.]}{}

%%fakesection Table of contents
% First add ToC to ToC
\makeatletter
\usepackage{etoolbox}
\pretocmd{\tableofcontents}{%
	\if@openright\cleardoublepage\else\clearpage\fi
	\pdfbookmark[0]{\contentsname}{toc}%
}{}{}%
\makeatother
\setcounter{tocdepth}{1}
\RedeclareSectionCommand[tocnumwidth=4.2em]{part}
\RedeclareSectionCommand[tocpagenumberwidth=2.2em,tocnumwidth=4.2em]{chapter}
\RedeclareSectionCommand[tocpagenumberwidth=2.2em,tocnumwidth=2.8em]{section}
% adjust tocpagenumberwidth manually for large page number: https://tex.stackexchange.com/a/502168

%%fakesection Asymptote definitions
\usepackage{patch-asy}
\numberwithin{asy}{chapter}
\renewcommand{\theasy}{\thechapter\Alph{asy}}
\begin{asydef}
	import extras;
	size(6cm);
	usepackage("amsmath");
	usepackage("amssymb");
	defaultpen(fontsize(11pt));
	settings.tex = "latex";
	settings.outformat = "pdf";
\end{asydef}
\def\asydir{asy}

%%fakesection Bibliography
\usepackage[backend=biber,backref=true,style=alphabetic]{biblatex}
\DeclareLabelalphaTemplate{
	\labelelement{
		\field[final]{shorthand}
		\field{label}
		\field[strwidth=2,strside=left]{labelname}
	}
	\labelelement{
		\field[strwidth=2,strside=right]{year}
	}
}
\DeclareFieldFormat{labelalpha}{\textbf{\scriptsize #1}}
\addbibresource{references.bib}
\addbibresource{images.bib}
%% stylistic biblatex choices
\DefineBibliographyStrings{english}{%
	backrefpage  = {cited p.}, % for single page number
	backrefpages = {cited pp.} % for multiple page numbers
}
\DeclareFieldFormat{journaltitle}{\mkbibemph{#1},} % italic journal title with comma
\DeclareFieldFormat[inbook,thesis]{title}{\mkbibemph{#1}\addperiod} % italic title with period
\DeclareFieldFormat[article]{title}{#1} % title of journal article is printed as normal text
\DeclareFieldFormat[article]{volume}{\textbf{#1}\addcolon\space}
\renewcommand{\mkbibnamegiven}[1]{\textsc{#1}}
\renewcommand{\mkbibnamefamily}[1]{\textsc{#1}}
\renewcommand{\mkbibnameprefix}[1]{\textsc{#1}}
\renewcommand{\mkbibnamesuffix}[1]{\textsc{#1}}
\renewcommand{\finentrypunct}{}

%%fakesection Mini ToC
\usepackage[tight]{minitoc}
\mtcsetfont{parttoc}{chapter}{\sffamily\bfseries}
\mtcsetfont{parttoc}{section}{\footnotesize\rmfamily\upshape\mdseries}
\mtcsetfont{parttoc}{subsection}{\footnotesize\rmfamily\upshape\mdseries}
%\mtcsetdepth{parttoc}{1}
\setcounter{parttocdepth}{1}
\renewcommand*{\partheadstartvskip}{\vspace*{20em}}
\renewcommand*{\partheadendvskip}{}
%\noptcrule
\renewcommand\beforeparttoc{\noindent{\bfseries \Large Part \thepart: Contents}}
%\hspace{\fill}\rule{0.95\linewidth}{2pt}\hspace{\fill}
\doparttoc[n]

%%fakesection Misc haxx
\pdfstringdefDisableCommands{\def\Spec{\text{Spec }}\def\sigma{σ}}

\def\asydir{}
\addbibresource{../../references.bib}
\renewcommand{\gim}{($\ast$)}

\begin{document}
\title{Singular Homology}
\maketitle

\chapter{Some Topological Constructions}
In this short chapter we briefly describe some common spaces and constructions
in topology that we haven't yet discussed.

\section{Spheres}
Recall that
\[ S^n = \left\{ (x_0, \dots, x_n)
	\mid x_0^2 + \dots + x_n^2 = 1 \right\} \subset \RR^{n+1} \]
is the surface of an $n$-sphere while
\[ D^{n+1} = \left\{ (x_0, \dots, x_n)
	\mid x_0^2 + \dots + x_n^2 \le 1 \right\} \subset \RR^{n+1} \]
is the corresponding \emph{closed disk}
(So for example, $D^2$ is a disk in a plane while $S^1$ is the unit circle.)
\begin{exercise}
	Show that the open disk $D^n \setminus S^{n-1}$
	is homeomorphic to $\RR^n$.
\end{exercise}

In particular, $S^0$ consists of two points,
while $D^1$ can be thought of as the interval $[-1,1]$.

\begin{center}
	\begin{asy}
		size(8cm);
		draw(dir(0)--dir(180), blue);
		dot(dir(0), red+4);
		dot(dir(180), red+4);
		label("$S^0$", dir(0), dir(90), red);
		label("$D^1$", dir(0)--dir(180), blue);
		add(shift(-4,0)*CC());
		unitsize(2cm);
		filldraw(unitcircle, lightblue+opacity(0.2), red);
		label("$D^2$", origin, blue);
		label("$S^1$", dir(45), dir(45), red);
	\end{asy}
\end{center}


\section{Quotient Topology}
\prototype{$D^n / S^{n-1} = S^n$, or the torus.}

Given a space $X$, we can \emph{identify} some of the points together
by any equivalence relation $\sim$;
for an $x \in X$ we denote its equivalence class by $[x]$.
Geometrically, this is the space achieved by welding together points
equivalent under $\sim$.

Formally,
\begin{definition}
	Let $X$ be a topological space, and $\sim$ an equivalence relation
	and the points of $X$.
	Then $X / {\sim}$ is the space whose
	\begin{itemize}
		\ii Points are equivalence classes of $X$, and
		\ii $U \subseteq X / {\sim}$ is open if and only if
		$\left\{ x \in X \text{ such that } [x] \in U  \right\}$
		is open in $X$.
	\end{itemize}
\end{definition}
As far as I can tell, this definition is mostly useless for intuition,
so here are some examples.

\begin{example}[Interval Modulo Endpoints]
	Suppose we take $D^1 = [-1, 1]$
	and quotient by the equivalence relation which identifies
	the endpoints $-1$ and $1$.
	(Formally, $x \sim y \iff (x=y) \text{ or } \{x,y\} = \{-1,1\}$.)
	In that case, we simply recover $S^1$:
	\begin{center}
		\begin{asy}
			size(8cm);
			draw(dir(0)--dir(180), blue);
			dot("$-1$", dir(0), dir(90), red+4);
			dot("$-1$", dir(180), dir(90), red+4);
			label("$D^1$", dir(0)--dir(180), blue);
			add(shift(-4,0)*CC());
			unitsize(2cm);
			draw(unitcircle, blue);
			label("$S^1 \approx D^1 / {\sim}$", dir(45), dir(45), blue);
			dot("$-1 \sim 1$", dir(90), dir(90), red);
		\end{asy}
	\end{center}
	Observe that a small neighborhood around $-1 \sim 1$ in the quotient space
	corresponds to two half-intervals at $-1$ and $1$ in the original space $D^1$.
	This should convince you the definition we gave is the right one.
\end{example}

\begin{example}[More Quotient Spaces]
	Convince yourself each of the following is correct:
	\begin{itemize}
		\ii Generalizing the previous example, $D^n$ modulo its boundary $S^{n-1}$ is $S^n$.
		\ii Given a square $ABCD$, suppose we identify segments $AB$ and $DC$ together.
		Then we get a cylinder. (Think elementary school, when you would tape
		up pieces of paper together to get cylinders.)
		\ii In the previous example, if we also identify $BC$ and $DA$ together,
		then we get a torus. (Imagine taking our cylinder and putting the two
		circles at the end together.)
		\ii Let $X = \RR$, and let $x \sim y$ if $y -x \in \ZZ$.
		Then $X / {\sim}$ is $S^1$ as well.
	\end{itemize}
\end{example}

One special case that we did above:
\begin{definition}
	Let $A \subseteq X$.
	Consider the equivalence relation which identifies
	all the points of $A$ with each other
	while leaving all remaining points inequivalent.
	(In other words, $x \sim y$ if $x=y$ or $x,y \in A$.)
	Then the resulting quotient space is denoted $X/A$.
\end{definition}

So in this notation, \[ D^n / S^{n-1} = S^n. \]

\begin{abuse}
	Note that I'm deliberately being sloppy, and saying
	``$D^n / S^{n-1} = S^n$'' or ``$D^n / S^{n-1}$ \emph{is} $S^n$'',
	when I really ought to say ``$D^n / S^{n-1}$ is homeomorphic to $S^n$''.
	This is a general theme in mathematics:
	objects which are homoeomorphic/isomorphic/etc.\ are generally
	not carefully distinguished from each other.
\end{abuse}

\section{Product Topology}
\prototype{$\RR \times \RR$ is $\RR^2$, $S^1 \times S^1$ is the torus.}

\begin{definition}
	Given topological spaces $X$ and $Y$,
	the product space $X \times Y$ is the space whose
	\begin{itemize}
		\ii Points are pairs $(x,y)$ with $x \in X$, $y \in Y$, and
		\ii Topology is given as follows: the \emph{basis} of
		the topology for $X \times Y$ is $U \times V$,
		for $U \subseteq X$ open and $V \subseteq Y$ open.
	\end{itemize}
\end{definition}
\begin{remark}
	It is not hard to show that, in fact,
	one need only consider basis elements for $U$ and $V$.
	That is to say,
	\[ \left\{ U \times V \mid
		U,V \text{ basis elements for } X,Y \right\} \]
	is also a basis for $X \times Y$.
\end{remark}

This does exactly what you think it would.
\begin{example}[The Unit Square]
	Let $X = [0,1]$ and consider $X \times X$.
	We of course expect this to be the unit square.
	Pictured below is an open set of $X \times X$ in the basis.
	\begin{center}
		\begin{asy}
		size(6cm);
		filldraw(unitsquare, opacity(0.2)+lightblue, black);

		pair B = (0,1);
		pair A = (1,0);
		fill(box(0.3*A+0.2*B,0.6*A+0.7*B), lightred+opacity(0.5));
		label("$U \times V$", (0.45,0.45), brown);

		draw(0.3*A--(0.3*A+B), heavygreen+dashed+1);
		draw(0.6*A--(0.6*A+B), heavygreen+dashed+1);
		draw(0.2*B--(0.2*B+A), heavycyan+dashed+1);
		draw(0.7*B--(0.7*B+A), heavycyan+dashed+1);

		draw( 0.3*A--0.6*A, heavygreen+2 );
		opendot( 0.3*A,  heavygreen+2);
		opendot( 0.6*A, heavygreen+2);
		label("$U$", 0.45*A, dir(-90), heavygreen);
		draw( 0.2*B--0.7*B, heavycyan+2 );
		opendot( 0.2*B, heavycyan+2);
		opendot( 0.7*B, heavycyan+2);
		label("$V$", 0.45*B, dir(180), heavycyan);
		\end{asy}
	\end{center}
\end{example}
\begin{exercise}
	Convince yourself this basis gives the same topology
	as the product metric on $X \times X$.
	So this is the ``right'' definition.
\end{exercise}

\begin{example}[More Product Spaces]
	\listhack
	\begin{enumerate}[(a)]
		\ii $\RR \times \RR$ is the Euclidean plane.
		\ii $S^1 \times [0,1]$ is a cylinder.
		\ii $S^1 \times S^1$ is a torus! (Why?)
	\end{enumerate}
\end{example}

\section{Disjoint Union and Wedge Sum}
\prototype{$S^1 \wedge S^1$ is the figure eight.}

The disjoint union of two spaces is geometrically exactly
what it sounds like: you just imagine the two spaces side by side.
For completeness, here is the formal definition.
\begin{definition}
	Let $X$ and $Y$ be two topological spaces.
	The \vocab{disjoint union}, denoted $X \amalg Y$, is defined by
	\begin{itemize}
		\ii The points are the disjoint union $X \sqcup Y$, and
		\ii A subset $U \subseteq X \sqcup Y$ is open if
		and only if $U \cap X$ and $U \cap Y$ are open.
	\end{itemize}
\end{definition}
\begin{exercise}
	Show that the disjoint union of two nonempty spaces is disconnected.
\end{exercise}

More interesting is the wedge sum, where two topological spaces $X$
and $Y$ are fused together only at a single base point.
\begin{definition}
	Let $X$ and $Y$ be topological spaces, and $x_0 \in X$ and $y_0 \in Y$
	be points.
	We define the equivalence relation $\sim$ by declaring $x_0 \sim y_0$ only.
	Then the \vocab{wedge sum} of two spaces is defined as
	\[ X \vee Y = (X \amalg Y) / {\sim}. \]
\end{definition}

\begin{example}
	[$S^1 \vee S^1$ is a Figure Eight]
	Let $X = S^1$ and $Y = S^1$,
	and let $x_0 \in X$ and $y_0 \in Y$ be any points.
	Then $X \vee Y$ is a ``figure eight'': it is two
	circles fused together at one point.
	\begin{center}
		\begin{asy}
			size(3cm);
			draw(shift(-1,0)*unitcircle);
			draw(shift(1,0)*unitcircle);
			dotfactor *= 1.4;
			dot(origin);
		\end{asy}
	\end{center}
\end{example}
\begin{abuse}
	We often don't mention $x_0$ and $y_0$ when they are understood
	(or irrelevant).  For example, from now on we will just
	write $S^1 \vee S^1$ for a figure eight.
\end{abuse}

\begin{remark}
	Annoyingly, in \LaTeX\ \verb+\wedge+ gives $\wedge$ instead
	of $\vee$ (which is \verb+\vee+).
	So this really should be called the ``vee product'', but too late.
\end{remark}


\section{CW Complexes}
Using this construction, we can start building some spaces.
One common way to do so is using a so-called \vocab{CW complex}.
Intuitively, a CW complex is built as follows:
\begin{itemize}
	\ii Start with a set of points $X^0$.
	\ii Define $X^1$ by taking some line segments (copies of $D^1$)
	and fusing the endpoints (copies of $S^0$) onto $X^0$.
	\ii Define $X^2$ by taking copies of $D^2$ (a disk)
	and welding its boundary (a copy of $S^1$) onto $X^1$.
	\ii Repeat inductively up until a finite stage $n$.
\end{itemize}
The resulting space $X$ is the CW-complex.
The set $X^n$ is called the \vocab{$n$-skeleton} of $X$.
Each $D^k$ is called a \vocab{$k$-cell}; it is customary to
denote it by $e_i^k$ where $i$ is some index.
\begin{abuse}
	Technically, most sources (like \cite{ref:hatcher}) allow one to
	repeat the process infinitely, giving infinite-dimensional spaces.
	We will not encounter any such spaces in the Napkin.
\end{abuse}

\begin{example}
	[$D^2$ with $2+2+1$ and $1+1+1$ Cells]
	\listhack
	\begin{enumerate}[(a)]
	\ii First, we start with $X^0$ having two points $e_a^0$ and $e_b^0$.
	Then, we join them with two $1$-cells $D^1$ (green),
	call them $e_c^1$ and $e_d^1$.
	The endpoints of each $1$-cell (the copy of $S^0$) get identified
	with distinct points of $X^0$; hence $X^1 \cong S^1$.
	Finally, we take a single $2$-cell $e^2$ (yellow) and weld it in,
	with its boundary fitting into the copy of $S^1$ that we just drew.
	This gives the figure on the left.

	\ii In fact, one can do this using just $1+1+1=3$ cells.
	Start with $X^0$ having a single point $e^0$.
	Then, use a single $1$-cell $e^1$, fusing its two endpoints
	into the single point of $X^0$.
	Then, one can fit in a copy of $S^1$ as before,
	giving $D^2$ as on the right.
	\end{enumerate}
	\begin{center}
		\begin{asy}
			size(4cm);
			filldraw(unitcircle, opacity(0.2)+yellow, heavygreen);
			dotfactor *= 1.4;
			dot(dir(90), blue);
			dot(dir(-90), blue);
			label("$e_a^0$", dir(90), dir(90), blue);
			label("$e_b^0$", dir(-90), dir(-90), blue);
			label("$e_c^1$", dir(0), dir(0), heavygreen);
			label("$e_d^1$", dir(180), dir(180), heavygreen);
			label("$e^2$", origin, origin);
		\end{asy}
		\qquad
		\begin{asy}
			size(4cm);
			filldraw(unitcircle, opacity(0.2)+yellow, heavygreen);
			dotfactor *= 1.4;
			dot(dir(90), blue);
			label("$e^0$", dir(90), dir(90), blue);
			label("$e^1$", dir(-90), dir(-90), heavygreen);
			label("$e^2$", origin, origin);
		\end{asy}
	\end{center}
\end{example}

\begin{example}
	[$S^n$ as a CW Complex]
	\listhack
	\begin{enumerate}[(a)]
		\ii One can obtain $S^n$ (for $n \ge 1$) with just two cells.
		Namely, take a single point $e^0$ for $X^0$, and to obtain $S^n$
		take $D^n$ and weld its entire boundary into $e^0$.

		We already saw this example in the beginning with $n=2$,
		when we saw that the sphere $S^2$ was the result when we fuse
		the boundary of a disk $D^2$ together.
		
		\ii Alternatively, one can do a ``hemisphere'' construction,
		by construction $S^n$ inductively using two cells in each dimension.
		So $S^0$ consists of two points, then $S^1$ is obtained
		by joining these two points by two segments ($1$-cells),
		and $S^2$ is obtained by gluing two hemispheres (each a $2$-cell)
		with $S^1$ as its equator.
	\end{enumerate}
\end{example}

Anyone who wants the formal definition of a CW complex can refer
to \cite[page 5]{ref:hatcher}.

\section{The Torus, Klein Bottle, $\RP^n$, $\CP^n$}
We now present four of the most import examples of CW complexes.

\subsection*{The Torus}
The \vocab{torus} can be formed by taking
a square and identifying the opposite edges in the same direction:
if you walk off the right edge, you re-appear at the corresponding
point in on the left edge.
(Think \emph{Asteroids} from Atari!)

\begin{center}
	\begin{asy}
		size(2cm);
		fill(unitsquare, yellow+opacity(0.2));
		pair C = (0,0);
		pair B = (1,0);
		pair A = (1,1);
		pair D = (0,1);
		draw(A--B, red, MidArrow);
		draw(B--C, blue, MidArrow);
		draw(D--C, red, MidArrow);
		draw(A--D, blue, MidArrow);
	\end{asy}
\end{center}

Thus the torus is $(\RR/\ZZ)^2 \cong S^1 \times S^1$.

Note that all four corners get identified together to a single point.  One
can realize the torus in $3$-space by treating the square as a sheet of paper,
taping together the left and right (red) edges to form a cylinder,
then bending the cylinder and fusing the top and bottom (blue) edges
to form the torus.
\begin{center}
	\includegraphics[width=0.8\textwidth]{Projection_color_torus.jpg} \\ \tiny
	\url{https://en.wikipedia.org/wiki/File:Projection_color_torus.jpg}
	(public domain)
\end{center}

The torus can be realized as a CW complex with
\begin{itemize}
	\ii A $0$-skeleton consisting of a single point,
	\ii A $1$-skeleton consisting of two $1$-cells $e^1_a$, $e^1_b$, and
	\begin{center}
		\begin{asy}
			unitsize(1cm);
			draw(shift(-1,0)*unitcircle, blue, MidArrow);
			draw(shift(1,0)*rotate(180)*unitcircle, red, MidArrow);
			label("$e^1_a$", 2*dir(180), dir(180), blue);
			label("$e^1_b$", 2*dir(0), dir(0), red);
			dotfactor *= 1.4;
			dot("$e^0$", origin, dir(0));
		\end{asy}
	\end{center}
	\ii A $2$-skeleton with a single $2$-cell $e^2$,
	whose circumference is divided into four parts,
	and welded onto the $1$-skeleton ``via $aba\inv b \inv$''.
	This means: wrap a quarter of the circumference around $e^1_a$,
	then another quarter around $e^1_b$,
	then the third quarter around $e^1_a$ but in the opposite direction,
	and the fourth quarter around $e^1_b$ again in the opposite direction as before.
	\begin{center}
		\begin{asy}
			size(3cm);
			fill(unitcircle, yellow+opacity(0.2));
			defaultpen(linewidth(1));
			draw(arc(origin, 1, 45, 135), blue, MidArrow);
			draw(arc(origin, 1, 315, 225), blue, MidArrow);
			draw(arc(origin, 1, 135, 225), red, MidArrow);
			draw(arc(origin, 1, 45, -45), red, MidArrow);
			label("$e^2$", origin, origin);
		\end{asy}
	\end{center}
\end{itemize}

\subsection*{The Klein Bottle}
The \vocab{Klein bottle} is defined similarly to
the torus, except one pair of edges is identified in the opposite manner,
as shown.

\begin{center}
	\begin{asy}
		size(2cm);
		fill(unitsquare, yellow+opacity(0.2));
		pair C = (0,0);
		pair B = (1,0);
		pair A = (1,1);
		pair D = (0,1);
		draw(A--B, red, MidArrow);
		draw(C--B, blue, MidArrow);
		draw(D--C, red, MidArrow);
		draw(A--D, blue, MidArrow);
	\end{asy}
\end{center}

Unlike the torus one cannot realize this in $3$-space
without self-intersecting. One can tape together the red edges
as before to get a cylinder, but to then fuse the resulting blue
circles in opposite directions is not possible in 3D.
Nevertheless, we often draw a picture in 3-dimensional space
in which we tacitly allow the cylinder to intersect itself.

\begin{center}
	\begin{minipage}[c]{0.5\textwidth}
	\includegraphics[width=\textwidth]{klein-fold.png}
	\end{minipage}
	\quad
	\begin{minipage}[c]{0.3\textwidth}
	\includegraphics[width=\textwidth]{KleinBottle-01.png}
	\end{minipage}

	\tiny
	\url{https://commons.wikimedia.org/wiki/File:Klein_Bottle_Folding_1.svg} \\
	\url{https://en.wikipedia.org/wiki/File:KleinBottle-01.png} \\
	(public domain)
\end{center}


Like the torus, the Klein bottle is realized as a CW complex with
\begin{itemize}
	\ii One $0$-cell,
	\ii Two $1$-cells $e^1_a$ and $e^1_b$, and
	\ii A single $2$-cell attached this time ``via $abab\inv$''.
\end{itemize}

\subsection*{Real Projective Space}
Let's start with $n=2$.
The space $\RP^2$ is obtained if we reverse both directions of
the square from before, as shown.

\begin{center}
	\begin{asy}
		size(2cm);
		fill(unitsquare, yellow+opacity(0.2));
		pair C = (0,0);
		pair B = (1,0);
		pair A = (1,1);
		pair D = (0,1);
		draw(B--A, red, MidArrow);
		draw(C--B, blue, MidArrow);
		draw(D--C, red, MidArrow);
		draw(A--D, blue, MidArrow);
	\end{asy}
\end{center}

However, once we do this the fact that the original
polygon is a square is kind of irrelevant;
we can combine a red and blue edge to get the single purple edge.
Equivalently, one can think of this as a circle with half
its circumference identified with the other half:

\begin{center}
	\begin{asy}
		size(3cm);
		dotfactor *= 2;
		fill(unitcircle, opacity(0.2)+yellow);
		draw(dir(-90)..dir(0)..dir(90), purple, MidArrow);
		draw(dir(90)..dir(180)..dir(-90), purple, MidArrow);
		dot(dir(90));
		dot(dir(-90));
		label("$\mathbb{RP}^2$", origin, origin);
	\end{asy}
	\qquad
	\begin{asy}
		size(3cm);
		dotfactor *= 2;
		draw(dir(-90)..dir(0)..dir(90));
		draw(dir(90)..dir(180)..dir(-90), dashed);
		fill(unitcircle, yellow+opacity(0.2));
		dot(dir(90));
		opendot(dir(-90));
		label("$\mathbb{RP}^2$", origin, origin);
	\end{asy}
\end{center}

The resulting space should be familiar to those of you who do
projective (Euclidean) geometry.
Indeed, there are several possible geometric interpretations:
\begin{itemize}
	\ii One can think of $\RP^2$ as the set of lines through the
	origin in $\RR^3$, with each line being a point in $\RP^2$.

	Of course, we can identify each line with a point on the unit sphere $S^2$,
	except for the property that two antipodal points actually 
	correspond to the same line, so that $\RP^2$ can be almost thought
	of as ``half a sphere''. Flattening it gives the picture above.

	\ii Imagine $\RR^2$, except augmented with ``points at infinity''.
	This means that we add some lines ``infinitely far away'',
	one for each possible direction of a line.
	Thus in $\RP^2$, any two lines indeed intersect
	(at a Euclidean point if they are not parallel, and at a line
	at infinity if they do).

	This gives an interpretation of $\RP^2$,
	where the boundary represents the \emph{line at infinity}
	through all of the points at infinity.
	Here we have used the fact that $\RR^2$
	and interior of $D^2$ are homeomorphic.
\end{itemize}
\begin{exercise}
	Observe that these formulations are equivalent
	by consider the plane $z=1$ in $\RR^3$,
	and intersecting each line in the first formulation with this plane.
\end{exercise}

We can also express $\RP^2$ using coordinates:
it is the set of triples $(x : y : z)$ of real numbers not all zero
up to scaling, meaning that 
\[ (x : y : z) = (\lambda x : \lambda y : \lambda z) \]
for any $\lambda \neq 0$.
Using the ``lines through the origin in $\RR^3$'' interpretation
makes it clear why this coordinate system gives the right space.
The points at infinity are those with $z = 0$,
and any point with $z \neq 0$ gives a Cartesian point since
\[ (x : y : z) = \left( \frac xz : \frac yz : 1 \right) \]
hence we can think of it as the Cartesian point $(\frac xz, \frac yz)$.

In this way we can actually define \vocab{real-projective $n$-space},
$\RP^n$ for any $n$, as either
\begin{enumerate}[(i)]
	\ii The set of lines through the origin in $\RR^{n+1}$,
	\ii Using $n+1$ coordinates as above, or
	\ii As $\RR^n$ augmented with points at infinity,
	which themselves form a copy of $\RP^{n-1}$.
\end{enumerate}

As a possibly helpful example, we draw all three pictures of $\RP^1$.
\begin{example}
	[Real Projective $1$-Space]
	$\RP^1$ can be thought of as $S^1$ modulo the relation
	the antipodal points are identified.
	Projecting onto a tangent line, we see that we get
	a copy of $\RR$ plus a single point at infinity, corresponding
	to the parallel line (drawn in cyan below).
	\begin{center}
		\begin{asy}
			size(8cm);
			filldraw(unitcircle, lightblue+opacity(0.2), heavyblue+opacity(0.4));
			label("$S^1$", dir(225), dir(225), lightblue);
			dot("$\vec 0$", origin, dir(45));
			pair X1 = (-2.1,1);
			pair X2 = (1.9,1);
			draw(X1--X2, heavyred, Arrows);
			dot("$0$", (0,1), dir(90), heavyred);
			dot("$1$", (1,1), dir(90), heavyred);
			pair P = extension( (0,1), (1,1), dir(250), dir(70) );
			dot("$0.36$", P, dir(90), heavyred);
			label("$\mathbb R$", X2, dir(105), heavyred);
			draw(L(dir(130),-dir(130),0.2,0.2), gray);
			draw(L(dir(250),-dir(250),0.2,0.2), gray);
			draw(L(dir(-20),-dir(-20),0.2,0.2), gray);
			draw(L(dir(0), -dir(0), 0.4,0.4), heavycyan+1);
		\end{asy}
	\end{center}
	Thus, the points of $\RP^1$ have two forms:
	\begin{itemize}
		\ii $(x:1)$, which we think of as $x \in \RR$ (in dark red above), and
		\ii $(1:0)$, which we think of as $1/0 = \infty$,
		corresponding to the cyan line above.
	\end{itemize}
	So, we can literally write
	\[ \RP^1 = \RR \cup \{\infty\}. \]
	Note that $\RP^1$ is also the boundary of $\RP^2$.
	In fact, note also that topologically we have
	\[ \RP^1 \cong S^1 \]
	since it is the ``real line with endpoints fused together''.
	\begin{center}
		\begin{asy}
			size(2cm);
			draw(unitcircle, heavyred);
			dot("$\infty$", dir(90), dir(90), heavycyan);
			dot("$0$", dir(-90), dir(-90), heavyred);
		\end{asy}
	\end{center}
\end{example}

Since $\RP^n$ is just ``$\RR^n$ (or $D^n$) with $\RP^{n-1}$ as its boundary'',
we can construct $\RP^n$ as a CW complex inductively,
with exactly one cell in each dimension.
Thus
\begin{moral}
	$\RP^n$ consists of one cell in each dimension.
\end{moral}
For example, in low dimensions:
\begin{example}[$\RP^n$ as a Cell Complex]
	\listhack
	\begin{enumerate}[(a)]
		\ii $\RP^0$ is a single point.
		\ii $\RP^1 \cong S^1$ is a circle, which as a CW complex
		is a $0$-cell plus a $1$-cell.
		\ii $\RP^2$ can be formed by taking a $2$-cell
		and wrapping its perimeter twice around a copy of $\RP^1$.
	\end{enumerate}
\end{example}

\subsection*{Complex Projective Space}
The \vocab{complex projective space} $\CP^n$ is
defined like $\RP^n$ with coordinates, via\
\[ (z_0 : z_1 : \dots : z_n) \]
under scaling.
The difference is that, as the name suggests, the coordinates
are complex numbers rather than real numbers.

As before, $\CP^n$ can be thought of as $\CC^n$ augmented
with some points at infinity (corresponding to $\CP^{n-1}$).
\begin{example}
	[Complex Projective Space]
	\listhack
	\begin{enumerate}[(a)]
		\ii $\CP^0$ is a single point.
		\ii $\CP^1$ is $\CC$ plus a single point at infinity
		(``complex infinity'' if you will).
		That means as before we can think of $\CP^1$ as
		\[ \CP^1 = \CC \cup \{\infty\}. \]
		So, imagine taking the complex plane and then adding
		a single point to encompass the entire boundary.
		The result is just sphere $S^2$.
	\end{enumerate}
	Here is a picture of $\CP^1$ with its coordinate system,
	idealized as the \vocab{Riemann sphere}.
	\begin{center}
		\includegraphics[width=0.9\textwidth]{earth.pdf}
	\end{center}
\end{example}

\begin{remark}
	[For Euclidean Geometers]
	You may recognize that while $\RP^2$ is the setting for projective geometry,
	inversion about a circle is done in $\CP^1$ instead.
	When one does an inversion sending generalized circles to generalized
	circles, there is only one point at infinity:
	this is why we work in $\CP^n$.
\end{remark}

Like $\RP^n$, $\CP^n$ is a CW complex, built inductively
by taking $\CC^n$ and welding its boundary onto $\CP^{n-1}$
The difference is that as topological spaces,
\[ \CC^n \cong \RR^{2n} \cong D^{2n}. \]
Thus, we attach the cells $D^0$, $D^2$, $D^4$ and so on
inductively to construct $\CP^n$.
Thus we see that
\begin{moral}
	$\CP^n$ consists of one cell in each \emph{even} dimension.
\end{moral}

\end{document}

\chapter{More on Computing Homology Groups}
\section{Excision}
\todo{relative homology groups, and excision}
\todo{wedge products}
\todo{invariance of dimension}


\section{Degree}
\prototype{$z \mapsto z^d$ has degree $d$.}
For any $n > 0$ map $f$ from $S^n$ to itself
\[ f_\ast : \underbrace{H^n(S^n)}_{\cong \ZZ} \to \underbrace{H^n(S^n)}_{\cong \ZZ} \]
which must be multiplication by some constant $d$.
This $d$ is called the \vocab{degree} of $f$, denoted $\deg f$.
\begin{ques}
	Show that $\deg(f \circ g) = \deg(f) \deg(g)$.
\end{ques}

\begin{example}
	[Degree]
	\listhack
	\begin{enumerate}[(a)]
		\ii For $n=1$, the map $z \mapsto z^k$ (viewing $S^1 \subseteq \CC$)
		has degree $k$.
		\ii A reflection map $(x_0, x_1, \dots, x_n) \mapsto (-x_0, x_1, \dots, x_n)$
		has degree $-1$; we won't prove this, but geometrically this should be clear.
		\ii The antipodal map $x \mapsto -x$ has degree $(-1)^{n+1}$
		since it's the composition of $n+1$ reflections as above.
		We denote this map by $-\id$.
	\end{enumerate}
\end{example}

Obviously, if $f$ and $g$ are homotopic, then $\deg f = \deg g$.
In fact, a theorem of Hopf says that this is a classifying invariant:
anytime $\deg f = \deg g$, we have that $f$ and $g$ are homotopic.

One nice application of this:
\begin{theorem}
	[Hairy Ball Theorem]
	Whenever $n$ is even, $S^n$ doesn't have a continuous field
	of nonzero tangent vectors.
\end{theorem}
\begin{proof}
	If the vectors are nonzero then WLOG they have norm $1$;
	that is for every $x$ we have an orthogonal unit vector $v(x)$.
	Then we can construct a homotopy map $F : S^n \times [0,1] \to S^n$ by
	\[ (x,t) \mapsto (\cos \pi t)x + (\sin \pi t) v(x). \]
	which gives a homotopy from $\id$ to $-\id$.
	So $\deg \id = \deg(-\id)$, which means $1 = (-1)^{n+1}$.
\end{proof}

\section{Cellular Homology}
Before starting, we state the following lemma.
\begin{lemma}
	[CW Homology Groups]
	Let $X$ be a CW complex. Then
	\begin{align*}
		H_k(X^n, X^{n-1}) &\cong
		\begin{cases}
			\ZZ^{\oplus\text{\#$n$-cells of $X$}} & k = n \\
			0 & \text{otherwise}.
		\end{cases} \\
		\intertext{and}
		H_k(X^n) &\cong
		\begin{cases}
			H_k(X) & k \le n-1 \\
			0 & k \ge n+1.
		\end{cases}
	\end{align*}
\end{lemma}
\begin{proof}
	I'll prove just the case where $X$ is finite-dimensional for simplicity.
	The first part is immediate by noting that $(X^n, X^{n-1})$ is a good pair
	and $X^n/X^{n-1}$ is a wedge sum of two spheres.
	For the second part, fix $k$ and note that, as long as $n \le k-1$ or $n \ge k+2$,
	\[
		\underbrace{H_{k+1}(X^n, X^{n-1})}_{=0}
		\to H_k(X^{n-1})
		\to H_k(X^n)
		\to \underbrace{H_{k}(X^n, X^{n-1})}_{=0}.
	\]
	So we have isomorphisms
	\[ H_k(X^{k-1}) \cong H_k(X^{k-2}) \cong \dots \cong H_k(X^0) = 0 \]
	and
	\[ H_k(X^{k+1}) \cong H_k(X^{k+2}) \cong \dots \cong H_k(X). \qedhere \]
\end{proof}

So, we know that the groups $H_k(X^n, X^{n-1})$ are super nice:
they are free abelian with basis given by the cells of $X$.

In light of this, let's look at our long exact sequence and try to
string together maps between these.
Consider the following diagram.

\begin{diagram}
	\small
	&& \underbrace{H_3(X^2)}_{=0} &&&&&&&& \\
	&& \dTo~0 &&&&&&&& \\
	\boxed{H_4(X^4, X^3)} & \rTo^{\partial_4} & H_3(X^3) & \rSurj
		& \underbrace{H_3(X^4)}_{\cong H_3(X)} &
		\rTo~0 & \underbrace{H_3(X^4, X^3)}_{= 0} &&&& \\
	& \rdDotted~{d_4} & \dTo~0 &&&&&&&& \\
	&& \boxed{H_3(X^3, X^2)} &&&& \underbrace{H_1(X^0)}_{=0} &&&& \\
	&& \dTo^{\partial_3} & \rdDotted~{d_3} &&& \dTo~0 &&&& \\
	\underbrace{H_2(X^1)}_{=0} & \rTo~0 & H_2(X^2) & \rInj &
		\boxed{H_2(X^2, X^1)} & \rTo^{\partial_2} & H_1(X^1) & \rSurj &
		\underbrace{H_1(X^2)}_{\cong H_1(X)} & \rTo~0 &
		\underbrace{H_1(X^2, X^1)}_{=0} \\
	&& \dSurj &&& \rdDotted~{d_2} & \dInj &&&& \\
	&& \underbrace{H_2(X^3)}_{\cong H_2(X)} &&&& \boxed{H_1(X^1, X^0)} &&&& \\
	&& \dTo~0 &&&& \dTo^{\partial_1} & \rdDotted~{d_1} &&& \\
	&& \underbrace{H_2(X^3, X^2)}_{=0} && \underbrace{H_0(\varnothing)}_{=0}
		& \rTo~0 & H_0(X^0) & \rInj & \boxed{H_0(X^0, \varnothing)} & \rTo^{\partial_0} & \dots \\
	&&&&&& \dSurj &&&& \\
	&&&&&& \underbrace{H_0(X^1)}_{\cong H_0(X)} &&&& \\
	&&&&&& \dTo~0 &&&& \\
	&&&&&& \underbrace{H_0(X^1, X^0)}_{=0} &&&& \\
\end{diagram}
Here we have $X^{-1} = \varnothing$.
The idea is that we have taken all the exact sequences generated by adjacent
skeletons, and strung them together at the groups $H_k(X^k)$,
with half the exact sequences being laid out vertically
and the other half horizontally.

In that case, composition generates a sequence of dotted maps
between the $H_k(X^k, X^{k-1})$ as shown.
\begin{ques}
	Show that the composition of two adjacent dotted arrows is zero.
\end{ques}

So the sequence of arrows $\dots \to H_4(X^4, X^3) \to H_3(X^3, X^2) \to \dots$
that we obtain is a chain complex, called the \vocab{cellular chain complex};
as mentioned before before all the homology groups are free,
but these ones are especially nice because for most reasonable CW complexes,
they are also finitely generated
(unlike the massive $C_\bullet(X)$ that we had earlier).
In other words, the $H_k(X^k, X^{k-1})$ are especially nice ``concrete'' free groups
that one can actually work with.

The other reason we care is that in fact:
\begin{theorem}[Cellular Chain Complex Gives $H_n(X)$]
	\label{thm:cellular_chase}
	The $k$th homology group of the cellular chain complex
	is isomorphic to $H_k(X)$.
\end{theorem}
\begin{proof}
	Follows from the diagram; \Cref{prob:diagram_chase}.
\end{proof}

In fact, one can describe explicitly what the maps $d_n$ are.
Recalling that $H_k(X^k, X^{k-1})$ has a basis the $k$-cells of $X$, we
can write the following formula.
\begin{theorem}
	[Cellular Boundary Formula]
	Let $e$ be a $k$-cell, and let $f_\beta$ denote all $(k-1)$-cells of $X$.
	Define $d_\beta$ to be the degree of the composed map
	\[ S^{n-1} = \partial D_\beta^n \xrightarrow{\text{attach}}
		X^{n-1} \surjto S_\beta^{n-1} \]
	where the first arrow is the attaching map for $e$
	and the second arrow is the quotient of collapsing $X^{n-1} - f_\beta$ to a point.
	Then \[ d_n : e \mapsto \sum_\beta d_\beta f_\beta \]
\end{theorem}

CP

Torus, Klein, RP2

\section\problemhead
\begin{problem}
	Show that a non-surjective map $f : S^n \to S^n$ has degree zero.
	\begin{hint}
		The space $S^n - \{x_0\}$ is contractible.
	\end{hint}
\end{problem}

\begin{problem}
	\label{prob:diagram_chase}
	Prove \Cref{thm:cellular_chase},
	showing that the homology groups of $X$
	coincide with the homology groups of the cellular chain complex.
	\begin{hint}
		You won't need to refer to any elements.
		Start with $H_2(X) \cong H_2(X^3) \cong
			H_2(X^2) / \ker \left[ H_2(X^2) \surjto H_2(X^3) \right]$, say.
		Take note of the marked injective and surjective arrows.
	\end{hint}
	\begin{sol}
		For concreteness, let's just look at the homology at $H_2(X^2, X^1)$
		and show it's isomorphic to $H_2(X)$.
		According to the diagram
		\begin{align*}
			H_2(X) &\cong H_2(X^3) \\
			&\cong H_2(X^2) / \ker \left[ H_2(X^2) \surjto H_2(X^3) \right] \\
			&\cong H_2(X^2) / \img \partial_3 \\
			&\cong \img\left[ H_2(X^2) \injto H_2(X^2, X^1) \right] / \img \partial_3 \\
			&\cong \ker(\partial_2) / \img\partial_3 \\
			&\cong \ker d_2 / \img d_3. \qedhere
		\end{align*}
	\end{sol}
\end{problem}

\begin{problem}[Moore Spaces]
	Let $G_1$, $G_2$, \dots, $G_N$ be a sequence of finitely generated abelian groups.
	Construct a space $X$ such that
	\[
		\wt H_n(X)
		\cong
		\begin{cases}
			G_n & 1 \le n \le N \\
			0 & \text{otherwise}.
		\end{cases}
	\]
\end{problem}

\chapter{Singular Cohomology}
\section{Motivation}
Here's the main way to motivate this chapter.
Let $X = \CP^2$ and $Y = S^2 \wedge S^4$.
\begin{exercise}
	Show that $H_n(X) \cong H_n(Y)$ for every $n$.
\end{exercise}
However, as it turns out $X$ and $Y$ are not homotopy equivalent --- so how can we tell them apart?

In this chapter, we'll define a \emph{cohomology group} $H^n(X)$ and $H^n(Y)$.
We will still have $H^n(X) \cong H^n(Y)$ for every $X$ and $Y$:
in fact, the $H^n$'s are completely determined by the $H_n$'s
by the so-called \emph{Universal Coefficient Theorem}
However, it turns out that one can take all the cohomology groups and put
them together to form a \emph{cohomology ring} $H^\bullet$.
We will then see that $H^\bullet(X) \not\cong H^\bullet(Y)$ as rings.

\section{Cochain Complexes}
\begin{definition}
A \vocab{cochain complex} $A^\bullet$ is algebraically the same as a chain complex:
it is a sequence of abelian groups
\[ \dots \taking{\delta} A^{n-1} \taking\delta A^n \taking\delta A^{n+1} \taking\delta \dots. \]
such that $\delta^2 = 0$.
Notation-wise, we're now using subscripts, and use $\delta$ rather $\partial$.
We can still define the homology groups by
\[ H_n(A^\bullet) = \ker\left( A^n \taking\delta A^{n+1} \right)
	/ \img\left( A^{n-1} \taking\delta A^n \right). \]
\end{definition}

\textbf{The most common way to get a cochain complex
is to \emph{dualize} a chain complex.}
Specifically, pick an abelian group $G$;
note that $\Hom(-, G)$ is a contravariant functor,
and thus takes every chain complex
\[ \dots \taking\partial A_{n+1} \taking\partial
	A_n \taking\partial A_{n-1} \taking\partial \dots \]
into a cochain complex: letting $A^n = \Hom(A_n, G)$ we obtain
\[ \dots \taking\delta A^{n-1} \taking\delta
	A^n \taking\delta A^{n+1} \taking\delta \dots. \]
where $\delta(A_n \taking{f} G) = A_{n+1} \taking\partial A \taking{f} G$.
\begin{definition}
	Given a chain complex $A_\bullet$ of abelian groups and a group $G$,
	we define the \vocab{cohomology groups} $H^n(A_\bullet; G)$ to be the
	homology groups of the dual $A^\bullet$ obtained by applying $\Hom(-,G)$.
\end{definition}
This is admittedly horrible naming:
the \emph{cohomology groups of a chain complex}
are the \emph{homology groups of the cochain complex}
dual to that chain complex.

\section{Cohomology of Spaces}
\prototype{$C^0(X;G)$ all functions $X \to G$ while $H^0(X)$ are those functions $X \to G$
constant on path components.}

The case of interest is our usual geometric situation, with $C_\bullet(X)$.
\begin{definition}
	For a space $X$ and abelian group $G$,
	we define $C^\bullet(X;G)$ to be the dual complex to $C_\bullet(X)$,
	called the \vocab{singular cochain complex} of $X$;
	its elements are called \vocab{cochains}.

	Then we define the \vocab{cohomology groups}
	of the space $X$ as 
	\[ H^n(X; G) \defeq H^n(C_\bullet(X); G) = H_n(C^\bullet(X;G)). \]
\end{definition}

\begin{example}
	[$C^0(X; G)$, $C^1(X; G)$, and $H^0(X;G)$]
	Of course the archetypal example is to dualize the singular chain complex
	of a space $X$. In this situation, elements of $C^n$ are called \emph{cochains}.
	\begin{itemize}
		\ii $C_0(X)$ is the free abelian group on $X$,
		so $C^0(X) = \Hom(C_0(X), G)$ is a function that
		takes every point of $X$ to an element of $G$.
		\ii $C_1(X)$ is the free abelian group on $1$-simplices in $X$.
		So $C^1(X)$ needs to take every $1$-simplex to an element of $G$.
	\end{itemize}
	Let's now try to understand $\delta : C^1(X) \to C^0(X)$.

	Now given a cochain $\phi \in C^0(X)$, i.e.\ a homomorphism $\phi : C^0(X) \to G$,
	what is $\delta\phi : C^1(X) \to G$?
	Answer: 
	\[ \delta\phi : [v_0, v_1] \mapsto \phi([v_0]) - \phi([v_1]). \]
	Hence, elements of $\ker(C^0 \taking\partial C^1)$ are those cochains
	that are \emph{constant on path-connected components}.
\end{example}
In particular, much like $H_0(X)$, we have \[ H^0(X) \cong G^{\oplus r} \]
if $X$ has $r$ path-connected components (where $r$ is finite\footnote{%
	Something funny happens if $X$ has \emph{infinitely} many path-connected components:
	say $X = \bigsqcup_\alpha X_\alpha$ over an infinite indexing set.
	In this case we have
	$H_0(X) = \bigoplus_\alpha G$ while $H^0(X) = \prod_\alpha G$.
	For homology we get a \emph{direct sum} while
	for cohomology we get a \emph{direct product}.

	These are actually different for infinite indexing sets.
	For general modules $\bigoplus_\alpha M_\alpha$ is \emph{defined} to only allow
	to have \emph{finitely many} zero terms.
	(This was never mentioned earlier in the Napkin,
	since I only ever defined $M \oplus N$ and extended it to finite direct sums.)
	No such restriction holds for $\prod_\alpha G_\alpha$ a product of groups.
	This corresponds to the fact that $C_0(X)$ is formal linear sums of $0$-chains
	(which, like all formal sums, are finite)
	from the path-connected components of $G$.
	But a cochain of $C^0(X)$ is a \emph{function}
	from each path-connected component of $X$ to $G$,
	where there is no restriction.
}).

To the best of my knowledge, the higher cohomology groups $H^n(X; G)$
(or even the cochain groups $C^n(X; G) = \Hom(C_n(X), G)$) are harder to describe concretely.

\begin{abuse}
	In this chapter the only cochain complexes
	we will consider are dual complexes as above.
	So, any time we write a chain complex $A^\bullet$ it is implicitly given
	by applying $\Hom(-,G)$ to $A_\bullet$.
\end{abuse}

\section{Cohomology of Spaces is Functorial}
We now check that the cohomology groups still exhibit the same nice functorial behavior.
First, let's categorize the previous results we had:

\begin{ques}
	Define $\catname{CoCmplx}$
	the category of cochain complexes.
\end{ques}

\begin{exercise}
	Interpret $\Hom(-,G)$ as a contravariant functor
	from \[ \Hom(-,G) : \catname{Cmplx}\op \to \catname{CoCmplx}. \]
	This means in particular that given a chain map $f : A_\bullet \to B_\bullet$,
	we naturally obtain a dual map $f^\vee : B^\bullet \to A^\bullet$.
\end{exercise}

\begin{ques}
	Interpret $H^n : \catname{CoCmplx} \to \catname{Grp}$ as a functor.
	Compose these to get a contravariant functor
	$H^n(-;G) : \catname{Cmplx}\op \to \catname{Grp}$.
\end{ques}

Then in exact analog to our result that $H_n : \catname{hTop} \to \catname{Grp}$ we have:
\begin{theorem}[$H^n (-;G): \catname{hTop}\op \to \catname{Grp}$]
	For every $n$, $H^n(-;G)$ is a contravariant functor
	from $\catname{hTop}\op$ to $\catname{Grp}$.
\end{theorem}
\begin{proof}
	The idea is to leverage the work we already did in constructing
	the prism operator earlier.
	First, we construct the entire sequence of functors
	from $\catname{Top}\op \to \catname{Grp}$:
	\begin{diagram}
		\catname{Top}\op & \rTo^{C_\bullet} & \catname{Cmplx}\op & \rTo^{\Hom(-;G)}
		& \catname{CoCmplx} & \rTo^{H^n} & \catname{Grp} \\
		X && C_\bullet(X) && C^\bullet(X;G) && H^n(X;G) \\
		\dTo^f &\rMapsto& \dTo^{f_\sharp} &\rMapsto&
		\uTo^{f^\sharp} &\rMapsto& \uTo^{f^\ast} \\
		Y && C_\bullet(Y) && C^\bullet(Y;G) && H^n(Y;G).
	\end{diagram}
	Here $f^\sharp = (f_\sharp)^\vee$, and $f^\ast$
	is the resulting induced map on homology groups of the cochain complex.

	So as before all we have to show is that $f \simeq g$,
	then $f^\ast = g^\ast$.
	Recall now that there is a prism operator such that
	$f_\sharp - g_\sharp = P \partial + \partial P$.
	If we apply the entire functor $\Hom(-;G)$ we get that
	$f^\sharp - g^\sharp = \delta P^\vee + P^\vee \delta$
	where $P^\vee : C^{n+1}(Y;G) \to C^n(X;G)$.
	So $f^\sharp$ and $g^\sharp$ are chain homotopic thus $f^\ast = g^\ast$.
\end{proof}


\section{Universal Coefficient Theorem}
We now wish to show that the cohomology groups are determined up to isomorphism
by the homology groups: given $H_n(A_\bullet)$, we can extract $H^n(A_\bullet; G)$.
This is achieved by the \emph{Universal Coefficient Theorem}.
\begin{theorem}
	[Universal Coefficient Theorem]
	Let $A_\bullet$ be a chain complex of \emph{free} abelian groups,
	and let $G$ be another abelian group.
	Then there is a natural short exact sequence
	\[
		0 \to \Ext(H_{n-1}(A_\bullet), G) \to H^n(A_\bullet; G)
		\taking{h} \Hom(H_n(A_\bullet), G) \to 0. \]
	In addition, this exact sequence is \emph{split}
	so in particular
	\[ H^n(C_\bullet; G) \cong \Ext(H_{n-1}(A_\bullet, G)
		\oplus \Hom(H_n(A_\bullet), G). \]
\end{theorem}
Fortunately, in our case of interest, $A_\bullet$ is $C_\bullet(X)$
which is by definition free.

There are two things we need to explain, what the map $h$ is and the map $\Ext$ is.

It's not too hard to guess how \[ h : H^n(A_\bullet; G) \to \Hom(H_n(A_\bullet), G) \] is defined.
An element of $H^n(A_\bullet;G)$ is represented by a function which sends a cycle
in $A_n$ to an element of $G$.
The content of the theorem is to show that $h$ is surjective with kernel $\Ext(H_{n-1}(A_\bullet), G)$.

What about $\Ext$?
It turns out that $\Ext(-,G)$ is the so-called \vocab{Ext functor}, defined as follows.
Let $H$ be an abelian group, and consider a \vocab{free resolution} of $H$,
by which we mean an exact sequence
\[ \dots \taking{f_2} F_1 \taking{f_1} F_0 \taking{f_0} H \to 0. \]
Then we can apply $\Hom(-,G)$ to get a cochain complex
\[ \dots \xleftarrow{f_2^\vee} \Hom(F_1, G) \xleftarrow{f_1}
	\Hom(F_0, G) \xleftarrow{f_0} \Hom(H,G) \to 0. \]
but \emph{this cochain complex need not be exact}
(in categorical terms, $\Hom(-,G)$ does not preserve exactness).
We define \[ \Ext(H,G) \defeq \ker(f_2^\vee) / \img(f_1^\vee) \]
and it's a theorem that this doesn't depend on the choice of the free resolution.
There's a lot of homological algebra that goes into this,
which I won't take the time to discuss;
but the upshot of the little bit that I did include is that the $\Ext$
functor is very easy to compute in practice, since
you can pick any free resolution you want and compute the above.

%By ``natural'', we mean that if $f : A_\bullet \to B_\bullet$ is a chain map,
%then we obtain a commutative diagram
%\begin{diagram}
%	0 & \rTo & \Ext(H_{n-1}(A_\bullet), G) & \rTo
%		& H^n(A_\bullet;G) & \rTo & \Hom(H_n(A_\bullet), G) & \rTo & 0 \\
%	& & \uTo^{ \Ext(f_\ast, G) } & & \uTo^{f^\ast} & & \uTo^{\Hom(f_\ast, G)} & & \\
%	0 & \rTo & \Ext(H_{n-1}(B_\bullet), G) & \rTo
%		& H^n(A_\bullet;G) & \rTo & \Hom(H_n(B_\bullet), G) & \rTo & 0 \\
%\end{diagram}
%where $f_\ast$ is the induced arrow $H_n(A_\bullet) \to H_n(B_\bullet)$.

\begin{lemma}
	[Computing the $\Ext$ Functor]
	For any abelian groups $G$, $H$, $H'$ we have
	\begin{enumerate}[(a)]
		\ii $\Ext(H \oplus H, G) = \Ext(H, G) \oplus \Ext(H', G)$.
		\ii $\Ext(H,G) = 0$ for $H$ free, and
		\ii $\Ext(\Zc n, G) = G / nG$.
	\end{enumerate}
\end{lemma}
\begin{proof}
	For (a), note that if $\dots \to F_1 \to F_0 \to H \to 0$
	and $\dots \to F_1' \to F_0' \to F_0' \to H' \to 0$ are free resolutions,
	then so is $F_1 \oplus F_1' \to F_0 \oplus F_0' \to H \oplus H' \to 0$.

	For (b), note that $0 \to H \to H \to 0$ is a free resolution.
	
	Part (c) follows by taking the free resolution
	\[ 0 \to \ZZ \taking{\times n} \ZZ \to \Zc n \to 0 \]
	and applying $\Hom(-,G)$ to it.
	\begin{ques}
		Finish the proof of (c) from here. \qedhere
	\end{ques}
\end{proof}

\begin{ques}
	Some $\Ext$ practice: compute
	$\Ext(\ZZ^{\oplus 2015}, G)$ and $\Ext(\Zc{30}, \Zc 4)$.
\end{ques}

\section{Example Computation of Cohomology Groups}
\prototype{Possibly $H^n(S^m)$.}
Before we start doing some computations, we take the time
to define reduced cohomology groups as well.
For $X$ nonempty we define the \vocab{reduced cohomology groups} by $\wt H^n(X;G)$
by $H^n(\wt C_\bullet(X); G)$.
To be concrete, the augmented singular chain complex is
\[ \dots \taking\partial C_1(X) \taking\partial C_0(X) \taking\eps \ZZ \to 0 \]
dualizes to
\[
	\dots \xleftarrow\delta C^1(X;G) \xleftarrow\delta C^0(X;G)
	\xleftarrow{\eps^\vee} \underbrace{\Hom(\ZZ, G)}_{\cong G}
	\leftarrow 0.
\]
Since the $\ZZ$ we add is also free,
the Universal Coefficient Theorem still applies.

\begin{ques}
	For each $n \ge 1$, $\wt H^n(X;G) \cong H^n(X;G)$.
\end{ques}

As for $\wt H^0(X;G)$ we use the Universal Coefficient Theorem;
since the homology at $\ZZ$ is trivial, the $\Ext$ term vanishes and
$\wt H^0(X;G) \cong \Hom(\wt H_0(X), G)$.
Geometrically, $\wt H^0(X;G)$ consists of functions $X \to G$
which are constant on path-connected components,
modulo the functions which are constant on all of $X$.

\begin{example}
	[Cohomolgy Groups of $S^m$]
	It is straightforward to compute $H^n(S^m)$ now:
	all the $\Ext$ terms vanish since $H_n(S^m)$ is always free,
	and hence we obtain that 
	\[ H^n(S^m) \cong \Hom(H_n(S^m), G) \cong
		\begin{cases}
			G & n=m, n=0 \\
			0 & \text{otherwise}.
		\end{cases}
	\]
	By UCT for reduced groups, we also have
	\[ \wt H^n(S^m) \cong \Hom(\wt H_n(S^m), G) \cong
		\begin{cases}
			G & n=m \\
			0 & \text{otherwise}.
		\end{cases}
	\]
	since $\Hom(\ZZ, G)$.
\end{example}

\begin{example}
	[Cohomolgy Groups of Torus]
	This example has no nonzero $\Ext$ terms either,
	since this time $H^n(S^1 \times S^1)$ is always free.
	So we obtain
	\[ H^n(S^1 \times S^1) \cong \Hom(H_n(S^1 \times S^1), G)
		\cong
		\begin{cases}
			G & n = 0,2 \\
			G^{\oplus 2} & n = 1.
		\end{cases}
	\]
\end{example}

\begin{example}
	[Cohomolgy Groups of Klein Bottle]
	This example will actually have $\Ext$ term.
	Recall that if $K$ is a Klein Bottle then its homology groups are
	$\ZZ$ in dimension $n=0$ and $\ZZ \oplus \Zc 2$ in $n=1$, and $0$ elsewhere.

	For $n=0$, we again just have $H^0(K;G) \cong \Hom(\ZZ, G) \cong G$.
	For $n=1$, the $\Ext$ term is $\Ext(H_0(K), G) \cong \Ext(\ZZ, G) = 0$
	so \[ H^1(K;G) \cong \Hom(\ZZ \oplus \Zc2, G) \cong G \oplus \Hom(\Zc2, G). \]
	We have that $\Hom(\Zc2,G)$ is the subgroup
	of elements of order $2$ in $G$ (and $0 \in G$).

	But for $n=2$, we have our first interesting $\Ext$ group:
	the exact sequence is
	\[ 0 \to \Ext(\ZZ \oplus \Zc 2, G) \to H^2(X;G) \to \underbrace{H_2(X)}_{=0} \to 0. \]
	Thus, we have
	\[ H^2(X;G) \cong \left( \Ext(\ZZ,G) \oplus \Ext(\Zc2,G) \right) \oplus 0
		\cong G/2G. \]
	All the higher groups vanish.
	In summary:
	\[
		H^n(X;G) \cong
		\begin{cases}
			G & n = 0 \\
			G \oplus \Hom(\Zc2, G) & n = 1 \\
			G/2G & n = 2 \\
			0 & n \ge 3. 
		\end{cases}
	\]
\end{example}

From the previous example we can extract the following observation.
\begin{lemma}
	[$0$th and $1$st Homology Groups are just Duals]
	For $n = 0$ and $n = 1$, we have
	\begin{align*}
		H^n(X;G) &\cong \Hom(H_n(X), G) \\
		\wt H^n(X;G) &\cong \Hom(\wt H_n(X), G).
	\end{align*}
\end{lemma}
\begin{proof}
	It's already been shown for $n=0$.
	For $n=1$, notice that $H_0(X)$ and $\wt H_0(X)$ are free,
	so the $\Ext$ term vanishes.
\end{proof}

\section\problemhead
\begin{sproblem}
	[Wedge Product Cohomology]
	For any $G$ and $n$ we have
	\[
		\wt H^n(X \vee Y; G)
		\cong
		\wt H^n(X; G) \oplus \wt H^n(Y; G).
	\]	
\end{sproblem}
\begin{sproblem}[$\Zc2$-Cohomology of $\RP^n$]
	Show that
	\[
		H^m(\RP^n, \Zc2)
		\cong
		\begin{cases}
			\ZZ & \text{$m=0$, or $m$ is odd and $m=n$} \\
			\Zc2 & \text{$0 < m < n$ and $m$ is odd} \\
			0 & \text{otherwise}.
		\end{cases}
	\]
\end{sproblem}

\chapter{The Cup Product}
I won't do much more in this chapter other than
introduce the cup product and say a few examples.
For an actual treatise, see \cite{ref:hatcher} or \cite{ref:maxim752}.

\section{Graded Rings}
\prototype{Polynomial rings are graded rings.}
In this chapter, the ring $R$ is commutative with identity;
but the remaining rings need not be commutative.
However, the word ``ring'' refers to a ring with identity,
or else we will use \vocab{pseudo-ring} to denote a ring that need not possess $1$.

\todo{define graded ring}

\todo{graded commutative}

\section{Cup Products}
Suppose now we replace our abelian group $G$ by a commutative ring $R$ with identity.
Then this gives us a way to multiply two cochains, as follows.
Suppose $\phi \in C^k(X;R)$ and $\psi \in C^\ell(X;R)$.
Then we can define their \vocab{cup product}
$\phi\smile\psi \in C^{k+\ell}(X;R)$ to be
\[
	(\phi\smile\psi)([v_0, \dots, v_{k+\ell}])
	= 
	\phi\left( [v_0, \dots, v_k] \right)
	\cdot
	\psi\left( [v_k, \dots, v_{k+\ell}] \right)
\]
where the multiplication is in $R$.

\begin{ques}
	Which $0$-cochain is the identity for $\smile$?
\end{ques}

This behaves well with respect to $\delta$:
\begin{lemma}[$\delta$ with Cup Products]
	We have
	$\delta(\phi\smile\psi) = \delta\phi\smile\psi + (-1)^k\phi\smile\delta\psi$.
\end{lemma}
\begin{proof}
	Direct $\sum$ computations.
\end{proof}
From this, it is not hard to see that we get an induced map
\[ H^k(X;R) \times H^\ell(X;R) \taking\smile H^{k+\ell}(X;R).  \]
We then define the \vocab{cohomology ring} $H^\bullet(X;R)$
whose elements are finite sums in 
\[ \bigoplus_{k \ge 0} H^k(X;R) \]
and with multiplication given by $\smile$.

Observe that $H^\bullet(X;R)$ is a \emph{graded ring}.
We say that an element $\alpha \in H^k(X;R)$ has dimension $k$,
and write $|\alpha|=k$.
It turns out (though we won't prove it; see \cite{ref:hatcher})
that the cup product is in fact graded commutative.

\begin{example}[Cohomology Ring of $S^n$]
	Consider $S^n$ for $n \ge 1$.
	The nontrivial cohomology groups are given by
	$H^0(S^n; \ZZ) \cong H^n(S^n, \ZZ) \cong \ZZ$.
	So as an abelian group
	\[ H^\bullet(S^n, \ZZ) \cong \ZZ \oplus \alpha \ZZ \]
	where $\alpha$ is the generator of $H^n(S^n, \ZZ)$.
	
	Now, observe that $|\alpha\smile\alpha| = 2n$, but
	since $H^{2n}(S^n, \ZZ) = 0$ we must have $\alpha\smile\alpha=0$.
	So even more succinctly,
	\[ H^\bullet(S^n, \ZZ) \cong \ZZ[\alpha]/(\alpha^2). \]
\end{example}

\begin{example}[Real and Complex Projective Space]
	Though we won't prove it, it turns out that
	\begin{align*}
		H^\bullet(\RP^n, \Zc2) \cong \Zc2[\alpha]/(\alpha^{n+1}) \\
		H^\bullet(\CP^n, \ZZ) \cong \ZZ[\beta]/(\beta^{n+1})
	\end{align*}
	where $|\alpha|=1$ is a generator of $H^1(\RP^n; \Zc2)$
	and $|\beta|=2$ is a generator of $H^2(\CP^n; \ZZ)$.
\end{example}

\section{Cup Product Structure}
In general, it's not so easy to figure out what the cup product looks like
for an arbitrary space.
\todo{wtf was being planned here}
\section{The Borsuk-Ulam Theorem}

\section\problemhead


