\documentclass[11pt]{scrreprt}
%%fakesection Load packages

\usepackage{lmodern}
\usepackage[pdfusetitle]{hyperref}
\ExplSyntaxOn
\sys_if_engine_luatex:T {
	\usepackage{luatex85}
}
\sys_if_engine_pdftex:T {
	\usepackage[T1]{fontenc}
}
\ExplSyntaxOff

% These are evan.sty
\usepackage{amsmath,amssymb,amsthm}
\usepackage{mathrsfs}
\usepackage[usenames,svgnames,dvipsnames]{xcolor}
\usepackage{textcomp}
\usepackage{enumerate}
\usepackage[textsize=scriptsize,shadow]{todonotes}
\usepackage{mathtools}
\usepackage{microtype}
\usepackage[normalem]{ulem}
\usepackage{stmaryrd}
\usepackage{wasysym}
\usepackage{multirow}
\usepackage{prerex}
\usepackage[nameinlink]{cleveref}
\usepackage{derivative}

%%fakesection evan.sty macros
%Small commands
%% Napkin commands
\newcommand{\prototype}[1]{
	\emph{{\color{red} Prototypical example for this section:} #1} \par\medskip
}
\newenvironment{moral}{%
	\begin{tcolorbox}[boxrule=0.4pt,colframe=green!70!black,sharp corners,
		standard jigsaw,opacityback=0,left=10pt,right=10pt,top=3pt,bottom=3pt,
		before skip=10pt,after skip=20pt]%
	\bfseries\color{green!50!black}}%
	{\end{tcolorbox}}

%%fakesection Links (hyperref loaded earlier implicitly)
\hypersetup{
	linkcolor={red!50!black},
	citecolor={green!50!black},
	urlcolor={blue!80!black},
	pdfkeywords={napkin,math},
	pdfsubject={web.evanchen.cc},
	colorlinks,
}

%%fakesection Commutative diagrams
\usepackage{tikz-cd}
\usetikzlibrary{arrows,arrows.meta}
% make a larger hook
% https://tex.stackexchange.com/questions/514451/how-to-define-a-new-hooked-arrow
\makeatletter
\pgfdeclarearrow{
	name=xGlyph,
	cache=false,
	bending mode=none,
	parameters={\tikzcd@glyph@len,\tikzcd@glyph@shorten},
	setup code={%
		\pgfarrowssettipend{\tikzcd@glyph@len\advance\pgf@x by\tikzcd@glyph@shorten}},
	defaults={
		glyph axis=axis_height,
		glyph length=+1.55ex,
		glyph shorten=+-0.1ex},
	drawing code={%
		\pgfpathrectangle{\pgfpoint{+0pt}{+-1.5ex}}{\pgfpoint{+\tikzcd@glyph@len}{+3ex}}%
		\pgfusepathqclip%
		\pgftransformxshift{+\tikzcd@glyph@len}%
		\pgftransformyshift{+-\tikzcd@glyph@axis}%
		\pgftext[right,base]{\tikzcd@glyph}}}
\makeatother
\tikzcdset{
	arrow style=tikz,
	diagrams={>={Latex}},
	tikzcd left hook/.tip={xGlyph[glyph math command=supset, swap, glyph axis = 5.7pt]},
	tikzcd right hook/.tip={xGlyph[glyph math command=supset, glyph axis = 5.7pt]},
	surjective head arrow /.tip = {tikzcd to[sep=-1.5pt]tikzcd to},
	surjective head/.style={
		-surjective head arrow
	}
}

%%fakesection Page layout
\usepackage[headsepline]{scrlayer-scrpage}
\renewcommand{\headfont}{}
\addtolength{\textheight}{3.14cm}
\setlength{\footskip}{0.5in}
\setlength{\headsep}{10pt}

\def\shortdate{\leavevmode\hbox{\the\year-\twodigits\month-\twodigits\day}}
\def\twodigits#1{\ifnum#1<10 0\fi\the#1}
\automark[chapter]{chapter}

\rohead{\footnotesize\thepage}
\rehead{\footnotesize \textbf{\sffamily Napkin}, by \emph{Evan Chen} (\napkinversion)}
\lehead{\footnotesize\thepage}
\lohead{\footnotesize \leftmark}
\chead{}
\rofoot{}
\refoot{}
\lefoot{}
\lofoot{}
%\cfoot{\pagemark}

%%fakesection Fancy section and chapter heads
\renewcommand*{\sectionformat}{\color{purple}\S\thesection\autodot\enskip}
\renewcommand*{\subsectionformat}{\color{purple}\S\thesubsection\autodot\enskip}
\newcommand{\problemhead}{A few harder problems to think about}
\renewcommand{\thesubsection}{\thesection.\roman{subsection}}

\addtokomafont{chapterprefix}{\raggedleft}
\RedeclareSectionCommand[beforeskip=0.5em]{chapter}
\renewcommand*{\chapterformat}{%
\mbox{\scalebox{1.5}{\chapappifchapterprefix{\nobreakspace}}%
\scalebox{2.718}{\color{purple}\thechapter\autodot}\enskip}}

\addtokomafont{partprefix}{\rmfamily}
\renewcommand*{\partformat}{\color{purple}\scalebox{2.5}{\thepart}}

%%fakesection Theorems
\usepackage{tcolorbox}
\tcbuselibrary{breakable,skins,hooks}

% patch tcolorbox to support continuation of a paragraph
% after box
% from https://tex.stackexchange.com/questions/568782/
\makeatletter
\tcbset{
	after app={%
		\ifx\tcb@drawcolorbox\tcb@drawcolorbox@breakable
		\else
		% add only when not breakabel
		\@endparenv
		\fi
	}
}

% for breakable
\appto\tcb@use@after@lastbox{\@endparenv\@doendpe}
\makeatother
% END patch tcolorbox for continuation of paragraph


\usepackage{thmtools}

\theoremstyle{definition}
\declaretheoremstyle[
	headfont=\sffamily\bfseries\color{MidnightBlue},
	headpunct={\\[3pt]},
	postheadspace={0pt},
]{thmtheorem}


\declaretheoremstyle[
	headfont=\bfseries\color{RawSienna},
	headpunct={\\[3pt]},
	postheadspace={0pt},
]{thmexample}

\tcbset{
	theorem box/.style={
		enhanced,
		arc=9pt,
		outer arc=10pt,
		colframe=blue,
		colback=TealBlue!5,
		boxrule=1pt,
		before skip=12pt,
		after skip=14pt,
		left=10pt,
		right=10pt
	},
	remark box/.style={
		boxrule=0pt,
		frame hidden,
		sharp corners,
		enhanced,
		borderline west={2pt}{0pt}{ForestGreen},
		before skip=8pt,
		colback=ForestGreen!5,
		after skip=12pt,
		breakable,
		left=10pt,
		right=10pt
	},
	example box/.style={
		enhanced,
		sharp corners,
		arc=9pt,
		outer arc=10pt,
		colframe=RawSienna,
		colback=Salmon!5,
		boxrule=0.5pt,
		before skip=12pt,
		after skip=14pt,
		breakable,
		top=6pt,
		bottom=8pt,
		breakable,
		left=10pt,
		right=10pt
	},
	ques box/.style={
		boxrule=0pt,
		frame hidden,
		enhanced,
		sharp corners,
		before skip=8pt,
		after skip=12pt,
		borderline west={3pt}{0pt}{black},
		colback=RedViolet!5!gray!5,
		breakable,
		left=10pt,
		right=10pt
	}
}

\declaretheorem[style=thmtheorem,name=Theorem,numberwithin=section]{theorem}
\tcolorboxenvironment{theorem}{theorem box}
\declaretheorem[style=thmtheorem,name=Lemma,sibling=theorem]{lemma}
\tcolorboxenvironment{lemma}{theorem box}
\declaretheorem[style=thmtheorem,name=Proposition,sibling=theorem]{proposition}
\tcolorboxenvironment{proposition}{theorem box}
\declaretheorem[style=thmtheorem,name=Corollary,sibling=theorem]{corollary}
\tcolorboxenvironment{corollary}{theorem box}
\declaretheorem[style=thmexample,name=Example,sibling=theorem]{example}
\tcolorboxenvironment{example}{example box}

\declaretheoremstyle[
	headfont=\bfseries\sffamily\color{ForestGreen!70!black},
	bodyfont=\normalfont,
	headpunct={ --- },
]{thmremark}
\declaretheoremstyle[
	headfont=\bfseries\sffamily\color{ForestGreen!70!black},
	bodyfont=\normalfont,
	headpunct={},
]{thmremark*}

\declaretheoremstyle[
	headfont=\bfseries,
	bodyfont=\normalfont\small
]{thmques}

\declaretheorem[name=Question,sibling=theorem,style=thmques]{ques}
\tcolorboxenvironment{ques}{ques box}
\declaretheorem[name=Exercise,sibling=theorem,style=thmques]{exercise}
\tcolorboxenvironment{exercise}{ques box}
\declaretheorem[name=Remark,sibling=theorem,style=thmremark]{remark}
\tcolorboxenvironment{remark}{remark box}
\declaretheorem[name=Remark,sibling=theorem,style=thmremark*]{remark*}
\tcolorboxenvironment{remark*}{remark box}
\declaretheorem[name=Step,style=thmremark]{step} % only used in Lebesgue int
\tcolorboxenvironment{step}{remark box}

\theoremstyle{definition}
\newtheorem{claim}[theorem]{Claim}
\newtheorem{definition}[theorem]{Definition}
\newtheorem{fact}[theorem]{Fact}
\newtheorem{abuse}[theorem]{Abuse of Notation}

\newtheorem{problem}{Problem}[chapter]
\renewcommand{\theproblem}{\thechapter\Alph{problem}}
\newtheorem{sproblem}[problem]{Problem}
\newtheorem{dproblem}[problem]{Problem}
\renewcommand{\thesproblem}{\theproblem$^{\star}$}
\renewcommand{\thedproblem}{\theproblem$^{\dagger}$}
\newcommand{\listhack}{$\empty$\vspace{-2em}}

%%fakesection Answers
\usepackage{answers}
\Newassociation{hint}{answeritem}{tex/backmatter/all-hints}
\Newassociation{sol}{answeritem}{tex/backmatter/all-solns}
\renewcommand{\solutionextension}{out}
\renewenvironment{answeritem}[1]{\item[\bfseries #1.]}{}

%%fakesection Table of contents
% First add ToC to ToC
\makeatletter
\usepackage{etoolbox}
\pretocmd{\tableofcontents}{%
	\if@openright\cleardoublepage\else\clearpage\fi
	\pdfbookmark[0]{\contentsname}{toc}%
}{}{}%
\makeatother
\setcounter{tocdepth}{1}
\RedeclareSectionCommand[tocnumwidth=4.2em]{part}
\RedeclareSectionCommand[tocpagenumberwidth=2.2em,tocnumwidth=4.2em]{chapter}
\RedeclareSectionCommand[tocpagenumberwidth=2.2em,tocnumwidth=2.8em]{section}
% adjust tocpagenumberwidth manually for large page number: https://tex.stackexchange.com/a/502168

%%fakesection Asymptote definitions
\usepackage{patch-asy}
\numberwithin{asy}{chapter}
\renewcommand{\theasy}{\thechapter\Alph{asy}}
\begin{asydef}
	import extras;
	size(6cm);
	usepackage("amsmath");
	usepackage("amssymb");
	defaultpen(fontsize(11pt));
	settings.tex = "latex";
	settings.outformat = "pdf";
\end{asydef}
\def\asydir{asy}

%%fakesection Bibliography
\usepackage[backend=biber,backref=true,style=alphabetic]{biblatex}
\DeclareLabelalphaTemplate{
	\labelelement{
		\field[final]{shorthand}
		\field{label}
		\field[strwidth=2,strside=left]{labelname}
	}
	\labelelement{
		\field[strwidth=2,strside=right]{year}
	}
}
\DeclareFieldFormat{labelalpha}{\textbf{\scriptsize #1}}
\addbibresource{references.bib}
\addbibresource{images.bib}
%% stylistic biblatex choices
\DefineBibliographyStrings{english}{%
	backrefpage  = {cited p.}, % for single page number
	backrefpages = {cited pp.} % for multiple page numbers
}
\DeclareFieldFormat{journaltitle}{\mkbibemph{#1},} % italic journal title with comma
\DeclareFieldFormat[inbook,thesis]{title}{\mkbibemph{#1}\addperiod} % italic title with period
\DeclareFieldFormat[article]{title}{#1} % title of journal article is printed as normal text
\DeclareFieldFormat[article]{volume}{\textbf{#1}\addcolon\space}
\renewcommand{\mkbibnamegiven}[1]{\textsc{#1}}
\renewcommand{\mkbibnamefamily}[1]{\textsc{#1}}
\renewcommand{\mkbibnameprefix}[1]{\textsc{#1}}
\renewcommand{\mkbibnamesuffix}[1]{\textsc{#1}}
\renewcommand{\finentrypunct}{}

%%fakesection Mini ToC
\usepackage[tight]{minitoc}
\mtcsetfont{parttoc}{chapter}{\sffamily\bfseries}
\mtcsetfont{parttoc}{section}{\footnotesize\rmfamily\upshape\mdseries}
\mtcsetfont{parttoc}{subsection}{\footnotesize\rmfamily\upshape\mdseries}
%\mtcsetdepth{parttoc}{1}
\setcounter{parttocdepth}{1}
\renewcommand*{\partheadstartvskip}{\vspace*{20em}}
\renewcommand*{\partheadendvskip}{}
%\noptcrule
\renewcommand\beforeparttoc{\noindent{\bfseries \Large Part \thepart: Contents}}
%\hspace{\fill}\rule{0.95\linewidth}{2pt}\hspace{\fill}
\doparttoc[n]

%%fakesection Misc haxx
\pdfstringdefDisableCommands{\def\Spec{\text{Spec }}\def\sigma{σ}}

\addbibresource{../../references.bib}

\usepackage{mathrsfs}
\newcommand{\CH}{\mathsf{CH}}
\newcommand{\ZFC}{\mathsf{ZFC}}

\newcommand{\Name}{\text{Name}}
\newcommand{\Po}{\mathbb P}
\newcommand{\nrank}{\opname{n-rank}} % ranks

\def\asydir{}

\begin{document}
\title{Set Theory}
\maketitle

\chapter{ZFC and Ordinal Numbers}
\section{The Ultimate Functional Equation}
definition nesting. . .

\section{The Axioms}

\section{Sets vs Classes}

\section{Ordinal Numbers}
Define
spiral picture

Induction

\section\problemhead
\begin{problem}
	Show that the class of ordinals $\On$ is a proper class.
\end{problem}


\chapter{Ordinals and Cardinals}
transfinite induction

ordinal arithmetic

definition of cardinals

cardinal arithmetic

cofinality: regular vs singular

\section\problemhead
Inaccessible


\chapter{Inner Models}
Meta

Tarski-Vaught

absoluteness

Skolem hulls $\implies$ Reflection

\section\problemhead


\chapter{Forcing}
We are now going to introduce Paul Cohen's technique of \vocab{forcing},
which we then use to break the Continuum Hypothesis.

Here is how it works.
Given a transitive model $M$ and a poset $\Po$ inside it,
we can consider a ``generic'' subset $G \subseteq \Po$, where $G$ is not in $M$.
Then, we are going to construct a bigger universe $M[G]$ which contains both $M$ and $G$.
(This notation is deliberately the same as $\ZZ[\sqrt2]$, for example -- in the algebra case,
we are taking $\ZZ$ and adding in a new element $\sqrt 2$, plus everything that can be generated from it.)
By choosing $\Po$ well, we can cause $M[G]$ to have desirable properties.

The models $M$ and $M[G]$ will share the same ordinals.
But one issue with this is that forcing may introduce some new bijections between cardinals of $M$
that were not there originally; this leads to a phenomenon called \emph{cardinal collapse}:
quite literally, cardinals in $M$ will no longer be cardinals in $M[G]$, and instead just an ordinal.
\missingfigure{universe showing cardinal collapse}

In the case of the Continuum Hypothesis, we'll introduce a $\Po$ such that
any generic subset $G$ will ``encode'' $\aleph_2^M$ real numbers.
We'll then show cardinal collapse does not occur, meaning $\aleph_2^{M[G]} = \aleph_2^M$.
Thus $M[G]$ will have $\aleph_2^{M[G]}$ real numbers, as desired.

\section{Setting Up Posets}
\prototype{Infinite Binary Tree}
Let $M$ be a transitive model of $\ZFC$.
Let $\Po = (\Po, \le) \in M$ be a poset with a maximal element $1_\Po$
which lives inside a model $M$.
The elements of $\Po$ are called \vocab{conditions};
because they will force things to be true in $M[G]$.

\begin{definition}
	A subset $D \subseteq \Po$ is \vocab{dense} if for all $p \in \Po$,
	there exists a $q  \in D$ such that $q \le p$.
\end{definition}
Examples of dense subsets include the entire $\Po$ as well
as any downwards ``slice''.

\begin{definition}
	For $p,q \in \Po$ we write $p \parallel q$,
	saying ``$p$ is \vocab{compatible} with $q$'',
	if there is exists $r \in \Po$ with $r \le p$ and $r \le q$.
	Otherwise, we say $p$ and $q$ are \vocab{incompatible}
	and write $p \perp q$.
\end{definition}
\begin{example}[Infinite Binary Tree]
	Let $\Po = 2^{<\omega}$ be the \vocab{infinite binary tree} shown below,
	extended to infinity in the obvious way:
	\begin{center}
		\begin{asy}
			size(8cm);
			pair P = Drawing("\varnothing", (0,4), dir(90));
			pair P0 = Drawing("0", (-5,2), 1.5*dir(90));
			pair P1 = Drawing("1", (5,2),  1.5*dir(90));
			pair P00 = Drawing("00", (-7,0), 1.4*dir(120));
			pair P01 = Drawing("01", (-3,0), 1.4*dir(60));
			pair P10 = Drawing("10", (3,0),  1.4*dir(120));
			pair P11 = Drawing("11", (7,0),  1.4*dir(60));

			pair P000 = Drawing("000", (-8,-3));
			pair P001 = Drawing("001", (-6,-3));
			pair P010 = Drawing("010", (-4,-3));
			pair P011 = Drawing("011", (-2,-3));

			pair P100 = Drawing("100", (2,-3));
			pair P101 = Drawing("101", (4,-3));
			pair P110 = Drawing("110", (6,-3));
			pair P111 = Drawing("111", (8,-3));

			label("$\vdots$", (-7,-3), dir(-90));
			label("$\vdots$", (-3,-3), dir(-90));
			label("$\vdots$", (3,-3), dir(-90));
			label("$\vdots$", (7,-3), dir(-90));

			draw(P01--P0--P00);
			draw(P11--P1--P10);
			draw(P0--P--P1);
			draw(P000--P00--P001);
			draw(P100--P10--P101);
			draw(P010--P01--P011);
			draw(P110--P11--P111);
		\end{asy}
	\end{center}

	\begin{enumerate}[(a)]
		\ii The maximal element $1_\Po$ is the empty string $\varnothing$.
		\ii $D = \{\text{all strings ending in $001$}\}$ is an example of a dense set.
		\ii No two elements of $\Po$ are compatible unless they are comparable.
	\end{enumerate}
\end{example}


Now, I can specify what it means to be ``generic''.
\begin{definition}
	A nonempty set $G \subseteq \Po$ is a \vocab{filter} if
	\begin{enumerate}[(a)]
		\ii The set $G$ is upwards-closed:
		$\forall p \in G (\forall q \ge p) (q \in G)$.
		\ii Any pair of elements in $G$ is compatible.
	\end{enumerate}
	We say $G$ is \vocab{$M$-generic} if for all $D$ which are \emph{in the model $M$},
	if $D$ is dense then $G \cap D \neq \varnothing$.
\end{definition}
\begin{ques}
	Show that if $G$ is a filter then $1_\Po \in G$.
\end{ques}
\begin{example}[Generic Filters on the Infinite Binary Tree]
	Let $\Po = 2^{<\omega}$.
	The generic filters on $\Po$ are sets of the following form:
	\[ \left\{ 0,\; b_1,\; b_1b_2,\; b_1b_2b_3,\; \dots \right\}. \]
	So every generic filter on $\Po$ correspond to a binary number $b = 0.b_1b_2b_3\dots$.

	It is harder to describe which reals correspond to generic filters,
	but they should really ``look random''.
	For example, the set of strings ending in $011$ is dense,
	so one should expect ``$011$'' to appear inside $b$,
	and more generally that $b$ should contain every binary string.
	So one would expect the binary expansion of $\pi-3$ might correspond to a generic,
	but not something like $0.010101\dots$.
	That's why we call them ``generic''.
\end{example}

\begin{center}
	\begin{asy}
		size(8cm);
		pair P = Drawing("\varnothing", red, (0,4), red, dir(90));
		pair P0 = Drawing("0", red, (-5,2), red, 1.5*dir(90));
		pair P1 = Drawing("1", (5,2),  1.5*dir(90));
		pair P00 = Drawing("00", (-7,0), 1.4*dir(120));
		pair P01 = Drawing("01", red, (-3,0), red, 1.4*dir(60));
		pair P10 = Drawing("10", (3,0),  1.4*dir(120));
		pair P11 = Drawing("11", (7,0),  1.4*dir(60));

		pair P000 = Drawing("000", (-8,-3));
		pair P001 = Drawing("001", (-6,-3));
		pair P010 = Drawing("010", red, (-4,-3), red);
		pair P011 = Drawing("011", (-2,-3));

		pair P100 = Drawing("100", (2,-3));
		pair P101 = Drawing("101", (4,-3));
		pair P110 = Drawing("110", (6,-3));
		pair P111 = Drawing("111", (8,-3));

		draw(P01--P0--P00);
		draw(P11--P1--P10);
		draw(P0--P--P1);
		draw(P000--P00--P001);
		draw(P100--P10--P101);
		draw(P010--P01--P011);
		draw(P110--P11--P111);

		draw(P--P0--P01--P010--(P010+2*dir(-90)), red+1.4);
		MP("G", P010+2*dir(-90), dir(-90), red);
	\end{asy}
\end{center}

\begin{exercise}
	Verify that these are every generic filter $2^{<\omega}$ has the form above.
	Show that conversely, a binary number gives a filter, but it need not be generic.
\end{exercise}

Notice that if $p \ge q$, then the sentence $q \in G$ tells us more information than the sentence $p \in G$.
In that sense $q$ is a \emph{stronger} condition.
In another sense $1_\Po$ is the weakest possible condition,
because it tells us nothing about $G$; we always have $1_\Po \in G$
since $G$ is upwards closed.

\section{More Properties of Posets}
We had better make sure that generic filters exist.
In fact this is kind of tricky, but for countable models it works:
\begin{lemma}[Rasiowa-Sikorski Lemma]
	Suppose $M$ is a \emph{countable} transitive model of $\ZFC$
	and $\Po$ is a partial order.
	Then there exists an $M$-generic filter $G$.
\end{lemma}
\begin{proof}
	Since $M$ is countable, there are only countably many dense sets (they live in $M$!),
	say \[ D_1, D_2, \ldots, D_n, \ldots \in M. \]
	Using Choice,
	let $p_1 \in D_1$, and then let $p_2 \le p_1$ such that $p_2 \in D_2$
	(this is possible since $D_2$ is dense), and so on.
	In this way we can inductively exhibit a chain
	\[ p_1 \ge p_2 \ge p_3 \ge \dots \]
	with $p_i \in D_i$ for every $i$.

	Hence, we want to generate a filter from the $\{p_i\}$.
	Just take the upwards closure -- let $G$ be the set of $q \in \Po$ such that $q \ge p_n$ for some $n$.
	By construction, $G$ is a filter (this is actually trivial).
	Moreover, $G$ intersects all the dense sets by construction.
	\todo{problem?}
\end{proof}
Fortunately, for breaking $\CH$ we would want $M$ to be countable anyways.
% This is really just the proof of the Baire category theorem.

The other thing we want to do to make sure we're on the right track is guarantee
that a generic set $G$ is not actually in $M$.
(Analogy: $\ZZ[3]$ is a really stupid extension.)
The condition that guarantees this is:

\begin{definition}
	A partial order $\Po$ is \vocab{splitting} if
	for all $p \in \Po$, there exists $q,r \le p$
	such that $q \perp r$.
\end{definition}
\begin{example}[Infinite Binary Tree is (Very) Splitting]
	The infinite binary tree is about as splitting as you can get.
	Given $p \in 2^{<\omega}$, just consider the two elements right under it.
\end{example}

\begin{lemma}[Splitting Posets Omit Generic Sets]
	Suppose $\Po$ is splitting.  Then if $F \subseteq \Po$ is a filter
	such that $F \in M$, then $\Po \setminus F$ is dense.
	In particular, if $G \subseteq \Po$ is generic, then $G \notin M$.
\end{lemma}
\begin{proof}
	Consider $p \notin \Po \setminus F \iff p \in F$.
	Then there exists $q, r \le p$ which are not compatible.
	Since $F$ is a filter it cannot contain both;
	we must have one of them outside $F$, say $q$.
	Hence every element of $p \in \Po \setminus (\Po \setminus F)$
	has an element $q \le p$ in $\Po \setminus F$.
	That's enough to prove $\Po \setminus F$ is dense.
	\begin{ques}
		Deduce the last assertion of the lemma about generic $G$. \qedhere
	\end{ques}
\end{proof}

\section{Names, and the Generic Extension}
We now define the \emph{names} associated to a poset $\Po$.

\begin{definition}
	Suppose $M$ is a transitive model of $\ZFC$, $\Po = (\Po, \le) \in M$ is a partial order.
	We define the hierarchy of \vocab{$\Po$-names} recursively by
	\begin{align*}
		\Name_0 &= \varnothing \\
		\Name_{\alpha+1} &= \PP(\Name_\alpha \times \Po) \\
		\Name_{\lambda} &= \bigcup_{\alpha < \lambda} \Name_\alpha.
	\end{align*}
	Finally, $\Name = \bigcup_\alpha \Name_\alpha$ denote the class of all $\Po$-names.
	% For $\tau \in \Name$, let $\nrank(\tau)$ be the least $\alpha$ such that $\tau \in \Name_\alpha$.
\end{definition}
(These $\Name_\alpha$'s are the analog of the $V_\alpha$'s:
each $\Name_\alpha$ is just the set of all names with rank $\le \alpha$.)

\begin{definition}
	For a filter $G$, we define the \vocab{interpretation} of $\tau$ by $G$,
	denote $\tau^G$, using the transfinite recursion
	\[ \tau^G
		= \left\{ \sigma^G
		\mid \left<\sigma, p\right> \in \tau
		\text{ and } p \in G\right\}. \]
	We then define the model
	\[ M[G] = \left\{ \tau^G \mid \tau \in \Name^M \right\}. \]
	In words, $M[G]$ is the interpretation of all the possible $\Po$-names
	(as computed by $M$).
\end{definition}

\textbf{You should think of a $\Po$-name as a ``fuzzy set''.}
Here's the idea.
Ordinary sets are collections of ordinary sets,
so fuzzy sets should be collections of fuzzy sets.
These fuzzy sets can be thought of like the Ghosts of Christmases yet to come:
they represent things that might be, rather than things that are certain.
In other words, they represent the possible futures of $M[G]$ for various choices of $G$.

Every fuzzy set has an element $p \in \Po$ pinned to it.
When it comes time to pass judgment,
we pick a generic $G$ and filter through the universe of $\Po$-names.
The fuzzy sets with an element of $G$ attached to it materialize into the real world,
while the fuzzy sets with elements outside of $G$ fade from existence.
The result is $M[G]$.

\begin{example}[First Few Levels of the Name Hierarchy]
	Let us compute
	\begin{align*}
		\Name_0 &= \varnothing \\
		\Name_1 &= \PP(\varnothing \times \Po) \\
		&= \{\varnothing\} \\
		\Name_2 &= \PP(\{\varnothing\} \times \Po) \\
		&= \PP\left( \left\{ 
			\left<\varnothing, p\right>
			\mid p \in \Po
		\right\} \right).
	\end{align*}
\end{example}
Compare the corresponding von Neuman universe.
\[ V_0 = \varnothing, \; V_1 = \{\varnothing\}, \;
V_2 = \left\{ \varnothing, \left\{ \varnothing \right\} \right\}. \]

\begin{example}[Example of an Interpretation]
	As we said earlier, $\Name_1 = \{\varnothing\}$.
	Now suppose
	\[ \tau =
		\left\{
			\left<\varnothing, p_1\right>,
			\left<\varnothing, p_2\right>,
			\dots
			\left<\varnothing, p_n\right>
		\right\} 
		\in \Name_2. \]
	Then 
	\[
		\tau^G
		= \left\{ \varnothing \mid
		\left<\varnothing, p\right> \in \tau \text{ and } p \in G\right\}
		=
		\begin{cases}
			\{\varnothing\} & \text{if some } p_i \in G \\
			\varnothing & \text{otherwise}.
		\end{cases}
	\]
	In particular, remembering that $G$ is nonempty we see that
	\[ \left\{ \tau^G \mid \tau \in \Name_2 \right\} = V_2^M. \]
	In fact, this holds for any natural number $n$, not just $2$.
\end{example}
So, $M[G]$ and $M$ agree on finite sets.

Now, we want to make sure $M[G]$ contains the elements of $M$.
To do this, we take advantage of the fact that $1_\Po$ must be in $G$, and define
for every $x \in M$ the set
\[ \check x = \left\{ \left<\check y, 1_\Po\right> \mid y \in x \right\} \]
by transfinite recursion.
Basically, $\check x$ is just a copy of $x$ where we add check marks and tag every element with $1_\Po$.

\begin{example}
	Compute $\check 0 = 0$ and $\check 1 = \left\{ \left<\check 0, 1_\Po\right> \right\}$.
	Thus \[ (\check 0)^G = 0 \quad\text{and}\quad (\check 1)^G = 1. \]
\end{example}
\begin{ques}
	Show that in general, $(\check x)^G = x$.
	(Rank induction.)
\end{ques}

However, we'd also like to cause $G$ to be in $M[G]$.
In fact, we can write down the name exactly:
\[ \dot G = \left\{ \left<\check p, p\right> \mid p \in \Po \right\}. \]
\begin{ques}
	Show that $(\dot G)^G = G$.
\end{ques}
\begin{ques}
	Verify that $M[G]$ is transitive:
	that is, if $\sigma^G \in \tau^G \in M[G]$, show that $\sigma^G \in M[G]$.
	(This is offensively easy.)
\end{ques}

In summary,
\begin{moral}
	$M[G]$ is a transitive model extending $M$ (it contains $G$).
\end{moral}

Moreover, it is reasonably well-behaved even if $G$ is just a filter.
Let's see what we can get off the bat.
\begin{lemma}[Properties Obtained from Filters]
	Let $M$ be a transitive model of $\ZFC$.
	If $G$ is a filter, then $M[G]$ is transitive
	and satisfies Extensionality, Foundation, EmptySet, Infinity, Pairing, and Union.
\end{lemma}
This leaves PowerSet, Comprehension, Replacement, and Choice.
\begin{proof}
	Hence, we get Extensionality and Foundation for free.
	Infinity and EmptySet follows from $M \subseteq M[G]$.

	For Pairing, suppose $\sigma_1^G, \sigma_2^G \in M[G]$.
	Then
	\[ \sigma = 
		\left\{ \left<\sigma_1, 1_\Po\right>, \left<\sigma_2, 1_\Po\right> \right\}
	\]
	works satisfies $\sigma^G = \{\sigma_1, \sigma_2\}$.
	(Note that we used Pairing in $M$.)
	Union is left as a problem, which you are encouraged to try now.
\end{proof}
Up to here, we don't need to know anything about when a sentence is true in $M[G]$;
all we had to do was contrive some names like $\check x$ or
$\left\{ \left<\sigma_1, 1_\Po\right>, \left<\sigma_2, 1_\Po\right> \right\}$
to get the facts we wanted.
But for the remaining axioms, we \emph{are} going to need this extra power
are true in $M[G]$.
For this, we have to introduce the Fundamental Theorem of Forcing.

\section{Fundamental Theorem of Forcing}
The model $M$ unfortunately has no idea what $G$ might be,
only that it is some generic filter.\footnote{%
	You might say this is a good thing, here's why.
	We're trying to show that $\neg \CH$ is consistent with $\ZFC$,
	and we've started with a model $M$ of the real universe $V$.
	But for all we know $\CH$ might be true in $V$ (what if $V=L$?),
	in which case it would also be true of $M$.

	Nonetheless, we boldly construct $M[G]$ an extension of the model $M$.
	In order for it to behave differently from $M$, it has to be out of reach of $M$.
	Conversely, if $M$ could compute everything about $M[G]$,
	then $M[G]$ would have to conform to $M$'s beliefs.

	That's why we worked so hard to make sure $G \in M[G]$ but $G \notin M$.
}
Nonetheless, we are going to define a relation $\Vdash$, called the \emph{forcing} relation.
Roughly, we are going to write
\[ p \Vdash \varphi(\sigma_1, \dots, \sigma_n) \]
where $p \in \Po$, $\sigma_1, \dots, \sigma_n \in M[G]$, if and only if the following holds:
\begin{quote}
	For \emph{any} generic $G$,
	if $p \in G$,
	then $M[G] \vDash \varphi(\sigma_1^G, \dots, \sigma_n^G)$.
\end{quote}
Note that $\Vdash$ is defined without reference to $G$:
it is something that $M$ can see.
We say $p$ \vocab{forces} the sentences $\varphi(\sigma_1, \dots, \sigma_n)$.
And miraculously, we can define this relation in such a way that the converse is true:
\emph{a sentence holds if and only if some $p$ forces it}.


\begin{theorem}
	[Fundamental Theorem of Forcing]
	Suppose $M$ is a transitive model of ZF.
	Let $\Po \in M$ be a poset, and $G \subseteq \PP$ is an $M$-generic filter.
	Then,
	\begin{enumerate}[(1)]
		\ii Consider $\sigma_1, \dots, \sigma_n \in \Name^M$,
		Then
		\[ M[G] \vDash \varphi[\sigma_1^G, \dots, \sigma_n^G] \]
		if and only if there exists a condition $p \in G$
		such that $p$ \emph{forces} the sentence $\varphi(\sigma_1, \dots, \sigma_n)$.
		We denote this by $p \Vdash \varphi(\sigma_1, \dots, \sigma_n)$.
		\ii This forcing relation is (uniformly) definable in $M$.
	\end{enumerate}
\end{theorem}

I'll tell you how the definition works in the following.

\section{(Optional) Defining the Relation}
Here's how we're going to go.
We'll define the most generous condition possible such that
the forcing works in one direction ($p \vDash \varphi(\sigma_1, \dots, \sigma_n)$ means
$M[G] \vDash \varphi[\sigma_1^G, \dots, \sigma_n^G]$).
We will then cross our fingers that the converse also works.

We proceed by induction on the formula complexity.
It turns out in this case that the atomic formula (base cases)
are hardest and themselves require induction on ranks.

For some motivation, let's consider how we should define $p \Vdash \tau_1 \in \tau_2$ given that we've already defined $p \vDash \tau_1 = \tau_2$.
We need to ensure this holds iff
\[ \forall \text{$M$-generic $G$ with $p \in G$}:
	\ M[G] \vDash \tau_1^G \in \tau_2^G. \]
So it suffices to ensure that any generic $G \ni p$ hits a condition $q$ which forces $\tau_1^G$ to \emph{equal} a member $\tau^G$ of $\tau_2^G$.
In other words, we want to choose the definition of $p \Vdash \tau_1 \in \tau_2$ to hold if and only if
\[
	\left\{ q \in \Po
	\mid \exists \left<\tau, r\right> \in \tau_2 
	\left( q \le r \land q \Vdash(\tau=\tau_1) \right)
	\right\}
\]
is dense below in $p$.
In other words, if the set is dense, then the generic must hit $q$, so it must hit $r$, meaning that $\left<\tau_r\right> \in \tau_2$ will get interpreted such that $\tau^G \in \tau_2^G$, and moreover the $q \in G$ will force $\tau_1 = \tau$.

Now let's write down the definition\dots
In what follows, the $\Vdash$ omits the $M$ and $\Po$.
\begin{definition}
	Let $M$ be a countable transitive model of ZFC.
	Let $\Po \in M$ be a partial order.
	For $p \in \Po$ and $\varphi(\sigma_1, \dots, \sigma_n)$ a formula in LST, we write $\tau \Vdash \varphi(\sigma_1, \dots, \sigma_n)$ to mean the following, defined by induction on formula complexity plus rank.
	\begin{enumerate}[(1)]
		\ii $p \Vdash \tau_1 = \tau_2$ means
		\begin{enumerate}[(i)]
			\ii For all $\left<\sigma_1, q_1\right> \in \tau_1$ the set
			\[ D_{\sigma_1, q_1}
				\defeq
				\left\{ r \mid
				r \le q_1 \lthen \exists \left<\sigma_2, q_2\right> \in \tau_2 \left( r \le q_2 \land r \Vdash (\sigma_1 = \sigma_2) \right)\right\}.
			\]
			is dense in $p$.
			(This encodes ``$\tau_1 \subseteq \tau_2$''.)
			\ii For all $\left<\sigma_2, q_2\right> \in \tau_2$,
			the set $D_{\sigma_2, q_2}$ defined similarly is dense below $p$.
		\end{enumerate}
		\ii $p \Vdash \tau_1 \in \tau_2$ means
		\[
		\left\{ q \in \Po
		\mid \exists \left<\tau, r\right> \in \tau_2 
		\left( q \le r \land q \Vdash(\tau=\tau_1) \right)
		\right\} \]
		is dense below $p$.
		\ii $p \Vdash \varphi \land \psi$ means $p \Vdash \varphi$ and $p \Vdash \psi$.
		\ii $p \Vdash \neg \varphi$ means $\forall q \le p$, $q \not\Vdash \varphi$.
		\ii $p \Vdash \exists x \varphi(x, \sigma_1, \dots, \sigma_n)$ means that the set
		\[
			\left\{ q \mid \exists \tau
				\left( q \Vdash \varphi(\tau, \sigma_1, \dots, \sigma_n \right)
			\right\}
		\]
		is dense below $p$.
	\end{enumerate}
\end{definition}
This is definable in $M$!
All we've referred to is $\Po$ and names, which are in $M$.
(Note that being dense is definable.)
Actually, in parts (3) through (5) of the definition above,
we use induction on formula complexity.
But in the atomic cases (1) and (2) we are doing induction on the ranks of the names.

So, the construction above gives us one direction (I've omitted tons of details, but\dots).

Now, how do we get the converse: that a sentence is true if and only if something forces it?
Well, by induction, we can actually show the following result:
\begin{lemma}[Consistency and Persistence]
	We have
	\begin{enumerate}[(1)]
		\ii (Consistency) If $p \Vdash \varphi$ and $q \le p$ then $q \Vdash \varphi$.
		\ii (Persistence) If $\left\{ q \mid q \Vdash \varphi \right\}$
		is dense below $p$ then $p \Vdash \varphi$.
	\end{enumerate}
\end{lemma}
These are just inductions on the five parts of the definition.
From this it also follows that
\begin{corollary}[Completeness]
	The set $\left\{ p \mid p \Vdash \varphi \text{ or } p \Vdash \neg\varphi \right\}$
	is dense.
\end{corollary}
\begin{proof}
	We claim that whenever $p \not\Vdash \varphi$ then
	for some $\ol p \le p$ we have $\ol p \Vdash \neg\varphi$;
	this will establish the corollary.

	By the contrapositive of the previous lemma,
	$\{q \mid q \Vdash \varphi\}$ is not dense below $p$,
	meaning for some $\ol p \le p$, every $q \le \ol p$ gives $q \not\Vdash \varphi$.
	By the definition of $p \vDash \neg\varphi$,
	we have $\ol p \vDash \neg\varphi$.
\end{proof}
And this gives the converse: the $M$-generic $G$ has to hit some condition
that passes judgment, one way or the other.
This completes the proof of the Fundamental Theorem.

\section{The Remaining Axioms}
\begin{theorem}[The Generic Extension Satisfies $\ZFC$]
	Suppose $M$ is a transitive model of $\ZFC$.
	Let $\Po \in M$ be a poset, and $G \subseteq \PP$ is an $M$-generic filter.
	Then \[ M[G] \vDash \ZFC. \]
\end{theorem}
\begin{proof}
	We'll just do Comprehension, as the other remaining axioms are similar.
	
	Suppose $\sigma^G, \sigma_1^G, \dots, \sigma_n^G \in M[G]$
	are a set and parameters, and
	$\varphi(x,x_1, \dots, x_n)$ is an LST formula.
	We want to show that the set
	\[ A = \left\{ 
		x \in \sigma^G \mid M[G] \vDash \varphi[x, \sigma_1^G, \dots, \sigma_n^G]
	\right\} \]
	is in $M[G]$; i.e.\ it is the interpretation of some name.

	Note that every element of $\sigma^G$ is of the form $\rho^G$
	for some $\rho \in \dom(\sigma)$ (a bit of abuse here,
	$\sigma$ is a bunch of pairs of names and $p$'s,
	and the domain is just the set of names).
	So by the Fundamental Theorem of Forcing, we may write
	\[ A = 
		\left\{ \rho^G \mid \rho \in \dom(\sigma)
			\text{ and }
			\exists p \in G
			\left( p \Vdash \rho \in \sigma
			\land \varphi(\rho, \sigma_1, \dots, \sigma_n)
			\right)
		\right\}.
	\]
	To show $A \in M[G]$ we have to write down a $\tau$
	such that the name $\tau^G$ coincides with $A$.
	We claim that
	\[
		\tau
		=
		\left\{ \left<\rho, p\right>
			\in \dom(\sigma) \times \Po \mid
			p \Vdash 
			\land \varphi(\rho, \sigma_1, \dots, \sigma_n)
		\right\}
	\]
	is the correct choice.
	It's actually clear that $\tau^G = A$ by construction;
	the ``content'' is showing that $\tau$ is in actually a name of $M$,
	which follows from Comprehension in $M$.

	So really, the point of the Fundamental Theorem of Forcing
	is just to let us write down this $\tau$;
	it lets us show that $\tau$ is in $\Name^M$
	without actually referencing $G$.
\end{proof}


\section\problemhead
RS Lemma

Union



%\begin{exercise}
%	Show that $\rank \sigma^G \le \nrank(\sigma)$ for any $\sigma \in \Name^M$.
%\end{exercise}

%\begin{exercise}
%	Check that
%	\begin{enumerate}[(1)]
%		\ii $(\check x)^G = x$.
%		\ii $(\dot G)^G = G$.
%	\end{enumerate}
%\end{exercise}



\chapter{The Failure of the Continuum Hypothesis}
We now use the technique of forcing to break the Contiuum Hypothesis by choosing a good poset $\Po$.

\section{Forcing $V \neq L$ is really easy}
As a small aside, to check we're on the right track we show the following result.

\begin{theorem}[$V \ne L$]
	Let $M$ be a countable transitive model of $\ZFC$.
	Let $\Po \in M$ be \emph{any} splitting poset,
	and let $G \subseteq \Po$ be $M$-generic.
	Then $M[G] \vDash (V \neq L)$.
\end{theorem}
\begin{proof}
	Since $L$ has a $\Sigma_1$ definition,
	we have \[ L^{M[G]} = L^M \subseteq M \subsetneq M[G] \]
	where the last part follows from $G \notin M[G]$.
\end{proof}

Thus $M[G] \vDash \ZFC + (V \ne L)$ for any splitting poset $\Po$,
and we are one step closer to breaking $\CH$.

\section{Adding in Reals}
Starting with a \emph{countable} transitive model $M$.

We want to choose $\Po \in M$ such that $(\aleph_2)^M$ many real numbers appear,
and then worry about cardinal collapse later.

Recall the earlier situation where we set $\Po$ to be the infinite complete binary tree; its nodes can be thought of as partial functions $n \to 2$ where $n < \omega$.
Then $G$ itself is a path down this tree; i.e.\ it can be encoded as a total function $G : \omega \to 2$,
and corresponds to a real number.

\begin{center}
	\begin{asy}
		size(8cm);
		pair P = Drawing("\varnothing", red, (0,4), red, dir(90));
		pair P0 = Drawing("0", red, (-5,2), red, 1.5*dir(90));
		pair P1 = Drawing("1", (5,2),  1.5*dir(90));
		pair P00 = Drawing("00", (-7,0), 1.4*dir(120));
		pair P01 = Drawing("01", red, (-3,0), red, 1.4*dir(60));
		pair P10 = Drawing("10", (3,0),  1.4*dir(120));
		pair P11 = Drawing("11", (7,0),  1.4*dir(60));

		pair P000 = Drawing("000", (-8,-3));
		pair P001 = Drawing("001", (-6,-3));
		pair P010 = Drawing("010", red, (-4,-3), red);
		pair P011 = Drawing("011", (-2,-3));

		pair P100 = Drawing("100", (2,-3));
		pair P101 = Drawing("101", (4,-3));
		pair P110 = Drawing("110", (6,-3));
		pair P111 = Drawing("111", (8,-3));

		draw(P01--P0--P00);
		draw(P11--P1--P10);
		draw(P0--P--P1);
		draw(P000--P00--P001);
		draw(P100--P10--P101);
		draw(P010--P01--P011);
		draw(P110--P11--P111);

		draw(P--P0--P01--P010--(P010+2*dir(-90)), red+1.4);
		MP("G", P010+2*dir(-90), dir(-90), red);
	\end{asy}
\end{center}

We want to do something similar, but with $\omega_2$ many real numbers instead of just one.
In light of this, consider in $M$ the following poset:
\[
	\Po = 
	\opname{Add} \left( \omega_2, \omega \right)
	\defeq
	\left( 
	\left\{ p : \omega_2 \times \omega \to 2,
		\dom(p) < \omega
	\right\},
	\supseteq
	\right).
\]
These elements (conditions) are ``partial functions'':
we take some finite subset of $\omega \times \omega_2$ and map it into $2=\{0,1\}$.
Moreover, we say $p \le q$ if $\dom(p) \supseteq \dom(q)$ and the two functions agree over $\dom(q)$.
\begin{ques}
	What is $1_\Po$ here?
\end{ques}

\begin{exercise}
	Show that a generic $G$ can be encoded as a function $\omega_2 \times \omega \to 2$.
\end{exercise}

%Let $G \subseteq \opname{Add}(\omega_2, \omega)$ be an $M$-generic.
%We claim that, like in the binary case, $G$ can be encoded as a function $\omega_2 \times \omega \to 2$.
%To see this, consider $\alpha \in \omega_2$ and $n \in \omega$; we have the dense set
%\[ D_{\alpha, n}
%	= \left\{ p \in \opname{Add}(\omega_2, \omega)
%	\mid (\alpha, n) \in \dom(p) \right\}
%\]
%(this is obviously dense, given any $p$ add in $(\alpha, n)$ if it's not in there already).
%So $G$ hits this dense set, meaning that for every $(\alpha, n)$ there's a function in $G$ which defines it.
%Using the fact that $G$ is upwards closed and a filter, we may as before we may interpret $G$ as a function $\omega_2 \times \omega \to 2$.

\begin{lemma}[$G$ encodes distinct real numbers]
	For $\alpha \in \omega_2$ define
	\[ G_\alpha = \left\{ n \mid G\left( \alpha,n \right) = 0 \right\} \in \PP(\NN). \]
	Then $G_\alpha \neq G_\beta$ for any $\alpha \neq \beta$.
\end{lemma}
\begin{proof}
	We claim that the set
	\[ D = \left\{ q \mid \exists n \in \omega :
		q\left( \alpha, n \right) \neq q\left( \beta, n \right)
		\text{ are both defined}
	\right\} \]
	is dense.
	\begin{ques}
		Check this.
		(Use the fact that the domains are all finite.)
	\end{ques}
%	This is pretty easy to see.
%	Consider $p \in \opname{Add}(\omega_2, \omega)$.
%	Then you can find an $n$ such that
%	neither $(\alpha, n)$ nor $(\beta, n)$ is defined,
%	just because $\dom(p)$ is finite.
%	Then you make $p'$ as $p$ plus $p'( (\alpha, n) ) = 1$
%	and $p'( (\beta, n) ) = 0$.
%	Hence the set is dense.

	Since $G$ is an $M$-generic it hits this dense set $D$.
	Hence $G_\alpha \neq G_\beta$.
\end{proof}

Since $G \in M[G]$ and $M[G] \vDash \ZFC$,
it follows that each $G_\alpha$ is in $M[G]$.
So there are at least $\aleph_2^M$ real numbers in $M$.
We are done once we can show there is no cardinal collapse.

\section{The Countable Chain Condition}
It remains to show that with $\Po = \opname{Add}(\omega, \omega_2)$, we have that
\[ \aleph_2^{M[G]} = \aleph_2^M. \]
In that case, since $M[G]$ will have $\aleph_2^M = \aleph_2^{M[G]}$ many reals, we will be done.

To do this, we'll rely on the following combinatorial property of $\Po$:

\begin{definition}
	We say that $A \subset \mathcal P$ is a \vocab{strong antichain}
	if for any distinct $p$ and $q$ in $A$, we have $p \perp q$.
\end{definition}
\begin{example}[Example of an Antichain]
	In the infinite binary tree, 
	the set $A = \{00, 01, 10, 11\}$ is a strong antichain
	(in fact maximal by inclusion).
\end{example}
This is stronger than the notion of ``antichain'' than you might be used to!\footnote{%
	In the context of forcing, some authors use ``antichain'' to refer to ``strong antichain''.
	I think this is lame.}
We don't merely require that every two elements are incomparable,
but that they are in fact \emph{incompatible}.
\begin{ques}
	Draw a finite poset and an antichain of it which is not strong.
\end{ques}

\begin{definition}
	A poset $\Po$ has the \vocab{$\kappa$-chain condition}
	(where $\kappa$ is a cardinal) if all strong antichains
	in $\Po$ have size less than $\kappa$.
	The special case $\kappa = \omega$ is called the \vocab{countable chain condition}.
\end{definition}

We are going to show that if the poset has the $\kappa$-chain condition
then it preserves all cardinals greater than $\kappa$.
% or was it > \kappa?
In particular, the countable chain condition will show that $\Po$ preserves all the cardinals.
Then, we'll show that $\opname{Add}(\omega, \omega_2)$ does indeed have this property.
This will complete the proof.

We isolate the following lemma.
\begin{lemma}[Possible Values Argument]
	Suppose $M$ is a transitive model of $\ZFC$ and $\Po$ is a partial order
	such that $\Po$ has the $\kappa$-chain condition in $M$.
	Let $X,Y \in M$ and let $f: X \to Y$
	be some function in $M[G]$, but $f \notin M$.

	Then there exists a function $F \in M$, with $F: X \to \PP(Y)$ and such that
	for any $x \in X$,
	\[ f(x) \in F(x) \quad\text{and}\quad \left\lvert F(x) \right\rvert^M < \kappa. \]
\end{lemma}
What this is saying is that if $f$ is some new function that's generated,
$M$ is still able to pin down the values of $f$ to at most $\kappa$ many values.

\begin{proof}
	The idea behind the proof is easy: any possible value of $f$ gives us some condition in
	the poset $\Po$ which forces it.
	Since distinct values must have incompatible conditions,
	the $\kappa$-chain condition guarantees
	there are at most $\kappa$ such values.

	Here are the details.
	Let $\dot f$, $\check X$, $\check Y$ be names for $f$, $X$, $Y$.
	Start with a condition $p$ such that $p$ forces the sentence
	``$\dot f$ is a function from $\check X$ to $\check Y$''; we'll work just below here.
	For each $x \in X$, we can consider (using the Axiom of Choice) a maximal antichain $A(x)$
	of incompatible conditions $q \le p$ which forces $f(x)$ to equal some value $y \in Y$.
	Then, we let $F(x)$ collect all the resulting $y$-values.
	These are all possible values, and there are less than $\kappa$ of them.
\end{proof}

\section{Preserving Cardinals}
\begin{definition}
	For $M$ a transitive model of ZFC and $\Po \in M$ a poset,
	we say $\Po$ \vocab{preserves cardinals} if
	$\forall G \subseteq \Po$ an $M$-generic,
	the model $M$ and $M[G]$ agree on the sentence ``$\kappa$ is a cardinal'' for every $\kappa$.

	In the same way we will talk about $\Po$ preserving cofinalities, et cetera.
\end{definition}

\begin{exercise}
	Let $M$ be a transitive model of ZFC.
	Let $\Po \in M$ be a poset.
	Show that the following are equivalent for each $\lambda$:
	\begin{enumerate}[(1)]
		\ii $\Po$ preserves cofinalities less than or equal to $\lambda$.
		\ii $\Po$ preserves regular cardinals less than or equal to $\lambda$.
	\end{enumerate}
	Moreover the same holds if we replace ``less than or equal to''
	by ``greater than or equal to''.
\end{exercise}
Thus, to show that $\Po$ preserves cardinality and cofinalities it suffices to show that $\Po$ preserves regularity.

\begin{theorem}
	Suppose $M$ is a transitive model of ZFC, and $\Po \in M$ is a poset.
	Suppose $M$ satisfies the sentence ``$\Po$ has the $\kappa$ chain condition and $\kappa$ is regular''.
	Then $\Po$ preserves cardinals and cofinalities greater than or equal to $\kappa$.
\end{theorem}
\begin{proof}
	It suffices to show that $\Po$ preserves regularity greater than or equal to $\kappa$.
	Consider $\lambda > \kappa$ which is regular in $M$,
	and suppose for contradiction that $\lambda$ is not regular in $M[G]$.
	That's the same as saying that there is a function $f \in M[G]$,
	$f : \ol \lambda \to \lambda$ cofinal, with $\ol \lambda < \lambda$.
	Then by the Possible Values Argument,
	there exists a function $F \in M$ from $\ol \lambda \to \PP(\lambda)$
	such that $f(\alpha) \in F(\alpha)$ and $\left\lvert F(\alpha) \right\rvert^M < \kappa$
	for every $\alpha$.

	Now we work in $M$ again.
	Note for each $\alpha \in \ol\lambda$, $F(\alpha)$ is bounded in $\lambda$ since $\lambda$ is regular in $M$ and greater than $\left\lvert F(\alpha) \right\rvert$.
	Now look at the function $\ol \lambda \to \lambda$ in $M$ by just
	\[ \alpha \mapsto \cup F(\alpha) < \lambda. \]
	This is cofinal in $M$, contradiction.
\end{proof}

\subsection{Infinite Combinatorics}
In particular, if $\Po$ has the countable chain condition then $\Po$ preserves all the cardinals (and cofinalities).
Therefore, it remains to show that $\opname{Add}(\omega, \omega_2)$ satisfies the countable chain condition.
And this is going to be infinite combinatorics.

\begin{definition}
	Suppose $C$ is an uncountable collection of finite sets.
	$C$ is a \textbf{$\Delta$-system} if there exists a \textbf{root} $R$
	with the condition that for any distinct $X$ and $Y$
	in $C$, we have $X \cap Y = R$.
\end{definition}

\begin{lemma}
	[$\Delta$-System Lemma] Suppose $C$ is an uncountable collection of finite sets.
	Then $\exists \ol C \subseteq C$ such that
	\begin{enumerate}[(1)]
		\ii $\ol C$ is uncountable.
		\ii $\ol C$ is a $\Delta$-system.
	\end{enumerate}
\end{lemma}
\begin{proof}
	There exists an integer $n$ such that $C$ has uncountably many guys of length $n$.
	So we can throw away all the other sets, and just assume that all sets in $C$ have size $n$.

	We now proceed by induction on $n$.
	The base case $n=1$ is trivial, since we can just take $R = \varnothing$.
	For the inductive step we consider two cases.

	First, assume there exists an $a \in C$ contained in uncountably many $F \in C$.
	Throw away all the other guys.
	Then we can just delete $a$, and apply the inductive hypothesis.

	Now assume that for every $a$, only countably many members of $C$ have $a$ in them.
	We claim we can even get a $\ol C$ with $R = \varnothing$.
	First, pick $F_0 \in C$.
	It's straightforward to construct an $F_1$ such that $F_1 \cap F_0 = \varnothing$.
	And we can just construct $F_2, F_3, \dots$
\end{proof}

\begin{lemma}
	For all $\kappa$, $\opname{Add}(\omega, \kappa)$ satisfies the countable chain condition.
\end{lemma}
\begin{proof}
	Assume not. Let
	\[ \left\{ p_\alpha : \alpha < \omega_1 \right\} \]
	be an antichain.  Let
	\[ C = \left\{ \dom(p_\alpha) : \alpha < \omega_1 \right\}. \]
	Let $\ol C \subseteq C$ be such that $\ol C$ is uncountable, and $\ol C$ is a $\Delta$-system which root $R$.
	Then let
	\[ B = \left\{ p_\alpha : \dom(p_\alpha) \in R \right\}. \]
	Each $p_\alpha \in B$ is a function $p_\alpha : R \to \{0,1\}$,
	so there are two that are the same.
\end{proof}

\end{document}
