\documentclass[11pt]{scrreprt}
%%fakesection Load packages

\usepackage{lmodern}
\usepackage[pdfusetitle]{hyperref}
\ExplSyntaxOn
\sys_if_engine_luatex:T {
	\usepackage{luatex85}
}
\sys_if_engine_pdftex:T {
	\usepackage[T1]{fontenc}
}
\ExplSyntaxOff

% These are evan.sty
\usepackage{amsmath,amssymb,amsthm}
\usepackage{mathrsfs}
\usepackage[usenames,svgnames,dvipsnames]{xcolor}
\usepackage{textcomp}
\usepackage{enumerate}
\usepackage[textsize=scriptsize,shadow]{todonotes}
\usepackage{mathtools}
\usepackage{microtype}
\usepackage[normalem]{ulem}
\usepackage{stmaryrd}
\usepackage{wasysym}
\usepackage{multirow}
\usepackage{prerex}
\usepackage[nameinlink]{cleveref}
\usepackage{derivative}

%%fakesection evan.sty macros
%Small commands
%% Napkin commands
\newcommand{\prototype}[1]{
	\emph{{\color{red} Prototypical example for this section:} #1} \par\medskip
}
\newenvironment{moral}{%
	\begin{tcolorbox}[boxrule=0.4pt,colframe=green!70!black,sharp corners,
		standard jigsaw,opacityback=0,left=10pt,right=10pt,top=3pt,bottom=3pt,
		before skip=10pt,after skip=20pt]%
	\bfseries\color{green!50!black}}%
	{\end{tcolorbox}}

%%fakesection Links (hyperref loaded earlier implicitly)
\hypersetup{
	linkcolor={red!50!black},
	citecolor={green!50!black},
	urlcolor={blue!80!black},
	pdfkeywords={napkin,math},
	pdfsubject={web.evanchen.cc},
	colorlinks,
}

%%fakesection Commutative diagrams
\usepackage{tikz-cd}
\usetikzlibrary{arrows,arrows.meta}
% make a larger hook
% https://tex.stackexchange.com/questions/514451/how-to-define-a-new-hooked-arrow
\makeatletter
\pgfdeclarearrow{
	name=xGlyph,
	cache=false,
	bending mode=none,
	parameters={\tikzcd@glyph@len,\tikzcd@glyph@shorten},
	setup code={%
		\pgfarrowssettipend{\tikzcd@glyph@len\advance\pgf@x by\tikzcd@glyph@shorten}},
	defaults={
		glyph axis=axis_height,
		glyph length=+1.55ex,
		glyph shorten=+-0.1ex},
	drawing code={%
		\pgfpathrectangle{\pgfpoint{+0pt}{+-1.5ex}}{\pgfpoint{+\tikzcd@glyph@len}{+3ex}}%
		\pgfusepathqclip%
		\pgftransformxshift{+\tikzcd@glyph@len}%
		\pgftransformyshift{+-\tikzcd@glyph@axis}%
		\pgftext[right,base]{\tikzcd@glyph}}}
\makeatother
\tikzcdset{
	arrow style=tikz,
	diagrams={>={Latex}},
	tikzcd left hook/.tip={xGlyph[glyph math command=supset, swap, glyph axis = 5.7pt]},
	tikzcd right hook/.tip={xGlyph[glyph math command=supset, glyph axis = 5.7pt]},
	surjective head arrow /.tip = {tikzcd to[sep=-1.5pt]tikzcd to},
	surjective head/.style={
		-surjective head arrow
	}
}

%%fakesection Page layout
\usepackage[headsepline]{scrlayer-scrpage}
\renewcommand{\headfont}{}
\addtolength{\textheight}{3.14cm}
\setlength{\footskip}{0.5in}
\setlength{\headsep}{10pt}

\def\shortdate{\leavevmode\hbox{\the\year-\twodigits\month-\twodigits\day}}
\def\twodigits#1{\ifnum#1<10 0\fi\the#1}
\automark[chapter]{chapter}

\rohead{\footnotesize\thepage}
\rehead{\footnotesize \textbf{\sffamily Napkin}, by \emph{Evan Chen} (\napkinversion)}
\lehead{\footnotesize\thepage}
\lohead{\footnotesize \leftmark}
\chead{}
\rofoot{}
\refoot{}
\lefoot{}
\lofoot{}
%\cfoot{\pagemark}

%%fakesection Fancy section and chapter heads
\renewcommand*{\sectionformat}{\color{purple}\S\thesection\autodot\enskip}
\renewcommand*{\subsectionformat}{\color{purple}\S\thesubsection\autodot\enskip}
\newcommand{\problemhead}{A few harder problems to think about}
\renewcommand{\thesubsection}{\thesection.\roman{subsection}}

\addtokomafont{chapterprefix}{\raggedleft}
\RedeclareSectionCommand[beforeskip=0.5em]{chapter}
\renewcommand*{\chapterformat}{%
\mbox{\scalebox{1.5}{\chapappifchapterprefix{\nobreakspace}}%
\scalebox{2.718}{\color{purple}\thechapter\autodot}\enskip}}

\addtokomafont{partprefix}{\rmfamily}
\renewcommand*{\partformat}{\color{purple}\scalebox{2.5}{\thepart}}

%%fakesection Theorems
\usepackage{tcolorbox}
\tcbuselibrary{breakable,skins,hooks}

% patch tcolorbox to support continuation of a paragraph
% after box
% from https://tex.stackexchange.com/questions/568782/
\makeatletter
\tcbset{
	after app={%
		\ifx\tcb@drawcolorbox\tcb@drawcolorbox@breakable
		\else
		% add only when not breakabel
		\@endparenv
		\fi
	}
}

% for breakable
\appto\tcb@use@after@lastbox{\@endparenv\@doendpe}
\makeatother
% END patch tcolorbox for continuation of paragraph


\usepackage{thmtools}

\theoremstyle{definition}
\declaretheoremstyle[
	headfont=\sffamily\bfseries\color{MidnightBlue},
	headpunct={\\[3pt]},
	postheadspace={0pt},
]{thmtheorem}


\declaretheoremstyle[
	headfont=\bfseries\color{RawSienna},
	headpunct={\\[3pt]},
	postheadspace={0pt},
]{thmexample}

\tcbset{
	theorem box/.style={
		enhanced,
		arc=9pt,
		outer arc=10pt,
		colframe=blue,
		colback=TealBlue!5,
		boxrule=1pt,
		before skip=12pt,
		after skip=14pt,
		left=10pt,
		right=10pt
	},
	remark box/.style={
		boxrule=0pt,
		frame hidden,
		sharp corners,
		enhanced,
		borderline west={2pt}{0pt}{ForestGreen},
		before skip=8pt,
		colback=ForestGreen!5,
		after skip=12pt,
		breakable,
		left=10pt,
		right=10pt
	},
	example box/.style={
		enhanced,
		sharp corners,
		arc=9pt,
		outer arc=10pt,
		colframe=RawSienna,
		colback=Salmon!5,
		boxrule=0.5pt,
		before skip=12pt,
		after skip=14pt,
		breakable,
		top=6pt,
		bottom=8pt,
		breakable,
		left=10pt,
		right=10pt
	},
	ques box/.style={
		boxrule=0pt,
		frame hidden,
		enhanced,
		sharp corners,
		before skip=8pt,
		after skip=12pt,
		borderline west={3pt}{0pt}{black},
		colback=RedViolet!5!gray!5,
		breakable,
		left=10pt,
		right=10pt
	}
}

\declaretheorem[style=thmtheorem,name=Theorem,numberwithin=section]{theorem}
\tcolorboxenvironment{theorem}{theorem box}
\declaretheorem[style=thmtheorem,name=Lemma,sibling=theorem]{lemma}
\tcolorboxenvironment{lemma}{theorem box}
\declaretheorem[style=thmtheorem,name=Proposition,sibling=theorem]{proposition}
\tcolorboxenvironment{proposition}{theorem box}
\declaretheorem[style=thmtheorem,name=Corollary,sibling=theorem]{corollary}
\tcolorboxenvironment{corollary}{theorem box}
\declaretheorem[style=thmexample,name=Example,sibling=theorem]{example}
\tcolorboxenvironment{example}{example box}

\declaretheoremstyle[
	headfont=\bfseries\sffamily\color{ForestGreen!70!black},
	bodyfont=\normalfont,
	headpunct={ --- },
]{thmremark}
\declaretheoremstyle[
	headfont=\bfseries\sffamily\color{ForestGreen!70!black},
	bodyfont=\normalfont,
	headpunct={},
]{thmremark*}

\declaretheoremstyle[
	headfont=\bfseries,
	bodyfont=\normalfont\small
]{thmques}

\declaretheorem[name=Question,sibling=theorem,style=thmques]{ques}
\tcolorboxenvironment{ques}{ques box}
\declaretheorem[name=Exercise,sibling=theorem,style=thmques]{exercise}
\tcolorboxenvironment{exercise}{ques box}
\declaretheorem[name=Remark,sibling=theorem,style=thmremark]{remark}
\tcolorboxenvironment{remark}{remark box}
\declaretheorem[name=Remark,sibling=theorem,style=thmremark*]{remark*}
\tcolorboxenvironment{remark*}{remark box}
\declaretheorem[name=Step,style=thmremark]{step} % only used in Lebesgue int
\tcolorboxenvironment{step}{remark box}

\theoremstyle{definition}
\newtheorem{claim}[theorem]{Claim}
\newtheorem{definition}[theorem]{Definition}
\newtheorem{fact}[theorem]{Fact}
\newtheorem{abuse}[theorem]{Abuse of Notation}

\newtheorem{problem}{Problem}[chapter]
\renewcommand{\theproblem}{\thechapter\Alph{problem}}
\newtheorem{sproblem}[problem]{Problem}
\newtheorem{dproblem}[problem]{Problem}
\renewcommand{\thesproblem}{\theproblem$^{\star}$}
\renewcommand{\thedproblem}{\theproblem$^{\dagger}$}
\newcommand{\listhack}{$\empty$\vspace{-2em}}

%%fakesection Answers
\usepackage{answers}
\Newassociation{hint}{answeritem}{tex/backmatter/all-hints}
\Newassociation{sol}{answeritem}{tex/backmatter/all-solns}
\renewcommand{\solutionextension}{out}
\renewenvironment{answeritem}[1]{\item[\bfseries #1.]}{}

%%fakesection Table of contents
% First add ToC to ToC
\makeatletter
\usepackage{etoolbox}
\pretocmd{\tableofcontents}{%
	\if@openright\cleardoublepage\else\clearpage\fi
	\pdfbookmark[0]{\contentsname}{toc}%
}{}{}%
\makeatother
\setcounter{tocdepth}{1}
\RedeclareSectionCommand[tocnumwidth=4.2em]{part}
\RedeclareSectionCommand[tocpagenumberwidth=2.2em,tocnumwidth=4.2em]{chapter}
\RedeclareSectionCommand[tocpagenumberwidth=2.2em,tocnumwidth=2.8em]{section}
% adjust tocpagenumberwidth manually for large page number: https://tex.stackexchange.com/a/502168

%%fakesection Asymptote definitions
\usepackage{patch-asy}
\numberwithin{asy}{chapter}
\renewcommand{\theasy}{\thechapter\Alph{asy}}
\begin{asydef}
	import extras;
	size(6cm);
	usepackage("amsmath");
	usepackage("amssymb");
	defaultpen(fontsize(11pt));
	settings.tex = "latex";
	settings.outformat = "pdf";
\end{asydef}
\def\asydir{asy}

%%fakesection Bibliography
\usepackage[backend=biber,backref=true,style=alphabetic]{biblatex}
\DeclareLabelalphaTemplate{
	\labelelement{
		\field[final]{shorthand}
		\field{label}
		\field[strwidth=2,strside=left]{labelname}
	}
	\labelelement{
		\field[strwidth=2,strside=right]{year}
	}
}
\DeclareFieldFormat{labelalpha}{\textbf{\scriptsize #1}}
\addbibresource{references.bib}
\addbibresource{images.bib}
%% stylistic biblatex choices
\DefineBibliographyStrings{english}{%
	backrefpage  = {cited p.}, % for single page number
	backrefpages = {cited pp.} % for multiple page numbers
}
\DeclareFieldFormat{journaltitle}{\mkbibemph{#1},} % italic journal title with comma
\DeclareFieldFormat[inbook,thesis]{title}{\mkbibemph{#1}\addperiod} % italic title with period
\DeclareFieldFormat[article]{title}{#1} % title of journal article is printed as normal text
\DeclareFieldFormat[article]{volume}{\textbf{#1}\addcolon\space}
\renewcommand{\mkbibnamegiven}[1]{\textsc{#1}}
\renewcommand{\mkbibnamefamily}[1]{\textsc{#1}}
\renewcommand{\mkbibnameprefix}[1]{\textsc{#1}}
\renewcommand{\mkbibnamesuffix}[1]{\textsc{#1}}
\renewcommand{\finentrypunct}{}

%%fakesection Mini ToC
\usepackage[tight]{minitoc}
\mtcsetfont{parttoc}{chapter}{\sffamily\bfseries}
\mtcsetfont{parttoc}{section}{\footnotesize\rmfamily\upshape\mdseries}
\mtcsetfont{parttoc}{subsection}{\footnotesize\rmfamily\upshape\mdseries}
%\mtcsetdepth{parttoc}{1}
\setcounter{parttocdepth}{1}
\renewcommand*{\partheadstartvskip}{\vspace*{20em}}
\renewcommand*{\partheadendvskip}{}
%\noptcrule
\renewcommand\beforeparttoc{\noindent{\bfseries \Large Part \thepart: Contents}}
%\hspace{\fill}\rule{0.95\linewidth}{2pt}\hspace{\fill}
\doparttoc[n]

%%fakesection Misc haxx
\pdfstringdefDisableCommands{\def\Spec{\text{Spec }}\def\sigma{σ}}

\def\asydir{}
\renewcommand{\gim}{*}
\renewcommand{\yod}{**}
\renewcommand{\kurumi}{***}

\begin{document}
\title{Algebraic Geometry II: Schemes}
\maketitle

\tableofcontents

\section{Blah}
\label{thm:reg_func_distinguish_open}
\label{ex:product_ring}

\chapter{Sheaves and ringed spaces}
Most of the complexity of the affine variety $V$ earlier comes from $\OO_V$.
This is a type of object called a ``sheaf''.
The purpose of this chapter is to completely define what this sheaf is,
and just what it is doing.

The typical example to keep in mind is a sheaf of
``functions with property $P$'' on a topological space $X$:
for every open set $U$, $\SF(U)$ gives us the ring of functions on $X$.
However, we will work very abstractly and only assume $\SF(U)$
is a ring, without an interpretation as ``functions''.

The payoff for this abstraction is that it will
allow us to define an arbitrary scheme in the next chapter.
Varieties use $\CC[x_1, x_2, \dots, x_n] / I$ as their ``ring of functions'',
and by using the fully general sheaf we will be replace this
with \emph{any} commutative ring.
In particular, we can use the case where $I$ is not radical, such as
$\CC[x] / (x^2)$; this gives the ``multiplicity''
behavior that we sought after all along.  

\section{Pre-sheaves}
\prototype{The sheaf of holomorphic (or regular, continuous,
differentiable, constant, whatever) functions.}

The proper generalization of our $\OO_V$ is a so-called sheaf of rings.
Recall that $\OO_V$ was a took \emph{open sets of $V$} to \emph{rings},
with the interpretation that $\OO_V(U)$ was a ``ring of functions''.

In light of this, we first make the following definition
for the general construct, in the language of categories.

\begin{definition}
	For a topological space $X$ let $\Opens(X)$ denote
	its open sets of $X$.
\end{definition}
\begin{definition}
	A \vocab{pre-sheaf} of rings on a space $X$ is a function
	\[ \SF : \Opens(X) \to \catname{Rings} \]
	meaning each open set gets associated with a ring $\SF(U)$.
	Each individual element of $\SF(U)$ is called a \vocab{section}.
	An element of $\SF(X)$ is called a \vocab{global section}.

	It is also equipped with the following information:
	for any $U_1 \subseteq U_2$ there is a \vocab{restriction map}
	\[ \res_{U_1,U_2} : \SF(U_2) \to \SF(U_1) \]
	such that $\res_{U,U}$ is the identity and
	whenever $U_1 \subseteq U_2 \subseteq U_3$ the diagram
	\begin{diagram}
		\SF(U_3) & \rTo^{\res_{U_2,U_3}} & \SF(U_2) \\
		& \rdTo_{\res_{U_1,U_3}} & \dTo_{\res_{U_1,U_2}} \\
		&& \SF(U_1)
	\end{diagram}
	commutes. (Restricting ``big to medium to small''
	is the same as ``big to small''.)
\end{definition}

\begin{abuse}
	If $s \in \mathscr F(U_2)$ is some section and $U_1 \subseteq U_2$,
	then rather than write $\res_{U_1,U_2}(s)$
	I will write $s\restrict{U_1}$ instead:
	``$s$ restricted to $U_1$''.
	This is abuse of notation because the section $s$ is just
	an element of some ring, and in the most abstract of cases
	may not have a natural interpretation as function.
\end{abuse}

\begin{example}[Examples of pre-sheaves]
	\listhack
	\begin{enumerate}[(a)]
		\ii For an affine variety $V$, $\OO_V$ is of course a sheaf,
		with $\OO_V(U)$ being the ring of regular functions on $U$.
		The restriction map just says that if $U_1 \subseteq U_2$,
		then a function $s \in \OO_V(U_2)$ can also be thought of as
		a function $s \restrict{U_1} \in \OO_V(U_1)$,
		hence the name ``restriction''.
		The commutativity of the diagram then follows.
		
		\ii Let $X \subseteq \RR^n$ be an open set.
		Then there is a sheaf of smooth/differentiable/etc.\ functions on $X$.
		In fact, one can do the same construction for any manifold $M$.

		\ii Similarly, if $X \subseteq \CC$ is open,
		we can construct a sheaf of holomorphic functions on $X$.
	\end{enumerate}
	In all these examples, the sections $s \in \SF(U)$
	are really functions on the space, but in general they need not be.
\end{example}

Now, we give a second, equivalent and far shorter definition of pre-sheaf:
\begin{abuse}
	By abuse of notation, $\Opens(X)$ will also be thought of as a
	posetal category by inclusion. Thus $\varnothing$ is an initial object
	and the entire space $X$ is a terminal object.
\end{abuse}
\begin{definition}
	A \vocab{pre-sheaf} of rings on $X$ is a contravariant functor
	\[ \SF : \Opens(X)\op \to \catname{Rings}. \]
\end{definition}
\begin{ques}
	Check that these definitions are equivalent.
\end{ques}
It is now clear that we can actually replace $\catname{Rings}$
with any category we want.

So we should think:
\begin{moral}
	Pre-sheaves should be thought of as
	``returning the ring of functions with a property $P$''.
\end{moral}

\section{Sheaves}
\prototype{Constant functions aren't pre-sheaves,
	but locally constant ones are.}

Now, the main idea next is that
\begin{moral}
	Sheaves are pre-sheaves for which $P$ is a \emph{local} property.
\end{moral}

The formal definition doesn't illuminate this as much as the examples
do, but I have to give it first for the examples to make sense.

\begin{definition}
	A \vocab{sheaf} $\mathscr F$ is a pre-sheaf satisfying the following
	two additional axioms:
	Suppose $U$ is covered by open sets $U_\alpha \subseteq U$. Then:
	\begin{enumerate}
		\ii (Identity) If $s, t \in \mathscr F(U)$ are sections,
		and $s\restrict{U_\alpha} = t\restrict{U_\alpha}$
		for all $\alpha$, then $s = t$.
		\ii (Collation) Consider sections
		$s_\alpha \in \mathscr(U_\alpha)$ for each $\alpha$.
		Suppose that 
		\[ s_\alpha \restrict{U_\alpha \cap U_\beta}
			= s_\beta \restrict{U_\alpha \cap U_\beta} \]
		for each $U_\alpha$ and $U_\beta$.
		Then we can find $s \in U$ such that
		$s \restrict{U_\alpha}  = s_\alpha$.
	\end{enumerate}
\end{definition}
This is best illustrated by picture: consider an open cover $U_1 \cup U_2$.
\begin{center}
	\begin{asy}
		size(4cm);
		filldraw(shift(-0.5,0)*unitcircle, lightred+opacity(0.3), red);
		filldraw(shift( 0.5,0)*unitcircle, lightblue+opacity(0.3), blue);
		label("$U_1$", (-0.5,0)+dir(135), dir(135), red);
		label("$U_2$", ( 0.5,0)+dir(45), dir(45), blue);
	\end{asy}
\end{center}
Then for a sheaf of functions, the axioms are saying that:
\begin{itemize}
	\ii If $s$ and $t$ are functions (with property $P$)
	on the whole space $U = U_1 \cup U_2$,
	and $s \restrict{U_1} = t \restrict{U_1}$,
	$s \restrict{U_2} = t \restrict{U_2}$,
	then $s = t$ on the entire union.
	This is clear.

	\ii If $s_1$ is a function with property $P$ on $U_1$
	and $s_2$ is a function with property $P$ on $U_2$,
	and the two functions agree on the overlap,
	then one can collate them to obtain a function $s$
	on the whole space:
	this is obvious, but \textbf{the catch is that the collated function
	needs to have property $P$ as well.}
	That's why it matters that $P$ is local.
\end{itemize}

The reason we need these axioms is that in our abstract definition of a sheaf,
the output of the sheaf is an abstract ring, which need not actually
have a concrete interpretation as ``functions on $X$'', even though
our examples will usually have this property.

Now for the examples, which are more enlightening:
\begin{example}
	[Examples and non-examples of sheaves]
	\listhack
	\begin{enumerate}[(a)]
		\ii Pre-sheaves of arbitrary / continuous / differentiable / smooth
		/ holomorphic functions are still sheaves.
		This is because to verify a function is continuous,
		one only needs to look at small neighborhoods at once.
		
		\ii For a complex variety $V$, $\OO_V$ is a sheaf,
		precisely because our definition was \emph{locally} quotients
		of polynomials.

		\ii The pre-sheaf of \emph{constant} real functions on a space $X$
		is \emph{not} a sheaf, because it fails the collation axiom.
		Namely, suppose that $U_1 \cap U_2 = \varnothing$.
		Then if $s_1$ is the constant function $1$ on $U_1$
		while $s_2$ is the constant function $2$ on $U_2$,
		then we cannot collate these to a constant function on $U_1 \cup U_2$.

		\ii On the other hand, \emph{locally constant} functions
		do produce a sheaf. (A function is locally constant
		if for every point it is constant on some neighborhood.)
	\end{enumerate}
	In fact, the sheaf in (c) is what is called a \emph{sheafification}
	of the pre-sheaf constant functions, which we define momentarily.
\end{example}

\section{Stalks}
\prototype{Germs of real smooth functions tell you the derivatives,
but germs of holomorphic functions determine the entire function.}
Let $\SF$ be a pre-sheaf.
If we have a function $s \in \SF(U)$ and a point $p \in U$,
then in general it doesn't make sense to ask what $s(p)$ is
(even though all our examples look like this),
because $\SF(U)$ is an arbitrary ring.
So, we will replace the notion of $s(p)$ with a so-called \emph{germ}.

\begin{definition}
	Let $\SF$ be a pre-sheaf of rings.
	For every point $p$ we define the \vocab{stalk} $\SF_p$ to be the set
	\[ \left\{ (s, U) \mid s \in \SF(U), p \in U \right\} \]
	modulo the relation $\sim$ that 
	\[ (s_1,U_1) \sim (s_2, U_2) \text{ if }
		s_1 \restrict{U_1 \cap U_2} = s_2 \restrict{U_1 \cap U_2}. \]
	The equivalence classes themselves are called \vocab{germs}.
\end{definition}

\begin{definition}
	The germ of a given $s \in \SF(U)$ at a point $p$
	is the equivalence class for $(s,U) \in \SF_p$.
	We denote this by $[s]_p$.
\end{definition}

\begin{remark}
	It is almost never useful to think of a germ as an ordered pair,
	since the set $U$ can get arbitrarily small.
	Instead, one should think of a germ as a
	``shred'' of some section near $p$.
\end{remark}

So what's happening is: We consider functions $s$ defined near $p$
and, since we only care about local behavior,
we identify any two functions agreeing on a neighborhood of $p$,
no matter how small.

Notice that the stalk is itself a ring as well:
for example, addition is done by
\[ 
	\left( s_1, U_1 \right) + \left( s_2, U_2 \right)
	=
	\left( s_1 \restrict{U_1 \cap U_2} + s_2 \restrict{U_1 \cap U_2},
	U_1 \cap U_2 \right).
\]

So the germ $[s]_p$ now plays the role of the ``value'' $s(p)$.
But actually, it carries much more information than that.

\begin{example}
	[Germs of real smooth functions]
	Let $X = \RR$ and let $\SF$ be the sheaf on $X$ of smooth functions
	(i.e.\ $\SF(U)$ is the set of smooth real-valued functions on $U$).

	Consider a global section, $s : \RR \to \RR$ (thus $s \in \SF(X)$)
	and its germ at $0$.
	\begin{enumerate}[(a)]
		\ii From the germ we can read off $s(0)$, obviously.
		\ii We can also find $s'(0)$, because the germ carries enough
		information to compute the limit $\lim_{h \to 0} \frac1h[s(h)-s(0)]$.
		\ii Similarly, we can compute the second derivative and so on.
		\ii However, we can't read off, say, $s(3)$ from the germ.
		For example, take
		\[
			s(x) = \begin{cases}
				e^{-\frac{1}{x-1}} & x > 1 \\
				0 & x \le 1.
			\end{cases}
		\]
		Note $s(3) = e^{-\half}$, but $[\text{zero function}]_0 = [s]_0$.
		So germs can't distinguish between the zero function and $s$.
	\end{enumerate}
\end{example}
\begin{example}
	[Germs of holomorphic functions]
	Holomorphic functions are very strange in this respect.
	Consider the sheaf $\SF$ on $\CC$ of \emph{holomorphic} functions.

	Take $s : \CC \to \CC$ a global section.
	Given the germ of $s$ at $0$, we can read off $s(0)$, $s'(0)$, et cetera.
	The miracle of complex analysis is that just knowing
	the derivatives of $s$ at zero is enough to reconstruct all of $s$:
	we can compute the Taylor series of $s$ now.
	\textbf{Thus germs of holomorphic functions determine the entire function};
	they ``carry much more information'' than their real counterparts.
	
	In particular, we can concretely describe the sheaf:
	\[
		\SF_p = \left\{
			\sum_{k \ge 0} c_k (z-p)^k
			\text{ convergent near $p$}
		\right\}.
	\]
	In particular, this includes germs of meromorphic functions,
	so long as there is no pole at $p$ itself.
\end{example}

And of course, our algebraic geometry example.
\begin{abuse}
	Rather than writing $(\OO_V)_p$ we will write $\OO_{V,p}$.
\end{abuse}
\begin{example}
	[Stalks of the sheaf on an affine variety]
	Let $V \subseteq \Aff^n$ be a variety, and assume $p \in V$.
	Then, a regular function $\varphi$ on $U \subseteq V$
	is supposed to be a function on $U$ that ``locally'' is a quotient
	of two functions in $\CC[V]$.

	However, \textbf{as far as the germ is concerned, we only care about
	whichever quotient applies near the point $p$}.
	In light of this, we only think about the representation at $p$,
	and ignore the local clause completely:
	thus the entire stalk can be thought of as
	\[
		\OO_{V,p} = 
		\left\{ \left( \tfrac fg , U \right) \mid 
			U \ni p, \; f,g \in \CC[V], \;
			\text{$g \neq 0$ on $U$} \right\}
	\]
	modulo the usual relations.

	Now, since we happen to be working with complex polynomials,
	we know that a rational function is determined by its
	behavior on any neighborhood of $p$
	(complex analysis forever!).
	Thus 
	\[
		\OO_{V,p} =
		\left\{ \tfrac fg \mid f,g \in \CC[V], \; g(p) \neq 0 \right\}.
	\]
	which don't vanish on $p$.
\end{example}

\begin{remark}
	[For category lovers]
	You may notice that $\SF_p$ seems to be
	``all the $\SF_p(U)$ coming together'', where $p \in U$.
	And in fact, $\SF_p(U)$ is the categorical \emph{limit}
	of the diagram formed by all the $\SF(U)$ such that $p \in U$.
	This is often written
	\[ \SF_p = \varinjlim_{U \ni p} \SF(U) \]
	Thus we can define stalks in any category with limits,
	though to be able to talk about germs the category needs
	to be concrete.
\end{remark}

\section{Sections ``are'' sequences of germs}
\prototype{A real function on $U$ is a sequence of
	real numbers $f(p)$ for each $p \in U$ satisfying some local condition.
	Analogously, a section $s \in \SF(U)$ is a sequence of germs
	satisfying some local compatibility condiiton.
}
Let $\SF$ be a sheaf on a space $X$ now.
The purpose of this section is to convince you that the correct way 
think about a section $s \in \SF(U)$ is as a sequence of germs
above every point $p \in U$.

How do we do this?
Given $s \in \SF(U)$ we consider its germ $[s]_p$ above every for $p \in U$.
To visualize, picture an open set $U \subseteq X$,
and above every point $p \in P$ imagine there is a little speck $[s]_p$,
which bundles the information of $s$ near $p$.
\begin{example}[Real functions vs.\ germs]
	Let $X$ be a space and let $\SF$ be the sheaf of smooth functions 
	Given a section $f \in \SF(U)$,
	\begin{itemize}
		\ii As a function, $f$ is just a choice of value $f(p) \in \RR$ at
		every point $p$, subject to a local ``smooth'' condition.

		\ii Let's now think of $f$ as a sequence of germs.
		At every point $p$ the germ $[f]_p \in \SF_p$ gives us the value $f(p)$
		as we described above. The germ packages even more data than this:
		from the germ $[f]_p$ alone we can for example compute $f'(p)$.
		Nonetheless we stretch the analogy and think of $f$
		as a choice of germ $[f]_p \in \SF_p$ at each point $p$.
	\end{itemize}
	Thus we can replace the notion of the value $f(p)$ with germ $[f]_p$.
	This is useful because in a general sheaf $\SF$, the notion $s(p)$
	is not defined while the notion $[s]_p$ is.
\end{example}

From the above example it's obvious that if we know each germ $[s]_p$
this should let us reconstruct the entire section $s$.
Let's check this from the sheaf axioms:
\begin{exercise}
	[Sections are determined by stalks]
	Let $\SF$ be a sheaf.
	Consider the natural map $\SF(U) \to \prod_{p \in U} \SF_p$ described above.
	Show that this map is injective, i.e.\
	the germs of $s$ at every point $p \in U$ determine the section $s$.
	(You will need the ``identity'' sheaf axiom, but not ``collating''.)
\end{exercise}

However, this map is clearly not surjective!
\begin{ques}
	Come up with a counterexample to break surjectivity.
	(This is like asking ``come up with a non-smooth function''.)
\end{ques}
Nonetheless we can describe the image:
we want a sequence of germs $(g_p)_{p \in U}$ such that near every germ $g_p$,
the germs $g_q$ are ``compatible'' with $g_p$.
We make this precise:
\begin{definition}
	Let $\SF$ be sheaf (or even a pre-sheaf) and $U$ open.
	A sequence $(g_p)_{p \in U}$ of germs
	($g_p \in \SF_p$) is said to be \vocab{compatible} if
	they can be ``locally collated'':
	\begin{quote}
		For any $p \in U$ there exists a neighborhood $U_p \ni p$
		and a section $s \in \SF(U_p)$ on it
		such that $[s]_q = g_q$ for each $q \in U_p$.
	\end{quote}
	Intuitively, the germs should ``collate together'' to some section near
	each \emph{individual} point $q$
	(but not necessarily to a section on all of $U$).
\end{definition}

We let the reader check this definition is what we want:
\begin{exercise}
	Prove that any choice of compatible germs over $U$
	collates together to a section of $U$.
	(You will need the ``collating'' sheaf axiom, but not ``identity''.)
\end{exercise}

Putting together the previous two exercise gives:
\begin{theorem}
	[Sections ``are'' just compatible germs]
	Let $\SF$ be a sheaf.
	There is a natural bijection between
	\begin{itemize}
		\ii sections of $\SF(U)$, and
		\ii sequences of compatible germs over $U$.
	\end{itemize}
\end{theorem}
This is in exact analogy to the way that e.g.\
a smooth real-valued function on $U$ is a choice
of real number $f(p) \in \RR$ at each point $p \in U$
satisfying a local smoothness condition.

Thus the notion of stalks is what lets us recover the viewpoint
that sections are ``functions''.  Therefore for theoretical purposes,
\begin{moral}
	You should usually think of a section $s \in \SF(U)$ as a sequence of germs.
\end{moral}
In particular, this makes restriction morphisms easy to deal with:
just truncate the sequence of germs!

\section{Sheafification}
\prototype{The pre-sheaf of locally constant functions
	becomes the sheaf of constant functions.}

Now that we have the language of germs,
we can define the so-called sheafification.
The idea is that if $\SF$ is the pre-sheaf of ``functions with property $P$''
then we want to associate a sheaf $\SF\sh$ of
``functions which are locally $P$'', which makes them into a sheaf.
We have already seen two examples of this:
\begin{example}
	[Sheafification]
	\listhack
	\begin{enumerate}[(a)]
		\ii If $X$ is a topological space,
		and $\SF$ is the pre-sheaf of constant functions on open sets of $X$,
		then $\SF\sh$ is the sheaf of locally constant functions.

		\ii If $V$ is an affine variety,
		and $\SF$ is the pre-sheaf of rational functions,
		then $\SF\sh$ is the sheaf of regular functions
		(which are locally rational).
	\end{enumerate}
\end{example}

So how do we encode ``locally $P$''?
Answer: we saw last time that for a sheaf $\SF$,
compatible germs biject to sections.
So we \emph{force} this to be true by defining:
\begin{definition}
	The \vocab{sheafification} $\SF\sh$ of a pre-sheaf $\SF$ is defined by
	\[ \SF\sh(U) =
		\left\{ \text{sequences of compatible
		germs $(g_p)_{p \in U}$} \right\}.  \]
\end{definition}
\begin{exercise}
	Reconcile this definition with the two examples we gave.
\end{exercise}
\begin{abuse}
	Technically I haven't told you what the restriction morphisms
	of $\SF\sh(U)$ are but, they are not hard to guess.
	I'll usually be equally sloppy in the future:
	when defining a sheaf $\SF$, I'll only say what $\SF(U)$ is,
	with the restriction morphisms $\SF(U_2) \to \SF(U_1)$ being implicit.
\end{abuse}
The construction is contrived so that given a section
$(g_p)_{p \in U} \in \SF\sh(U)$ the germ at a point $p$ is $g_p$:
\begin{lemma}
	[Pre-sheaves and sheaves have the same stalk]
	\label{lem:pre_sheaf_stalk}
	Let $\SF$ be a pre-sheaf and $\SF\sh$ its sheafification.
	Let $q \in U$ with $U$ an open set.
	Then there is an isomorphism
	\[ (\SF\sh)_q \cong \SF_q. \]
\end{lemma}
\begin{proof}
	A germ in $(\SF\sh)_q$ looks like 
	$\left( (g_p)_{p \in U}, U \right)$,
	where $g_p = (s_p, U_p)$ are themselves germs of $\SF_p$.
	Then the isomorphism is given by
	\[ \left( (g_p)_{p \in U}, U \right) \mapsto g_q \in \SF_q. \]
	The inverse map is given by for each $g = (s,U) \in \SF_q$ by
	\[ g \mapsto \left( (g)_{p \in U}, U \right) \in (\SF\sh)_q \]
	i.e.\ the sequence of germs is the constant sequence.
\end{proof}

We'll later see that if $\SF$ is already a sheaf, then $\SF\sh = \SF$.

\section{Morphisms of sheaves}
First, recall that a sheaf is a contravariant functor (pre-sheaf)
with extra conditions. In light of this, it is not hard to guess
the definition of a morphism of pre-sheaves:
\begin{definition}
	A \vocab{morphism of (pre-)sheaves} $\alpha : \SF \to \SG$ on the same
	space $X$ is a \textbf{natural transformation} of the underlying functors.
	Isomorphism of sheaves is defined in the usual way.
\end{definition}
\begin{ques}
	Show that this amounts to: for each $U \subseteq X$ we need to specify
	a morphism $\alpha_U : \SF(U) \to \SG(U)$ such that the diagram
	\begin{diagram}
		\SF(U_2) & \rTo^{\alpha_{U_2}} & \SG(U_2) \\
		\dTo^{\res_{U_1, U_2}} && \dTo_{\res_{U_1, U_2}} \\
		\SF(U_1) & \rTo_{\alpha_{U_1}} & \SG(U_1)
	\end{diagram}
	commutes any time that $U_1 \subseteq U_2$.
\end{ques}
However, in the sheaf case we like stalks more than sections because
they are theoretically easier to think about.
And in fact:
\begin{proposition}
	[Morphisms determined by stalks]
	A morphism of sheaves $\alpha : \SF \to \SG$ induces a morphism of stalks
	\[ \alpha_p : \SF_p \to \SG_p \]
	for every point $p \in X$.
	Moreover, the sequence $(\alpha_p)_{p \in X}$ determines $\alpha$ uniquely.
\end{proposition}
\begin{proof}
	The morphism $\alpha_p$ itself is just
	$(s, U) \xmapsto{\alpha_p} (\alpha_U(s), U)$.
	\begin{ques}
		Show this is well-defined.
	\end{ques}
	Now suppose $\alpha , \beta : \SF \to \SG$ satisfy $\alpha_p = \beta_p$
	for every $p$. We want to show $\alpha_U(s) = \beta_U(s)$
	for every $s \in U$.
	\begin{ques}
		Verify this using the description of sections
		as sequences of germs. \qedhere
	\end{ques}
\end{proof}
Thus a morphism of sheaves can be instead modelled as a morphism
of all the stalks. We will see later on that this viewpoint is quite useful.


\section{Local rings, and locally ringed spaces}
\prototype{Smooth functions $f$ on $X \subseteq \RR^n$ have invertible
germs at $p$ unless $f(p) = 0$.}
The stalks of the examples we produced above are special types
of rings, called \emph{local rings}.
Algebraically, the definition of these is:
\begin{definition}
	A \vocab{local ring} $R$ is a ring with exactly one maximal ideal.
\end{definition}
\begin{exercise}
	Show that a ring $R$ is a local ring if there exists
	a proper ideal $\mm \subsetneq R$ such that
	all elements of $R \setminus \mm$ are units.
	(Bonus: prove the converse.)
\end{exercise}

% Wikipedia has a good explanation here
To see why this definition applies to the stalks above,
we need to identify what the maximal ideal is.
Let's go back to the example of $X = \RR$ and $\SF(U)$ the smooth functions,
and consider the stalk $\SF_{p}$, where $p \in X$.
Define the ideal $\mm_p$ to be the set of germs $(s,U)$ for which $s(p) = 0$.

Then $\mm_p$ is maximal: we have an exact sequence
\[ 0 \to \mm_p \to \SF(U) \taking{(s,U) \mapsto s(p)} \RR \to 0 \]
and so $\SF(U) / \mm_p \cong \RR$, which is a field.

It remains to check there are no nonzero maximal ideals.
Now note that if $s \notin \mm_p$,
then $s$ is nonzero in some neighborhood of $p$,
then one can construct the function $1/s$ in a neighborhood of $p$.
So \textbf{every element of $\SF_p \setminus \mm_p$ is a unit};
$\mm_p$ is in fact the only maximal ideal!

More generally,
\begin{moral}
	If $\SF$ consists of ``field-valued functions'',
	the stalk $\SF_p$ probably has a maximal ideal
	consisting of the germs vanishing at $p$.
\end{moral}
The discussion above implies, for example, the following.
\begin{proposition}
	[Stalks are often local rings]
	The stalks of each of the following types of sheaves are local rings:
	\begin{enumerate}[(a)]
		\ii Sheaves of continuous real/complex functions on a topological space
		\ii Sheaves of smooth functions on any manifold
		\ii Regular functions on an algebraic variety $V$.
	\end{enumerate}
\end{proposition}

We can now define:
\begin{definition}
	A \vocab{ringed space} is a topological space $X$ equipped
	with a sheaf $\OO_X$ of rings.
	Suppose that for every point $p$, the stalk $\OO_{X,p}$
	is a local ring.
	Then we say that $\OO_X$ is a \vocab{locally ringed space}.
	We denote the maximal ideals by $\mm_{X,p}$.
\end{definition}

In particular, in the previous chapter we showed that every
affine variety could be built into a locally ringed space. Hooray!
\begin{abuse}
	A ringed space $(X, \OO_X)$ is abbreviated to just $X$,
	while $p \in X$ means ``$p$ is in the topological space $X$''.
\end{abuse}


\section{Morphisms of (locally) ringed spaces}
Finally, it remains to define a morphism of locally ringed space.
To do this we have to build up in several steps.

\begin{remark}
	I always secretly felt that one can probably get away with just
	suspending belief and knowing ``there is a reasonable definition
	of morphisms of locally ringed spaces''.
	Daring readers are welcome to try this approach.
\end{remark}

\subsection*{Morphisms of ringed spaces}
Suppose we have ringed spaces $X = (X, \OO_X)$ and $Y = (Y, \OO_Y)$,
and we want to define a map
\[ f : X \to Y. \]

On the level of spaces, $f : X \to Y$ should of course be a continuous map.
But we also want to $f$ to do something with $\OO_X$ and $\OO_Y$.
However, we only defined how to take a morphism of sheaves
that act on the \emph{same} space!
Thus we impose the following definition,
which lets us \emph{push} the sheaf on $X$ to a sheaf on $Y$.
Picture:
\begin{diagram}
	X && \rTo^f & Y && X && \rTo^f & Y \\
	\Opens(X)\op && \lTo^{f\pre} & \Opens(Y)\op
	&& f\pre(U) && \lMapsto^{f\pre} & U & \\
	& \rdMapsto(3,2)_{\SF} && \dMapsto_{f_\ast \SF}
		&& & \rdMapsto(3,2)_{\SF} && \dMapsto_{f_\ast \SF} & \\
	&&& \catname{Rings}
		&& && \SF(f\pre(U)) & = (f_\ast \SF)(U) & \in \catname{Rings}
\end{diagram}

\begin{definition}
	Let $\SF$ be a sheaf on $X$, and $f : X \to Y$ a continuous map.
	The \vocab{pushforward sheaf} $f_\ast \SF$ on $Y$ is defined by
	\[ (f_\ast \SF)(U) = \SF(f\pre(U)) \qquad \forall U \subseteq Y. \]
	This makes sense, since $f\pre(U)$ is open in $X$.
\end{definition}
As $f\pre$ is a functor $\Opens(Y)\op \to \Opens(X)\op$,
\textbf{the pushforward $f_\ast\SF$ is
just the composition of these two functors}.

\begin{ques}
	Technically $f_\ast\SF$ is supposed to be a
	functor $\Opens(Y)\op \to \catname{Rings}$, so it also needs
	to come with restriction arrows. What are they?
\end{ques}

I haven't actually checked that $f_\ast \SF$ is a sheaf
(as opposed to a pre-sheaf), but this isn't hard to do.

Now we can define a morphism of ringed spaces.
In addition to the topological spaces,
we also include a natural transformation between
the two structure sheafs, after pushing the one on $X$ forward.
\begin{definition}
	A \vocab{morphism of ringed spaces} $f : (X, \OO_X) \to (Y, \OO_Y)$
	consists of a continuous map of topological spaces $f : X \to Y$,
	as well as a morphism of sheaves $f^\ast : \OO_Y \to (f_\ast \OO_X)$.
\end{definition}
The latter is thus a map
$f^\ast_U : \OO_Y(U) \to (f_\ast \OO_X)(U) = \OO_X(f\pre(U))$
for every $U \subseteq Y$, satisfying
the axioms for a natural transformation.

\subsection*{Morphisms of locally ringed spaces}
Last step!
Suppose now that $(X, \OO_X)$ and $(Y, \OO_Y)$ are locally ringed spaces.
Thus we need to deal with some information about the stalks.

Given a morphism of ringed spaces $f : (X, \OO_X) \to (Y, \OO_Y)$,
we can actually use the $f^\ast_U$ above to induce maps on the stalks,
as follows. For a point $p \in X$, construct the map
\begin{diagram}
	f^\ast_P :& \OO_{Y, f(p)} & \rTo && \OO_{X, f(p)} \\
	& (s, U) & \rMapsto && (f^\ast_U(s), f\pre(U)).
\end{diagram}
where $s \in \OO_Y(U)$.

\begin{definition}
	Let $R$ and $S$ be local rings with maximal ideals $\mm_R$ and $\mm_S$.
	A \vocab{morphism of local rings} is a homomorphism of rings
	$\psi : R \to S$ such that $\psi\pre(\mm_S) = \mm_R$.
\end{definition}
\begin{definition}
	A \vocab{morphism of locally ringed spaces}
	is a morphism of ringed spaces $f : (X, \OO_X) \to (Y, \OO_Y)$ such that
	for every point $p$ the induced map of stalks is a morphism of local rings.
\end{definition}
Recalling that $\mm_{X,p}$ is the maximal ideal of $\OO_X$ at $p$,
the new condition is saying that
\[ (f^\ast_P)\pre (\mm_{X,p}) = \mm_{Y,f(p)}. \]
Concretely, if $g$ is a function which vanishes on $f(p)$,
then its ``pullback'' $f^\ast_U(g)$ vanishes on $p$.

This completes the definition of a morphism of locally ringed spaces.
Isomorphisms of (locally) ringed spaces are defined in the usual way.

\section\problemhead
\begin{problem}
	Prove that if $\SF$ is a sheaf, then $\SF \cong \SF\sh$.
\end{problem}

\begin{sproblem}
	[Global section functor]
	We consider the category $\catname{ShRing}_X$ of sheaves of rings on a space $X$.
	The \vocab{global section functor}, denoted $\Gamma(X,-)$,
	which sends each sheaf $\SF$ to the ring $\SF(X)$.
	How does the global section functor behave on arrows?
	\begin{hint}
		Given a morphism of sheaves $\alpha : \SF \to \SG$
		read off a map $\SF(X) \to \SG(X)$.
	\end{hint}
\end{sproblem}

\begin{sproblem}
	\label{prob:finite_sheaf}
	Suppose $X$ is a finite topological space equipped with the discrete topology
	(i.e.\ a finite set of points).
	Let $\SF$ be a sheaf on $X$.
	Show that for any open set $U$, we have a ring isomorphism
	\[ \SF(U) \cong \prod_{p \in U} \SF_p. \]
	Here the product is the product ring as defined in \Cref{ex:product_ring}.
	\begin{hint}
		We have $\SF_p \cong \SF(\{p\})$.
		If $X$ is discrete, then all sequences of germs are compatible.
	\end{hint}
\end{sproblem}

\chapter{Schemes}
Now that we understand sheaves well, we can readily define the scheme.
It will be a locally ringed space, so we need to define
\begin{itemize}
	\ii The set of points,
	\ii The topology on it, and
	\ii The structure sheaf on it.
\end{itemize}

In the case of $\Aff^n$, we used $\CC^N$ as the set of points
and $\CC[x_1, \dots, x_n]$ but then remarked that the set of points
of $\CC^n$ corresponded to the maximal ideals of $\CC[x_1, \dots, x_n]$.

In an \emph{affine scheme}, we will take an \emph{arbitrary} ring $R$,
and generate the entire structure from just $R$ itself.
The final result is called $\Spec R$, the \vocab{spectrum} of $R$.
The affine varieties $\VV(I)$ we met earlier will just be
$\CC[x_1, \dots, x_n] / I$, but now we will be able to take
\emph{any} ideal $I$, thus finally completing the table at the end
of the ``affine variety'' chapter.

In particular, $\Spec \CC[x] / (x^2)$ will be the double we sought for so long.

\section{The set of points}
\prototype{$\Spec \CC[x_1, \dots, x_n] / I$.}

First surprise, for a ring $R$:
\begin{moral}
	$\Spec R$ is defined as the set of prime ideals of $R$.
\end{moral}

This might be a little surprising, since we might have guessed
that $\Spec R$ should just have the maximal ideals.
What do the remaining ideals correspond to?
The answer is that they will be so-called \emph{generic points}
points which are ``somewhere'' in the space, but nowhere in particular.

\begin{remark}
	As usual $R$ itself is not a prime ideal, but $(0)$
	does if $R$ is an integral domain.
\end{remark}

\begin{example}
	[Examples of spectrums]
	\listhack
	\begin{enumerate}[(a)]
		\ii $\Spec \CC[x]$ consists of a point $(x-a)$ for every $a \in \CC$,
		which correspond to what we geometrically think of as $\Aff^1$.
		In additionally consists of a point $(0)$,
		which we think of as a ``generic point'', nowhere in particular.

		\ii $\Spec \CC[x,y]$ consists of points $(x-a,y-b)$
		(which are the maximal ideals) as well as $(0)$ again, a generic
		point that is thought of as ``somewhere in $\CC^2$,
		but nowhere in particular''.
		It also consists of generic points corresponding to irreducible
		polynomials $f(x,y)$, for example $(y-x^2)$,
		which is a ``generic point on the parabola''.

		\ii If $k$ is a field, $\Spec k$ is a single point,
		since the only maximal ideal of $k$ is $(0)$.
	\end{enumerate}
\end{example}
\begin{example}
	[Complex affine varieties]
	Let $I \subseteq \CC[x_1, \dots, x_n]$ be an ideal.
	Then \[ \Spec \CC[x_1, \dots, x_n] /I \] contains a
	point for every closed irreducible subvariety of $\VV(I)$.
	So in addition to the ``geometric points'' we have
	``generic points'' along each of the varieties.
\end{example}
\begin{example}
	[More examples of spectrums]
	\listhack
	\begin{enumerate}[(a)]
		\ii $\Spec \ZZ$ consists of a point for every prime $p$,
		plus a generic point that is somewhere, but no where in particular.

		\ii $\Spec \CC[x] / (x^2)$ has only $(x)$ as a prime ideal.
		The ideal $(0)$ is not prime since $0 = x \cdot x$.
		Thus as a \emph{topological space},
		$\Spec \CC[x] / (x^2)$ is a single point.
		
		\ii $\Spec \Zc{60}$ consists of three points.
		What are they?
	\end{enumerate}
\end{example}

\section{The Zariski topology of the spectrum}
\prototype{Still $\Spec \CC[x_1, \dots, x_n] / I$.}

Now, we endow a topology on $\Spec R$.
Since the points on $\Spec R$ are the prime ideals, we continue
the analogy by thinking of the points $f$ as functions on $\Spec R$. That is,
\begin{definition}
	Let $f \in R$ and $\pp \in \Spec R$.
	Then the \vocab{value} of $f$ at $\pp$ is defined to be $f \pmod{\pp}$.
	We denote it $f(\pp)$.
\end{definition}
\begin{example}
	[Vanishing locii in $\Aff^n$]
	Suppose $R = \CC[x_1, \dots, x_n]$,
	and $\pp = (x_1-a_1, x_2-a_2, \dots, x_n-a_n)$ is a maximal ideal of $R$.
	Then for a polynomial $f \in \CC$,
	\[ f \pmod \pp = f(a_1, \dots, a_n) \]
	with the identification that $\CC/\pp \cong \CC$.
\end{example}
Indeed if you replace $R$ with $\CC[x_1, \dots, x_n]$
and $\Spec R$ with $\Aff^n$ in everything that follows,
then everything will be clear.

\begin{definition}
	Let $f \in R$. We define the \vocab{vanishing locus} of $f$ to be
	\[ \VV(f) = \left\{ \pp \in \Spec R \mid f(\pp) = 0 \right\}
		= \left\{ \pp \in \Spec R \mid f \in \pp \right\}. \]
	More generally, just as in the affine case,
	we define the vanishing locus for an ideal $I$ as
	\begin{align*}
		\VV(I) &= \left\{ \pp \in \Spec R \mid f(\pp)=0 \forall f \in I \right\} \\
		&= \left\{ \pp \in \Spec R \mid f \in \pp \; \forall f \in I \right\} \\
		&= \left\{ p \in \Spec R \mid I \subseteq \pp \right\}.
	\end{align*}
	Finally, we define the \vocab{Zariski topology} on $\Spec R$
	by declaring that the sets of the form $\VV(I)$ are closed.
\end{definition}

Now, the following topological notion will come in handy:
\begin{definition}
	A point $p \in X$ is a \vocab{closed point} if the set $\{p\}$ is closed.
\end{definition}
\begin{ques}
	Show that $\pp \in \Spec R$ is a closed point
	if and only if $\pp$ is a maximal ideal.
\end{ques}
Therefore the Zariski topology lets us refer back to the old ``geometric''
as just the closed points.
\begin{example}
	[Generic points, continued]
	Let $R = \CC[x,y]$ and let $\pp = (y-x^2) \in \Spec \RR$;
	this is the ``generic point'' on a parabola.
	It is not closed, but we can compute its closure:
	\[
		\ol{\{\pp\}}
		= \VV(\pp) = \left\{ \qq \in \Spec R \mid \qq \supseteq \pp \right\}.
	\]
	This closure contains the point $\pp$ as well
	as several maximal ideals $\qq$, such as $(x-2,y-4)$ and $(x-3,y-9)$.
	In other words, the closure of the ``generic point'' of the parabola
	is literally the set of all points that are actually on the parabola
	(including generic points).

	That means the way to picture $\pp$ is a point that 
	is ``somewhere on the parabola'', but nowhere in particular.
	It makes sense then that if we take the closure,
	we get the entire parabola,
	since $\pp$ ``could have been'' any of those points.
\end{example}
The previous example illustrates the following important observation:
\begin{exercise}
	Let $I \subsetneq R$ be a proper ideal.
	Construct a bijection between the maximal ideals of $R$ containing $I$,
	and the maximal ideals of $R/I$.
	(\cite{ref:vakil} labels this exercise as
	``Essential Algebra Exercise (Mandatory if you haven't done it before)''.)
\end{exercise}

\begin{example}
	[The generic point of the $y$-axis isn't on the $x$-axis]
	Let $R = \CC[x,y]$ again.
	Consider $\VV(y)$, which is the $x$-axis of $\Spec R$.
	Then consider $\pp = (x)$, which is the generic point on the $y$-axis.
	Observe that
	\[ \pp \notin \VV(y). \]
	The geometric way of saying this is that a \emph{generic point}
	on the $y$-axis does not lie on the $x$-axis.
\end{example}

Finally, as before, we can define distinguished open sets,
which form a basis of the Zariski topology.
\begin{definition}
	Let $f \in \Spec R$.
	Then $D(f)$ is the set of $\pp$ such that $f(\pp) \neq 0$,
	a \vocab{distinguished open set}.
	These open sets form a basis for the Zariski topology on $\Spec R$.
\end{definition}

\section{The structure sheaf}
\prototype{Still $\CC[x_1, \dots, x_n] / I$.}

We have now endowed $\Spec R$ with the Zariski topology,
and so all that remains is to put a sheaf $\OO_{\Spec R}$ on it.
To do this we want a notion of ``regular functions'' as before.

In order for this to make sense, we have to talk about rational
functions in a meaningful way.
If $R$ was an integral domain, then we could just use the field of fractions;
for example if $R = \CC[x_1, \dots, x_n]$ then we could just
look at rational quotients of polynomials.

Unfortunately, in the general situation $R$ may not be an integral domain:
for example, the ring $R = \CC[x] / (x^2)$ corresponding to the double point.
So we will need to define something a little different:
we will construct the \emph{localization} of $R$ at a set $S$,
which we think of as the ``set of allowed denominators''.

\begin{definition}
	Let $S \subseteq R$, where $R$ is a ring,
	and assume $S$ is closed under multiplication.
	Then the \vocab{localization of $R$ at $S$}, denoted $S\inv R$,
	is defined as the set of fractions
	\[ \left\{ r/s \mid r \in R, s \in S \right\} \]
	where we declare two fractions $r_1 / s_1 = r_2 / s_2$ 
	to be equal if 
	\[ \exists s \in S : \quad s(r_1s_2 - r_2s_1) = 0. \]
\end{definition}
In particular, if $0 \in S$ then $S\inv R$ is the trivial ring.
So we usually only take situations where $0 \notin S$.
\begin{ques}
	Assume $R$ is an integral domain and $S = R \setminus \{0\}$.
	Show that $S\inv R$ is just the field of fractions.
\end{ques}

\begin{example}
	[Why the extra $s$?]
	The reason we need the condition $s(r_1s_2 - r_2s_1) = 0$
	rather than the simpler $r_1s_2 - r_2s_1 = 0$ is that
	otherwise the equivalence relation may fail to be transitive.
	Here is a counterexample: take
	\[ R = \Zc{12} \qquad S = \{ 0, 4, 8 \}. \]
	Then we have for example
	\[ \frac12 = \frac24 = \frac64 = \frac32. \]
	So we need to have $\frac12=\frac32$ which is only true
	with the first definition.

	Of course, if $R$ is an integral domain (and $0\notin S$)
	then this is a moot point.
\end{example}

The most important special case is the localization at a prime ideal.
\begin{definition}
	Let $R$ be a ring and $\pp$ a prime ideal.
	Then $R_\pp$ is defined to be $S\inv R$ for $S = R\setminus\pp$.
	We call this \vocab{localization at $\pp$}.
	Addition is defined in the obvious way.
\end{definition}
\begin{ques}
	Why is $S$ multiplicative closed in the above definition?
\end{ques}
Thus,
\begin{moral}
	If $R$ is functions on the space $\Spec R$,
	we think of $R_\pp$ as rational quotients $f/g$ where $g(\pp) \neq 0$.
\end{moral}
In particular, if $R = \CC[x_1, \dots, x_n]$
then this is precisely the definition of rational function from before!

Now, we can define the sheaf as ``locally rational'' functions.
This is done by a sheafification.
First, let $\SF$ be the pre-sheaf of ``globally rational'' functions:
i.e.\ we define $\SF(U)$ to be the following localization of $R$
to the functions vanishing outside $U$:
\[
	\SF(U) = \left\{ 
		\frac fg \mid f, g \in R
		\text{ and } g(\pp) \neq 0 \; \forall \pp \in U
	\right\}
	= \left(R \setminus \bigcup_{\pp \in U} \pp \right)\inv R.
\]
For every $\pp \in U$ we can view $f/g$ as an element in $R_\pp$
(since $g(\pp) \ne 0$).
As one might expect this is an isomorphism
\begin{lemma}[Stalks of the ``globally rational'' pre-sheaf]
	\label{lem:global_rational_stalk}
	The stalk of $\SF$ defined above $\SF_\pp$ is isomorphic to $R_\pp$.
\end{lemma}
\begin{proof}
	There is an obvious map $\SF_\pp \to R_\pp$ on germs by
	\[
		\left(U, f/g \in \SF(U) \right)
		\mapsto f/g \in R_\pp . \]
	(Note the $f/g$ on the left lives in
	$\SF(U) = \left(R \setminus \bigcup_{\pp \in U} \pp \right)\inv R$
	but the one on the right lives in $R_\pp$).
	Now suppose $(U_1, f_1 / g_1)$ and $(U_2, f_2 / g_2)$
	are germs with $f_1/g_1 = f_2/g_2 \in R_\pp$.
	\begin{exercise}
		Now, show both germs are equal to $(U_1 \cap U_2 \cap D(h), f_1 / g_1)$
		with $D(h)$ the distinguished open set.
	\end{exercise}
	It is also surjective, since given $f/g \in R_\pp$ we take $U = D(g)$
	the distinguished open set for $g$.
\end{proof}

Then, we set \[ \OO_{\Spec R} = \SF\sh. \]
In fact, we can even write out the definition of the sheafification,
by viewing the germ at each point as an element of $R_\pp$.
\begin{definition}
	Let $R$ be a ring. Then $\Spec R$ is made into a ringed space by setting
	\[ \OO_{\Spec R}(U) 
		= \left\{ (f_\pp \in R_\pp)_{\pp \in U}
		\text{ which are locally $f/g$} \right\}. \]
	That is, it consists of sequence $(f_\pp)_{\pp \in U}$, with
	each $f_\pp \in R_\pp$, such that for every point $\pp$ there
	is a neighborhood $U_\pp$ and an $f,g \in R$ such that
	$f_\qq = \frac fg \in R_\qq$ for all $\qq \in U_\pp$.
\end{definition}

\section{Properties of affine schemes}
Now that we're done defining an affine scheme,
we state some important results about them.

\begin{definition}
	For $g \in R$, we define the \vocab{localization of $R$ at $g$},
	denoted $R_g$ to be $\{1, g, g^2, g^3, \dots\}\inv R$.
	(Note that $\left\{ 1, g, g^2, \dots \right\}$ is multiplicatively closed.)
\end{definition}
This is admittedly somewhat terrible notation, since $R_\pp$
is the localization with multiplicative set $R \setminus \pp$,
while $R_g$ is the localization with multiplicative set $\{1,g,g^2,\dots\}$;
these two are quite different beasts!

\begin{example}
	[Localization at an element]
	Let $R = \CC[x,y,z]$ and let $g = x$.
	Then
	\[ R_g = \left\{ \frac{P(x,y,z)}{x^n} \mid
		P \in \CC[x,y,z], \; n \ge 0 \right\}. \]
\end{example}
\begin{theorem}
	[On the affine structure sheaf]
	Let $R$ be a ring and $\Spec R$ the associated affine scheme.
	\begin{enumerate}[(a)]
		\ii Let $\pp$ be a prime ideal.
		Then $\OO_{\Spec R, \pp} \cong R_\pp$.
		\ii Let $D(g)$ be a distinguished open set.
		Then $\OO_{\Spec S}(D(g)) \cong R_g$.
	\end{enumerate}
\end{theorem}
This matches the results that we've seen when $\Spec R$ is an affine variety.
\begin{proof}
	Part (a) follows by \Cref{lem:global_rational_stalk}
	and \Cref{lem:pre_sheaf_stalk}.
	For part (b), we need the following:
	\begin{ques}
		If $I$ and $J$ are ideals of a ring $R$,
		then $\VV(I) \subseteq \VV(J)$ if and only if
		$\sqrt{J} \subseteq \sqrt{I}$.
		(Use the fact that $\sqrt I = \cap_{I \subseteq \pp} \pp$.)
	\end{ques}
	Then we can repeat the proof of \Cref{thm:reg_func_distinguish_open}.
\end{proof}

We now state the most important theorem.
In fact, this theorem is the justification that our definition
of a scheme is the correct one.
\begin{theorem}
	[Affine schemes and commutative rings are the same category]
	Let $R$ and $S$ be rings.
	There is a natural bijection between maps of schemes $\Spec R \to \Spec S$	
	and ring homomorphisms $\psi : S \to R$.
\end{theorem}
\begin{proof}
	First, we need to do the construction:
	given a map of schemes $f : \Spec R \to \Spec S$ 
	we need to construct a homomorphism $S \to R$.
	This should be the easy part, because $f$ has a lot of data.
	Indeed, recall that for every $U \subseteq \Spec S$
	there is supposed to be a map
	\[ f^\ast_U : \OO_{\Spec S} (U) \to \OO_{\Spec R}(f\pre(U)). \]
	If we take $U$ to be the entire space $\Spec S$ we now have a ring homomorphism
	\[ S = \OO_{\Spec S} (\Spec S) \to \OO_{\Spec R} (\Spec R) = R. \]
	which is one part of the construction.

	In fact, this is just the global section functor
	applied to $\OO_{\Spec S} \to f_\ast \OO_{\Spec R}$.

	The more involved direction is building from a homomorphism
	$\psi : S \to R$ a map $f : \Spec R \to \Spec S$.
	We define it by:
	\begin{itemize}
		\ii On points, $f(\pp) = \psi\pre(\pp)$ for each prime ideal $\pp$.
		One can check $\psi\pre(\pp)$ is indeed prime
		and that $f$ is continuous in the Zariski topology.
		\ii We first construct a map on the stalks of the sheaf as follows:
		for every prime ideal $\qq$ in $R$, let $\psi``(\qq) = \pp$ be its image
		(so that $f(\qq) = \pp$) and consider the map
		\[ \psi_\qq : \OO_{\Spec S, \qq} = S_{\qq}
			\to R_{\psi``(\qq)} = \OO_{\Spec R, \psi``(\qq)}
			\quad\text{by}\quad f/g \mapsto \psi(f)/\psi(g). \]
		Again, one can check this is well-defined and a map of local rings.
		This is called the localization of $\psi$ at $\qq$.
		\ii Finally, we use the above map to construct an
		$f^\ast_U : \OO_{\Spec S}(U) \to \OO_{\Spec R}(f\pre(U))$
		for every open set  $U \subseteq \Spec S$.
		Since $\OO_{\Spec S}$ is a sheafification, we think of its
		sections as sequences of compatible germs $(g_\qq)_{\qq \in U}$
		and then map it via the map $\psi_\qq$ above.
	\end{itemize}
	One then has to check that everything is well-defined.
	This is left as an exercise to the diligent reader;
	for an actual proof see \cite[Proposition 6.3.2]{ref:vakil}.
\end{proof}
\begin{remark}
	Categorically, this says that the ``global section functor''
	on the category $\catname{AffSch}$ of affine schemes is a
	fully faithful functor to $\catname{CRing}$.
	Moreover, it is surjective on objects
	(the correct term here is ``essentially surjective on objects'').
\end{remark}
To be more philosophical,
\begin{moral}
	The category of affine schemes and the
	category of commutative rings are \emph{exactly the same},
	down to the morphisms between two pairs of objects.
\end{moral}

\section{Schemes}
\begin{definition}
	A \vocab{scheme} is a locally ringed space $(X, \OO_X)$
	with an open affine cover $\{U_\alpha\}$ of $X$
	such that each pair $(U_\alpha, \OO_X \restrict{U_\alpha})$
	is isomorphic to an affine scheme.
\end{definition}
Hooray!

\section{Projective scheme}
\prototype{Projective varieties, in the same way.}
The most important class of schemes which are not affine are
\emph{projective} schemes.
The complete the obvious analogy:
\[
	\frac{\text{Affine variety}}{\text{Projective variety}}
	= 
	\frac{\text{Affine scheme}}{\text{Projective scheme}}.
\]
Let $S$ be \emph{any} (commutative) graded ring, like $\CC[x_0, \dots, x_n]$.
\begin{definition}
	We define $\Proj S$, the \vocab{projective scheme over $S$}:
	\begin{itemize}
		\ii As a set, $\Proj S$ consists of \emph{homogeneous prime ideals}
		$\pp$ which do not contain $S^+$.
		\ii If $I \subseteq S$ is homogeneous, then
		we let $\Vp(I) = \{ \pp \in \Proj S \mid \pp \subseteq I \}$.
		Then the \vocab{Zariski topology} is imposed by declaring 
		sets of the form $\Vp(I)$ to be closed.
		\ii We now define a pre-sheaf $\SF$ on $\Proj S$ by
		\[ \SF(U) = 
			\left\{ \frac{f}{g} \mid 
			g(\pp) \neq 0 \; \forall \pp \in U \text{ and }
			\deg f = \deg g \right\}.
		\]
		In other words, the rational functions are quotients $f/g$
		where $f$ and $g$ are \emph{homogeneous of the same degree}.
		Then we let \[ \OO_{\Proj S} = \SF\sh \] be the sheafification.
	\end{itemize}
\end{definition}
\begin{definition}
	The \vocab{distinguished open sets} $D(f)$ of the $\Proj S$
	are defined as $\left\{ \pp \in \Proj S : f(\pp) \neq 0 \right\}$,
	as before; these form a basis for the Zariski topology of $\Proj S$.
\end{definition}
Now, we want analogous results as we did for affine structure sheaf.
So, we define a slightly modified localization:
\begin{definition}
	Let $S$ be a graded ring.
	\begin{enumerate}[(i)]
		\ii For a prime ideal $\pp$, let
		\[ S_{(\pp)} = \left\{ \frac fg \mid g(\pp) \neq 0 \text{ and }
			\deg f = \deg g \right\} \]
		denote the elements of $S_\pp$ with ``degree zero''.
		\ii For any homogeneous $g \in S$ of degree $d$, let
		\[ S_{(g)} = \left\{ \frac{f}{g^r} \mid 
			\deg f = r \deg g \right\} \]
		denote the elements of $S_g$ with ``degree zero''.
	\end{enumerate}
\end{definition}

\begin{theorem}
	[On the projective structure sheaf]
	Let $S$ be a graded ring and let $\Proj S$ the associated projective scheme.
	\begin{enumerate}[(a)]
		\ii Let $\pp \in \Proj S$.
		Then $\OO_{\Proj S, \pp} \cong S_{(\pp)}$.
		\ii Suppose $g$ is homogeneous with $\deg g > 0$. Then
		\[ D(g) \cong \Spec S_{(g)} \]
		as locally ringed spaces.
		In particular, $\OO_{\Proj S}(D(g)) \cong S_{(g)}$.
	\end{enumerate}
\end{theorem}
\begin{ques}
	Conclude that $\Proj S$ is a scheme.
\end{ques}

Of course, the archetypal example is that
\[ \Proj \CC[x_0, x_1, \dots, x_n] / I \]
corresponds to the projective subvariety of $\CP^n$
cut out by $I$ (when $I$ is radical).
In the general case of an arbitrary ideal $I$, we
call such schemes \vocab{projective subscheme} of $\CP^n$
For example, the ``double point'' is given by $\Proj[x_0,x_1]/(x_0^2)$.

\begin{remark}
	No comment yet on what the global sections of $\OO_{\Proj S}(\Proj S)$ are.
	(The theorem above requires $\deg g > 0$, so we cannot just take $g=1$.)
	One might hope that in general $\OO_{\Proj S}(\Proj S) \cong S^0$
	in analogy to our complex projective varieties, but
	one needs some additional assumptions on $S$ for this to hold.
\end{remark}

\section\problemhead
\begin{problem}
	Describe the points of $\Spec \RR[x,y]$.
	\begin{hint}
		Galois conjugates.
	\end{hint}
\end{problem}

\begin{dproblem}
	[Chinese remainder theorem]
	Consider $X = \Spec \Zc{60}$, which as a topological space has three points.
	By considering $\OO_X(X)$ prove the Chinese theorem
	\[ \Zc{60} \cong \Zc{4} \times \Zc{3} \times \Zc{5}. \]
	\begin{hint}
		Appeal to \Cref{prob:finite_sheaf}.
	\end{hint}
\end{dproblem}

\begin{problem}
	Given an affine scheme $X = \Spec R$,
	show that there is a unique morphism of schemes $X \to \Spec \ZZ$,
	and describe where it sends points of $X$.
	\begin{hint}
		Use the proof that $\catname{AffSch} \simeq \catname{CRing}$.
	\end{hint}
	\begin{sol}
		$\pp$ gets sent to the characteristic of the field $\OO_{X,\pp} / \mm_{X,\pp}$.
	\end{sol}
\end{problem}

\chapter{More on sheaves}
Previously, we defined a sheaf so that we could equip
it onto a scheme and thus produce a ringed space.
However, we will now proceed to define some additional sheaves on a scheme $X$,
and study the relations between them.

\section{Sheaves of $\OO_X$-modules}
\prototype{Twisting Sheaves}
Up until now we have mostly been working with sheaves of rings,
but now we want to instead think about sheaves of modules related
to a scheme $X$. 
\begin{definition}
	Let $(X, \OO_X)$ be a ringed space.
	A \vocab{sheaf of $\OO_X$-modules} is a sheaf $\SF$
	of abelian groups such that $\SF(U)$ is also an $\OO_X(U)$-module.
	The module structures are also required to be compatible
	with the restriction maps.
\end{definition}
\begin{example}
	[Twisting sheaves]
	Let $X = \Proj S$ be a projective scheme for a graded ring $S$
	(e.g.\ a projective subscheme).
	For any integer $m$ we consider a pre-sheaf
	\[ \SF^n = \left\{ \frac fg \mid g(\pp)\neq 0 \;\forall p \in X 
		\text{ and } \deg f - \deg g = n \right\}.
	\]
	Intuitively, these are degree $n$ ``functions''.
	Then $\OO_X(n) = (\SF^n)\sh$;
	we call these the \vocab{twisting sheaves}.

	For example, if $S = \CC[x_0, \dots, x_n]$ then
	\begin{enumerate}[(a)]
		\ii $\OO_X(0) \cong \OO_X$, by definition.
		\ii A degree $n$ polynomial defines a global section of $\OO_X(n)$.
		\ii $\OO_X(m)$ has no nonzero global sections if $m < 0$.
		\ii Suppose $X = \CP^1$ has coordinates $(s:t)$. Then
		\[ \frac 1s \in \OO_X(-1)(U_0) \]
		is an example of a (non-global) section of $\OO_X(-3)$.
		\ii Suppose $X = \CP^2$ has coordinates $(x:y:z)$. Then
		\[ \frac{1}{xyz} \in \OO_X(-1)(U_0 \cap U_1 \cap U_2) \]
		is an example of a (non-global) section of $\OO_X(-3)$.
	\end{enumerate}
\end{example}

Observe that the stalks of such a sheaf are abelian groups.

\section{Sheaves of $\OO_X$-modules make an abelian category}
I won't take the time to do this properly,
since I won't really use much of it.
See \cite{ref:vakil} for a proper exposition.

Fix a ringed space $(X, \OO_X)$.
It's easy to see there is a category of sheaves of $\OO_X$-modules.
(the objects are sheaves and the morphisms are, well, morphisms of sheaves).
To see that it is abelian, we need to do the following:
\begin{itemize}
	\ii Identify the zero object of the category.
	This is constant sheaf $0$, which gives $\SF(U) = \{0\}$ for every $U$.
	(This more or less follows from the fact that $\{0\}$
	is the zero object of the abelian category $\catname{AbGrp}$.)
	\ii Show that we can add two morphisms.
	This is done in the obvious manner.
	\ii Show that any morphism has a ``kernel sheaf'' and ``cokernel sheaf''.
	The definition is not too bad: given $\alpha : \SF \to \SG$ we define the
	pre-sheaves $\ker\alpha$ and $\coker'\alpha$ by
	\begin{align*}
		\ker(\alpha)(U) &= \ker(\alpha_U) \\
		\coker'(\alpha)(U) &= \coker(\alpha_U).
	\end{align*}
	It turns out that $\ker(\alpha)$ is already a sheaf.
	But $\coker'(\alpha)$ is only a pre-sheaf and in general not a sheaf.
	(Again, see \cite{ref:vakil} or \cite{ref:gathmann} for why.)
	Thus we have to define $\coker(\alpha) = \coker'(\alpha) \sh$.
	
	The tricky part, done in \cite{ref:vakil} and the one which I'm omitting,
	is showing that this kernel and cokernel sheaf satisfy
	the correct universal properties.
\end{itemize}

From this we deduce
\begin{theorem}
	[Sheaves form an abelian category]
	Let $(X, \OO_X)$ be a ringed space.
	The category of sheaves of $\OO_X$-modules forms an abelian category.
\end{theorem}

In particular, we can talk about whether a sequence of sheaves
$\SF \to \SG \to \SH$ is exact: just use the categorical definition.
However, if possible we want to avoid using the long-winded definition.
So, I will just quote the relevant results:
\begin{theorem}
	[Exactness can be checked at stalks]
	A sequence of sheaves of $\OO_X$-modules
	$\SF \to \SG \to \SH$ is exact if and only if
	for every point $p \in X$,
	the induced map of stalks $\SF_p \to \SG_p \to \SH_p$ is exact.
\end{theorem}
Thus, we \emph{really} like stalks; they behave as nicely as possible.
\begin{ques}
	Deduce that $\SF \to \SG$ is injective / surjective / an isomorphism
	if and only if $\SF_p \to \SG_p$ is injective / surjective / an isomorphism
	for every point $p$.
\end{ques}

If we insist on sections, the only useful result we have is:
\begin{theorem}
	[Monic/epic on sections]
	Let $\alpha : \SF \to \SG$ be a morphism of $\OO_X$-modules.
	\begin{enumerate}[(a)]
		\ii The map $\alpha$ is monic if and only if $\SF(U) \to \SG(U)$
		is injective for every open $U$.
		\ii The map $\alpha$ is epic if $\SF(U) \to \SG(U)$
		is surjective for every open $U$,
		but the converse is not true in general.
	\end{enumerate}
\end{theorem}

In addition, we can do the following constructions.
\begin{definition}
	If $\SF$ and $\SG$ are sheaves of $\OO_X$-modules, then
	\begin{enumerate}[(a)]
		\ii The \vocab{direct sum} $\SF \oplus \SG$ is defined
		on open sets by \[ U \mapsto \SF(U) \oplus \SG(U). \]
		(This is already a sheaf, so we don't need to sheafify.)
		\ii The \vocab{tensor product} $\SF \otimes \SG$ is defined
		by taking the sheafification of the pre-sheaf
		\[ U \mapsto \SF(U) \otimes_{\OO_X(U)} \SG(U). \]
		\ii The \vocab{dual} $\SF^\vee$ of $\SF$ is defined
		by taking the sheafification of the pre-sheaf
		\[ U \mapsto \Hom_{\OO_X(U) \text{ module}} (\SF(U), \OO_X(U)). \]
	\end{enumerate}
\end{definition}
\begin{example}[More on twisting sheaves]
	Consider the twisting sheaves of $X = \CP^N$.
	\begin{enumerate}[(a)]
		\ii $\OO_X(n) \otimes \OO_X(m) \cong \OO_X(m+n)$.
		What do you think the isomorphism is?
		\ii $\OO_X(n)^\vee \cong \OO_X(-n)$.
	\end{enumerate}
\end{example}

\section{Quasi-coherent sheaves}
Example: exponential sequence

Define quasi-coherent (easier to work with)

Kahler differential module, show the Euler sequence

\chapter{Divisors and line bundles}

\end{document}
