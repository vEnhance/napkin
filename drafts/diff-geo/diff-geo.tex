\documentclass[11pt]{scrreprt}
%%fakesection Load packages

\usepackage{lmodern}
\usepackage[pdfusetitle]{hyperref}
\ExplSyntaxOn
\sys_if_engine_luatex:T {
	\usepackage{luatex85}
}
\sys_if_engine_pdftex:T {
	\usepackage[T1]{fontenc}
}
\ExplSyntaxOff

% These are evan.sty
\usepackage{amsmath,amssymb,amsthm}
\usepackage{mathrsfs}
\usepackage[usenames,svgnames,dvipsnames]{xcolor}
\usepackage{textcomp}
\usepackage{enumerate}
\usepackage[textsize=scriptsize,shadow]{todonotes}
\usepackage{mathtools}
\usepackage{microtype}
\usepackage[normalem]{ulem}
\usepackage{stmaryrd}
\usepackage{wasysym}
\usepackage{multirow}
\usepackage{prerex}
\usepackage[nameinlink]{cleveref}
\usepackage{derivative}

%%fakesection evan.sty macros
%Small commands
%% Napkin commands
\newcommand{\prototype}[1]{
	\emph{{\color{red} Prototypical example for this section:} #1} \par\medskip
}
\newenvironment{moral}{%
	\begin{tcolorbox}[boxrule=0.4pt,colframe=green!70!black,sharp corners,
		standard jigsaw,opacityback=0,left=10pt,right=10pt,top=3pt,bottom=3pt,
		before skip=10pt,after skip=20pt]%
	\bfseries\color{green!50!black}}%
	{\end{tcolorbox}}

%%fakesection Links (hyperref loaded earlier implicitly)
\hypersetup{
	linkcolor={red!50!black},
	citecolor={green!50!black},
	urlcolor={blue!80!black},
	pdfkeywords={napkin,math},
	pdfsubject={web.evanchen.cc},
	colorlinks,
}

%%fakesection Commutative diagrams
\usepackage{tikz-cd}
\usetikzlibrary{arrows,arrows.meta}
% make a larger hook
% https://tex.stackexchange.com/questions/514451/how-to-define-a-new-hooked-arrow
\makeatletter
\pgfdeclarearrow{
	name=xGlyph,
	cache=false,
	bending mode=none,
	parameters={\tikzcd@glyph@len,\tikzcd@glyph@shorten},
	setup code={%
		\pgfarrowssettipend{\tikzcd@glyph@len\advance\pgf@x by\tikzcd@glyph@shorten}},
	defaults={
		glyph axis=axis_height,
		glyph length=+1.55ex,
		glyph shorten=+-0.1ex},
	drawing code={%
		\pgfpathrectangle{\pgfpoint{+0pt}{+-1.5ex}}{\pgfpoint{+\tikzcd@glyph@len}{+3ex}}%
		\pgfusepathqclip%
		\pgftransformxshift{+\tikzcd@glyph@len}%
		\pgftransformyshift{+-\tikzcd@glyph@axis}%
		\pgftext[right,base]{\tikzcd@glyph}}}
\makeatother
\tikzcdset{
	arrow style=tikz,
	diagrams={>={Latex}},
	tikzcd left hook/.tip={xGlyph[glyph math command=supset, swap, glyph axis = 5.7pt]},
	tikzcd right hook/.tip={xGlyph[glyph math command=supset, glyph axis = 5.7pt]},
	surjective head arrow /.tip = {tikzcd to[sep=-1.5pt]tikzcd to},
	surjective head/.style={
		-surjective head arrow
	}
}

%%fakesection Page layout
\usepackage[headsepline]{scrlayer-scrpage}
\renewcommand{\headfont}{}
\addtolength{\textheight}{3.14cm}
\setlength{\footskip}{0.5in}
\setlength{\headsep}{10pt}

\def\shortdate{\leavevmode\hbox{\the\year-\twodigits\month-\twodigits\day}}
\def\twodigits#1{\ifnum#1<10 0\fi\the#1}
\automark[chapter]{chapter}

\rohead{\footnotesize\thepage}
\rehead{\footnotesize \textbf{\sffamily Napkin}, by \emph{Evan Chen} (\napkinversion)}
\lehead{\footnotesize\thepage}
\lohead{\footnotesize \leftmark}
\chead{}
\rofoot{}
\refoot{}
\lefoot{}
\lofoot{}
%\cfoot{\pagemark}

%%fakesection Fancy section and chapter heads
\renewcommand*{\sectionformat}{\color{purple}\S\thesection\autodot\enskip}
\renewcommand*{\subsectionformat}{\color{purple}\S\thesubsection\autodot\enskip}
\newcommand{\problemhead}{A few harder problems to think about}
\renewcommand{\thesubsection}{\thesection.\roman{subsection}}

\addtokomafont{chapterprefix}{\raggedleft}
\RedeclareSectionCommand[beforeskip=0.5em]{chapter}
\renewcommand*{\chapterformat}{%
\mbox{\scalebox{1.5}{\chapappifchapterprefix{\nobreakspace}}%
\scalebox{2.718}{\color{purple}\thechapter\autodot}\enskip}}

\addtokomafont{partprefix}{\rmfamily}
\renewcommand*{\partformat}{\color{purple}\scalebox{2.5}{\thepart}}

%%fakesection Theorems
\usepackage{tcolorbox}
\tcbuselibrary{breakable,skins,hooks}

% patch tcolorbox to support continuation of a paragraph
% after box
% from https://tex.stackexchange.com/questions/568782/
\makeatletter
\tcbset{
	after app={%
		\ifx\tcb@drawcolorbox\tcb@drawcolorbox@breakable
		\else
		% add only when not breakabel
		\@endparenv
		\fi
	}
}

% for breakable
\appto\tcb@use@after@lastbox{\@endparenv\@doendpe}
\makeatother
% END patch tcolorbox for continuation of paragraph


\usepackage{thmtools}

\theoremstyle{definition}
\declaretheoremstyle[
	headfont=\sffamily\bfseries\color{MidnightBlue},
	headpunct={\\[3pt]},
	postheadspace={0pt},
]{thmtheorem}


\declaretheoremstyle[
	headfont=\bfseries\color{RawSienna},
	headpunct={\\[3pt]},
	postheadspace={0pt},
]{thmexample}

\tcbset{
	theorem box/.style={
		enhanced,
		arc=9pt,
		outer arc=10pt,
		colframe=blue,
		colback=TealBlue!5,
		boxrule=1pt,
		before skip=12pt,
		after skip=14pt,
		left=10pt,
		right=10pt
	},
	remark box/.style={
		boxrule=0pt,
		frame hidden,
		sharp corners,
		enhanced,
		borderline west={2pt}{0pt}{ForestGreen},
		before skip=8pt,
		colback=ForestGreen!5,
		after skip=12pt,
		breakable,
		left=10pt,
		right=10pt
	},
	example box/.style={
		enhanced,
		sharp corners,
		arc=9pt,
		outer arc=10pt,
		colframe=RawSienna,
		colback=Salmon!5,
		boxrule=0.5pt,
		before skip=12pt,
		after skip=14pt,
		breakable,
		top=6pt,
		bottom=8pt,
		breakable,
		left=10pt,
		right=10pt
	},
	ques box/.style={
		boxrule=0pt,
		frame hidden,
		enhanced,
		sharp corners,
		before skip=8pt,
		after skip=12pt,
		borderline west={3pt}{0pt}{black},
		colback=RedViolet!5!gray!5,
		breakable,
		left=10pt,
		right=10pt
	}
}

\declaretheorem[style=thmtheorem,name=Theorem,numberwithin=section]{theorem}
\tcolorboxenvironment{theorem}{theorem box}
\declaretheorem[style=thmtheorem,name=Lemma,sibling=theorem]{lemma}
\tcolorboxenvironment{lemma}{theorem box}
\declaretheorem[style=thmtheorem,name=Proposition,sibling=theorem]{proposition}
\tcolorboxenvironment{proposition}{theorem box}
\declaretheorem[style=thmtheorem,name=Corollary,sibling=theorem]{corollary}
\tcolorboxenvironment{corollary}{theorem box}
\declaretheorem[style=thmexample,name=Example,sibling=theorem]{example}
\tcolorboxenvironment{example}{example box}

\declaretheoremstyle[
	headfont=\bfseries\sffamily\color{ForestGreen!70!black},
	bodyfont=\normalfont,
	headpunct={ --- },
]{thmremark}
\declaretheoremstyle[
	headfont=\bfseries\sffamily\color{ForestGreen!70!black},
	bodyfont=\normalfont,
	headpunct={},
]{thmremark*}

\declaretheoremstyle[
	headfont=\bfseries,
	bodyfont=\normalfont\small
]{thmques}

\declaretheorem[name=Question,sibling=theorem,style=thmques]{ques}
\tcolorboxenvironment{ques}{ques box}
\declaretheorem[name=Exercise,sibling=theorem,style=thmques]{exercise}
\tcolorboxenvironment{exercise}{ques box}
\declaretheorem[name=Remark,sibling=theorem,style=thmremark]{remark}
\tcolorboxenvironment{remark}{remark box}
\declaretheorem[name=Remark,sibling=theorem,style=thmremark*]{remark*}
\tcolorboxenvironment{remark*}{remark box}
\declaretheorem[name=Step,style=thmremark]{step} % only used in Lebesgue int
\tcolorboxenvironment{step}{remark box}

\theoremstyle{definition}
\newtheorem{claim}[theorem]{Claim}
\newtheorem{definition}[theorem]{Definition}
\newtheorem{fact}[theorem]{Fact}
\newtheorem{abuse}[theorem]{Abuse of Notation}

\newtheorem{problem}{Problem}[chapter]
\renewcommand{\theproblem}{\thechapter\Alph{problem}}
\newtheorem{sproblem}[problem]{Problem}
\newtheorem{dproblem}[problem]{Problem}
\renewcommand{\thesproblem}{\theproblem$^{\star}$}
\renewcommand{\thedproblem}{\theproblem$^{\dagger}$}
\newcommand{\listhack}{$\empty$\vspace{-2em}}

%%fakesection Answers
\usepackage{answers}
\Newassociation{hint}{answeritem}{tex/backmatter/all-hints}
\Newassociation{sol}{answeritem}{tex/backmatter/all-solns}
\renewcommand{\solutionextension}{out}
\renewenvironment{answeritem}[1]{\item[\bfseries #1.]}{}

%%fakesection Table of contents
% First add ToC to ToC
\makeatletter
\usepackage{etoolbox}
\pretocmd{\tableofcontents}{%
	\if@openright\cleardoublepage\else\clearpage\fi
	\pdfbookmark[0]{\contentsname}{toc}%
}{}{}%
\makeatother
\setcounter{tocdepth}{1}
\RedeclareSectionCommand[tocnumwidth=4.2em]{part}
\RedeclareSectionCommand[tocpagenumberwidth=2.2em,tocnumwidth=4.2em]{chapter}
\RedeclareSectionCommand[tocpagenumberwidth=2.2em,tocnumwidth=2.8em]{section}
% adjust tocpagenumberwidth manually for large page number: https://tex.stackexchange.com/a/502168

%%fakesection Asymptote definitions
\usepackage{patch-asy}
\numberwithin{asy}{chapter}
\renewcommand{\theasy}{\thechapter\Alph{asy}}
\begin{asydef}
	import extras;
	size(6cm);
	usepackage("amsmath");
	usepackage("amssymb");
	defaultpen(fontsize(11pt));
	settings.tex = "latex";
	settings.outformat = "pdf";
\end{asydef}
\def\asydir{asy}

%%fakesection Bibliography
\usepackage[backend=biber,backref=true,style=alphabetic]{biblatex}
\DeclareLabelalphaTemplate{
	\labelelement{
		\field[final]{shorthand}
		\field{label}
		\field[strwidth=2,strside=left]{labelname}
	}
	\labelelement{
		\field[strwidth=2,strside=right]{year}
	}
}
\DeclareFieldFormat{labelalpha}{\textbf{\scriptsize #1}}
\addbibresource{references.bib}
\addbibresource{images.bib}
%% stylistic biblatex choices
\DefineBibliographyStrings{english}{%
	backrefpage  = {cited p.}, % for single page number
	backrefpages = {cited pp.} % for multiple page numbers
}
\DeclareFieldFormat{journaltitle}{\mkbibemph{#1},} % italic journal title with comma
\DeclareFieldFormat[inbook,thesis]{title}{\mkbibemph{#1}\addperiod} % italic title with period
\DeclareFieldFormat[article]{title}{#1} % title of journal article is printed as normal text
\DeclareFieldFormat[article]{volume}{\textbf{#1}\addcolon\space}
\renewcommand{\mkbibnamegiven}[1]{\textsc{#1}}
\renewcommand{\mkbibnamefamily}[1]{\textsc{#1}}
\renewcommand{\mkbibnameprefix}[1]{\textsc{#1}}
\renewcommand{\mkbibnamesuffix}[1]{\textsc{#1}}
\renewcommand{\finentrypunct}{}

%%fakesection Mini ToC
\usepackage[tight]{minitoc}
\mtcsetfont{parttoc}{chapter}{\sffamily\bfseries}
\mtcsetfont{parttoc}{section}{\footnotesize\rmfamily\upshape\mdseries}
\mtcsetfont{parttoc}{subsection}{\footnotesize\rmfamily\upshape\mdseries}
%\mtcsetdepth{parttoc}{1}
\setcounter{parttocdepth}{1}
\renewcommand*{\partheadstartvskip}{\vspace*{20em}}
\renewcommand*{\partheadendvskip}{}
%\noptcrule
\renewcommand\beforeparttoc{\noindent{\bfseries \Large Part \thepart: Contents}}
%\hspace{\fill}\rule{0.95\linewidth}{2pt}\hspace{\fill}
\doparttoc[n]

%%fakesection Misc haxx
\pdfstringdefDisableCommands{\def\Spec{\text{Spec }}\def\sigma{σ}}

\addbibresource{../../references.bib}
\def\asydir{}

\newcommand{\fpartial}[2]{\frac{\partial #1}{\partial #2}}

\begin{document}
\title{Diff Geo}
\maketitle

\chapter{Multivariable Calculus Done Correctly}
As I have ranted about before, linear algebra is done wrong
by the extensive use of matrices to obscure the structure of a linear map.
Similar problems occur with multivariable calculus, so here I would like to set 
the record straight.

Once that's done, I will tell you what a differential form is,
and you'll finally know all those stupid $dx$'s and $dy$'s really mean.
(They weren't just there for decoration!)

Since we are doing this chapter using linear algebra,
it's imperative you're comfortable with linear maps,
and in particular the dual space $V^\vee$ which we will repeatedly use.

\todo{manifolds}

\section{Preliminaries}
\prototype{$V = \RR^n$, and $\norm{(x_1, \dots, x_n)} = \sqrt{x_1^2 + \dots + x_n^2}$.}
In this chapter, all vector spaces are \vocab{normed} and finite-dimensional over $\RR$.
By ``normed'' I mean there is an ``absolute value'' function $\norm{-}_V : V \to \RR_{\ge 0}$
satisfying the triangle inequality and scaling by constants: 
\[
	\norm{v_1+v_2}_V \le \norm{v_1}_V+\norm{v_2}_V
	\qquad\text{and}\qquad
	\norm{cv}_V = c\norm{v}_V \quad\text{(for $c \ge 0$)}.
\]
In particular, $\norm{0_V}_V = 0$.

The norm can be used as a metric on $V$ (by taking $d(x,y) = \norm{x-y}$), thus we can talk about continuous maps.
The typical example, of course, is $V = \RR^n$ as above.


\section{The Total Derivative}
\prototype{If $f(x,y) = x^2+y^2$, then $(Df)_{(x,y)} = 2xe_1^\vee + 2ye_2^\vee$.}
First, let $f : (a,b) \to \RR$.
You might recall from high school calculus that for every point $p \in \RR$,
we defined $f'(p)$ as the derivative at the point $p$ (if it existed), which we interpreted as the \emph{slope} of
the ``tangent line''.

\begin{center}
	\begin{asy}
		import graph;
		size(150,0);

		real f(real x) {return 3-2/(x+2.5);}
		graph.xaxis("$x$");
		graph.yaxis();
		draw(graph(f,-2,2,operator ..), heavygray, Arrows);

		real p = -1;
		real h = 1000 * (f(p+0.001)-f(p));
		real r = 0.9;
		draw( (p+r,f(p)+r*h)--(p-r,f(p)-r*h), red);
		dot( (p, f(p)) );
		draw( (p, f(p))--(p,0), dashed);
		dot("$p$", (p, 0), dir(-90));
		label("$f'(p)$", (p+r/2, f(p) + h*r/2), dir(115));
	\end{asy}
\end{center}

That's fine, but I claim that the ``better'' way to interpret
the derivative at that point is as a \emph{linear map},
that is, as a \emph{function}.
If $f'(p) = 1.5$,
then the derivative tells me that if I move $\eps$ away from $p$
then I should expect $f$ to change by about $1.5\eps$.
In other words,
\begin{moral}
The derivative of $f$ at $p$ approximates $f$ near $p$ by a \emph{linear function}.
\end{moral}

What about more generally?
Suppose I have a function like $f : \RR^2 \to \RR$, say 
\[ f(x,y) = x^2+y^2 \]
for concreteness or something.
For a point $p \in \RR^2$, the ``derivative'' of $f$ at $p$ ought to represent a linear map
that approximates $f$ at that point $p$.
That means I want a linear map $T : \RR^2 \to \RR$ such that
\[ f(p + v) \approx f(p) + T(v) \]
for small displacements $v \in \RR^2$.

Even more generally, if $f : U \to W$ with $U \subseteq V$ open,
then the derivative at $p \in U$ ought to be so that
\[ f(p + v) \approx f(p) + T(v) \in W. \]
(We need $U$ open so that for small enough $v$, $p+v \in U$ as well.)
In fact this is exactly what we're doing earlier with $f'(p)$ in high school.

\missingfigure{2D image}

The only difference is that, by an unfortunate coincidence,
a linear map $\RR \to \RR$ can be represented by just its slope.
And in the unending quest to make everything a number so that it can be AP tested,
we immediately forgot all about what we were trying to do in the first place
and just defined the derivative of $f$ to be a \emph{number} instead of a \emph{function}.

\begin{moral}
	The fundamental idea of Calculus is the local approximation of functions by linear functions.
	The derivative does exactly this.
\end{moral}
Jean Dieudonn\'e as quoted in \cite{ref:pugh} continues:
\begin{quote}
	In the classical teaching of Calculus, this idea is immediately obscured
	by the accidental fact that, on a one-dimensional vector space,
	there is a one-to-one correspondence between linear forms and numbers,
	and therefore the derivative at a point is defined as a number instead of a linear form.
	This \textbf{slavish subservience to the shibboleth of numerical interpretation at any cost}
	becomes much worse . . .
\end{quote}

So let's do this right.
The only thing that we have to do is say what ``$\approx$'' means, and for
this we use the norm of the vector space.
\begin{definition}
	Let $U \subseteq V$ be open.
	Let $f : U \to W$ be a continuous function, and $p \in U$.
	Suppose there exists a linear map $T : V \to W$ such that
	\[
		\lim_{\norm{v} \to 0}
		\frac{\norm{f(p + v) - f(p) - T(v)}_W}{\norm{v}_V} = 0.
	\]
	Then $T$ is the \vocab{total derivative} of $f$ at $p$.
	We denote this by $(Df)_p$, and say $f$ is \vocab{differentiable at $p$}.

	If $(Df)_p$ exists at every point, we say $f$ is \vocab{differentiable}.
\end{definition}

\begin{ques}
	Check that $V = W = \RR$, this is equivalent to the single-variable definition.
	(What are the linear maps from $V$ to $W$?)
\end{ques}
\begin{example}
	Let $V = \RR^2$ with basis $e_1$, $e_2$ and let $W = \RR$,
	and let $f\left( xe_1+ye_2 \right) = x^2+y^2$.  Let $p = ae_1 + be_2$.
	Then, we claim that \[ (Df)_p : \RR^2 \to \RR \quad\text{by}\quad
	v \mapsto 2a \cdot e_1^\vee(v) + 2b \cdot e_2^\vee(v). \]
\end{example}
Here, the notation $e_1^\vee$ and $e_2^\vee$ makes sense,
because by definition $(Df)_p \in V^\vee$: it's a function from $V$ to $\RR$!

Let's check this manually with the limit definition.
Set $v = xe_1 + ye_2$, and note that the norm on $V$ is $\norm{(x,y)}_V = \sqrt{x^2+y^2}$
while the norm on $W$ is just the absolute value $\norm{c}_W = \left\lvert c \right\rvert$.
Then we compute
\begin{align*}
	\frac{\norm{f(p + v) - f(p) - T(v)}_W}{\norm{v}_V} 
	&= \frac{\left\lvert (a+x)^2 + (b+y)^2 - (a^2+b^2) - (2ax+2by) \right\rvert}{\sqrt{x^2+y^2}} \\
	&= \frac{x^2+y^2}{\sqrt{x^2+y^2}} \\
	&= \sqrt{x^2+y^2} \\
	&\to 0
\end{align*}
as $\norm{v} \to 0$.
Thus, for $p = ae_1 + be_2$ we indeed have $(Df)_p = 2a \cdot e_1^\vee + 2b \cdot e_2^\vee$.

\begin{remark}
	As usual, differentiability implies continuity.
\end{remark}

\section{The Projection Principle}
Before proceeding I need to say something really important.
\begin{theorem}[Projection Principle]
	Let $W$ be an $n$-dimensional real vector space with basis $w_1, \dots, w_n$.
	Then a continuous function $f : V \to W$
	amounts to a $n$-tuple of continuous $f_1, f_2, \dots, f_n : V \to \RR$
	\[ f(v) = f_1(v)w_1 + \dots + f_n(v)w_n. \]
\end{theorem}
\begin{proof}
	Obvious.
\end{proof}
The theorem remains true if one replaces ``continuous'' by ``differentiable'', ``smooth'', ``arbitrary'',
or most other reasonable words. Translation:
\begin{moral}
To think about a function $f : V \to \RR^{n}$,
it suffices to think about each coordinate separately.
\end{moral}
For this reason, we'll most often be interested in functions $f : V \to \RR$.
That's why the dual space $V^\vee$ is so important.

\section{Total and Partial Derivatives}
\prototype{If $f(x,y) = x^2+y^2$, then $(Df) : (x,y) \mapsto 2xe_1^\vee + 2ye_2^\vee$, and
$\fpartial fx = 2x$, $\fpartial fy = 2y$.}
Let $U \subseteq V$ be open and let $V$ have a basis $e_1$, \dots, $e_n$.
Suppose $f : U \to \RR$ is a function which is differentiable everywhere,
meaning $(Df)_p \in V^\vee$ exists for every $p$.
In that case, one can consider $Df$ as \emph{itself} a function:
\begin{align*}
	Df : U &\to V^\vee \\
	p &\mapsto (Df)_p.
\end{align*}
\textbf{This is a little crazy}: to every \emph{point} in $U$ we associate a \emph{function} in $V^\vee$.
We say $Df$ is the \vocab{total derivative} of $f$ at $p$.

Let's apply the projection principle now to $Df$.
Since we picked a basis $e_1$, \dots, $e_n$ of $V$,
there is a corresponding dual basis
$e_1^\vee$, $e_2^\vee$, \dots, $e_n^\vee$.
The Projection Principle tells us that $Df$ can thus be thought of as just $n$ functions, so we can write
\[ Df = \psi_1 e_1^\vee + \dots + \psi_n e_n^\vee.  \]
In fact, we can even describe what the $\psi_i$ are.
\begin{definition}
	The \vocab{$i^{\text{th}}$ partial derivative} of $f : U \to \RR$, denoted 
	\[ \fpartial{f}{e_i}: U \to \RR \]
	is defined by
	\[
		\fpartial{f}{e_i} (p)
		\defeq \lim_{t \to 0} \frac{f(p + te_i) - f(v)}{t}.
	\]
\end{definition}
You can think of it as ``$f'$ along $e_i$''.
\begin{ques}
	Check that if $Df$ exists, then \[ (Df)_p(e_i) = \fpartial{f}{e_i}(p). \]
\end{ques}
\begin{remark}
	Of course you can write down a definition of $\fpartial{f}{v}$
	for any $v$ (rather than just the $e_i$).
\end{remark}

From the above remarks, we can derive that
\[
	\boxed{
	Df =
	\frac{\partial f}{\partial e_1} \cdot e_1^\vee
	+ \dots + 
	\frac{\partial f}{\partial e_n} \cdot e_n^\vee .
	}
\]
and so given a basis of $V$, we can think of $Df$ as just
the $n$ partials.
\begin{remark}
Keep in mind that each $\frac{\partial f}{\partial e_i}$ is a function from $U$ to the \emph{reals}.
That is to say,
\[
	(Df)_p =
	\underbrace{\frac{\partial f}{\partial e_1}(p)}_{\in \RR} \cdot e_1^\vee
	+ \dots + 
	\underbrace{\frac{\partial f}{\partial e_n}(p)}_{\in \RR} \cdot e_n^\vee
	\in V^\vee.
\]
\end{remark}


\begin{example}
	Let $f : \RR^2 \to \RR$ by $(x,y) \mapsto x^2+y^2$.
	Then in our new language, 
	\[ Df : (x,y) \mapsto 2x \cdot e_1^\vee + 2y \cdot e_2^\vee \]
	(here $e_1 = (1,0)$, $e_2=(0,1)$).
	Thus the partials are
	\[
		\frac{\partial f}{\partial x} : (x,y) \mapsto 2x \in \RR
		\quad\text{and}\quad
		\frac{\partial f}{\partial y} : (x,y) \mapsto 2y \in \RR
	\]
\end{example}

With all that said, I haven't really said much about how to
find the total derivative itself.
For example, if I told you
\[ f(x,y) = x \sin y + x^2y^4 \]
you might want to be able to compute $Df$ without going through
that horrible limit definition I told you about earlier.

Fortunately, it turns out you already know how to compute partial derivatives,
because you had to take AP Calculus at some point in your life.
It turns out for most reasonable functions, this is all you'll ever need.
\begin{theorem}[Continuous Partials Implies Differentiable]
	Let $f : U \to W$ and suppose that $\fpartial{f}{e_i}$ is defined
	for each $i$ and moreover is \emph{continuous}.
	Then $f$ is differentiable and given by
	\[ Df = \sum \fpartial{f}{e_i} \cdot e_i^\vee. \]
\end{theorem}
\begin{proof}
	Not going to write out the details, but\dots
	given $v = t_1e_1 + \dots + t_ne_n$,
	the idea is to just walk from $p$ to $p+t_1e_1$, $p+t_1e_1+t_2e_2$, \dots,
	up to $p+t_1e_1+t_2e_2+\dots+t_ne_n = p+v$,
	picking up the partial derivatives on the way.
	Do some calculation.
\end{proof}

\begin{remark}
	The continuous condition cannot be dropped. The function
	\[
			f(x,y)
		=
		\begin{cases}
			\frac{xy}{x^2+y^2} & (x,y) \neq (0,0) \\
			0 & (x,y) = (0,0).
		\end{cases}
	\]
	is the classical counterexample -- the total derivative $Df$ does not exist at zero,
	even though both partials do.
\end{remark}

\begin{example}
	[Actually Computing a Total Derivative]
	Let $f(x,y) = x \sin y + x^2y^4$. Then
	\begin{align*}
		\fpartial fx (x,y) &= \sin y + y^4 \cdot 2x \\
		\fpartial fy (x,y) &= x \cos y + x^2 \cdot 4y^3.
	\end{align*}
	Then $Df = \fpartial fx e_1^\vee + \fpartial fy e_2^\vee$,
	which I won't bother to write out.
\end{example}

The example $f(x,y) = x^2+y^2$ is the same thing.
That being said, who cares about $x \sin y + x^2y^4$ anyways?


\section{(Optional) Gradients}
Lovers of matrices and vectors often call $(Df)_p$ the \vocab{gradient}
of $f$ at $p$, and denote it $(\nabla f)_p$.
It makes sense 
\todo{dot product, canonical isomorphism}



\section{Problems}



\end{document}
