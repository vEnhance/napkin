\documentclass[11pt]{scrreprt}
%%fakesection Load packages

\usepackage{lmodern}
\usepackage[pdfusetitle]{hyperref}
\ExplSyntaxOn
\sys_if_engine_luatex:T {
	\usepackage{luatex85}
}
\sys_if_engine_pdftex:T {
	\usepackage[T1]{fontenc}
}
\ExplSyntaxOff

% These are evan.sty
\usepackage{amsmath,amssymb,amsthm}
\usepackage{mathrsfs}
\usepackage[usenames,svgnames,dvipsnames]{xcolor}
\usepackage{textcomp}
\usepackage{enumerate}
\usepackage[textsize=scriptsize,shadow]{todonotes}
\usepackage{mathtools}
\usepackage{microtype}
\usepackage[normalem]{ulem}
\usepackage{stmaryrd}
\usepackage{wasysym}
\usepackage{multirow}
\usepackage{prerex}
\usepackage[nameinlink]{cleveref}
\usepackage{derivative}

%%fakesection evan.sty macros
%Small commands
%% Napkin commands
\newcommand{\prototype}[1]{
	\emph{{\color{red} Prototypical example for this section:} #1} \par\medskip
}
\newenvironment{moral}{%
	\begin{tcolorbox}[boxrule=0.4pt,colframe=green!70!black,sharp corners,
		standard jigsaw,opacityback=0,left=10pt,right=10pt,top=3pt,bottom=3pt,
		before skip=10pt,after skip=20pt]%
	\bfseries\color{green!50!black}}%
	{\end{tcolorbox}}

%%fakesection Links (hyperref loaded earlier implicitly)
\hypersetup{
	linkcolor={red!50!black},
	citecolor={green!50!black},
	urlcolor={blue!80!black},
	pdfkeywords={napkin,math},
	pdfsubject={web.evanchen.cc},
	colorlinks,
}

%%fakesection Commutative diagrams
\usepackage{tikz-cd}
\usetikzlibrary{arrows,arrows.meta}
% make a larger hook
% https://tex.stackexchange.com/questions/514451/how-to-define-a-new-hooked-arrow
\makeatletter
\pgfdeclarearrow{
	name=xGlyph,
	cache=false,
	bending mode=none,
	parameters={\tikzcd@glyph@len,\tikzcd@glyph@shorten},
	setup code={%
		\pgfarrowssettipend{\tikzcd@glyph@len\advance\pgf@x by\tikzcd@glyph@shorten}},
	defaults={
		glyph axis=axis_height,
		glyph length=+1.55ex,
		glyph shorten=+-0.1ex},
	drawing code={%
		\pgfpathrectangle{\pgfpoint{+0pt}{+-1.5ex}}{\pgfpoint{+\tikzcd@glyph@len}{+3ex}}%
		\pgfusepathqclip%
		\pgftransformxshift{+\tikzcd@glyph@len}%
		\pgftransformyshift{+-\tikzcd@glyph@axis}%
		\pgftext[right,base]{\tikzcd@glyph}}}
\makeatother
\tikzcdset{
	arrow style=tikz,
	diagrams={>={Latex}},
	tikzcd left hook/.tip={xGlyph[glyph math command=supset, swap, glyph axis = 5.7pt]},
	tikzcd right hook/.tip={xGlyph[glyph math command=supset, glyph axis = 5.7pt]},
	surjective head arrow /.tip = {tikzcd to[sep=-1.5pt]tikzcd to},
	surjective head/.style={
		-surjective head arrow
	}
}

%%fakesection Page layout
\usepackage[headsepline]{scrlayer-scrpage}
\renewcommand{\headfont}{}
\addtolength{\textheight}{3.14cm}
\setlength{\footskip}{0.5in}
\setlength{\headsep}{10pt}

\def\shortdate{\leavevmode\hbox{\the\year-\twodigits\month-\twodigits\day}}
\def\twodigits#1{\ifnum#1<10 0\fi\the#1}
\automark[chapter]{chapter}

\rohead{\footnotesize\thepage}
\rehead{\footnotesize \textbf{\sffamily Napkin}, by \emph{Evan Chen} (\napkinversion)}
\lehead{\footnotesize\thepage}
\lohead{\footnotesize \leftmark}
\chead{}
\rofoot{}
\refoot{}
\lefoot{}
\lofoot{}
%\cfoot{\pagemark}

%%fakesection Fancy section and chapter heads
\renewcommand*{\sectionformat}{\color{purple}\S\thesection\autodot\enskip}
\renewcommand*{\subsectionformat}{\color{purple}\S\thesubsection\autodot\enskip}
\newcommand{\problemhead}{A few harder problems to think about}
\renewcommand{\thesubsection}{\thesection.\roman{subsection}}

\addtokomafont{chapterprefix}{\raggedleft}
\RedeclareSectionCommand[beforeskip=0.5em]{chapter}
\renewcommand*{\chapterformat}{%
\mbox{\scalebox{1.5}{\chapappifchapterprefix{\nobreakspace}}%
\scalebox{2.718}{\color{purple}\thechapter\autodot}\enskip}}

\addtokomafont{partprefix}{\rmfamily}
\renewcommand*{\partformat}{\color{purple}\scalebox{2.5}{\thepart}}

%%fakesection Theorems
\usepackage{tcolorbox}
\tcbuselibrary{breakable,skins,hooks}

% patch tcolorbox to support continuation of a paragraph
% after box
% from https://tex.stackexchange.com/questions/568782/
\makeatletter
\tcbset{
	after app={%
		\ifx\tcb@drawcolorbox\tcb@drawcolorbox@breakable
		\else
		% add only when not breakabel
		\@endparenv
		\fi
	}
}

% for breakable
\appto\tcb@use@after@lastbox{\@endparenv\@doendpe}
\makeatother
% END patch tcolorbox for continuation of paragraph


\usepackage{thmtools}

\theoremstyle{definition}
\declaretheoremstyle[
	headfont=\sffamily\bfseries\color{MidnightBlue},
	headpunct={\\[3pt]},
	postheadspace={0pt},
]{thmtheorem}


\declaretheoremstyle[
	headfont=\bfseries\color{RawSienna},
	headpunct={\\[3pt]},
	postheadspace={0pt},
]{thmexample}

\tcbset{
	theorem box/.style={
		enhanced,
		arc=9pt,
		outer arc=10pt,
		colframe=blue,
		colback=TealBlue!5,
		boxrule=1pt,
		before skip=12pt,
		after skip=14pt,
		left=10pt,
		right=10pt
	},
	remark box/.style={
		boxrule=0pt,
		frame hidden,
		sharp corners,
		enhanced,
		borderline west={2pt}{0pt}{ForestGreen},
		before skip=8pt,
		colback=ForestGreen!5,
		after skip=12pt,
		breakable,
		left=10pt,
		right=10pt
	},
	example box/.style={
		enhanced,
		sharp corners,
		arc=9pt,
		outer arc=10pt,
		colframe=RawSienna,
		colback=Salmon!5,
		boxrule=0.5pt,
		before skip=12pt,
		after skip=14pt,
		breakable,
		top=6pt,
		bottom=8pt,
		breakable,
		left=10pt,
		right=10pt
	},
	ques box/.style={
		boxrule=0pt,
		frame hidden,
		enhanced,
		sharp corners,
		before skip=8pt,
		after skip=12pt,
		borderline west={3pt}{0pt}{black},
		colback=RedViolet!5!gray!5,
		breakable,
		left=10pt,
		right=10pt
	}
}

\declaretheorem[style=thmtheorem,name=Theorem,numberwithin=section]{theorem}
\tcolorboxenvironment{theorem}{theorem box}
\declaretheorem[style=thmtheorem,name=Lemma,sibling=theorem]{lemma}
\tcolorboxenvironment{lemma}{theorem box}
\declaretheorem[style=thmtheorem,name=Proposition,sibling=theorem]{proposition}
\tcolorboxenvironment{proposition}{theorem box}
\declaretheorem[style=thmtheorem,name=Corollary,sibling=theorem]{corollary}
\tcolorboxenvironment{corollary}{theorem box}
\declaretheorem[style=thmexample,name=Example,sibling=theorem]{example}
\tcolorboxenvironment{example}{example box}

\declaretheoremstyle[
	headfont=\bfseries\sffamily\color{ForestGreen!70!black},
	bodyfont=\normalfont,
	headpunct={ --- },
]{thmremark}
\declaretheoremstyle[
	headfont=\bfseries\sffamily\color{ForestGreen!70!black},
	bodyfont=\normalfont,
	headpunct={},
]{thmremark*}

\declaretheoremstyle[
	headfont=\bfseries,
	bodyfont=\normalfont\small
]{thmques}

\declaretheorem[name=Question,sibling=theorem,style=thmques]{ques}
\tcolorboxenvironment{ques}{ques box}
\declaretheorem[name=Exercise,sibling=theorem,style=thmques]{exercise}
\tcolorboxenvironment{exercise}{ques box}
\declaretheorem[name=Remark,sibling=theorem,style=thmremark]{remark}
\tcolorboxenvironment{remark}{remark box}
\declaretheorem[name=Remark,sibling=theorem,style=thmremark*]{remark*}
\tcolorboxenvironment{remark*}{remark box}
\declaretheorem[name=Step,style=thmremark]{step} % only used in Lebesgue int
\tcolorboxenvironment{step}{remark box}

\theoremstyle{definition}
\newtheorem{claim}[theorem]{Claim}
\newtheorem{definition}[theorem]{Definition}
\newtheorem{fact}[theorem]{Fact}
\newtheorem{abuse}[theorem]{Abuse of Notation}

\newtheorem{problem}{Problem}[chapter]
\renewcommand{\theproblem}{\thechapter\Alph{problem}}
\newtheorem{sproblem}[problem]{Problem}
\newtheorem{dproblem}[problem]{Problem}
\renewcommand{\thesproblem}{\theproblem$^{\star}$}
\renewcommand{\thedproblem}{\theproblem$^{\dagger}$}
\newcommand{\listhack}{$\empty$\vspace{-2em}}

%%fakesection Answers
\usepackage{answers}
\Newassociation{hint}{answeritem}{tex/backmatter/all-hints}
\Newassociation{sol}{answeritem}{tex/backmatter/all-solns}
\renewcommand{\solutionextension}{out}
\renewenvironment{answeritem}[1]{\item[\bfseries #1.]}{}

%%fakesection Table of contents
% First add ToC to ToC
\makeatletter
\usepackage{etoolbox}
\pretocmd{\tableofcontents}{%
	\if@openright\cleardoublepage\else\clearpage\fi
	\pdfbookmark[0]{\contentsname}{toc}%
}{}{}%
\makeatother
\setcounter{tocdepth}{1}
\RedeclareSectionCommand[tocnumwidth=4.2em]{part}
\RedeclareSectionCommand[tocpagenumberwidth=2.2em,tocnumwidth=4.2em]{chapter}
\RedeclareSectionCommand[tocpagenumberwidth=2.2em,tocnumwidth=2.8em]{section}
% adjust tocpagenumberwidth manually for large page number: https://tex.stackexchange.com/a/502168

%%fakesection Asymptote definitions
\usepackage{patch-asy}
\numberwithin{asy}{chapter}
\renewcommand{\theasy}{\thechapter\Alph{asy}}
\begin{asydef}
	import extras;
	size(6cm);
	usepackage("amsmath");
	usepackage("amssymb");
	defaultpen(fontsize(11pt));
	settings.tex = "latex";
	settings.outformat = "pdf";
\end{asydef}
\def\asydir{asy}

%%fakesection Bibliography
\usepackage[backend=biber,backref=true,style=alphabetic]{biblatex}
\DeclareLabelalphaTemplate{
	\labelelement{
		\field[final]{shorthand}
		\field{label}
		\field[strwidth=2,strside=left]{labelname}
	}
	\labelelement{
		\field[strwidth=2,strside=right]{year}
	}
}
\DeclareFieldFormat{labelalpha}{\textbf{\scriptsize #1}}
\addbibresource{references.bib}
\addbibresource{images.bib}
%% stylistic biblatex choices
\DefineBibliographyStrings{english}{%
	backrefpage  = {cited p.}, % for single page number
	backrefpages = {cited pp.} % for multiple page numbers
}
\DeclareFieldFormat{journaltitle}{\mkbibemph{#1},} % italic journal title with comma
\DeclareFieldFormat[inbook,thesis]{title}{\mkbibemph{#1}\addperiod} % italic title with period
\DeclareFieldFormat[article]{title}{#1} % title of journal article is printed as normal text
\DeclareFieldFormat[article]{volume}{\textbf{#1}\addcolon\space}
\renewcommand{\mkbibnamegiven}[1]{\textsc{#1}}
\renewcommand{\mkbibnamefamily}[1]{\textsc{#1}}
\renewcommand{\mkbibnameprefix}[1]{\textsc{#1}}
\renewcommand{\mkbibnamesuffix}[1]{\textsc{#1}}
\renewcommand{\finentrypunct}{}

%%fakesection Mini ToC
\usepackage[tight]{minitoc}
\mtcsetfont{parttoc}{chapter}{\sffamily\bfseries}
\mtcsetfont{parttoc}{section}{\footnotesize\rmfamily\upshape\mdseries}
\mtcsetfont{parttoc}{subsection}{\footnotesize\rmfamily\upshape\mdseries}
%\mtcsetdepth{parttoc}{1}
\setcounter{parttocdepth}{1}
\renewcommand*{\partheadstartvskip}{\vspace*{20em}}
\renewcommand*{\partheadendvskip}{}
%\noptcrule
\renewcommand\beforeparttoc{\noindent{\bfseries \Large Part \thepart: Contents}}
%\hspace{\fill}\rule{0.95\linewidth}{2pt}\hspace{\fill}
\doparttoc[n]

%%fakesection Misc haxx
\pdfstringdefDisableCommands{\def\Spec{\text{Spec }}\def\sigma{σ}}

\def\asydir{}
\addbibresource{../../references.bib}
\renewcommand{\gim}{($\ast$)}


\begin{document}
\title{Representation Theory}
\maketitle

\chapter{Representations of Algebras}
In the 19th century, a group was literally defined
as a subset of $\GL(n)$ or of $S_n$: the word ``group'' hadn't been invented yet.
Only much later did the abstract definition of a group was given,
an abstract set $G$ which was an object in its own right.

While this abstraction is good for some reasons,
it is often also useful to work with concrete representations.
This is the subject of representation theory.
Linear algebra is easier than abstract algebra,
so if we can take a group $G$ and represent it concretely
as a set of matrices in $\opname{GL}(n)$,
this makes them easier to study.
This is the \emph{representation theory of groups}:
how can we take a group and represent its elements as matrices?

\section{Algebras}
\prototype{$k[x_1, \dots, x_n]$ and $k[G]$.}
Rather than working directly with groups from the beginning,
it will be more convenient to deal with so-called $k$-algebras.
This setting is more natural and general than that of groups,
so once we develop the theory of algebras well enough,
it will be fairly painless to specialize to the case of groups.

Colloquially,
\begin{moral}
	An associative $k$-algebra is
	a (possibly noncommutative) ring with a copy of $k$ inside it.
	It is thus a $k$-vector space.
\end{moral}
In particular this makes such an algebra it into a $k$-vector space.
I'll present examples before the definition:
\begin{example}
	[Examples of $k$-Algebras]
	Let $k$ be any field.
	\begin{enumerate}[(a)]
		\ii The field $k$ itself is a $k$-algebra.
		\ii The polynomial ring $k[x_1, \dots, x_n]$.
		\ii The set of $n \times n$ matrices with entries in $k$,
		which we denote by $\Mat_n(k)$.
		Note the multiplication here is not commutative.
		\ii The set $\Mat(V)$ of linear operators $T : V \to V$,
		with multiplication given by the composition of operators.
		(Here $V$ is some vector space over $k$.)
		This is really the same as the previous example.
	\end{enumerate}
\end{example}
\begin{definition}
	Let $k$ be a field.
	A \vocab{$k$-algebra} $A$ is a ring, \emph{possibly noncommutative},
	equipped with an ring homomorphism $k \to A$
	(whose image is the ``copy of $k$'').
	Thus we can consider $k$ as a subset of $A$, and
	we then additionally require $\lambda \cdot a = a \cdot \lambda$
	for each $\lambda0 \in k$ and $a \in A$.

	If the multiplication operation is also commutative,
	then we say $A$ is a \vocab{commutative} algebra.
\end{definition}
\begin{definition}
	Equivalently, a \vocab{$k$-algebra} $A$ is a
	$k$-\emph{vector space} which also has a associative,
	bilinear multiplication operation (with an identity $1_A$).
	The ``copy of $k$'' is obtained by considering elements
	$\lambda 1_A$ for each $\lambda \in k$
	(i.e.\ scaling the identity by the elements of $k$,
	taking advantage of the vector space structure).
\end{definition}

\begin{abuse}
	Some authors don't require $A$ to be associative or have unit,
	so to them what we have just defined is an
	``associative algebra with $1$''.
	However, this is needlessly wordy for our purposes.
\end{abuse}

\begin{example}
	[Group Algebra]
	The \vocab{group algebra} $k[G]$ is the $k$-vector space
	whose \emph{basis elements} are the elements of a group $G$,
	and where the product of two basis elements is the group multiplication.
	For example, suppose $G = \Zc 2 = \{1_G, x\}$.
	Then
	\[ k[G] = \left\{ a1_G + bx \mid a,b \in k \right\} \]
	with multiplication given by
	\[ (a1_G + bx)(c1_G+dx) = (ac+bd)1_G + (bc+ad)x. \]
\end{example}
\begin{ques}
	When is $k[G]$ commutative?
\end{ques}
The example $k[G]$ is very important,
because (as we will soon see) a representation of the algebra $k[G]$
amounts to a representation of the group $G$ itself.

It is worth mentioning at this point that:
\begin{definition}
	A \vocab{homomorphism} of $k$-algebras $A$, $B$ is a
	linear map $T : A \to B$ which respects multiplication
	(i.e.\ $T(xy) = T(x)T(y)$) and which sends $1_A$ to $1_B$.
	In other words, $T$ is both a homomorphism as a ring and as a vector space.
\end{definition}
\begin{definition}
	Given $k$-algebras $A$ and $B$, the \vocab{direct sum} $A \oplus B$
	is defined as pairs $a + b$, where addition is done in the obvious way,
	but we declare $ab = 0$ for any $a \in A$ and $b \in B$.
\end{definition}
\begin{ques}
	Show that $1_A + 1_B$ is the multiplicative identity of $A \oplus B$.
\end{ques}

\section{Representations}
\prototype{$k[S_3]$ acting on $k^{\oplus 3}$ is my favorite.}

\begin{definition}
	A \vocab{representation} of a $k$-algebra $A$ consists of
	\begin{enumerate}[(i)]
		\ii A $k$-vector space $V$, and
		\ii An \emph{action} $\cdot$ of $A$ on $k$: thus, for every $a \in A$
		we can take $v \in V$ and act on it to get $a \cdot v$.
		This satisfies the usual axioms:
		\begin{itemize}
			\ii $(a+b) \cdot v = a \cdot v + b \cdot v$,
			and $(ab) \cdot v = a(b \cdot v)$.
			\ii $\lambda \cdot v = \lambda v$ for $\lambda \in k$.
			In particular, $1_A \cdot v = v$.
		\end{itemize}
	\end{enumerate}
	In other words, a representation is a \textbf{left $A$-module} $V$.
\end{definition}

\begin{definition}
	The action of $A$ can be more succinctly described as saying
	that there is a \emph{$k$-algebra homomorphism} $\rho : A \to \Mat(V)$.
	(So $a \cdot v = \rho(a)(v)$.)
	Thus we can also define a \vocab{representation} of $A$ as a pair
	\[ \left( V, \rho : A \to \Mat(V) \right). \]
\end{definition}
\begin{remark}
	This is completely analogous to how a group action $G$ on a set $X$
	with $n$ elements just amounts to a group homomorphism $G \to S_n$.
\end{remark}
From this perspective, what we are really trying to do is:
\begin{moral}
	If $A$ is an algebra,
	we are trying to \emph{represent}
	the elements of $A$ as matrices.
\end{moral}

\begin{abuse}
	While a representation is a pair $(V, \rho)$
	of both the vector space $V$ and the action $\rho$,
	we frequently will just abbreviate it to ``$V$''.
	This is probably one of the worst abuses I will commit.
\end{abuse}

\begin{example}
	[Representations of $\Mat(V)$]
	\listhack
	\begin{enumerate}[(a)]
		\ii Let $A = \Mat_2(\RR)$.
		Then there is a representation $(\RR^{\oplus 2}, \rho)$
		where a matrix $a \in A$ just acts by $a \cdot v = \rho(a)(v) = a(v)$.

		\ii More generally, given a vector space $V$ over any field $k$,
		there is an obvious representation of $A = \Mat(V)$
		by $a \cdot v = \rho(a)(v) = a(v)$ (since $a \in \Mat(V)$).

		From the matrix perspective: if $A = \Mat(V)$,
		then we can just represent $A$ as matrices over $V$.

		\ii There are other representations of $A = \Mat_2(\RR)$.
		A silly example is the representation $(\RR^{\oplus 4}, \rho)$ given by
		\[
			\rho : 
			\begin{pmatrix} a & b \\ c & d \end{pmatrix} 
			\mapsto
			\begin{pmatrix} a & b & 0 & 0 \\ c & d & 0 & 0 \\
				0 & 0 & a & b \\ 0 & 0 & c & d \end{pmatrix} .
		\]
		More abstractly, viewing $\RR^{\oplus 4}$ as
		$(\RR^{\oplus 2}) \oplus (\RR^{\oplus 2})$,
		this is $a \cdot (v_1,v_2) = (a \cdot v_1, a \cdot v_2)$.
	\end{enumerate}
\end{example}

\begin{example}
	[Representations of Polynomials Algebras]
	\listhack
	\begin{enumerate}[(a)]
		\ii Let $A = k$.
		Then a representation of $k$ is just any $k$-vector space $V$.

		\ii If $A = k[x]$,
		then a representation $(V, \rho)$ of $A$
		amounts to a vector space $V$ plus the choice of
		a linear operator $T \in \Mat(V)$ (the image $\rho(x)$).

		\ii If $A = k[x] / (x^2)$
		then a representation $(V, \rho)$ of $A$
		amounts to a vector space $V$ plus the choice of
		a linear operator $T \in \Mat(V)$ satisfying $T^2 = 0$.

		\ii We can create arbitrary ``functional equations'' with this pattern.
		For example, if $A = k[x,y] / (x^2 - x+y, y^4)$
		then representing $A$ by $V$ amounts to finding operators
		$S, T \in \Mat(V)$ satisfying $S^2 = S-T$ and $T^4 = 0$.
	\end{enumerate}
\end{example}


\begin{example}
	[Representations of Groups]
	\listhack
	\begin{enumerate}[(a)]
		\ii Let $A = \RR[S_3]$.
		Then let 
		\[ V = \RR^{\oplus 3} = \{ (x,y,z) \mid x,y,z \in \RR \}. \]
		We can let $A$ act on $V$ as follows:
		given a permutation $\pi \in S_3$, we permute the corresponding
		coordinates in $V$.
		So for example, if 
		\[ \text{If } \pi = (1 \; 2)
		\text{ then } \pi \cdot (x,y,z) = (y,x,z). \]
		This extends linearly to let $A$ act on $V$,
		by permuting the coordinates.

		From the matrix perspective, what we are doing
		is representing the permutations in $S_3$
		as permutation matrices on $k^{\oplus 3}$, like
		\[ (1 \; 2)
		\mapsto \begin{pmatrix} 0&1&0 \\ 1&0&0 \\ 0&0&1 \end{pmatrix}. \]
		
		\ii More generally, let $A = k[G]$.
		Then a representation $(V, \rho)$ of $A$ amounts to 
		a group homomorphism $\psi : G \to \GL(V)$.
	\end{enumerate}
\end{example}
\begin{example}[Regular Representation]
	Any $k$-algebra $A$ is a representation $(A, \rho)$ over itself,
	with $a \cdot b = \rho(a)(b) = ab$ (i.e.\ multiplication given by $A$).
	This is called the \vocab{regular representation}, denoted $\Reg(A)$.
\end{example}

\begin{example}
	[Direct Sums of Representations]
	Let $A$ be $k$-algebra and let $V = (V, \rho_1)$ and $W = (W, \rho_2)$
	be two representations of $A$.
	Then $V \oplus W$ is a representation, with action $\rho$ by
	\[ a \cdot (v,w) = (a \cdot v, a \cdot w). \]
	In terms of matrices, this looks like
	\[ \rho(a) = \begin{pmatrix}
			\rho_1(a) & 0 \\ 0 & \rho_2(a)
		\end{pmatrix}. \]
	This is called the \vocab{direct sum} of $V$ and $W$.
	We have already seen this example earlier in the special case
	when we let $A = \Mat_2(\RR)$ act on $\RR^{\oplus 4}$.
\end{example}

Direct sums also come up when we play with algebras.
\begin{proposition}[Representations of $A \oplus B$ are $V_A \oplus V_B$]
	\label{prop:rep_direct_sum}
	Let $A$ and $B$ be $k$-algebras.
	Then every representation of $A \oplus B$ is of the form 
	\[ V_A \oplus V_B \]
	where $V_A$ and $V_B$ are representations of $A$ and $B$, respectively.
\end{proposition}
\begin{proof}[Sketch of Proof]
	Let $(V, \rho)$ be a representation of $A \oplus B$.
	For any $v \in V$, $\rho(1_A+1_B)v = \rho(1_A)v + \rho(1_B)v$.
	One can then set $V_A = \{ \rho(1_A)v \mid v \in V \}$
	and $V_B = \{ \rho(1_B)v \mid v \in V \}$.
	These are disjoint, since if $\rho(1_A) v = \rho(1_B) v'$,
	we have $\rho(1_A)v = \rho(1_A1_A)v = \rho(1_A1_B) v' = 0_V$,
	and similarly for the other side.
\end{proof}

\section{Irreducible and Indecomposable Representations}
\prototype{$k[S_3]$ decomposes as the sum of two spaces.}

One of the goals of representation theory will be to classify
all possible representations of an algebra $A$.
If we want to have a hope of doing this,
then we want to discard ``silly'' representations such as
\[
	\rho : 
	\begin{pmatrix} a & b \\ c & d \end{pmatrix} 
	\mapsto
	\begin{pmatrix} a & b & 0 & 0 \\ c & d & 0 & 0 \\
		0 & 0 & a & b \\ 0 & 0 & c & d \end{pmatrix} .
\]
and focus our attention instead on ``irreducible'' representations.
This motivates the following definition.

\begin{definition}
	Let $V$ be a representation of $A$.
	A \vocab{subrepresentation} $W \subset V$ is a subspace $W$
	with the property that for any $a \in A$ and $w \in W$,
	$a \cdot w = w$.
	In other words, this subspace is invariant under actions by $A$.
\end{definition}
Thus if $V \neq W_1 \oplus W_2$ for representations $W_1$, $W_2$
then $W_1$ and $W_2$ are subrepresentations.

\begin{definition}
	If $V$ has no proper subrepresentations then it is \vocab{irreducible}.
	If $V \neq W_1 \oplus W_2$ for proper subrepresentations $W_1$, $W_2$,
	then we say it is \vocab{indecomposable}.
\end{definition}

Why do we need two parts of the definition?
Unfortunately, if $W$ is a subrepresentation of $V$,
then it is not necessarily the case that we can find a
supplementary vector space $W'$ such that $V = W \oplus W'$.
Put another way, if $V$ is reducible, we know that it has a subrepresentation,
but a decomposition requires \emph{two} subrepresentations.
Here is a standard counterexample:
\begin{exercise}
	Let $A = \RR[x]$, and $V = \RR^{\oplus 2}$ the representation with action
	\[ \rho(x) = \begin{pmatrix} 1 & 1 \\ 0 & 1 \end{pmatrix}. \]
	Show that the only subrepresentation is $W = \{ (t,0) \mid t \in \RR \}$.
	So $V$ is not irreducible, but it is indecomposable.
\end{exercise}

Here is a slightly more optimistic example,
and the ``prototypical example'' that you should keep in mind.
\begin{example}[Representation of $S_n$ decomposes]
	Let $A = \RR[S_3]$ again,
	with the action on 
	\[ V = \RR^{\oplus 3} = \{ (x,y,z) \mid x,y,z \in \RR \}. \]
	Consider the two subspaces
	\begin{align*}
		W_1 &= \left\{ (t,t,t) \mid t \in \RR \right\} \\
		W_2 &= \left\{ (x,y,z) \mid x+y+z = 0 \right\}.
	\end{align*}
	Note $V = W_1 \oplus W_2$ as vector spaces.
	But each of $W_1$ and $W_2$ is a subrepresentation
	(since the action of $A$ keeps each $W_i$ in place),
	so $V = W_1 \oplus W_2$ as representations.
\end{example}

\begin{exercise}
	Let $A = \Mat_d(k)$ and consider the obvious representation $k^{\oplus d}$
	that we described earlier. Show that it is irreducible.
	(This is obvious if you understand the definitions well enough.)
\end{exercise}

\section{Morphisms of Representations}
We now proceed to define the morphisms between representations.

\begin{definition}
	Let $(V, \rho_1)$ and $(W, \rho_2)$ be representations of $A$.
	An \vocab{intertwining operator}, or \vocab{morphism}, is a
	linear map $T : V \to W$ such that
	\[ T(a \cdot v) = a \cdot T(v) \]
	for any $a \in A$, $v \in V$.
	(Note the first $\cdot$ is the action $\rho_1$
	and the second $\cdot$ is the action of $\rho_2$.)
	This is exactly what you expect if you think that $V$ and $W$
	are ``left $A$-modules''.
	If $T$ is invertible, then it is an \vocab{isomorphism} of representations
	and we say $V \cong W$.
\end{definition}
\begin{remark}
	[For Commutative Diagram Lovers]
	The condition $T(a \cdot v) = a \cdot T(v)$ can be read as saying that
	\begin{diagram}
		V & \rTo^{\rho_1(a)} & V \\
		\dTo^T & & \dTo_T \\
		W & \rTo^{\rho_2(a)} & W
	\end{diagram}
	commutes for any $a \in A$.
\end{remark}

\begin{remark}
	[For Category Lovers]
	A representation is just a ``bilinear'' functor from an
	abelian one-object category $\{\ast\}$ (so $\Hom(\ast, \ast) \cong A$)
	to the abelian category $\catname{Vect}_k$.
	Then an intertwining operator is just a \emph{natural transformation}.
\end{remark}

Here are some examples of intertwining operators.
\begin{example}[Intertwining Operators]
	\listhack
	\begin{enumerate}[(a)]
		\ii For any $\lambda \in k$, the scalar map $T(v) = \lambda v$
		is intertwining.
		\ii If $W \subseteq V$ is a subrepresentation,
		then the inclusion $W \injto V$ is a subrepresentation.
		\ii The projection map $V_1 \oplus V_2 \surjto V_1$
		is an intertwining operator.
		\ii Let $V  = \RR^{\oplus 2}$ 
		and represent $A = k[x]$ by $(V, \rho)$ where
		\[ \rho(x) = \begin{pmatrix} 0 & 1 \\ -1 & 0 \end{pmatrix}. \]
		Thus $\rho(x)$ is rotation by $90\dg$ around the origin.
		Let $T$ be rotation by $30\dg$.
		Then $T : V \to V$ is intertwining (the rotations commute).
	\end{enumerate}
\end{example}

\begin{exercise}[Kernel and Image Are Subrepresentations]
	Let $T : V \to W$ be an intertwining operator.
	\begin{enumerate}[(a)]
		\ii Show that $\ker T \subseteq V$ is a subrepresentation of $V$.
		\ii Show that $\img T \subseteq W$ is a subrepresentation of $W$.
	\end{enumerate}
\end{exercise}

This gives us the famous Schur's Lemma.
\begin{theorem}
	[Schur's Lemma]
	Let $V$ and $W$ be representations of a $k$-algebra $A$.
	Let $T : V \to W$ be a \emph{nonzero} intertwining operator.
	Then
	\begin{enumerate}[(a)]
		\ii If $V$ is irreducible, then $T$ is injective.
		\ii If $W$ is irreducible, then $T$ is surjective.
	\end{enumerate}
	In particular if both $V$ and $W$ are irreducible then $T$
	is either zero or an isomorphism.
\end{theorem}
An important special case is if $k$ is algebraically closed:
then the only intertwining operators $V \to V$
are the trivial constant ones.
\begin{theorem}
	[Schur's Lemma for Algebraically Closed Fields]
	Let $k$ be an algebraically closed field.
	Let $V$ and $V$ be irreducible representations of a $k$-algebra $A$.
	Then any intertwining operator $T : V \to V$ is multiplication by a scalar.
\end{theorem}
\begin{exercise}
	Use the fact that $T$ has an eigenvalue $\lambda$ to
	deduce this from Schur's Lemma.
	(Consider $T - \lambda \cdot \id_V$, and use Schur to deduce it's zero.)
\end{exercise}
We have already seen the counterexample of rotation by $90\dg$ for $k = \RR$;
this was the same counterexample we gave to the assertion that all linear maps
have eigenvalues.

\section{The Representations of $\Mat_d(k)$}
To give an example of the kind of progress already possible,
we prove the following theorem.
\begin{theorem}
	[Representations of $\Mat_d(k)$]
	\label{thm:rep_1mat}
	Let $k$ be any field, $d$ be a positive integer and
	let $X = k^{\oplus d}$ be the obvious representation of $A = \Mat_d(k)$.
	Then the only finite-dimensional representations
	of $\Mat_d(k)$ are $W^{\oplus n}$
	for some positive integer $n$ (up to isomorphism).
	In particular, it is irreducible if and only if $n=1$.
\end{theorem}
For concreteness, I'll just sketch the case $d=2$,
since the same proof applies verbatim to other situations.
This shows that the examples of representations of $\Mat_2(\RR)$
we gave earlier are the only ones.

As we've said this is essentially a functional equation.
The algebra $A = \Mat_2(k)$ has basis given by four matrices
\[
	E_1 = \begin{pmatrix} 1 & 0 \\ 0 & 0 \end{pmatrix},
	\qquad
	E_2 = \begin{pmatrix} 0 & 0 \\ 0 & 1 \end{pmatrix},
	\qquad
	E_3 = \begin{pmatrix} 0 & 1 \\ 0 & 0 \end{pmatrix},
	\qquad
	E_4 = \begin{pmatrix} 0 & 0 \\ 1 & 0 \end{pmatrix}
\]
satisfying relations like $E_1 + E_2 = \id$, $E_i^2 = E_i$, $E_1E_2 = 0$, etc.
So let $V$ be a representation of $A$, and let $M_i = \rho(E_i)$ for each $i$;
we want to classify the possible matrices $M_i$ on $V$
satisfying the same functional equations.
This is because, for example,
\[ \id_V = \rho(\id_A) = \rho(E_1+E_2) = M_1 + M_2. \]
By the same token $M_1M_3 = M_3$.
Proceeding in a similar way, we can obtain the following multiplication table.
\[
	\begin{array}{r|llll}
		\times & M_1 & M_2 & M_3 & M_4 \\ \hline
		M_1 & M_1 & 0 & M_3 & 0 \\
		M_2 & 0 & M_2 & 0 & M_4 \\
		M_3 & 0 & M_3 & 0 & M_1 \\
		M_4 & M_4 & 0 & M_2 & 0
	\end{array}
	\qquad \text{and} \qquad
	M_1 + M_2 = \id_V
\]
Note that each $M_i$ is a linear operator $V \to V$;
for all we know, it could have hundreds of entries.
Nonetheless, given the multiplication table of the basis $E_i$
we get the corresponding table for the $M_i$.

So, in short, the problem is as follows:
\begin{moral}
	Find all vector spaces $V$ and quadruples of matrices $M_i$
	satisfying the multiplication table above.
\end{moral}

Let $W_1$ be the image of $M_1 : V \to V$ and $W_2$ the image of $M_2 : V \to V$.
\begin{claim}
	$V = W_1 \oplus W_2$.
\end{claim}
\begin{proof}
	First, note that for any $v \in V$ we have
	\[ v = \rho(\id)(v) = (M_1+M_2)v = M_1v + M_2v. \]
	Moreover, we have that $W_1 \cap W_2 = \{0\}$, because if
	$M_1v_1 = M_2v_2$ then $M_1v_1 = M_1(M_1v_1) = M_1(M_2v_2) = 0$.
\end{proof}
\begin{claim}
	$W_1 \cong W_2$.
\end{claim}
\begin{proof}
	Check that the maps $W_1 \taking{\times M_4} W_2$
	and $W_2 \taking{\times M_3} W_1$ are
	well-defined and mutually inverse.
\end{proof}
Now, let $e_1, \dots, e_n$ be basis elements of $W_1$;
thus $M_4e_1$, \dots, $M_4e_n$ are basis elements of $W_2$.
However, each $\{e_j, M_4e_j\}$ forms a basis of a subrepresentation
isomorphic to $X = k^{\oplus 2}$ (what's the isomorphism?).

This finally implies that all representations of $A$
are of the form $W^{\oplus n}$.
In particular, $W$ is irreducible because there are no representations
of smaller dimension at all!

\section\problemhead

\begin{sproblem}
	\label{prob:reg_mat}
	Let $(V, \rho)$ be a representation of $A$.
	Then $\Mat(V)$ is a representation of $A$
	with action given by $a \cdot T = \rho(a) \circ T$, for $T \in \Mat(V)$.
	\begin{enumerate}[(a)]
		\ii Show that $\rho : \Reg(A) \to \Mat(V)$ is an intertwining operator.
		\ii If $V$ is $d$-dimensional, show that $\Mat(V) \cong V^{\oplus d}$
		as representations of $A$.
	\end{enumerate}
	\begin{hint}
		For part (b), pick a basis and do $T \mapsto (T(e_1), \dots, T(e_n))$.
	\end{hint}
\end{sproblem}

\begin{dproblem}
	[Schur's Lemma for Commutative Algebras]
	Let $A$ be a \emph{commutative} algebra over an algebraically closed field $k$.
	Prove that any irreducible representation of $A$ is one-dimensional.
	\begin{hint}
		For any $a \in A$, the map $v \mapsto a \cdot v$ is intertwining.
	\end{hint}
\end{dproblem}

\begin{sproblem}
	\label{prob:regA_intertwine}
	Find all intertwining operators $T : \Reg(A) \to \Reg(A)$,
	where $A$ is a fixed algebra.
\end{sproblem}

\chapter{Semisimple Algebras}
In what follows, assume the field $k$ is algebraically closed.

A \vocab{completely reducible} representation is one which
can be written as a direct sum of irreducible representations.
We like these types of representations, because they don't have the
undesired behavior of being indecomposable while being reducible.

Now, we name the good algebras:
a finite-dimensional algebra $A$ is \vocab{semisimple}
if all its finite-dimensional representations are completely reducible.
The culminating point of the chapter is when we
prove that $A$ is semisimple if and only if
$A \cong \bigoplus_i \Mat(V_i)$,
where the $V_i$ are the irreducible representations of $A$
(yes, there are only finitely many!).

The goal for the chapter is to examine these in more detail:
we'll find conditions on algebras $A$ which cause all its representations
to be completely reducible.


\section{Schur's Lemma Continued}
\prototype{For $V$ irreducible,
	$\Homrep(V^{\oplus 2}, V^{\oplus 2}) \simeq k^{\oplus 4}$.}
\begin{definition}
	For an algebra $A$ and representations $V$ and $W$,
	we let $\Homrep(V,W)$ be the set of intertwining operators between them.
	(It is also a $k$-algebra.)
\end{definition}

By Schur's Lemma, we already know that if $V$ and $W$ are irreducible,
then
\[
	\Homrep(V,W) \simeq
	\begin{cases}
		k & \text{if $V \not\cong W$} \\
		0 & \text{if $V \not\cong W$}.
	\end{cases}
\]
Can we say anything more?
For example, it also tells us that 
\[ \Homrep(V, V^{\oplus 2}) = k^{\oplus 2}. \]
The possible maps are $v \mapsto (c_1v_1, c_2v_2)$ for some choice of $c_1, c_2 \in k$.

More generally, suppose $V$ is irreducible and consider
\[ \Homrep(V^{\oplus m}, V^{\oplus n}). \]
Intertwining operators are determined completely
$T : V^{\oplus m} \to V^{\oplus n}$ by the $mn$ choices of compositions
\begin{diagram}
	V & \rInj & V^{\oplus m} & \rTo^T & V^{\oplus n} & \rSurj & V
\end{diagram}
where the first arrow is inclusion to the $i$th component of $V^{\oplus m}$
(for $1 \le i \le m$) and the second arrow is inclusion to the $j$th
component of $V^{\oplus n}$ (for $1 \le j \le n$).
However, by Schur's Lemma on each of these compositions,
we know they must be constant.

Thus, $\Homrep(V^{\oplus n}, V^{\oplus m})$ consist of $n \times m$ ``matrices''
of constants, and the map is provided by
\[
	\begin{pmatrix}
		c_{11} & c_{12} & \dots & c_{1(n-1)} & c_{1n} \\
		c_{21} & c_{22} & \dots & c_{2(n-1)} & c_{1n} \\
		\vdots & \vdots & \ddots & \vdots & \vdots \\
		c_{m1} & c_{m2} & \dots & c_{m(n-1)} & c_{mn}
	\end{pmatrix}
	\begin{pmatrix} v_1 \\ v_2 \\ \vdots \\ v_n \end{pmatrix}
	\in V^{\oplus n}
\]
where the $c_{ij} \in k$ but $v_i \in V$; note the type mismatch!
This is \emph{not} just a linear map $V^{\oplus n_i} \to V^{\oplus m_i}$;
rather, the outputs are $m$ \emph{linear combinations} of the inputs.

More generally, we have the following result.
\begin{theorem}
	[Schur's Lemma for Completely Reducible Representations]
	\label{thm:compred_schur}
	Let $V$ and $W$ be completely reducible representations,
	and set $V = \bigoplus V_i^{\oplus n_i}$, $W = \bigoplus V_i^{\oplus m_i}$
	for integers $n_i, m_i \ge 0$, where each $V_i$ is irreducible.
	Then
	\[ \Homrep(V, W)
		\simeq \bigoplus_i \Mat_{n_i \times m_i}(k) \]
	meaning that an intertwining operator $T : V \to W$
	amounts to, for each $i$, an $n_i \times m_i$ matrix of constants
	which gives a map $V_i^{\oplus n_i} \to V_i^{\oplus m_i}$.
\end{theorem}

\begin{corollary}
	[Subrepresentations of Completely Reducible Representations]
	\label{cor:subrep_schur}
	Let $V = \bigoplus V_i^{\oplus n_i}$ be completely reducible.
	Then any subrepresentation $W$ of $V$ is isomorphic
	to $\bigoplus V_i^{\oplus m_i}$ where $m_i \le n_i$ for each $i$,
	and the inclusion $W \injto V$ is given
	by the direct sum of inclusion $V_i^{\oplus m_i} \injto V_i^{\oplus n_i}$,
	which are $n_i \times m_i$ matrices.
\end{corollary}
\begin{proof}
	Apply Schur's Lemma to the inclusion $W \injto V$.
\end{proof}



\section{Density Theorem}
We are going to take advantage of the previous result to prove that
finite-dimensional algebras have finitely many irreducible representations.

\begin{theorem}
	[Density]
	Let $V_1$, \dots, $V_r$ be pairwise nonisomorphic
	finite-dimensional representations of $A$.
	Let $(V, \rho)$ be the representation of $A$
	by $A = V_1 \oplus \dots \oplus V_r$.
	Then the map of representations
	\[ \rho : A \to \bigoplus_{i=1}^r \Mat(V_i) \]
	is surjective.
\end{theorem}
\begin{proof}
	First, by \Cref{prob:reg_mat}, we have
	\[ \bigoplus_{i=1}^r \Mat(V_i) \cong \bigoplus_i V_i^{\oplus d_i} \]
	so we can consider this instead; it will be easier to work with.

	First, we handle the case $r = 1$.
	Fix a basis $e_1$, \dots, $e_n$ of $V = V_1$.
	Assuming for contradiction that the map is not surjective.
	Then there is a map of representations (by $\rho$ and the isomorphism)
	$\Reg(A) \to V^{\oplus n}$ given by $a \mapsto (a \cdot e_1, \dots, a \cdot e_n)$.
	By hypothesis is not surjective:
	its image is a \emph{proper} subrepresentation of $V^{\oplus n}$.
	Assume its image is isomorphic to $V^{\oplus m}$ for $m < n$,
	so by \Cref{thm:compred_schur} there is a matrix of constants $X$ with
	\begin{diagram}
		\Reg(A) & \rTo & V^{\oplus n} & \lInj^{X \cdot -} & V^{\oplus r} \\
		a & \rMapsto & (a \cdot e_1, \dots, a \cdot e_n) && \\
		1_A & \rMapsto & (e_1, \dots, e_n) & \lMapsto & (v_1, \dots, v_m)
	\end{diagram}
	where the two arrows in the top row have the same image;
	hence the pre-image $(v_1, \dots, v_m)$ of $(e_1, \dots, e_n)$ can be found.
	But since $r < n$ we can find constants $c_1, \dots, c_n$ not all zero
	such that $X$ applied to the column vector $(c_1, \dots, c_n)$ is zero:
	\[
		\sum_{i=1}^n c_ie_i
		=
		\begin{pmatrix} c_1 & \dots & c_n \end{pmatrix}
		\begin{pmatrix} e_1 \\ \vdots \\ e_n \end{pmatrix}
		=
		\begin{pmatrix} c_1 & \dots & c_n \end{pmatrix}
		X
		\begin{pmatrix} v_1 \\ \vdots \\ v_m \end{pmatrix}
		= 0
	\]
	contradicting the fact that $e_i$ are linearly independent.
	Hence we conclude the theorem for $r=1$.

	As for $r \ge 2$, the image of $\rho$ is necessarily of the form
	$\bigoplus_i V_i^{\oplus r_i}$ (by \Cref{cor:subrep_schur})
	and by the above $r_i = \dim V_i$ for each $i$.
\end{proof}

In particular, this tells us that
\begin{corollary}
	[Finiteness of Number of Representations]
	Any finite-dimensional algebra $A$
	has at most finitely many finite-dimensional irreducible representations.
	\label{cor:finiteness}
\end{corollary}
\begin{proof}
	If $V_i$ are such irreducible representations
	then $A \surjto \bigoplus_i V_i^{\oplus \dim V_i}$,
	hence we have the inequality $\sum (\dim V_i)^2 \le \dim A$.
\end{proof}

\section{Semisimple Algebras}

\begin{definition}
	A finite-dimensional algebra $A$ is a \vocab{semisimple}
	if every finite-dimensional representation of $A$ is completely reducible.
\end{definition}

\begin{theorem}
	[Semisimple Algebras]
	Let $A$ be a finite-dimensional algebra.
	Then the following are equivalent:
	\begin{enumerate}[(i)]
		\ii $A \cong \bigoplus_i \Mat_{d_i}(k)$ for some $d_i$.
		\ii $A$ is semisimple.
		\ii $\Reg(A)$ is completely reducible.
	\end{enumerate}
\end{theorem}
\begin{proof}
	(i) $\implies$ (ii) follows
	from \Cref{thm:rep_1mat} and \Cref{prop:rep_direct_sum}.
	(ii) $\implies$ (iii) is tautological.

	To see (iii) $\implies$ (i), we use the following clever trick.
	Consider
	\[ \Homrep(\Reg(A), \Reg(A)). \]
	On one hand, by \Cref{prob:regA_intertwine},
	it is isomorphic to $A\op$ ($A$ with opposite multiplication),
	because the only intertwining operators $\Reg(A) \to \Reg(A)$
	are those of the form $- \cdot a$.
	On the other hand, suppose that we have set
	$ \Reg(A) = \bigoplus_i V_i^{n_i} $.
	By \Cref{thm:compred_schur}, we have
	\[ A\op \simeq \Homrep(\Reg(A), \Reg(A))
		= \bigoplus_i \Mat_{n_i \times n_i}(k). \]
	But $\Mat_n(k)\op \simeq \Mat_n(k)$ (just by transposing),
	so we recover the desired conclusion.
\end{proof}

In fact, if we combine the above result with \Cref{cor:finiteness},
we also have the following.
\begin{theorem}
	[Sum of Squares Formula]
	For a finite-dimensional algebra $A$ we have
	\[ \sum_{i} \dim(V_i)^2 \le \dim A \]
	where the $V_i$ are the irreducible representations of $A$;
	equality holds exactly when $A$ is semisimple,
	in which case \[ A \cong \bigoplus_i \Mat(V_i). \]
\end{theorem}
\begin{proof}
	The inequality was already mentioned in \Cref{cor:finiteness}.
	It is equality if and only if the map $\rho : A \to \bigoplus_i \Mat(V_i)$
	is an isomorphism; this means all $V_i$ are present.
\end{proof}


\end{document}
