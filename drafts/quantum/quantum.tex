\documentclass[11pt]{scrreprt}
%%fakesection Load packages

\usepackage{lmodern}
\usepackage[pdfusetitle]{hyperref}
\ExplSyntaxOn
\sys_if_engine_luatex:T {
	\usepackage{luatex85}
}
\sys_if_engine_pdftex:T {
	\usepackage[T1]{fontenc}
}
\ExplSyntaxOff

% These are evan.sty
\usepackage{amsmath,amssymb,amsthm}
\usepackage{mathrsfs}
\usepackage[usenames,svgnames,dvipsnames]{xcolor}
\usepackage{textcomp}
\usepackage{enumerate}
\usepackage[textsize=scriptsize,shadow]{todonotes}
\usepackage{mathtools}
\usepackage{microtype}
\usepackage[normalem]{ulem}
\usepackage{stmaryrd}
\usepackage{wasysym}
\usepackage{multirow}
\usepackage{prerex}
\usepackage[nameinlink]{cleveref}
\usepackage{derivative}

%%fakesection evan.sty macros
%Small commands
%% Napkin commands
\newcommand{\prototype}[1]{
	\emph{{\color{red} Prototypical example for this section:} #1} \par\medskip
}
\newenvironment{moral}{%
	\begin{tcolorbox}[boxrule=0.4pt,colframe=green!70!black,sharp corners,
		standard jigsaw,opacityback=0,left=10pt,right=10pt,top=3pt,bottom=3pt,
		before skip=10pt,after skip=20pt]%
	\bfseries\color{green!50!black}}%
	{\end{tcolorbox}}

%%fakesection Links (hyperref loaded earlier implicitly)
\hypersetup{
	linkcolor={red!50!black},
	citecolor={green!50!black},
	urlcolor={blue!80!black},
	pdfkeywords={napkin,math},
	pdfsubject={web.evanchen.cc},
	colorlinks,
}

%%fakesection Commutative diagrams
\usepackage{tikz-cd}
\usetikzlibrary{arrows,arrows.meta}
% make a larger hook
% https://tex.stackexchange.com/questions/514451/how-to-define-a-new-hooked-arrow
\makeatletter
\pgfdeclarearrow{
	name=xGlyph,
	cache=false,
	bending mode=none,
	parameters={\tikzcd@glyph@len,\tikzcd@glyph@shorten},
	setup code={%
		\pgfarrowssettipend{\tikzcd@glyph@len\advance\pgf@x by\tikzcd@glyph@shorten}},
	defaults={
		glyph axis=axis_height,
		glyph length=+1.55ex,
		glyph shorten=+-0.1ex},
	drawing code={%
		\pgfpathrectangle{\pgfpoint{+0pt}{+-1.5ex}}{\pgfpoint{+\tikzcd@glyph@len}{+3ex}}%
		\pgfusepathqclip%
		\pgftransformxshift{+\tikzcd@glyph@len}%
		\pgftransformyshift{+-\tikzcd@glyph@axis}%
		\pgftext[right,base]{\tikzcd@glyph}}}
\makeatother
\tikzcdset{
	arrow style=tikz,
	diagrams={>={Latex}},
	tikzcd left hook/.tip={xGlyph[glyph math command=supset, swap, glyph axis = 5.7pt]},
	tikzcd right hook/.tip={xGlyph[glyph math command=supset, glyph axis = 5.7pt]},
	surjective head arrow /.tip = {tikzcd to[sep=-1.5pt]tikzcd to},
	surjective head/.style={
		-surjective head arrow
	}
}

%%fakesection Page layout
\usepackage[headsepline]{scrlayer-scrpage}
\renewcommand{\headfont}{}
\addtolength{\textheight}{3.14cm}
\setlength{\footskip}{0.5in}
\setlength{\headsep}{10pt}

\def\shortdate{\leavevmode\hbox{\the\year-\twodigits\month-\twodigits\day}}
\def\twodigits#1{\ifnum#1<10 0\fi\the#1}
\automark[chapter]{chapter}

\rohead{\footnotesize\thepage}
\rehead{\footnotesize \textbf{\sffamily Napkin}, by \emph{Evan Chen} (\napkinversion)}
\lehead{\footnotesize\thepage}
\lohead{\footnotesize \leftmark}
\chead{}
\rofoot{}
\refoot{}
\lefoot{}
\lofoot{}
%\cfoot{\pagemark}

%%fakesection Fancy section and chapter heads
\renewcommand*{\sectionformat}{\color{purple}\S\thesection\autodot\enskip}
\renewcommand*{\subsectionformat}{\color{purple}\S\thesubsection\autodot\enskip}
\newcommand{\problemhead}{A few harder problems to think about}
\renewcommand{\thesubsection}{\thesection.\roman{subsection}}

\addtokomafont{chapterprefix}{\raggedleft}
\RedeclareSectionCommand[beforeskip=0.5em]{chapter}
\renewcommand*{\chapterformat}{%
\mbox{\scalebox{1.5}{\chapappifchapterprefix{\nobreakspace}}%
\scalebox{2.718}{\color{purple}\thechapter\autodot}\enskip}}

\addtokomafont{partprefix}{\rmfamily}
\renewcommand*{\partformat}{\color{purple}\scalebox{2.5}{\thepart}}

%%fakesection Theorems
\usepackage{tcolorbox}
\tcbuselibrary{breakable,skins,hooks}

% patch tcolorbox to support continuation of a paragraph
% after box
% from https://tex.stackexchange.com/questions/568782/
\makeatletter
\tcbset{
	after app={%
		\ifx\tcb@drawcolorbox\tcb@drawcolorbox@breakable
		\else
		% add only when not breakabel
		\@endparenv
		\fi
	}
}

% for breakable
\appto\tcb@use@after@lastbox{\@endparenv\@doendpe}
\makeatother
% END patch tcolorbox for continuation of paragraph


\usepackage{thmtools}

\theoremstyle{definition}
\declaretheoremstyle[
	headfont=\sffamily\bfseries\color{MidnightBlue},
	headpunct={\\[3pt]},
	postheadspace={0pt},
]{thmtheorem}


\declaretheoremstyle[
	headfont=\bfseries\color{RawSienna},
	headpunct={\\[3pt]},
	postheadspace={0pt},
]{thmexample}

\tcbset{
	theorem box/.style={
		enhanced,
		arc=9pt,
		outer arc=10pt,
		colframe=blue,
		colback=TealBlue!5,
		boxrule=1pt,
		before skip=12pt,
		after skip=14pt,
		left=10pt,
		right=10pt
	},
	remark box/.style={
		boxrule=0pt,
		frame hidden,
		sharp corners,
		enhanced,
		borderline west={2pt}{0pt}{ForestGreen},
		before skip=8pt,
		colback=ForestGreen!5,
		after skip=12pt,
		breakable,
		left=10pt,
		right=10pt
	},
	example box/.style={
		enhanced,
		sharp corners,
		arc=9pt,
		outer arc=10pt,
		colframe=RawSienna,
		colback=Salmon!5,
		boxrule=0.5pt,
		before skip=12pt,
		after skip=14pt,
		breakable,
		top=6pt,
		bottom=8pt,
		breakable,
		left=10pt,
		right=10pt
	},
	ques box/.style={
		boxrule=0pt,
		frame hidden,
		enhanced,
		sharp corners,
		before skip=8pt,
		after skip=12pt,
		borderline west={3pt}{0pt}{black},
		colback=RedViolet!5!gray!5,
		breakable,
		left=10pt,
		right=10pt
	}
}

\declaretheorem[style=thmtheorem,name=Theorem,numberwithin=section]{theorem}
\tcolorboxenvironment{theorem}{theorem box}
\declaretheorem[style=thmtheorem,name=Lemma,sibling=theorem]{lemma}
\tcolorboxenvironment{lemma}{theorem box}
\declaretheorem[style=thmtheorem,name=Proposition,sibling=theorem]{proposition}
\tcolorboxenvironment{proposition}{theorem box}
\declaretheorem[style=thmtheorem,name=Corollary,sibling=theorem]{corollary}
\tcolorboxenvironment{corollary}{theorem box}
\declaretheorem[style=thmexample,name=Example,sibling=theorem]{example}
\tcolorboxenvironment{example}{example box}

\declaretheoremstyle[
	headfont=\bfseries\sffamily\color{ForestGreen!70!black},
	bodyfont=\normalfont,
	headpunct={ --- },
]{thmremark}
\declaretheoremstyle[
	headfont=\bfseries\sffamily\color{ForestGreen!70!black},
	bodyfont=\normalfont,
	headpunct={},
]{thmremark*}

\declaretheoremstyle[
	headfont=\bfseries,
	bodyfont=\normalfont\small
]{thmques}

\declaretheorem[name=Question,sibling=theorem,style=thmques]{ques}
\tcolorboxenvironment{ques}{ques box}
\declaretheorem[name=Exercise,sibling=theorem,style=thmques]{exercise}
\tcolorboxenvironment{exercise}{ques box}
\declaretheorem[name=Remark,sibling=theorem,style=thmremark]{remark}
\tcolorboxenvironment{remark}{remark box}
\declaretheorem[name=Remark,sibling=theorem,style=thmremark*]{remark*}
\tcolorboxenvironment{remark*}{remark box}
\declaretheorem[name=Step,style=thmremark]{step} % only used in Lebesgue int
\tcolorboxenvironment{step}{remark box}

\theoremstyle{definition}
\newtheorem{claim}[theorem]{Claim}
\newtheorem{definition}[theorem]{Definition}
\newtheorem{fact}[theorem]{Fact}
\newtheorem{abuse}[theorem]{Abuse of Notation}

\newtheorem{problem}{Problem}[chapter]
\renewcommand{\theproblem}{\thechapter\Alph{problem}}
\newtheorem{sproblem}[problem]{Problem}
\newtheorem{dproblem}[problem]{Problem}
\renewcommand{\thesproblem}{\theproblem$^{\star}$}
\renewcommand{\thedproblem}{\theproblem$^{\dagger}$}
\newcommand{\listhack}{$\empty$\vspace{-2em}}

%%fakesection Answers
\usepackage{answers}
\Newassociation{hint}{answeritem}{tex/backmatter/all-hints}
\Newassociation{sol}{answeritem}{tex/backmatter/all-solns}
\renewcommand{\solutionextension}{out}
\renewenvironment{answeritem}[1]{\item[\bfseries #1.]}{}

%%fakesection Table of contents
% First add ToC to ToC
\makeatletter
\usepackage{etoolbox}
\pretocmd{\tableofcontents}{%
	\if@openright\cleardoublepage\else\clearpage\fi
	\pdfbookmark[0]{\contentsname}{toc}%
}{}{}%
\makeatother
\setcounter{tocdepth}{1}
\RedeclareSectionCommand[tocnumwidth=4.2em]{part}
\RedeclareSectionCommand[tocpagenumberwidth=2.2em,tocnumwidth=4.2em]{chapter}
\RedeclareSectionCommand[tocpagenumberwidth=2.2em,tocnumwidth=2.8em]{section}
% adjust tocpagenumberwidth manually for large page number: https://tex.stackexchange.com/a/502168

%%fakesection Asymptote definitions
\usepackage{patch-asy}
\numberwithin{asy}{chapter}
\renewcommand{\theasy}{\thechapter\Alph{asy}}
\begin{asydef}
	import extras;
	size(6cm);
	usepackage("amsmath");
	usepackage("amssymb");
	defaultpen(fontsize(11pt));
	settings.tex = "latex";
	settings.outformat = "pdf";
\end{asydef}
\def\asydir{asy}

%%fakesection Bibliography
\usepackage[backend=biber,backref=true,style=alphabetic]{biblatex}
\DeclareLabelalphaTemplate{
	\labelelement{
		\field[final]{shorthand}
		\field{label}
		\field[strwidth=2,strside=left]{labelname}
	}
	\labelelement{
		\field[strwidth=2,strside=right]{year}
	}
}
\DeclareFieldFormat{labelalpha}{\textbf{\scriptsize #1}}
\addbibresource{references.bib}
\addbibresource{images.bib}
%% stylistic biblatex choices
\DefineBibliographyStrings{english}{%
	backrefpage  = {cited p.}, % for single page number
	backrefpages = {cited pp.} % for multiple page numbers
}
\DeclareFieldFormat{journaltitle}{\mkbibemph{#1},} % italic journal title with comma
\DeclareFieldFormat[inbook,thesis]{title}{\mkbibemph{#1}\addperiod} % italic title with period
\DeclareFieldFormat[article]{title}{#1} % title of journal article is printed as normal text
\DeclareFieldFormat[article]{volume}{\textbf{#1}\addcolon\space}
\renewcommand{\mkbibnamegiven}[1]{\textsc{#1}}
\renewcommand{\mkbibnamefamily}[1]{\textsc{#1}}
\renewcommand{\mkbibnameprefix}[1]{\textsc{#1}}
\renewcommand{\mkbibnamesuffix}[1]{\textsc{#1}}
\renewcommand{\finentrypunct}{}

%%fakesection Mini ToC
\usepackage[tight]{minitoc}
\mtcsetfont{parttoc}{chapter}{\sffamily\bfseries}
\mtcsetfont{parttoc}{section}{\footnotesize\rmfamily\upshape\mdseries}
\mtcsetfont{parttoc}{subsection}{\footnotesize\rmfamily\upshape\mdseries}
%\mtcsetdepth{parttoc}{1}
\setcounter{parttocdepth}{1}
\renewcommand*{\partheadstartvskip}{\vspace*{20em}}
\renewcommand*{\partheadendvskip}{}
%\noptcrule
\renewcommand\beforeparttoc{\noindent{\bfseries \Large Part \thepart: Contents}}
%\hspace{\fill}\rule{0.95\linewidth}{2pt}\hspace{\fill}
\doparttoc[n]

%%fakesection Misc haxx
\pdfstringdefDisableCommands{\def\Spec{\text{Spec }}\def\sigma{σ}}

\input{../../tex/Qcircuit}
\def\asydir{}
\addbibresource{../../references.bib}
\renewcommand{\gim}{($\ast$)}

\begin{document}
\title{Quantum Algorithms}
\maketitle

\chapter{Quantum States and Measurements}
In this chapter we'll explain how to set up quantum states using
linear algebra. This will allow me to talk about quantum \emph{circuits}
in the next chapter, which will set the stage for Shor's algorithm.

I won't do very much physics (read: none at all).
That is, I'll only state what the physical reality is in terms
of linear algebras, and defer the philosophy of why this is true
to your neighborhood ``Philosophy of Quantum Mechanics'' class
(which is a ``social science'' class at MIT!).

\section{Bra-Ket Notation}
Physicists have their own notation for vectors:
whereas I previously used something like $v$, $e_1$, and so on,
in this chapter you'll see the infamous \vocab{bra-ket} notation:
a vector will be denoted by $\ket\bullet$, where $\bullet$ is some
variable name: unlike in math or Python, this can include
numbers, symbols, Unicode characters, whatever you like.
This is called a ``ket''.
To pay a homage to physicists everywhere,
we'll use this notation for this chapter too.

For us:
\begin{definition}
	A \vocab{Hilbert space} $H$ is a finite-dimensional
	complex inner product space.
\end{definition}
\begin{abuse}
	In actual physics, almost all Hilbert spaces used
	are infinite-dimensional, in which case there is 
	the additional requirement that $H$ be a complete metric
	space with respect to its norm. But finite-dimensional spaces
	will suffice for the purposes here
	(and automatically satisfy the completeness property).
\end{abuse}

If $\dim H = n$, then its orthonormal basis elements are often denoted
\[ \ket0, \ket1, \dots, \ket{n-1} \]
(instead of $e_i$)
and a generic element of $H$ denoted by
\[ \ket\psi, \ket\phi, \dots \]
and various other Greek letters.

Now for any $\ket\psi \in H$,
we can consider the canonical dual element in $H^\vee$
(since $H$ has an inner form), which we denote by $\bra\psi$ (a ``bra'').
For example, if $\dim H = 2$ then we can write
\[ \ket\psi = \begin{pmatrix} \alpha \\ \beta \end{pmatrix} \]
in an orthonormal basis, in which case
\[ \bra\psi = \begin{pmatrix} \ol\alpha & \ol\beta \end{pmatrix}. \]
We even can write dot products succinctly in this notation:
if $\ket\phi = \begin{pmatrix} \gamma \\ \delta \end{pmatrix}$,
then the dot product of $\ket\phi$ and $\ket\psi$ is given by
\[
	\braket{\psi|\phi}
	= \cvec{\ol\alpha & \ol\beta} \cvec{\gamma \\ \delta}
	= \ol\alpha\gamma + \ol\beta \delta.
\]
So we will use the notation $\braket{\psi|\phi}$
instead of the more mathematical $\left< \ket\psi, \ket\phi \right>$.
In particular, the squared norm of $\ket\psi$ is just $\braket{\psi|\psi}$.
Concretely, for $\dim H = 2$ we have
$\braket{\psi|\psi} = |\alpha^2| + |\beta|^2$.

\section{The State Space}
If you think that's weird, well, it gets worse.

In classical computation, a bit is either $0$ or $1$.
More generally, we can think of a classical space of $n$
possible states $0$, \dots, $n-1$.
Thus in the classical situation, the space of possible states
is just a discrete set with $n$ elements.

In quantum computation, a \vocab{qubit} is instead
any \emph{complex linear combination} of $0$ and $1$.
To be precise, consider the normed complex vector space
\[ H = \CC^{\oplus 2} \]
and denote the orthonormal basis elements by $\ket0$ and $\ket1$.
Then a \emph{qubit} is a nonzero element $\ket\psi \in H$,
so that it can be written in the form
\[ \ket\psi = \alpha \ket 0 + \beta \ket 1 \]
where $\alpha$ and $\beta$ are not both zero.
Typically, we normalize so that $\ket\psi$ has norm $1$:
\[ \braket{\psi|\psi} = 1 \iff |\alpha|^2 + |\beta|^2 = 1. \]
In particular, we can recover the ``classical'' situation
with $\ket 0 \in H$ and $\ket 1 \in H$,
but now we have some ``intermediate'' states,
such as \[ \frac{1}{\sqrt2} \left(\ket 0 + \ket 1 \right). \]
Philosophically, what has happened is that:
\begin{moral}
	Instead of allowing just the states $\ket 0$ and $\ket 1$,
	we allow any complex linear combination of them.
\end{moral}
More generally, if $\dim H = n$,
then the possible states are nonzero elements
\[ c_0\ket0 + c_1\ket1 + \dots + c_{n-1}\ket{n-1} \]
which we usually normalize so that
$|c_0|^2 + |c_1|^2 + \dots + |c_{n-1}|^2 = 1$.

\section{Observations}
\prototype{$\id$ corresponds to not making a measurement
	since all its eigenvalues are equal,
	but any operator with distinct eigenvalues will cause collapse.}
If you think that's weird, well, it gets worse.
First, some linear algebra:
\begin{definition}
	Let $V$ be a finite-dimensional inner product space.
	For a map $T: V \to V$, the following conditions are equivalent:
	\begin{itemize}
		\ii $\left< Tx, y\right> = \left< x, Ty \right>$
		for any $x,y \in V$.
		\ii $T = T^\dagger$.
	\end{itemize}
	A map $T$ satisfying these conditions is called \vocab{Hermitian}.
\end{definition}
\begin{ques}
	Show that $T$ is normal.
\end{ques}
Thus, we know that $T$ is diagonalizable
with respect to the inner form, so for a suitable basis we
can write it in an orthonormal basis as
\[
	T = \begin{pmatrix}
		\lambda_0 & 0 & \dots & 0 \\
		0 & \lambda_1 & \dots & 0 \\
		\vdots & \vdots & \ddots & \vdots \\
		0 & 0 & \dots & \lambda_{n-1}
	\end{pmatrix}.
\]
As we've said, this is fantastic:
not only do we have a basis of eigenvectors,
but the eignvectors are pairwise orthogonal,
and so they form an orthonormal basis of $V$.
\begin{ques}
	Show that all eigenvalues of $T$ are real.
	($T = T^\dagger$.)
\end{ques}

Back to quantum computation.
Suppose we have a state $\ket\psi \in H$, where $\dim H = 2$;
we haven't distinguished a particular basis yet,
so we just have a nonzero vector.
Then the way observations work (and this is physics, so you'll have to
take my word for it) is as follows:
\begin{moral}
	Pick a Hermitian operator $T : H \to H$;
	then observations of $T$ return eigenvalues of $T$.
\end{moral}
To be precise:
\begin{itemize}
	\ii Pick a Hermitian operator $T : H \to H$,
	which is called the \vocab{observable}.
	\ii Consider its eigenvalues $\lambda_0$, \dots, $\lambda_{n-1}$
	and corresponding eigenvectors $\ket{0}_T$, \dots, $\ket{n-1}_T$.
	Tacitly we may assume that $\ket{0}_T$, \dots, $\ket{n-1}_T$ form
	an orthonormal basis of $H$.
	(The subscript $T$ is here to distinguish the eigenvectors of $T$
	from the basis elements of $H$.)
	\ii Write $\ket\psi$ in the orthonormal basis as
	\[ c_0\ket0_T + c_1\ket1_T + \dots + c_{n-1}\ket{n-1}_T. \]
	\ii Then the probability of observing $\lambda_i$ is
	\[ \frac{|c_i|^2}{|c_0|^2 + \dots + |c_{n-1}|^2}. \]
	This is called making an \vocab{observation along $T$}.
\end{itemize}
Note that in particular, for any nonzero constant $c$,
$\ket\psi$ and $c\ket\psi$ are indistinguishable,
which is why we like to normalize $\ket\psi$.
But the queerest thing of all is what happens to $\ket\psi$:
by measuring it, we actually destroy information.
This behavior is called \vocab{quantum collapse}.
\begin{itemize}
	\ii Suppose for simplicity that we observe $\ket\psi$
	with $T$ and obtain an eigenvalue $\lambda$,
	and that $\ket{i}_T$ is the only eigenvector with this eigenvalue.
	Then, the state $\ket\psi$ \emph{collapses} to just the state
	$c_i \ket{i}_T$: all the other information is destroyed.
	(In fact, we may as well say it collapses to $\ket{i}_T$,
	since again constant factors are not relevant.)

	\ii	More generally, if we observe $\lambda$,
	consider the generalized eigenspace $H_\lambda$, i.e.\
	(i.e.\ the span of eigenvectors with the same eigenvalue).
	Then the physical state $\ket\psi$ has been changed as well:
	it has now been projected onto the eigenspace $H_\lambda$.
	In still other words, after observation, the state collapses to
	\[
		\sum_{\substack{0 \le i \le n \\ \substack{\lambda_i = \lambda}}}
		c_i \ket{i}_T.
	\]
\end{itemize}
In other words,
\begin{moral}
	When we make a measurement,
	the coefficients from different eigenspaces are destroyed.
\end{moral}
Why does this happen? Beats me\dots physics (and hence real life) is weird.
But anyways, an example.
\begin{example}
	[Quantum Measurement of a state $\ket\psi$]
	Let $H = \CC^{\oplus 2}$ with orthonormal basis $\ket0$ and $\ket1$
	and consider the state
	\[
		\ket\psi
		= \frac{i}{\sqrt5} \ket 0
		+ \frac{2}{\sqrt5} \ket 1
		= \pair{i/\sqrt5}{2/\sqrt5} \in H.
	\]
	\begin{enumerate}[(a)]
		\ii Let \[ T = \begin{pmatrix} 1 & 0 \\ 0 & -1 \end{pmatrix}. \]
		This has eigenvectors $\ket0 = \ket0_T$ and $\ket1 = \ket1_T$,
		with eigenvalues $+1$ and $-1$.  So if we measure $\ket\psi$ to $T$,
		we get $+1$ with probability $1/5$ and $-1$ with probability $4/5$.
		After this measurement, the original state collapses to
		$\ket0$ if we measured $0$, and $\ket1$ if we measured $1$.
		So we never learn the original probabilities.

		\ii Now consider $T = \id$, and arbitrarily
		pick two orthonormal eigenvectors $\ket0_T$, $\ket1_T$;
		thus $\psi = c_0\ket0_T + c_1\ket1_T$.
		Since all eigenvalues of $T$ are $+1$,
		our measurement will always be $+1$ no matter what we do.
		But there is also no collapsing,
		because none of the coefficients get destroyed.

		\ii Now consider
		\[ T = \begin{pmatrix} 0 & 7 \\ 7 & 0 \end{pmatrix}. \]
		The two normalized eigenvectors are
		\[ \ket0_T = \frac{1}{\sqrt2}\pair11
		\qquad \ket1_T = \frac{1}{\sqrt2}\pair1{-1} \]
		with eigenvalues $+7$ and $-7$ respectively. In this basis, we have
		\[
			\ket\psi = \frac{2+i}{\sqrt{10}}\ket0_T
			+ \frac{-2+i}{\sqrt{10}}\ket1_T. \]
		So we get $+7$ with probability $\half$ and $-7$
		with probability $\half$, and after the measurement,
		$\ket\psi$ collapses to one of $\ket0_T$ and $\ket1_T$.
	\end{enumerate}
\end{example}
\begin{ques}
	Suppose we measure $\ket\psi$ with $T$ and get $\lambda$.
	What happens if we measure with $T$ again?
\end{ques}

For $H = \CC^{\oplus 2}$ we can come up with more classes of examples using
the so-called \vocab{Pauli matrices}.
These are the three Hermitian matrices
\[
	\sigma_z = \begin{pmatrix} 1 & 0 \\ 0 & -1 \end{pmatrix}
	\qquad
	\sigma_x = \begin{pmatrix} 0 & 1 \\ 1 & 0 \end{pmatrix}
	\qquad
	\sigma_y = \begin{pmatrix} 0 & i \\ -i & 0 \end{pmatrix}.
\]
\begin{ques}
	Show that these three matrices, plus the identity matrix,
	form a basis for the set of Hermitian $2 \times 2$ matrices.
\end{ques}
Their normalized eigenvectors are
\[ \zup = \ket0 = \pair10 \qquad \zdown = \ket1 = \pair01 \]
\[ \xup = \frac{1}{\sqrt2}\pair11
	\qquad \xdown = \frac{1}{\sqrt2}\pair1{-1} \]
\[ \yup = \frac{1}{\sqrt2}\pair1i
	\qquad \ydown = \frac{1}{\sqrt2}\pair1{-i} \]
which we call ``$z$-up'', ``$z$-down'',
``$x$-up'', ``$x$-down'', ``$y$-up'', ``$y$-down''.
So, given a state $\ket\psi \in \CC^{\oplus 2}$
we can make a measurement with respect to any of these three bases
by using the corresponding Pauli matrix.

In light of this, the previous examples were (a) measuring along $\sigma_z$,
(b) measuring along $\id$, and (c) measuring along $7\sigma_x$.

Notice that if we are given a state $\ket\psi$,
and are told in advance that it is either $\xup$ or $\xdown$
(or any other orthogonal states)
then we are in what is more or less a classical situation.
Specifically, if we make a measurement along $\sigma_x$,
then we find out which state that $\ket\psi$ was in (with 100\% certainty),
and the state does not undergo any collapse.
Thus, orthogonal states are reliably distinguishable.

\section{Entanglement}
\prototype{Singlet state: spooky action at a distance.}
If you think that's weird, well, it gets worse.

Qubits don't just act independently:
they can talk to each other by means of a \emph{tensor product}.
Explicitly, consider \[ H = \CC^{\oplus 2} \otimes \CC^{\oplus 2} \]
endowed with the norm described in \Cref{prob:inner_prod_tensor}.
One should think of this as a qubit $A$ in a space $H_A$
along with a second qubit $B$ in a different space $H_B$,
which have been allowed to interact in some way,
and $H = H_A \otimes H_B$ is the set of possible states of \emph{both} qubits.
Thus
\[ 
	\ket{0}_A \otimes \ket{0}_B, \quad
	\ket{0}_A \otimes \ket{1}_B, \quad
	\ket{1}_A \otimes \ket{0}_B, \quad
	\ket{1}_A \otimes \ket{1}_B
\]
is an orthonormal basis of $H$;
here $\ket{i}_A$ is the basis of the first $\CC^{\oplus 2}$
while $\ket{i}_B$ is the basis of the second $\CC^{\oplus 2}$,
so these vectors should be thought of as ``unrelated''
just as with any tensor product.
The pure tensors mean exactly what you want:
for example $\ket0_A \otimes \ket1_B$ means
``$0$ for qubit $A$ and $1$ for qubit $B$''.

As before, a measurement of a state in $H$ requires
a Hermitian map $H \to H$.
In particular, if we only want to measure the qubit $B$ along $M_B$,
we can use the operator \[ \id_A \otimes M_B. \]
The eigenvalues of this operator coincide with the ones for $M_B$,
and the eigenspace for $\lambda$ will be the $H_A \otimes (H_B)_\lambda$,
so when we take the projection the $A$ qubit will be unaffected.

This does what you would hope for pure tensors in $H$:
\begin{example}[Two Non-Entangled Qubits]
	Suppose we have qubit $A$ in the state
	$\frac{i}{\sqrt5}\ket0_A + \frac{2}{\sqrt5}\ket1_A$
	and qubit $B$ in the state
	$\frac{1}{\sqrt2} \ket0_B + \frac{1}{\sqrt2}\ket1_B$.
	So, the two qubits in tandem are represented by the pure tensor
	\[
		\ket\psi
		=
		\left( \frac{i}{\sqrt5}\ket0_A + \frac{2}{\sqrt5}\ket1_A \right)
		\otimes
		\left( \frac{1}{\sqrt2} \ket0_B + \frac{1}{\sqrt2}\ket1_B \right).
	\]
	Suppose we measure $\ket\psi$ along 
	\[ M = \id_A \otimes \sigma_z^B. \]
	The eigenspace decomposition is
	\begin{itemize}
		\ii $+1$ for the span of $\ket0_A \otimes \ket0_B$ and
		$\ket1_A \otimes \ket0_B$, and
		\ii $-1$ for the span of $\ket0_A \otimes \ket1_B$ and
		$\ket1_A \otimes \ket1_B$.
	\end{itemize}
	(We could have used other bases, like $\xup_A \otimes \ket0_B$ and
	$\xdown_A \otimes \ket0_B$ for the first eigenpsace, but it doesn't matter.)
	Expanding $\ket\psi$ in the four-element basis, we find that
	we'll get the first eigenspace with probability
	\[ \left|\frac{i}{\sqrt{10}}\right|^2
	+ \left|\frac{2}{\sqrt{10}}\right|^2 = \half. \]
	and the second eigenspace with probability $\half$ as well.
	(Note how the coefficients for $A$ don't do anything!)
	After the measurement, we destroy the coefficients of the other eigenspace;
	thus (after re-normalization) we obtain the collapsed state
	\[ \left( \frac{i}{\sqrt5}\ket0_A + \frac{2}{\sqrt5}\ket1_A \right)
		\otimes \ket0_B
		\qquad\text{or}\qquad
		\left( \frac{i}{\sqrt5}\ket0_A + \frac{2}{\sqrt5}\ket1_A \right)
		\otimes \ket1_B
	\]
	again with 50\% probability each.
\end{example}
So this model lets us more or less work with the two qubits independently:
when we make the measurement, we just make sure to not touch the other qubit
(which corresponds to the identity operator).

\begin{exercise}
	Show that if $\id_A \otimes \sigma_x^B$ is applied to the $\ket\psi$
	in this example, there is no collapse at all.
	What's the result of this measurement?
\end{exercise}

Since the $\otimes$ is getting cumbersome to write,
we make the following abbreviation.
\begin{abuse}
	From now on $\ket 0_A \otimes \ket 0_B$ will be abbreviated
	to just $\ket{00}$, and similarly for $\ket{01}$, $\ket{10}$, $\ket{11}$.
\end{abuse}

\begin{example}
	[Simultaneously Measuring A General $2$-Qubit State]
	\label{ex:simult_measurement}
	Consider a normalized state $\ket\psi$ in
	$H = \CC^{\oplus 2} \otimes \CC^{\oplus 2}$, say
	\[ \ket\psi = \alpha\ket{00} + \beta\ket{01}
		+ \gamma\ket{10} + \delta\ket{11}. \]
	We can make a measurement along the diagonal matrix
	$T : H \to H$ with 
	\[ T(\ket{00}) = 0\ket{00}, \quad
	T(\ket{01}) = 1\ket{01}, \quad
	T(\ket{10}) = 2\ket{10}, \quad
	T(\ket{11}) = 3\ket{11}. \]
	Thus we get each of the eigenvalues $0$, $1$, $2$, $3$
	with probability $|\alpha|^2$, $|\beta|^2$, $|\gamma|^2$, $|\delta|^2$.
	So if we like we can make ``simultaneous'' measurements on two qubits
	in the same way that we make measurements on one qubit.
\end{example}

However, some states behave very weirdly.
\begin{example}[The Singlet State]
	Consider the state
	\[
		\ket{\Psi_-}
		=
		\frac{1}{\sqrt2} \ket{01}
		- \frac{1}{\sqrt2} \ket{10}
	\]
	which is called the \vocab{singlet state}.
	One can see that $\ket{\Psi_-}$ is not a simple tensor,
	which means that it doesn't just consists of two qubits side by side:
	the qubits in $H_A$ and $H_B$ have become \emph{entangled}.

	Now, what happens if we measure just the qubit $A$?
	This corresponds to making the measurement
	\[ T = \sigma_z^A \otimes \id_B. \]
	The eigenspace decomposition of $T$ can be described as:
	\begin{itemize}
		\ii The span of $\ket{00}$ and $\ket{01}$, with eigenvalue $+1$.
		\ii The span of $\ket{10}$ and $\ket{10}$, with eigenvalue $-1$.
	\end{itemize}
	So one of two things will happen:
	\begin{itemize}
		\ii With probability $\half$, we measure $+1$
		and the collapsed state is $\ket{01}$.
		\ii With probability $\half$, we measure $-1$
		and the collapsed state is $\ket{10}$.
	\end{itemize}
	But now we see that measurement along $A$ has told us what the
	state of the bit $B$ is completely!
\end{example}
By solely looking at measurements on $A$, we learn $B$;
this paradox is called \emph{spooky action at a distance},
or in Einstein's tongue, \vocab{spukhafte Fernwirkung}.  
Thus,
\begin{moral}
	In tensor products of Hilbert spaces,
	states which are not pure tensors
	correspond to ``entangled'' states.
\end{moral}

What this really means is that the qubits cannot be described independently;
the state of the system must be given as a whole.
That's what entangled states mean: the qubits somehow depend on each other.

\section\problemhead

\begin{problem}
	We measure $\ket{\Psi_-}$ by $\sigma_x^A \otimes \id_B$,
	and hence obtain either $+1$ or $-1$.
	Determine the state of qubit $B$ from this measurement.
	\begin{hint}
		Rewrite $\ket{\Psi_-} = -\frac{1}{\sqrt2}
		\left( \xup_A\otimes\xdown_B - \xdown_A\xup_B \right)$.
	\end{hint}
	\begin{sol}
		By a straightforward computation, 
		we have $\ket{\Psi_-} = -\frac{1}{\sqrt2}
		\left( \xup_A\otimes\xdown_B - \xdown_A\xup_B \right)$.
		Now, $\xup_A \otimes \xup_B$, $\xup_A \otimes \xdown_B$
		span one eigenspace of $\sigma_x^A \otimes \id_B$,
		and $\xdown_A \otimes \xup_B$, $\xdown_A \otimes \xdown_B$
		span the other. So this is the same as before:
		$+1$ gives $\xdown_B$ and $-1$ gives $\xdown_A$.
	\end{sol}
\end{problem}

\begin{problem}
	[Greenberger-Horne-Zeilinger Paradox]
	Consider the state in $(\CC^{\oplus2})^{\otimes 3}$
	\[
		\ket{\Psi}_{\text{GHZ}}
		=
		\frac{1}{\sqrt2}
		\left( 
		\ket0_A \ket0_B \ket0_C
		- \ket1_A \ket1_B \ket1_C \right).
	\]
	Find the value of the measurements along each of
	\[ \sigma_y^A \otimes \sigma_y^B \otimes \sigma_x^C , \quad
		\sigma_y^A \otimes \sigma_x^B \otimes \sigma_y^C, \quad
		\sigma_x^A \otimes \sigma_y^B \otimes \sigma_y^C, \quad
		\sigma_x^A \otimes \sigma_x^B \otimes \sigma_x^C.
	\]
	As for the paradox: what happens if you multiply all these measurement together?
	\begin{hint}
		$-1$, $1$, $1$, $1$.
		When we multiply them all together,
		we get that $\id^A \otimes \id^B \otimes \id^C$
		has measurement $-1$, which is the paradox.
		What this means is that the values of the measurements
		are created when we make the observation,
		and not prepared in advance.
	\end{hint}
\end{problem}

\chapter{Quantum Circuits}
Now that we've discussed qubits, we can talk about how to use them in circuits.
The key change --- and the reason that quantum circuits can do things that
classical circuits cannot --- is the fact that we are allowing
linear combinations of $0$ and $1$.

\section{Classical Logic Gates}
In classical logic, we build circuits which take in some bits for input,
and output some more bits for input.
These circuits are built out of individual logic gates.
For example, the \vocab{AND gate} can be pictured as follows.
\[
	\Qcircuit @C=0.8em @R=.7em {
		\lstick{0} &\qw & \multigate{1}{\textsc{and}} & \rstick{0} \qw \\
		\lstick{0} &\qw & \ghost{\textsc{and}} & 
	}
	\hspace{4.5em}
	\Qcircuit @C=0.8em @R=.7em {
		\lstick{0} &\qw & \multigate{1}{\textsc{and}} & \rstick{0} \qw \\
		\lstick{1} &\qw & \ghost{\textsc{and}} & 
	}
	\hspace{4.5em}
	\Qcircuit @C=0.8em @R=.7em {
		\lstick{1} &\qw & \multigate{1}{\textsc{and}} & \rstick{0} \qw \\
		\lstick{0} &\qw & \ghost{\textsc{and}} & 
	}
	\hspace{4.5em}
	\Qcircuit @C=0.8em @R=.7em {
		\lstick{1} &\qw & \multigate{1}{\textsc{and}} & \rstick{1} \qw \\
		\lstick{1} &\qw & \ghost{\textsc{and}} & 
	}
\]
One can also represent the AND gate using the ``truth table'':
\[
	\begin{array}{|cc|c|}
		\hline
		A & B & A \text{ and } B \\ \hline
		0 & 0 & 0 \\
		0 & 1 & 0 \\
		1 & 0 & 0 \\
		1 & 1 & 1 \\
		\hline
	\end{array}
\]
Similarly, we have the \vocab{OR gate} and the \vocab{NOT gate}:
\[
	\begin{array}{|cc|c|}
		\hline
		A & B & A \text{ or } B \\ \hline
		0 & 0 & 0 \\
		0 & 1 & 1 \\
		1 & 0 & 1 \\
		1 & 1 & 1 \\
		\hline
	\end{array}
	\qquad
	\begin{array}{|c|c|}
		\hline
		A & \text{not } A \\ \hline
		0 & 1 \\
		1 & 0 \\
		\hline
	\end{array}
\]
We also have a so-called \vocab{COPY gate}, which duplicates a bit.
\[
	\Qcircuit @C=0.8em @R=.7em {
		\lstick{0} & \qw & \multigate{1}{\textsc{copy}} & \rstick{0} \qw & \\
		&& & \rstick{0} \qw & 
	}
	\qquad\qquad
	\Qcircuit @C=0.8em @R=.7em {
		\lstick{1} & \qw & \multigate{1}{\textsc{copy}} & \rstick{1} \qw & \\
		&& & \rstick{1} \qw & 
	}
\]
Of course, the first theorem you learn about these gates is that:
\begin{theorem}
	[AND, OR, NOT, COPY Are Universal]
	The set of four gates AND, OR, NOT, COPY is universal in the sense that
	any boolean function $f : \{0,1\}^n \to \{0,1\}$ 
	can be implemented as a circuit using only these gates.
\end{theorem}
\begin{proof}
	Somewhat silly: we essentially write down a circuit that OR's across
	all input strings in $f\pre(1)$.
	For example, suppose we have $n=3$ and want to simulate the function
	$f(abc)$ with $f(011) = f(110) = 1$ and $0$ otherwise.
	Then the corresponding Boolean expression for $f$ is simply
	\[
		f(abc) = 
		\left[ \text{(not $a$) and $b$ and $c$} \right]
		\text{ or }
		\left[ \text{$a$ and $b$ and (not $c$)} \right].
	\]
	Clearly, one can do the same for any other $f$,
	and implement this logic into a circuit.
\end{proof}
\begin{remark}
	Since
	$x \text{ and } y = \text{not } ( (\text{not $x$}) \text{ or } (\text{not $y$}))$,
	it follows that in fact, we can dispense with the AND gate.
\end{remark}

\section{Reversible Classical Logic}
\prototype{CNOT gate, Toffoli gate.}

For the purposes of quantum mechanics, this is not enough.
To carry through the analogy we in fact need gates that are \vocab{reversible},
meaning the gates are bijections from the input space to the output space.
In particular, such gates must take the same number of input and output gates.
\begin{example}[Reversible Gates]
	\listhack
	\begin{enumerate}[(a)]
		\ii None of the gates AND, OR, COPY are reversible for dimension reasons.
		\ii The NOT gate, however, is reversible:
		it is a bijection $\{0,1\} \to \{0,1\}$.
	\end{enumerate}
\end{example}
\begin{example}
	[The CNOT Gate]
	The controlled-NOT gate, or the \vocab{CNOT} gate,
	is a reversible $2$-bit gate with the following truth table.
	\[
		\begin{array}{|rr|rr|}
			 \hline
			 \multicolumn{2}{|c|}{\text{In}} & \multicolumn{2}{|c|}{\text{Out}} \\
			 \hline
			 0 & 0 & 0 & 0 \\ 
			 1 & 0 & 1 & 1 \\ 
			 0 & 1 & 0 & 1 \\ 
			 1 & 1 & 1 & 0 \\ \hline
		\end{array}
	\]
	In other words, this gate XOR's the first bit to the second bit,
	while leaving the first bit unchanged.
	It is depicted as follows.
	\[
		\Qcircuit @C=1em @R=.7em {
			\lstick{x} & \ctrl{1} & \rstick{x} \qw \\
			\lstick{y} & \targ & \rstick{x+y \mod 2} \qw 
		}
	\]
	The first dot is called the ``control'',
	while the $\oplus$ is the ``negation'' operation:
	the first bit controls whether the second bit gets flipped or not.
	Thus, a typical application might be as follows.
	\[
		\Qcircuit @C=1em @R=.7em {
			\lstick{1} & \ctrl{1} & \rstick{1} \qw \\
			\lstick{0} & \targ & \rstick{1} \qw
		}
	\]
\end{example}
So, NOT and CNOT are the only nontrivial reversible gates on two bits.

We now need a different definition of universal for our reversible gates.
\begin{definition}
	A set of reversible gates can \vocab{simulate} a Boolean function $f(x_1 \dots x_n)$,
	if one can implement a circuit which takes
	\begin{itemize}
		\ii As input, $x_1 \dots x_n$ plus some fixed bits set to $0$ or $1$,
		called \vocab{ancilla bits}\footnote{%
			The English word ``ancilla'' means ``maid''.}.
		\ii As output, the input bits $x_1, \dots, x_n$,
		the output bit $f(x_1, \dots, x_n)$,
		and possibly some extra bits (called \vocab{garbage bits}).
	\end{itemize}
	The gate(s) are \vocab{universal} if they can simulate any Boolean function.
\end{definition}
For example, the CNOT gate can simulate the NOT gate,
using a single ancilla bit $1$ but with no garbage,
according to the following circuit.
\[
	\Qcircuit @C=1em @R=.7em {
		\lstick{x} & \ctrl{1} & \rstick{x} \qw \\
		\lstick{1} & \targ & \rstick{\text{not } x} \qw
	}
\]
Unfortunately, it is not universal.
\begin{proposition}
	[CNOT $\not\Rightarrow$ AND]
	The CNOT gate cannot simulate the boolean function ``$x \text{ and } y$''.
\end{proposition}
\begin{proof}[Sketch of Proof]
	One can see that any function simulated using only CNOT gates
	must be of the form \[ a_1x_1 + a_2x_2 + \dots + a_nx_n \pmod 2 \]
	because CNOT is the map $(x,y) \mapsto (x, x+y)$.
	Thus, even with ancilla bits, we can only create functions
	of the form $ax+by+c \pmod 2$ for fixed $a$, $b$, $c$.
	The AND gate is not of this form.
\end{proof}

So, we need at least a three-qubit gate.
The most commonly used one is the following.
\begin{definition}
	The three-bit \vocab{Toffoli gate}, also called the CCNOT gate, is given by
	\[
		\Qcircuit @C=1em @R=.7em {
			\lstick{x} & \ctrl{1} & \rstick{x} \qw \\
			\lstick{y} & \ctrl{1} & \rstick{y} \qw \\
			\lstick{z} & \targ & \rstick{z + xy \pmod 2} \qw
		}
	\]
	So the Toffoli has two controls, and toggles the last bit if and only if 
	both of the control bits are $1$.
\end{definition}
This replacement is sufficient.
\begin{theorem}
	[Toffoli Gate is Universal]
	The Toffoli gate is universal.
\end{theorem}
\begin{proof}
	We will show it can reversibly simulate AND, NOT, hence OR,
	which we know is enough to show universality.

	For the AND gate, we draw the circuit
	\[
		\Qcircuit @C=1em @R=.7em {
			\lstick{x} & \ctrl{1} & \rstick{x} \qw \\
			\lstick{y} & \ctrl{1} & \rstick{y} \qw \\
			\lstick{1} & \targ & \rstick{x \text{ and } y} \qw
		}
	\]
	with one ancilla bit, and no garbage bits.

	For the NOT gate, we use two ancilla $1$ bits and two garbage bits:
	\[
		\Qcircuit @C=1em @R=.7em {
			\lstick{1} & \ctrl{1} & \rstick{1} \qw \\
			\lstick{1} & \ctrl{1} & \rstick{1} \qw \\
			\lstick{z} & \targ & \rstick{\text{not } z} \qw
		}
	\]
	This completes the proof.
\end{proof}

Hence, in theory we can create any classical circuit we desire
using the Toffoli gate alone.
Of course, this could require exponentially many gates for even the
simplest of functions.
Fortunately, this is NO BIG DEAL because I'm a math major,
and having $2^n$ gates is a problem best left for the CS majors.

\section{Quantum Logic Gates}
In quantum mechanics, since we can have \emph{linear combinations} of basis
elements, our logic gates will instead consist of \emph{linear maps}.
Moreover, in quantum computation, gates are always reversible,
which was why we took the time in the previous section to show
that we can still do simulate any function when restricted to reversible gates
(e.g.\ using the Toffoli gate).

First, some linear algebra:
\begin{definition}
	Let $V$ be a finite dimensional inner product space.
	Then for a map $U : V \to V$, the following are equivalent:
	\begin{itemize}
		\ii $\left< U(x), U(y) \right> = \left< x,y \right>$ for $x,y \in V$.
		\ii $U^\dagger$ is the inverse of $U$.
		\ii $\norm{x} = \norm{U(x)}$ for $x \in V$.
	\end{itemize}
	The map $U$ is called \vocab{unitary}
	if it satisfies these equivalent conditions.
\end{definition}

Then
\begin{moral}
	Quantum logic gates are unitary matrices.
\end{moral}
In particular, unlike the classical situation,
quantum gates are always reversible
(and hence they always take the same number of input and output bits).

For example, consider the CNOT gate.
Its quantum analog should be a unitary map $\UCNOT : H \to H$,
where $H = \CC^{\oplus 2} \otimes \CC^{\oplus 2}$,
given on basis elements by
\[
	\UCNOT(\ket{00}) = \ket{00}, \quad
	\UCNOT(\ket{01}) = \ket{01}
\]
\[
	\UCNOT(\ket{10}) = \ket{11}, \quad
	\UCNOT(\ket{11}) = \ket{10}.
\]
So pictorially, the quantum CNOT gate is given by
\[
	\Qcircuit @C=0.8em @R=.7em {
		\lstick{\ket0} & \ctrl{1} & \rstick{\ket0} \qw \\
		\lstick{\ket0} & \targ & \rstick{\ket0} \qw \\
	}
	\hspace{6em}
	\Qcircuit @C=0.8em @R=.7em {
		\lstick{\ket0} & \ctrl{1} & \rstick{\ket0} \qw \\
		\lstick{\ket1} & \targ & \rstick{\ket1} \qw \\
	}
	\hspace{6em}
	\Qcircuit @C=0.8em @R=.7em {
		\lstick{\ket1} & \ctrl{1} & \rstick{\ket1} \qw \\
		\lstick{\ket0} & \targ & \rstick{\ket1} \qw \\
	}
	\hspace{6em}
	\Qcircuit @C=0.8em @R=.7em {
		\lstick{\ket1} & \ctrl{1} & \rstick{\ket1} \qw \\
		\lstick{\ket1} & \targ & \rstick{\ket0} \qw \\
	}
\]
OK, so what?
The whole point of quantum mechanics is that we allow linear
qubits to be in linear combinations of $\ket0$ and $\ket1$,
too, and this will produce interesting results.
For example, let's take $\xdown = \frac{1}{\sqrt2} (\ket0-\ket1)$
and plug it into the top, with $\ket 1$ on the bottom, and see what happens:
\[
	\UCNOT \left( \xdown \otimes \ket1 \right)
	= \UCNOT \left( \frac{1}{\sqrt2} (\ket{01}-\ket{11}) \right)
	= \frac{1}{\sqrt2} \left( \ket{01}-\ket{10} \right)
	= \ket{\Psi_-}
\]
which is the fully entanngled \emph{singlet state}! Picture:
\[
	\Qcircuit @C=0.8em @R=.7em {
		\lstick{\xdown} & \ctrl{1} & \rstick{\ket{\Psi_-}} \qw \\
		\lstick{\ket1} & \targ & \rstick{} \qw \\
	}
\]

Thus, when we input mixed states into our quantum gates,
the outputs are often entangled states,
even when the original inputs are not entangled.

\begin{example}
	[More Examples of Quantum Gates]
	\listhack
	\begin{enumerate}[(a)]
		\ii Every reversible classical gate that we encountered before
		has a quantum analog obtained in the same way as CNOT:
		by specifying the values on basis elements.
		For example, there is a quantum Tofolli gate which
		for example sends
		\[
			\Qcircuit @C=0.8em @R=.7em {
				\lstick{\ket1} & \ctrl{1} & \rstick{\ket1} \qw \\
				\lstick{\ket1} & \ctrl{1} & \rstick{\ket1} \qw \\
				\lstick{\ket0} & \targ & \rstick{\ket1} \qw.
			}
		\]
		\ii The \vocab{Hadamard gate} on one qubit is a rotation given by
		\[
			\begin{pmatrix}
				\frac{1}{\sqrt2} & \frac{1}{\sqrt2} \\
				\frac{1}{\sqrt2} & -\frac{1}{\sqrt2}
			\end{pmatrix}.
		\]
		Thus, it sends $\ket0$ to $\xup$ and $\ket1$ to $\xdown$.
		Note that the Hadamard gate is its own inverse.
		It is depicted by an ``$H$'' box.
		\[
			\Qcircuit @C=0.8em @R=.7em {
				\lstick{\ket0} & \gate{H} & \rstick{\xup} \qw
			}
		\]
		\ii More generally, if $U$ is a $2 \times 2$ unitary matrix
		(i.e.\ a map $\CC^{\oplus 2} \to \CC^{\oplus 2}$) then
		there is \vocab{$U$-rotation gate} similar to the previous one,
		which applies $U$ to the input.
		\[
			\Qcircuit @C=0.8em @R=.7em {
				\lstick{\ket\psi} & \gate{U} & \rstick{U\ket\psi} \qw
			}
		\]
		For example, the classical NOT gate is represented by $U = \sigma_x$.
		\ii A \vocab{controlled $U$-rotation gate} generalizes the CNOT gate.
		Let $U : \CC^{\oplus 2} \to \CC^{\oplus 2}$ be a rotation gate,
		and let $H = \CC^{\oplus 2} \otimes \CC^{\oplus 2}$ be a $2$-qubit space.
		Then the controlled $U$ gate has the following circuit diagrams.
		\[
			\Qcircuit @C=0.8em @R=.7em {
				\lstick{\ket0} & \ctrl{1} & \rstick{\ket0} \qw \\
				\lstick{\ket\psi} & \gate{U} & \rstick{\ket\psi} \qw
			}
			\hspace{8em}
			\Qcircuit @C=0.8em @R=.7em {
				\lstick{\ket1} & \ctrl{1} & \rstick{\ket1} \qw \\
				\lstick{\ket\psi} & \gate{U} & \rstick{U\ket\psi} \qw
			}
		\]
		Thus, $U$ is applied when the controlling bit is $1$,
		and CNOT is the special case $U = \sigma_x$.  As before,
		we get interesting behavior if the control is mixed.
	\end{enumerate}
\end{example}

And now, some more counterintuitive quantum behavior.
Suppose we try to use CNOT as a copy, with truth table.
\[
	\begin{array}{|rr|rr|}
		 \hline
		 \multicolumn{2}{|c|}{\text{In}} & \multicolumn{2}{|c|}{\text{Out}} \\
		 \hline
		 0 & 0 & 0 & 0 \\ 
		 1 & 0 & 1 & 1 \\ 
		 0 & 1 & 0 & 1 \\ 
		 1 & 1 & 1 & 0 \\ \hline
	\end{array}
\]
The point of this gate is to be used with a garbage $0$ at the bottom
to try and simulate a ``copy'' operation.
So indeed, one can check that
\[
	\Qcircuit @C=0.8em @R=.7em {
		\lstick{\ket0} & \multigate{1}{U} & \rstick{\ket0} \qw \\
		\lstick{\ket0} & \ghost{U} & \rstick{\ket0} \qw
	}
	\hspace{8em}
	\Qcircuit @C=0.8em @R=.7em {
		\lstick{\ket1} & \multigate{1}{U} & \rstick{\ket1} \qw \\
		\lstick{\ket0} & \ghost{U} & \rstick{\ket1} \qw
	}
\]
Thus we can copy $\ket0$ and $\ket1$.
But as we've already seen if we input $\xdown \otimes \ket0$ into $U$,
we end up with the entangled state $\ket{\Psi_-}$
which is decisively \emph{not} the $\xdown \otimes \xdown$ we wanted.
And in fact, the so-called \vocab{No-Cloning Theorem} implies
that it's impossible to duplicate an arbitrary $\ket\psi$;
the best we can do is copy specific orthogonal states as in the classical case.
See also \Cref{prob:baby_no_clone}.

\section{Deutsch-Jozsa Algorithm}
The Deutsch-Jozsa algorithm is the first example of a nontrivial
quantum algorithm which cannot be performed classically:
it is a ``proof of concept'' that would later inspire Grover's search algorithm
and Shor's factoring algorithm.

The problem is as follows: we're given a function $f : \{0,1\}^n \to \{0,1\}$,
and promised that the function $f$ is either
\begin{itemize}
	\ii A constant function, or
	\ii A balanced function, meaning that exactly half the inputs map to
	$0$ and half the inputs map to $1$.
\end{itemize}
The function $f$ is given in the form of a reversible black box $U_f$ which
is the control of a NOT gate, so it can be represented as the circuit diagram
\[
	\Qcircuit @C=1em @R=0.7em {
		\lstick{\ket{x_1x_2 \dots x_n}} & /^n \qw & \multigate{1}{U_f} &
			\rstick{\ket{x_1x_2\dots x_n}} \qw \\
		\lstick{\ket{y}} & \qw & \ghost{U_f} & \rstick{\ket{y+f(x) \mod 2}}\qw \\
	}
\]
i.e.\ if $f(x_1, \dots, x_n) = 0$ then the gate does nothing,
otherwise the gate flips the $y$ bit at the bottom.
The slash with the $n$ indicates that the top of the input really consists
of $n$ qubits, not just the one qubit drawn,
and so the black box $U_f$ is a map on $n+1$ qubits.

The problem is to determine,
with as few calls to the black box $U_f$ as possible,
whether $f$ is balanced or constant.

\begin{ques}
	Classically, show that in the worst case we may need
	up to $2^{n-1}+1$ calls to the function $f$ to answer the question.
\end{ques}

So with only classical tools, it would take $O(2^n)$ queries to determine
whether $f$ is balanced or constant.
However,
\begin{theorem}
	[Deutsch-Jozsa]
	The Deutsch-Jozsa problem can be determined in a quantum circuit
	with only a single call to the black box.
\end{theorem}
\begin{proof}
	For concreteness, we do the case $n=1$ explicitly;
	the general case is contained in \Cref{prob:deutsch_jozsa}.
	We claim that the necessary circuit is
	\[
		\Qcircuit @C=1em @R=0.7em {
			\lstick{\ket0} & \gate{H} & \multigate{1}{U_f} & \gate{H} & \meter \qw \\
			\lstick{\ket1} & \gate{H} & \ghost{U_f} & \qw & \\
		}
	\]
	Here the $H$'s are Hadamard gates, and meter at the end of the rightmost wire
	indicates that we make a measurement along the usual $\ket0$, $\ket1$ basis.
	This is not a typo! Even though classically the top wire is just
	a repeat of the input information,
	we are about to see that it's the top we want to measure.

	Note that after the two Hadamard operations, the state we get is
	\begin{align*}
		\ket{01} &\xmapsto{H^{\otimes 2}}
		\left( \frac{1}{\sqrt2}(\ket0+\ket1) \right)
		\otimes
		\left( \frac{1}{\sqrt2}(\ket0-\ket1) \right) \\
		&=
		\half \Big( \ket0\otimes\big(\ket0-\ket1\big) 
		\; + \; \ket1\otimes\big(\ket0-\ket1\big) \Big).
	\end{align*}
	So after applying $U_f$, we obtain
	\[
		\half\Big( 
		\ket0\otimes\big(\ket{0+f(0)}-\ket{1+f(0)}\big)
		\; + \; \ket1\otimes\big(\ket{0+f(1)}-\ket{1+f(1)}\big)
		\Big)
	\]
	where the modulo $2$ has been left implicit.
	Now, observe that the effect of going from
	$\ket0-\ket1$ to $\ket{0+f(x)}-\ket{1+f(x)}$ is merely
	to either keep the state the same (if $f(x)=0$)
	or to negate it (if $f(x)=1$).
	So we can simplify and factor to get
	\[
		\half
		\left( (-1)^{f(0)}\ket0 + (-1)^{f(1)}\ket1 \right)
		\otimes
		\left( \ket0-\ket1 \right).
	\]
	Thus, the picture so far is:
	\[
		\Qcircuit @C=1em @R=0.7em {
			\lstick{\ket0} & \gate{H} & \multigate{1}{U_f} &
				\rstick{\frac{1}{\sqrt2}%
				\Big((-1)^{f(0)}\ket0+(-1)^{f(1)}\ket1\Big)} \qw \\
			\lstick{\ket1} & \gate{H} & \ghost{U_f} &
				\rstick{\frac{1}{\sqrt2}(\ket0-\ket1)} \qw
		}
	\]
	In particular, the resulting state is not entangled,
	and we can simply discard the last qubit (!).
	Now observe:
	\begin{itemize}
		\ii If $f$ is constant, then the upper-most state is $\pm\xup$.
		\ii If $f$ is balanced, then the upper-most state is $\pm\xdown$.
	\end{itemize}
	So simply doing a measurement along $\sigma_x$ will give us the answer.
	Equivalently, perform another $H$ gate
	(so that $H\xup = \ket0$, $H\xdown = \ket1$)
	and measuring along $\sigma_z$ in the usual $\ket0$, $\ket1$ basis.
	Thus for $n=1$ we only need a single call to the oracle.
\end{proof}


\section\problemhead
\begin{problem}[Fredkin Gate]
	The \vocab{Fredkin gate} (also called the controlled swap, or CSWAP gate)
	is the three-bit gate with the following truth table:
	\[
	\begin{array}{|rrr|rrr|}
		 \hline
		 \multicolumn{3}{|c|}{\text{In}} & \multicolumn{3}{|c|}{\text{Out}} \\
		 \hline
		 0 & 0 & 0 & 0 & 0 & 0 \\
		 0 & 0 & 1 & 0 & 0 & 1 \\
		 0 & 1 & 0 & 0 & 1 & 0 \\
		 0 & 1 & 1 & 0 & 1 & 1 \\
		 1 & 0 & 0 & 1 & 0 & 0 \\
		 1 & 0 & 1 & 1 & 1 & 0 \\
		 1 & 1 & 0 & 1 & 0 & 1 \\
		 1 & 1 & 1 & 1 & 1 & 1 \\\hline
	 \end{array}
	\]
	Thus the gate swaps the last two input bits whenever the first bit is $1$.
	Show that this gate is also reversible and universal.
	\begin{hint}
		One way is to create CCNOT using a few Fredkin gates.
	\end{hint}
	\begin{sol}
		To show the Fredkin gate is universal
		it suffices to reversibly create a CCNOT gate with it.
		We write the system
		\begin{align*}
			(z,\neg z,-) &= \opname{Fred}(z,1,0) \\
			(x,a,-) &= \opname{Fred}(x,1,0) \\
			(y,b,-) &= \opname{Fred}(y,a,0) \\
			(-,c,-) &= \opname{Fred}(b,0,1) \\
			(-,d,-) &= \opname{Fred}(c, z, \neg z).
		\end{align*}
		Direct computation shows that $d = z+xy\pmod 2$.
	\end{sol}
\end{problem}

\begin{problem}
	[Baby No-Cloning Theorem]
	\label{prob:baby_no_clone}
	Show that there is no unitary map $U$ on two qubits
	which sends $U(\ket\psi \otimes \ket0) = \ket\psi \otimes \ket\psi$
	for any qubit $\ket\psi$, i.e. the following circuit diagram is impossible.
	\[
	\Qcircuit @C=0.8em @R=.7em {
		\lstick{\ket\psi} & \multigate{1}{U} & \rstick{\ket\psi} \qw \\
		\lstick{\ket0} & \ghost{U} & \rstick{\ket\psi} \qw
	}
	\]
	\begin{hint}
		Plug in $\ket\psi=\ket0$, $\ket\psi=\ket1$, $\ket\psi = \xup$
		and derive a contradiction.
	\end{hint}
\end{problem}

\begin{problem}[Deutsch-Jozsa]
	\label{prob:deutsch_jozsa}
	Given the black box $U_f$ described in the Deutsch-Jozsa algorithm,
	consider the following circuit.
	\[
		\Qcircuit @C=1em @R=0.7em {
			\lstick{\ket{0\dots0}} & /^n \qw & \gate{H^{\otimes n}} & \multigate{1}{U_f}
				& \gate{H^{\otimes n}} & \meter \qw \\
			\lstick{\ket1} & \qw & \gate{H} & \ghost{U_f} & \qw & \\
		}
	\]
	That is, take $n$ copies of $\ket 0$, apply the Hadamard rotation to all of them,
	apply $U_f$, reverse the Hadamard to all $n$ input bits
	(again discarding the last bit), then measure all $n$ bits
	in the $\ket0$/$\ket1$ basis (as in \Cref{ex:simult_measurement}).

	Show that the probability of measuring $\ket{0\dots0}$
	is $1$ if $f$ is constant and $0$ if $f$ is balanced.
	\begin{hint}
		First show that the box sends
		$\ket{x_1} \otimes \dots \otimes \ket{x_m} \otimes \xdown$
		to $(-1)^{f(x_1, \dots, x_m)}
		(\ket{x_1} \otimes \dots \otimes \ket{x_m} \otimes \xdown)$.
	\end{hint}
	\begin{sol}
		Put $\xdown = \frac{1}{\sqrt2} (\ket0-\ket1)$.
		Then we have that $U_f$ sends
		\[
			\ket{x_1}  \dots  \ket{x_m}  \ket 0 
			- \ket{x_1}  \dots  \ket{x_m}  \ket 1 
			\xmapsto{U_f}
			\pm \ket{x_1}  \dots  \ket{x_m}  \ket 0 
			\mp \ket{x_1}  \dots  \ket{x_m}  \ket 1 
		\]
		the sign being $+$, $-$ exactly when $f(x_1, \dots, x_m) = 1$.

		Now, upon inputting $\ket0 \dots \ket0 \ket1$, we find that $H^{\otimes m+1}$ maps it to
		\[ 2^{-n/2} \sum_{x_1, \dots, x_n} \ket{x_1} \dots \ket{x_n} \xdown.  \]
		Then the image under $U_f$ is
		\[ 2^{-n/2} \sum_{x_1, \dots, x_n} (-1)^{f(x_1, \dots, x_n)} \ket{x_1} \dots \ket{x_n} \xdown.  \]
		We now discard the last qubit, leaving us with
		\[ 2^{-n/2} \sum_{x_1, \dots, x_n} (-1)^{f(x_1, \dots, x_n)} \ket{x_1} \dots \ket{x_n}.  \]
		Applying $H^{\otimes m}$ to this, we get
		\[ 2^{-n/2} \sum_{x_1, \dots, x_n} (-1)^{f(x_1, \dots, x_n)}
			\cdot
			\left(
			2^{-n/2}
			\sum_{y_1, \dots, y_n}
			(-1)^{x_1y_1 + \dots + x_ny_n}
			\ket{y_1} \ket{y_2} \dots \ket{y_n}
			\right)
		\]
		since $H\ket0 = \frac{1}{\sqrt2}(\ket0+\ket1)$
		while $H\ket1 = \frac{1}{\sqrt2}(\ket0-\ket1)$,
		so minus signs arise exactly if $x_i = 0$ and $y_i = 0$ simultaneously,
		hence the term $(-1)^{x_1y_1 + \dots + x_ny_n}$.
		Swapping the order of summation, we get
		\[ 
			2^{-n}
			\sum_{y_1, \dots, y_n}
			C(y_1, \dots, y_n)
			\ket{y_1} \ket{y_2} \dots \ket{y_n}
		\]
		where $C_{y_1, \dots, y_n}  = \sum_{x_1, \dots, x_n} (-1)^{f(x_1, \dots, x_n)+x_1y_1 + \dots + x_ny_n}$.
		Now, we finally consider two cases.
		\begin{itemize}
			\ii If $f$ is the constant function, then we find that
			\[
				C(y_1, \dots, y_n) = 
				\begin{cases}
					\pm 1 &  y_1 = \dots = y_n = 0 \\
					0 & \text{otherwise}.
				\end{cases}
			\]
			To see this, note that the result is clear for $y_1 = \dots = y_n = 0$;
			otherwise, if WLOG $y_1 = 1$, then the terms for $x_1 = 0$ exactly cancel
			the terms for $x_1 = 0$, pair by pair.
			Thus in this state, the measurements all result in $\ket0 \dots \ket0$.

			\ii On the other hand if $f$ is balanced, we derive that
			\[ C(0, \dots, 0) = 0. \]
			Thus \emph{no} measurements result in $\ket 0 \dots \ket 0$.
		\end{itemize}
		In this way, we can tell whether $f$ is balanced or not.
	\end{sol}
\end{problem}

\begin{dproblem}[Barenco et al, 1995; arXiv:quant-ph/9503016v1]
	Let
	\[
		P = \begin{pmatrix} 1 & 0 \\ 0& i \end{pmatrix} 
		\qquad
		Q = \frac{1}{\sqrt2}\begin{pmatrix} 1 & -i \\ -i & 1 \end{pmatrix}
	\]
	Verify that the quantum Toffoli gate can be implemented
	using just controlled rotations via the circuit
	\[
		\Qcircuit @R=1em @C=0.7em {
			\lstick{\ket{x_1}} & \qw & \ctrl{2} & \ctrl{1} & \ctrl{1} & \qw & \ctrl{1} & \qw \\
			\lstick{\ket{x_2}} & \ctrl{1} & \qw & \gate{P} & \targ & \ctrl{1} & \targ & \qw \\
			\lstick{\ket{x_3}} & \gate{Q} & \gate{Q} & \qw & \qw & \gate{Q^\dagger} & \qw & \qw
		}
	\]
	This was a big surprise to researchers when discovered,
	because classical reversible logic requires three-bit gates (e.g. Toffoli, Fredkind).
	\begin{hint}
		This is direct computation.
	\end{hint}
\end{dproblem}



\chapter{Shor's Algorithm}

\end{document}
